
\section{MET Templates}
\label{sec:templates}

The premise of this data driven technique is that MET in \Z plus jets events
is produced by the hadronic recoil system and {\it not} by the leptons making up the \Z.
Therefore, the basic idea of the MET template method is to measure the MET distribution in 
a control sample which has no true MET and the same general attributes regarding
fake MET as in \Z plus jets events.
In our case, we choose a photon-like sample. These are not necessarily photons, but jets with predominantly 
electromagnetic energy deposition in a good fiducial volume. This ensures that 
they are well measured and do not contribute to fake MET.

Both the control sample and the \Z plus jets background
consist of a well measured object (either a photon or a leptonically decaying \Z) which recoils 
against a system of hadronic jets. The MET in these events 
is then dominated by mismeasurements of the hadronic system. To account for kinematic 
differences between the hadronic systems in the control vs. signal samples, 
we measure the MET distributions in the control sample in bins of the number of jets and the 
scalar sum of jet \pt. 

In order to track conditions which change over the course of data-taking
(most notably the increased pile-up) and to increase the statistics at large photon \pt, we
use templates obtained from multiple photon triggers ``stitched'' together. 
Each photon event enters the template for the
highest \pt photon trigger which fired in the event. The resulting MET 
distributions in each bin of njets, sumjetpt and photon trigger are then 
normalized to unit area, yielding an array of MET templates. 

Each \Z event is then assigned one such 
unit area template based on its number of jets, the scalar sum of jet \pt and the \Z \pt. 
The \Z \pt enters only in choosing which bin of photon trigger to use. 
This is done because the \Z \pt is the analogous variable in the \Z sample 
to the photon \pt. 
In order to account for the different prescales of the different photon triggers (which 
change during running), 
each template is filled with a weight equal to the prescale of the corresponding trigger.
After this procedure, the templates are normalized to unity.
If this is not done, the relative contribution of different parts of the photon
\pt specturm will not be sampled properly. In other words, this procedure corrects and 
smoothes the photon \pt distribution.% so that it matches the \Z \pt distribution.
The ranges of \Z \pt chosen for each photon trigger 
take into account the turn-on of the photon trigger and are shown below.
%{\bf Claudio: currently the Z pt is used to determine which trigger template set to use. You argued earlier that instead
%\bf we should use the pt of the hadronic recoil. I have verified that this makes a negligible difference, but if you prefer,
%\bf I can switch to using hadronic recoil pt.} 

%Below was commented out, but since I notice this procedure has a large effect due to the large prescale, I think it's important to include this text.

%In order to select a photon sample whose \pt distribution matches the distribution of \Z \pt
% in our preselection, a combination of photon triggers is used.
% must be used (see below). 
%In order to account for varying prescales of the triggers in different runs,
% separate templates are created for each trigger line. 
%Each event enters the template for the highest \pt trigger which fires in the event,
% and the template is filled with a weight equal to the prescale of the corresponding trigger.
%After this procedure, the templates are normalized to unity.

The sum of the templates for all selected \Z events then forms the 
prediction of the MET distribution for the \Z sample. Integrating this prediction for our 
signal regions  thus provides a data driven prediction for the \Z plus jets yields in the 
signal regions. 
%The prediction is then formed by selecting the photon trigger based on the \pt of the \Z in the event which is being predicted

While there is in principle a small contribution from \ttbar in the \Z sample from 
which we predict the MET distribution, it is only $\approx$2.6\% of the total sample used
(MC expectation),
as shown in table \ref{preselyieldtable}, and the data driven prediction (see Fig. 
\ref{fig:pfmet_eemm}) estimates that the \ttbar contribution to the loose signal 
region is $\approx$0.3\% of the total \Z yield. As the MET measurement in these events 
does not enter the MET prediction, this small non-Z contribution is negligible.

%The MET templates are derived from a set of photon triggered samples. 
The Njet binning used is 2 jets and $\ge$ 3 jets. 
The sum jet \pt binning is defined by the boundaries {0, 30, 60, 90, 120, 150, 250, 5000}~GeV.
The photon triggers, and the ranges of \Z \pt for which they are selected, are (where the ``\verb=v*='' stands for a verion number which increments with trigger menu changes):

\begin{itemize}
\item \verb=HLT_Photon20_CaloIdVL_IsoL_v*= (\Z \pt $<$ 33~GeV)
\item \verb=HLT_Photon30_CaloIdVL_IsoL_v*= (33 GeV $<$ \Z \pt $<$ 55~GeV)
\item \verb=HLT_Photon50_CaloIdVL_IsoL_v*= (55 GeV $<$ \Z \pt $<$ 81~GeV)
\item \verb=HLT_Photon75_CaloIdVL_IsoL_v*= (\Z \pt $>$ 81~GeV)
\end{itemize}

The HLT requirements (as described in \cite{ref:eghlt}) on the above triggers are listed in appendix~\ref{app:photrig}.

%\begin{itemize}
%\item CaloIdVL: \\
%  H/E $<$ 0.15 (0.1), $\sigma_{i\eta i\eta} <$ 0.024 (0.04) in barrel (endcap)
%\item IsoL: \\
%  Ecal ET $<$ 5.5 + 0.012*ET, %\\
%  Hcal ET $<$ 3.5 + 0.005*ET, %\\
%  Trk PT $<$ 3.5 + 0.002*ET 
%\end{itemize}

%last year for the thresholds we used trigger value +2. This year I observe a bit slower turn-on:

%  //notes on turn on curves from drawing photon pts from data babies
%  //the 20 trig is nearly fully efficient at 21, but use 22 to be conservative
%  //the 30 trig is nearly fully efficient at 32, but use 33 to be conservative
%  //the 50 trig is nearly fully efficient at 54, but use 55 to be conservative
%  //the 75 trig is nearly fully efficient at 80, but use 81 to be conservative

We show all the templates used in appendix~\ref{sec:appendix_templates}.
