\section{MET Templates}
\label{sec:templates}

The background from SM $Z$ production accompanied by artificial MET from detector mismeasurements
is estimated using the novel MET templates technique.
The premise of this data driven technique is that MET in \Z plus jets events
is produced by the hadronic recoil system and {\it not} by the leptons making up the \Z.
Therefore, the MET distribution in these events can be modeled using a control sample
which has no true MET and the same general attributes regarding fake MET as in \Z plus jets events.
We use two complementary control samples: one consisting of photon plus jets events, and one
consisting of QCD multijet events. 

Photon plus jets events are selected from a set of single photon
triggers with varying photon \pt\ thresholds, and QCD multijet events are selected from a set
of single jet triggers with varying jet \pt\ thresholds. 
In the \Z plus jets events as well as in both control samples, the MET is dominated by mismeasurements
of the hadronic system. To account for kinematic 
differences between the hadronic systems in the control vs. signal samples, 
we measure the MET distributions in the control samples in bins of the number of jets and the 
scalar sum of the transverse energies of the jets (\Ht), separately for each of the single photon and single jet trigger thresholds.
Each MET distribution is normalized to unit area, yielding an
array of MET templates. Each \Z event is then assigned one such unit area template based on its number of jets and \Ht.
The trigger threshold is chosen based on the \Z \pt\ (leading jet \pt) for the photon plus
jets (QCD multijets) control sample.
The sum of the templates for all selected \Z events then forms the 
prediction of the MET distribution for the \Z sample. Integrating this prediction for our 
signal regions  thus provides a data driven prediction for the \Z plus jets yields in the 
signal regions. 

We use 2 separate control samples, since each has relative advantages. The photon plus jets events have a topology
which is more similar to the \Z plus jets events, since both consist of a well-measured
object recoiling against a system of hadronic jets. Possible contributions of the photon
to the MET in the event are eliminated in the QCD multijet sample. We have verified that
the predictions from the 2 control samples are consistent within their uncertainties, and
choose to use the prediction from the photon plus jets sample.
By using two independent control samples, we are able to illustrate
the robustness of the MET templates method and to cross check the data driven background 
prediction.

The systematic uncertainty in the prediction from the photon plus jets MET templates
originates from 2 sources. Possible contributions to the MET from the photon are assessed
by varying the photon selection, which leads to a relative difference in the predicted
background of 15\%. The effect of the difference between the distributions of hadronic recoil \pt\
in the control vs. signal samples is estimated by reweighting the photon plus jets events such
that the hadronic recoil \pt distribution matches that in \Z plus jets events, leading to a relative
difference in the predicted background of 20\%. The total uncertainty in the predicted background
from the MET templates method is 25\%.
