\section{MET Templates}
\label{sec:templates}

The background from SM $Z$ production accompanied by artificial MET from detector mismeasurements
is estimated using the MET templates technique.
The premise of this data driven technique is that MET in \Z plus jets events
is produced by the hadronic recoil system and {\it not} by the leptons making up the \Z.
Therefore, the MET distribution in these events can be modeled using a control sample
which has no true MET and is accompanied by a similar hadronic system as in \Z plus jets events.
We use two complementary control samples: one consisting of photon plus jets events, and one
consisting of QCD multijet events. 

In the \Z plus jets events as well as in both control samples, the MET is dominated by mismeasurements
of the hadronic system. To account for kinematic 
differences between the hadronic systems in the control vs. signal samples, 
the expected MET distribution for a \Z plus jets event is obtained from the MET distribution
for photon plus jets or QCD multijet events of the same jet multiplicity and scalar sum 
of jet transverse energies.
We use 2 separate control samples, since each has relative advantages. The photon plus jets events have a topology
which is more similar to the \Z plus jets events, since both consist of a well-measured
object recoiling against a system of hadronic jets. Possible contributions of the photon
to the MET in the event are eliminated in the QCD multijet sample. We have verified that
the predictions from the 2 control samples are consistent within their uncertainties, and
choose to use the prediction from the photon plus jets sample.
By using two independent control samples, we are able to illustrate
the robustness of the MET templates method and to cross check the data driven background 
prediction.

The systematic uncertainty in the prediction from the photon plus jets MET templates
originates from 2 sources. Possible contributions to the MET from the photon are assessed
by varying the photon selection, which leads to a relative difference in the predicted
background of 15\%. The effect of the difference between the distributions of hadronic recoil \pt\
in the control vs. signal samples is estimated by reweighting the photon plus jets events such
that the hadronic recoil \pt distribution matches that in \Z plus jets events, leading to a relative
difference in the predicted background of 20\%. The total uncertainty in the predicted background
from the MET templates method is 25\%.
