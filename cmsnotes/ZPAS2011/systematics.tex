\section{Acceptance and Efficiency Systematic Uncertainties}
\label{sec:systematics}

The acceptance and efficiency, as well as the systematic uncertainties in these quantities, 
depend on the signal model.
For some of the individual uncertainties, it is reasonable to quote values 
based on SM control samples with kinematic properties similar to the SUSY benchmark models. 
For others that depend strongly on the kinematic properties of the event, the systematic
uncertainties must be quoted model by model.

The systematic uncertainty in the lepton acceptance consists
of two parts: the trigger efficiency uncertainty and the 
identification and isolation uncertainty. The trigger efficiency 
for two leptons of $\pt>10\GeVc$, with one lepton of 
$\pt>20\GeVc$ is measured using samples of $Z \to \ell\ell$, 
with an uncertainty of 2\%. We verify that the MC reproduces the lepton identification and isolation efficiencies in data using
samples of $Z \to \ell\ell$; the data and MC efficiencies are found to be consistent within 2\%.

Another significant source of systematic uncertainty is 
associated with the jet and MET energy scale.  The impact
of this uncertainty is final-state dependent.  Final
states characterized by very large MET are 
less sensitive than final states where the MET
is typically close to the minimum requirement applied to this quantitys.  To be more quantitative,
we have used the method of Ref.~\cite{ref:top} to evaluate
the systematic uncertainties in the acceptance for $t\bar{t}$ 
and for the two benchmark SUSY points using a 5\% uncertainty in the hadronic 
energy scale~\cite{ref:jes}.
The signal efficiency uncertainty for the region MET $>$ 100 GeV is 16\%, 3\% and 4\% for \ttbar, LM4 and LM8, respectively.
The signal efficiency uncertainty for the region MET $>$ 200 GeV is 33\%, 8\% and 9\% for \ttbar, LM4 and LM8, respectively.

The uncertainty in the integrated luminosity is 4\%~\cite{ref:lumi}.
