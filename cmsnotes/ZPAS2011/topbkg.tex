\section{Top Background Estimation}
\label{sec:topbkg}

The \ttbar\ contribution to the signal region is estimated using an opposite-flavor (OF) subtraction technique.
This technique takes advantage of the fact that the \ttbar\ yield in the 
OF final state ($e\mu$) is the same as in the same-flavor (SF) final state
($ee+\mu\mu$), modulo differences in efficiency in the $e$ vs. $\mu$ selection.
Hence the \ttbar\ yield in the same-flavor final state can be estimated
using the corresponding yield in the opposite-flavor final state. 
Other backgrounds for which the lepton flavors are
uncorrelated (for example, $W^+W^-$ and DY$\rightarrow \tau^+\tau^-$) will also be included in
this estimate.

To predict the SF yield in a signal region defined by a requirement on the MET, we take the 
OF yield passing the same MET requirement. This yield is corrected for using the ratio of
muon to electron selection efficiencies $\epsilon_{\mu e}=1.07 \pm 0.03$.
This quantity is evalued as the square root of the ratio of $Z\to\mu^+\mu^-$ to $Z\to e^+e^-$
events ind data, with no jets or MET requirements. To improve the statistical precision
of the background estimate, we do not require the OF events to lie in the $Z$ mass region,
and we apply a scale factor $K=0.16 \pm 0.01$ accounting for the fraction of \ttbar\ events
which lie in the region $81 < \mathrm{M(\ell\ell)} < 101$\GeVcc, extracted from MC.

Backgrounds from pair production of vector bosons are negligible compared to $t\bar{t}$.
Backgrounds from fake leptons are negligible due to the requirement of two \pt$ > 20$~GeV leptons
in the \Z mass window, accompanied by jets and large MET.
