
\section{Introduction}

In this note we describe a search for physics beyond the standard model (BSM) 
in a sample of proton-proton collisions at a centre-of-mass energy of 7~TeV. 
The data sample was collected with the Compact Muon Solenoid (CMS) detector~\cite{JINST} at 
the Large Hadron Collider (LHC) in 2011
and corresponds to an integrated luminosity of \lumi.
We search for new physics in events containing
opposite sign isolated lepton pairs ($ee$, $e\mu$, and $\mu\mu$).
The main sources of high \pt\ isolated dileptons at CMS are Drell Yan and \ttbar.
Here we concentrate on dileptons with invariant mass consistent
with $Z \to ee$ and $Z \to \mu\mu$.  A separate search for new physics in the non-\Z
sample is described in~\cite{ref:ospaper}.

We search for new physics in the final state of \Z plus two or more jets plus missing 
transverse energy (MET). We reconstruct the \Z boson
in its decay to $e^+e^-$ or $\mu^+\mu^-$. Our signal regions are defined as 
MET $>$ \signalmetl~GeV (loose signal region) and MET $>$ \signalmett~GeV 
(tight signal region). We use data driven techniques to predict the
standard model (SM) backgrounds in these search regions. 
Contributions from Drell-Yan production combined with detector mis-measurements that 
produce fake MET are modeled via the MET templates  method~\cite{ref:templates1,ref:templates2}
based on control samples of photon plus jets and QCD multijets events.
Top pair production backgrounds, as well as other backgrounds for which the lepton
flavors are uncorrelated such as $W^+W^-$ and DY$\rightarrow\tau^+\tau^-$, are 
modeled via $e^\pm\mu^\mp$ subtraction.

As leptonically decaying \Z bosons provide a signature that has very little background, 
they provide a clean final state in which to search for new physics. 
Because new physics is expected to be connected to the SM Electroweak sector, 
it is likely that new particles will couple to W and Z bosons. 
For example, in mSUGRA, low $M_{1/2}$ can lead to a significant branching fraction 
for $\chi_2^0 \rightarrow Z \chi_1^0$. 
In addition, we are motivated by the existence of dark matter to search for new physics with MET.
Enhanced MET is a feature of many new physics scenarios, and R-parity conserving SUSY 
again provides a popular example. The main challenge of this search is therefore to 
understand the tail of the fake MET distribution in \Z plus jets events.

%The search is not optimized in the context of any particular BSM physics, e.g. specific SUSY model.
No specific BSM  physics scenario, e.g.\ a particular  SUSY model, has
been  used  to  optimize  the  search.  In  order  to  illustrate  the
sensitivity of  the search, a  simplified and practical model  of SUSY
breaking,  the  constrained minimal  supersymmetric  extension of  the
standard  model  (CMSSM)~\cite{CMSSM,CMSSM2}, is  used.  The CMSSM  is
described by  five parameters: the  universal scalar and  gaugino mass
parameters   ($m_0$  and   $m_{1/2}$,  respectively),   the  universal
trilinear soft SUSY breaking parameter  $A_0$, the ratio of the vacuum
expectation values  of the two  Higgs doublets ($\tan\beta$),  and the
sign of  the Higgs mixing  parameter $\mu$.  Throughout the  note, two
CMSSM parameter sets, referred to  as LM4 and LM8~\cite{TDR}, are used
to  illustrate possible  CMSSM yields.   In these  scenarios, opposite
sign  leptons are  produced  in  the decays  of $Z$  bosons, which  are
produced  in  the  cascade  decays  of heavy,  colored  objects.   The
parameter  values defining  LM4  (LM8) are  $m_0  = 210~(500)$\GeVcc,
$m_{1/2} = 285~(300) \GeVcc$,  $A_0 = 0~(-300)\GeV$; both LM4 and
LM8 have $\tan\beta = 10$ and $\mu  > 0$. 
In this analysis, the  LM4 and LM8 scenarios serve as benchmarks
which may  be used to allow  comparison of the  sensitivity with other
analyses.
