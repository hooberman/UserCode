
In order to search for a possible signal from stop decays giving rise to a signature of \ttbar\ with additional \met\
from the LSPs, it is necessary to determine the composition of the SM backgrounds in the signal region. 
This section details the methods pursued to estimate the background in the signal sample and describes the 
procedure to estimate the systematic uncertainties. The general strategy is to use the MC prediction for the 
backgrounds after applying corrections derived from data. 

The most important background to a stop signal arises from SM \ttbar. The \ttbar\ background may be 
separated into contributions containing a single lepton \ttlj\ and two leptons \ttll. As described in this section, 
the \ttll\ background is the dominant process satisfying the event selection, contributing $\sim 80\%$ of the 
signal sample defined with the benchmark selection of \met\ $>$ 100~\GeV and \mt\ $>$ 150~\GeV. This 
background has large true \met\ and consequently larger \mt\ due to the presence of two neutrinos.
Additional contributions to the single lepton sample arise from \wjets\ and single top. The combination of 
all single lepton backgrounds, \ttlj, \wjets\ and single top, comprises $\sim 15\%$ of the signal sample. 
Finally, other background sources such as dibosons, \dy\ + jets, in addition to rarer processes such as \ttbar\ 
produced in association with a vector boson and tribosons, provide a combined contribution to the signal sample 
at the level of $\sim 5\%$.
Finally, the QCD background contribution is small, particularly in the signal sample, with a large \met\ requirement.

