\section{Overview and Analysis Strategy}
\label{sec:overview}

We are searching for a $t\bar{t}\chi^0\chi^0$ or $W \ell b W \ell \bar{b} \chi^0 \chi^0$ final state
(after top decay in the first mode, the final states are actually the same).  So to first order 
this is ``$t\bar{t} +$ extra \met''.  

We work in the $\ell +$ jets final state, where the main background is $t\bar{t}$.  We look for 
\met inconsistent with $W \to \ell \nu$.  We do this by concentrating on the $\ell \nu$ transverse
mass ($M_T$), since except for resolution effects, $M_T < M_W$ for $W \to \ell \nu$.  Thus, the
initial analysis is simply a counting experiment in the tail of the $M_T$ distribution.  

The event selection is one-and-only-one high $P_T$ isolated lepton, four or more jets, and
some moderate \met cut.  At least one of the jets has to be btagged to reduce $W+$ jets.
The event sample is then dominated by $t\bar{t}$, but there are also contributions from $W+$ jets,
single top, dibosons, etc.

In order to be sensitive to $\widetilde{t}\widetilde{t}$ production, the background in the $M_T$
tail has to be controlled at the level of 10\% or better. So this is (almost) a precision measurement.

The $t\bar{t}$ events in the $M_T$ tail can be broken up into two categories: 
(i) $t\bar{t} \to \ell $+ jets and (ii) $t\bar{t} \to \ell^+ \ell^-$ where one of the two
leptons is not found by the second-lepton-veto (here the second lepton can be a hadronically
decaying $\tau$).
 For a reasonable $M_T$ cut, say $M_T >$ 150 GeV, the dilepton background is of order 80\% of 
the total.  This is because in dileptons there are two neutrinos from $W$ decay, thus $M_T$
is not bounded by $M_W$.  This is a very important point: while it is true that we are looking in
the tail of $M_T$, the bulk of the background events end up there not because of some exotic
\met reconstruction failure, but because of well understood physics processes.  This means that 
the background estimate can be taken from Monte Carlo (MC), after carefully accounting for possible
data/MC differences.  Sophisticated fully ``data driven'' techniques are not really needed.

Another important point is that in order to minimize systematic uncertainties, the MC background
predictions are always normalized to the bulk of the $t\bar{t}$ data, ie, events passing all of the 
requirements but with $M_T \approx 80$ GeV.
This removes uncertainties
due to $\sigma(t\bar{t})$, lepton ID, trigger efficiency, luminosity, etc.   

The $\ell +$ jets background, which is dominated by 
$t\bar{t} \to \ell $+ jets, but also includes some $W +$ jets as well as single top,
is estimated as follows:
\begin{enumerate}
\item We select a control sample of events passing all cuts, but anti-btagged.  This is 
sample is now dominated by $W +$ jets.  The sample is used to understand the
$M_T$ tail in $\ell +$ jets processes. 
\item In MC we measure the ratio of the number of $\ell +$ jets events in the $M_T$ tail to
the number of events with $M_T \approx$ 80 GeV.  This ratio turns out to be pretty much the
same for all sources of $\ell +$ jets.
\item In data we measure the same ratio but after correcting for the $t\bar{t} \to$ dilepton
contribution, as well as dibosons etc.  The dilepton contribution is taken from MC after 
the correction described below.  
\item We compare the two ratios, as well as the shapes of the data and MC $M_T$ distributions. 
If they do not agree, we try to figure out why and fix it.  If they agree well enough, we define a 
data MC scale factor (SF) which is the ratio of the  ratios defined in step 2 and 3, keeping track of the 
uncertainty.  
\item We next perform the full selection in $t\bar{t} \to \ell +$ jets MC, and measure this ratio
again (which should be the same as that in step 2).
\item We perform the full selection in data.  We count the events with $M_T \approx 80$ GeV, we
subtract off the dilepton contribution, we multiply the subtracted event count by the ratio from step 5 (or from 
step 2), and also by the data/MC SF from step 4.  The result is the prediction for the $\ell +$ jets BG in 
the $M_T$ tail.
\end{enumerate}

The dilepton background can be broken up into many components depending
on the characteristics of the 2nd (undetected) lepton
\begin{itemize}
\item 3-prong hadronic tau decay
\item 1-prong hadronic tau decay 
\item $e$ or $\mu$ possibly from $\tau$ decay
\end{itemize}
We have currently no veto against 3-prong taus.  For the other two categories, we explicitely 
veto events with additional electrons and muons above 10 GeV , and
we veto events with an isolated track of $P_T > 10$ GeV.  This also rejects 1-prong taus
(it turns out that the explicit $e$ or $\mu$ veto is redundant with the isolated track veto).
Therefore the latter two categories can be broken into 
\begin{itemize}
\item out of acceptance $(|\eta| > 2.50)$
\item $P_T < 10$ GeV
\item $P_T > 10$ GeV, but survives the additional lepton/track isolation veto
\end{itemize}
Monte Carlo studies indicate that there is no dominant contribution: it is ``a little bit of this,
and a little bit of that''.

The high $M_T$ dilepton backgrounds come from MC, but their rate is normalized to the 
$M_T \approx 80$ GeV peak.  In other to perform this normalization in data, the $W +$ jets
events in the $M_T$ peak have to be subtracted off.  This introduces a systematic uncertainty.

There are two types of effects that can influence the MC dilepton prediction: physics effects 
and instrumental effects.  We discuss these next, starting from physics.

First of all, many of our $t\bar{t}$ MC samples (eg: MadGraph) have
 BR$(W \to \ell \nu)=\frac{1}{9} = 0.1111$.
PDG says BR$(W \to \ell \nu) = 0.1080 \pm 0.0009$.  This difference matter, so the $t\bar{t}$ MC 
must be corrected to account for this.

Second, our selection is $\ell +$ 4 or more jets.  A dilepton event passes the selection only if there are 
two additional jet from ISR, or one jet from ISR and one jet which is reconstructed from the 
unidentified lepton, {\it e.g.}, a three-prong tau.  Therefore, all MC dilepton $t\bar{t}$ samples used
in the analysis must have their jet multiplicity corrected (if necessary) to agree with what is 
seen in $t\bar{t}$ data.  We use a data control sample of well identified dilepton events with
\met and at least two jets as a template to ``adjust'' the $N_{jet}$ distribution of the $t\bar{t} \to$
dileptons MC samples.

The final physics effect has to do with the modeling of $t\bar{t}$ production and decay.  Different
MC models could in principle result in different BG predictions.  Therefore we use several different 
$t\bar{t}$ MC samples using different generators and dfferent parameters, to test the stability
of the dilepton BG prediction.  All these predictions {\bf after} corrections for branching ratio
and $N_{jet}$ dependence, are compared to each other.  The spread is a measure of the systematic
uncertainty associated with the $t\bar{t}$ generator modeling.

The main instrumental effect is associated with the underefficiency of the 2nd lepton veto.
We use tag-and-probe to compare the isolated track veto performance in $Z + 4$ jet data and 
MC, and we extract corrections if necessary.  Note that the performance of the isolated track veto 
is not exactly the same on $e/\mu$ and on one prong hadronic tau decays.  This is because
the pions from one-prong taus are often accompanied by $\pi^0$'s that can then result in extra 
tracks due to phton conversions.  We let the simulation take care of that.  Similarly, at the moment
we also let the simulation take care of the possibility of a hadronic tau ``disappearing'' in the
detector due to nuclear interaction of the pion.

The sample of events failing the last isolated track veto is an important control sample to 
check that we are doing the right thing.

Note that JES uncertainties are effectively ``calibrated away'' by the $N_{jet}$ rescaling described 
above.  

Finally, there are possible improvements to this basic analysis strategy that can be added in the future:
\begin{itemize}
\item Move from counting experiment to shape analysis.  But first, we need to get the counting
experiment under control.
\item Add an explicit three prong tau veto
\item Do something to require that three of the jets in the event be consistent with $t \to Wb, W \to q\bar{q}$.
This could help rejecting some of the dilepton BG; however, it would also loose efficiency for 
the $\widetilde{t} \to b \chi^+$ mode
\item Consider the $M(\ell b)$ variable, which is not bounded by $M_{top}$ in $\widetilde{t} \to b \chi^+$
\end{itemize}