\section{Overview and Strategy for Background Determination}
\label{sec:overview}

We are searching for a $t\bar{t}\chi^0\chi^0$ or $W b W \bar{b} \chi^0 \chi^0$ final state
(after top decay in the first mode, the final states are actually the same).  So to first order 
this is ``$t\bar{t} +$ extra \met''.  

We work in the $\ell +$ jets final state, where the main background is $t\bar{t}$.  We look for 
\met\ inconsistent with $W \to \ell \nu$.  We do this by concentrating on the $\ell \nu$ transverse
mass ($M_T$), since except for resolution and W-off-shell effects, $M_T < M_W$ for $W \to \ell \nu$.  Thus, the
initial analysis is simply a counting experiment in the tail of the $M_T$ distribution.  

The event selection is one-and-only-one high \pt\ isolated lepton, four or more jets, and
an \met\ cut.  At least one of the jets has to be btagged to reduce $W+$ jets.
The event sample is then dominated by $t\bar{t}$, but there are also contributions from $W+$ jets,
single top, dibosons, as well as rare SM processes such as $ttW$.

% In order to be sensitive to $\widetilde{t}\widetilde{t}$ production, the background in the $M_T$
% tail has to be controlled at the level of 10\% or better. So this is (almost) a precision measurement.

The $t\bar{t}$ events in the $M_T$ tail can be broken up into two categories: 
(i) $t\bar{t} \to \ell $+ jets and (ii) $t\bar{t} \to \ell^+ \ell^-$ where one of the two
leptons is not found by the second-lepton-veto (here the second lepton can be a hadronically
decaying $\tau$).
 For a reasonable $M_T$ cut, say $M_T >$ 150 GeV, the dilepton background is of order 80\% of 
the total.  This is because in dileptons there are two neutrinos from $W$ decay, thus $M_T$
is not bounded by $M_W$.  This is a very important point: while it is true that we are looking in
the tail of $M_T$, the bulk of the background events end up there not because of some exotic
\met\ reconstruction failure, but because of well understood physics processes.  This means that 
the background estimate can be taken from Monte Carlo (MC) 
after carefully accounting for possible
data/MC differences.   

The search is performed in a number of Signal Regions (SRs) defined 
by minimum requirements on \met\  and $M_T$.  The SRs
are defined in Section~\ref{sec:SR}.

In Section~\ref{sec:CR} we will describe the analysis of various Control Regions
(CRs)  that are used to test the Monte Carlo model and, if necessary,
to extract data/MC scale factors.  In this section we give a
general description of the procedure.  The details of how the
final background prediction is assembled are given in Section~\ref{sec:bkg_pred}.



% Sophisticated fully ``data driven'' techniques are not really needed.

One general point is that in order to minimize systematic uncertainties, the MC background
predictions are whenever possible normalized to the bulk of the $t\bar{t}$ data, ie, events passing all of the 
requirements but with $M_T \approx 80$ GeV.
This (mostly) removes uncertainties
due to $\sigma(t\bar{t})$, lepton ID, trigger efficiency, luminosity, etc.   

\subsection{$\ell +$ jets background}
\label{sec:ljbg-general}

The $\ell +$ jets background is dominated by 
$t\bar{t} \to \ell $+ jets, but also includes some $W +$ jets as well as single top.
The MC input used in the background estimation
is the ratio of the number of events with $M_T$ in the signal region
to the number of events with $M_T \approx 80$ GeV.
This ratio is (possibly) corrected by a data/MC scale factor obtained
from a study of CRs, as outlined below.

Note that the ratio described above is actually different for 
$t\bar{t}$/single top and $W +$ jets.  This is because in $W$ events
there is a significant contribution to the $M_T$ tail from very off-shell
$W$.
This contribution is much smaller in top events because $M(\ell \nu)$
cannot excees $M_{top}-M_b$.

For $W +$ jets the ability of the Monte Carlo to model this ratio
($R_{wjet}$) is tested in a sample of $\ell +$ jets enriched in 
$W +$ jets by the application of a b-veto.
The equivalent ratio for top events ($R_{top}$) is validated in a sample of well
identified $Z \to \ell \ell$ with one lepton added to the \met\
calculation.
This sample is well suited to testing the resolution effects on 
the $M_T$ tail, since off-shell effects are eliminated by the $Z$-mass
requirement.

Note that the fact that the ratios are different for 
$t\bar{t}$/single top and $W +$ jets introduces a systematic
uncertainty in the background calculation because one needs
to know the relative fractions of these two components in 
$M_T \approx 80$ GeV lepton $+$ jets sample.


\subsection{Dilepton background}
\label{sec:dil-general}

To suppress dilepton backgrounds, we veto events with an isolated track of \pt $>$ 10 GeV. 
Being the common feature for electron, muon, and one-prong
tau decays, this veto is highly efficient for rejecting 
$t\bar{t}$ to dilepton events. The remaining dilepton background can be classified into the following categories:

%The dilepton background can be broken up into many components depending
%on the characteristics of the 2nd (undetected) lepton
%\begin{itemize}
%\item 3-prong hadronic tau decay
%\item 1-prong hadronic tau decay 
%\item $e$ or $\mu$ possibly from $\tau$ decay
%\end{itemize}
%We have currently no veto against 3-prong taus.  For the other two categories, we explicitely 
%veto events %with additional electrons and muons above 10 GeV , and we veto events 
%with an isolated track of \pt\ $>$ 10 GeV.  This rejects electrons and muons (either from $W\to e/\mu$ or
%$W\to \tau\to e/\mu$) and 1-prong tau decays.
%(it turns out that the explicit $e$ or $\mu$ veto is redundant with the isolated track veto).
%Therefore the latter two categories can be broken into 
\begin{itemize}
\item lepton is out of acceptance $(|\eta| > 2.50)$
\item lepton has \pt\ $<$ 10 GeV, and is inside the acceptance
\item lepton has \pt\ $>$ 10 GeV, is inside the acceptance, but survives the additional isolated track veto
\end{itemize}

%Monte Carlo studies indicate that there is no dominant contribution: it is ``a little bit of this,
%and a little bit of that''.

The last category includes 3-prong tau decays as well as electrons and muons from W decay that fail the isolation requirement.
Monte Carlo studies indicate that these three components populate the $M_T$ tail in the proportions of roughly  6\%, 47\%, 47\%. 
We note that at present we do not attempt to veto 3-prong tau decays as they are only 16\% of the total dilepton background according to the MC.

The high $M_T$ dilepton backgrounds come from MC, but their rate is normalized to the 
$M_T \approx 80$ GeV peak.  In order to perform this normalization in
data, the non-$t\bar{t}$ (eg, $W +$ jets)
events in the $M_T$ peak have to be subtracted off.  This also introduces a systematic uncertainty.

There are two types of effects that can influence the MC dilepton prediction: physics effects 
and instrumental effects.  We discuss these next, starting from physics.

First of all, many of our $t\bar{t}$ MC samples (eg: MadGraph) have
 BR$(W \to \ell \nu)=\frac{1}{9} = 0.1111$.
PDG says BR$(W \to \ell \nu) = 0.1080 \pm 0.0009$.  This difference matters, so the $t\bar{t}$ MC 
must be corrected to account for this.

Second, our selection is $\ell +$ 4 or more jets.  A dilepton event passes the selection only if there are 
two additional jets from ISR, or one jet from ISR and one jet which is reconstructed from the 
unidentified lepton, {\it e.g.}, a three-prong tau.  Therefore, all MC dilepton $t\bar{t}$ samples used
in the analysis must have their jet multiplicity corrected (if necessary) to agree with what is 
seen in $t\bar{t}$ data.  We use a data control sample of well identified dilepton events with
\met\ and at least two jets as a template to ``adjust'' the $N_{jet}$ distribution of the $t\bar{t} \to$
dileptons MC samples.

The final physics effect has to do with the modeling of $t\bar{t}$ production and decay.  Different
MC models could in principle result in different BG predictions.  Therefore we use several different 
$t\bar{t}$ MC samples using different generators and different parameters, to test the stability
of the dilepton BG prediction.  All these predictions, {\bf after} corrections for branching ratio
and $N_{jet}$ dependence, are compared to each other.  The spread is a measure of the systematic
uncertainty associated with the $t\bar{t}$ generator modeling.

The main instrumental effect is associated with the efficiency of the isolated track veto.
We use tag-and-probe to compare the isolated track veto performance in $Z + 4$ jet data and 
MC, and we extract corrections if necessary.  Note that the performance of the isolated track veto 
is not exactly the same on $e/\mu$ and on one prong hadronic tau decays.  This is because
the pions from one-prong taus are often accompanied by $\pi^0$'s that can then result in extra 
tracks due to phton conversions.  We let the simulation take care of that.  
Note that JES uncertainties are effectively ``calibrated away'' by the $N_{jet}$ rescaling described above.  

%Similarly, at the moment
%we also let the simulation take care of the possibility of a hadronic tau ``disappearing'' in the
%detector due to nuclear interaction of the pion.

%The sample of events failing the last isolated track veto is an important control sample to 
%check that we are doing the right thing.

\subsection{Other backgrounds}
\label{sec:other-general}
Other backgrounds are $tW$, $ttV$, dibosons, tribosons, Drell Yan.
These  are small.  They are taken from MC with appropriate scale
factors
for trigger efficiency, etc.


\subsection{Future improvements}
\label{sec:improvements-general}
Finally, there are possible improvements to this basic analysis strategy that can be added in the future:
\begin{itemize}
\item Move from counting experiment to shape analysis.  But first, we need to get the counting
experiment under control.
\item Add an explicit three prong tau veto
\item Do something to require that three of the jets in the event be consistent with $t \to Wb, W \to q\bar{q}$.
%This could help rejecting some of the dilepton BG; however, it would also loose efficiency for 
%the $\widetilde{t} \to b \chi^+$ mode
This could help reject some of the dilepton BG in the search for $\widetilde{t} \to t \chi^0$, 
but is not applicable to the $\widetilde{t} \to b \chi^+$ search.
\item Consider the $M(\ell b)$ variable, which is not bounded by $M_{top}$ in $\widetilde{t} \to b \chi^+$
\end{itemize}
