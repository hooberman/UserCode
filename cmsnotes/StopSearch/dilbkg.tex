%\clearpage

%\section{Background Estimation Technique}

%Using dilepton triggers 
%Data and MC comparison with only efficiency Data/MC SFs from OS analysis
%SF(e-mu) = 0.95
%SF(e-e) = 1.
%SF(mu-mu) = 0.90
%need to find uncertainties on these numbers
%Jet multiplicity trends similar for e-mu and dimuon samples,
%dielectron is flatter
%
%K3 = SF3/SF2 = 0.92 ± 0.03
%K4 = SF4/SF2 = 0.83 ± 0.04

%njets_all_dl_dimu.pdf
%njets_all_dl_diel.pdf
%njets_all_dl_mueg.pdf


The dominant background to the signal sample comes from \ttll\
events. Unlike in the case of single lepton events, \ttll\ events do not have a kinematic edge at $\mt \sim \mW$
due to the presence of a second neutrino. As a result, these events satisfy the
selection criteria due to real \met\ and do not depend on resolution
effects or mis-measurements of the \met. An isolated track veto is
used to reject events with a second lepton and since the dilepton
contribution corresponds to a smaller fraction of the signal, the veto
rejects more background than signal. 

The \ttll\ background is comprised of events where the second lepton
satisfies the isolated track veto, because it falls outside the
acceptance, both in $\eta$ or in $\pt$, because it is non-isolated, or
because it decays to a hadronic tau that is not explicitly rejected. 

PUT TABLE AND EXPLAIN IN DETAIL WHY DIFFERENT EVENTS ARE MISSED


large \met\ and \mt\ requirements due to the presence of true \met\

The SM background in the signal region defined by requirements of large \met\ and \mt\ is estimated using MC.
The MC is validated using data control samples, which are used to derive data-to-MC scale factors and corresponding uncertainties.
We consider three samples:

\begin{itemize}
\item Dilepton sample (exactly 2 selected leptons): used to correct the \njets\ distribution in \ttll\ MC, which is not necessarily well-modelled since ISR jets 
are needed to satisfy the \njets $\geq$ 4 requirement defining the signal region;
\item Inclusive sample (at least 1 selected lepton): used to define a scale factor which corrects for effects of integrated luminosity, \ttbar\ cross section,
jet energy scale and jet selection efficiencies, lepton selection and trigger efficiencies; 
{\bf THING 1 we are uncertain about: should this sample be fully inclusive and require AT LEAST 1 lepton, or should we veto events with a 2nd lepton and remove only the isolated track veto}.
\item Signal sample (exactly 1 selected lepton): this is the sample used to define the signal region. In addition, this sample is used to determine a scale 
factor accounting for possible data vs. MC discrepancies in the isolated track fake rate for backgrounds which have only 1 genuine lepton.
\end{itemize}

\subsection{Step 1: Use dilepton control sample to correct the \njets\ distributon in \ttll\ MC}

The dilepton control sample is defined by the following requirements:
\begin{itemize}
\item Exactly 2 selected electrons or muons with \pt $>$ 20 GeV
\item \met\ $>$ 50 GeV
\item $\geq1$ b-tagged jet
\end{itemize}

This sample is dominated by \ttll. The distribution of \njets\ for data and MC passing this selection is displayed in Fig.~\ref{fig:dilepton_njets}. 
We use this distribution to derive scale factors which reweight the \ttll\ MC \njets\ distribution to match the data. We define the following
quantities

\begin{itemize}
\item $N_{2}=$ data yield minus non-dilepton \ttbar\ MC yield for \njets\ $\leq$ 2
\item $N_{3}=$ data yield minus non-dilepton \ttbar\ MC yield for \njets\ = 3
\item $N_{4}=$ data yield minus non-dilepton \ttbar\ MC yield for \njets\ $\geq$ 4
\item $M_{2}=$ dilepton \ttbar\ MC yield for \njets\ $\leq$ 2
\item $M_{3}=$ dilepton \ttbar\ MC yield for \njets\ = 3
\item $M_{4}=$ dilepton \ttbar\ MC yield for \njets\ $\geq$ 4
\end{itemize}

We use these yields to define 3 scale factors, which quantify the data/MC ratio in the 3 \njets\ bins:

\begin{itemize}
\item $SF_2 = N_2 / M_2$
\item $SF_3 = N_3 / M_3$
\item $SF_4 = N_4 / M_4$
\end{itemize}

And finally, we define the scale factors $K_3$ and $K_4$:

\begin{itemize}
\item $K_3 = SF_3 / SF_2$
\item $K_4 = SF_4 / SF_2$
\end{itemize}

The scale factor $K_3$ is extracted from dilepton \ttbar\ events with \njets = 3, which have exactly 1 ISR jet.
The scale factor $K_4$ is extracted from dilepton \ttbar\ events with \njets $\geq$ 4, which have at least 2 ISR jets.
Both of these scale factors are needed since dilepton \ttbar\ events which fall in our signal region (including
the \njets $\geq$ 4 requirement) may require exactly 1 ISR jet, in the case that the second lepton is reconstructed
as a jet, or at least 2 ISR jets, in the case that the second lepton is not reconstructed as a jet. These scale
factors are applied to the dilepton \ttbar\ MC only. For a given MC event, we determine whether to use $K_3$ or $K_4$
by counting the number of reconstructed jets in the event ($N_{\rm{jets}}^R$) , and subtracting off any reconstructed 
jet which is matched to the second lepton at generator level ($N_{\rm{jets}}^\ell$); $N_{\rm{jets}}^{\rm{cor}} = N_{\rm{jets}}^R - N_{\rm{jets}}^\ell$.
For events with $N_{\rm{jets}}^{\rm{cor}}=3$ the factor $K_3$ is applied, while for events with $N_{\rm{jets}}^{\rm{cor}}\geq4$ the factor $K_4$ is applied.
For all subsequent steps, the scale factors $K_3$ and $K_4$ have been applied to the \ttll\ MC.
 
\subsection{Step 2: Use the pre-veto sample to define a data-to-MC scale factor}

The pre-veto sample is defined by the following requirements

\begin{itemize}
\item At least 1 selected electron (muon) with \pt $>$ 30 GeV and $|\eta|<2.5$ ($|\eta|<2.1$)
\item At least 4 selected jets, with at least 1 b-tagged jet
\item \met\ $>$ 50 GeV
\end{itemize}

Thus all selection criteria are applied with the exception of the veto on events containing an isolated track. 
This sample is dominated by \ttlj\, with secondary contributions from \wjets\ and \ttll\ . 
%The largest contribution to this sample is \ttlj, but \wjets\ and \ttll\ also have significant contributions.
This sample is used to define an overall data over MC scale factor ($SF$) in the peak control region, 
that accounts for differences between the data and the MC due to the luminosity estimate, the lepton 
efficiencies and so on and is thus applied to the full MC cocktail. 
To do so we define for the pre-veto sample (labeled `all'):

\begin{itemize}
\item $N_{\rm{peak}}^{\rm{all}}$ = data yield in the peak region $60<\mt<100$ GeV
\item $M_{\rm{peak}}^{\rm{all}}$ = MC yield in the peak region $60<\mt<100$ GeV
\item $SF^{\rm{all}} = N_{\rm{peak}}^{\rm{all}} / M_{\rm{peak}}^{\rm{all}}$
\end{itemize}

For all subsequent steps, the scale factor $SF^{\rm{all}}$ is applied to all MC contributions.

\subsection{Step 3: Isolated Track Veto Efficiency Correction}

The signal sample is defined by the same requirements as the pre-veto sample, except that we veto events containing
an isolated track. The background is the inclusive sample can be split into 2 contributions:

\begin{itemize}
\item Dilepton background: mostly \ttll.
\item Single lepton background: mostly \ttlj\ and \wjets;
\end{itemize}

The isolated track veto impacts these 2 contributions in different ways. For the dilepton background, the veto
rejects events which have a genuine 2nd lepton, so applying the isolated track veto scales the dilepton background
by the efficiency $\epsilon(trk)$ to identify the isolated track. For the single lepton background, the veto rejects events
which do not have a genuine 2nd lepton but which have a ``fake'' isolated track, so the isolated track veto scales
the single lepton background by (1-FR), where FR is the ``fake rate'' to identify an isolated track in events which 
contain no genuine 2nd lepton. 

The isolated track efficiency $\epsilon$ can be measured in data and MC using $Z\to\ell\ell$ tag-and-probe studies. 
We therefore use the tag-and-probe studies to apply a correction to the \ttll\ MC. 
A measurement of the FR in data is non-trivial, but we can account for differences between the data and the MC by scaling 
only the single lepton background in the \mt\ peak region after applying the isolated track veto.

In detail, the procedure to correct the dilepton background is:

\begin{itemize}
\item Using tag-and-probe studies, we plot the distribution of {\bf absolute} track isolation for identified probe electrons
and muons {\bf TODO: need to compare the e vs. $\mu$ track iso distributions, they might differ due to e$\to$e$\gamma$}.
\item We verify that the distribution of absolute track isolation does not depend on the \pt\ of the probe lepton.
This is due to the fact that this isolation is from ambient PU and jet activity in the event, which is uncorrelated with
the lepton \pt {\bf TODO: verify this in data and MC.}.
\item Our requirement is {\bf relative} track isolation $<$ 0.1. For a given \ttll\ MC event, we determine the \pt of the 2nd
lepton and translate this to find the corresponding requirement on the {\bf absolute} track isolation, which is simply $0.1\times$\pt.
\item We measure the efficiency to satisfy this requirement in data and MC, and define a scale-factor $SF_{\epsilon(trk)}$ which
is the ratio of the data-to-MC efficiencies. This scale-factor is applied to the \ttll\ MC event.
\item {\bf THING 2 we are unsure about: we can measure this SF for electrons and for muons, but we can't measure it for hadronic 
tracks from $\tau$ decays. Verena has showed that the absolute track isolation distribution in hadronic $\tau$ tracks is harder due 
to $\pi^0\to\gamma\gamma$ with $\gamma\to e^+e^-$.}
\end{itemize} 

At this point we are done applying scale factors to the dilepton background. We next apply a scale factor to the single lepton background. We use the following quantities:

%Use the signal sample to apply a correction factor to the single lepton background

\begin{itemize}
\item $N_{\rm{peak}}^{\rm{veto}}$ =  data yield in the peak region $60<\mt<100$ GeV
\item $M^{\ell,\rm{veto}}_{\rm{peak}}$ =  single lepton background MC in the peak region $60<\mt<100$ GeV
\item $M^{\ell\ell,\rm{veto}}_{\rm{peak}}$ =  dilepton background MC in the peak region $60<\mt<100$ GeV
\item $SF_{\ell}^{\rm{veto}} = (N_{\rm{peak}}^{\rm{veto}} - M^{\ell\ell,\rm{veto}}_{\rm{peak}} SF_{\epsilon(trk)} )/ M^{\ell,\rm{veto}}_{\rm{peak}}$
\end{itemize}

The scale factor $SR_{\ell}$ is applied to the single lepton background to account for potential data vs. MC discrepancies 
in the isolated track veto.

To summarize, the dilepton and single lepton background prediction in the signal region (\mt\ $>150$ GeV) are given by

\begin{itemize}
\item $\rm{P}^{\ell\ell}_{\rm{sig}} = SF^{\rm{all}} \times SF_{\epsilon(trk)} \times M^{\ell \ell}_{\rm{sig}}$
\item $\rm{P}^{\ell}_{\rm{sig}} = SF^{\rm{all}} \times SF_{\ell}^{\rm{veto}} \times M^{\ell}_{\rm{sig}}$
\end{itemize}

where $M_{\rm{sig}}$ correspond to the Monte Carlo predictions in the tail of the distributions and the $SF$ terms are data over MC scale factors to account 
for differences between the data and the MC for the following effects

\begin{itemize}
\item $SF^{\rm{all}} \rightarrow $ effects impacting the normalization with the exception of the isolated track veto i.e. luminosity, 
lepton identification and trigger efficiencies $\dots$
\item $SF_{\epsilon(trk)} \rightarrow $ the track veto efficiency for real second leptons
\item $SF_{\ell}^{\rm{veto}} \rightarrow $ the inefficiency introduced by the track veto on events without a second lepton (1-FR defined previously)
\end{itemize}

%SF_{\ell}^{\rm{post-veto}}$ 
%\item $M^{\ell,\rm{sig}}_{\rm{peak}}$ =  single lepton background MC in the peak region $60<\mt<100$ GeV



