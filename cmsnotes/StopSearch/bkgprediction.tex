
%\section{Background Estimate Derivation}

[THIS IS WHERE THE NUMBERS TO DERIVE THE BACKGROUND ESTIMATES ARE DUMPED AND THE CALCULATION EXPLAINED]

Here we give the details of how we arrive at the background prediction
in a given signal region.  Here we concentrate on the method used
to arrive at the central value of the background prediction.  The
systematic
uncertainties will be discussed in Section~\ref{sec:systematics}.

As mentioned in Section~\ref{sec:overview}, we normalize the main
$t\bar{t}$
background to the $M_T$ peak.  This is actually a bit tricky because 
we want to minimize the effect of the isolated track veto on 
lepton $+$ jets events, which may not be terribly well reproduced.
Thus, we define two normaliztion region in the $M_T$ peak ($60 < M_T < 100$~GeV),
one before and one after the application of the isolated track veto.

The dominant dilepton background is normalized to the pre-veto normalization
region.  A pre-veto scale factor ($SF_{pre}$) is defined as a common scale
factors that needs to be applied to the $t\bar{t}$, single-top, and 
$W +$ jets
MC to make the data yield in the pre-veto normalization agree with
the MC prediction (the small rare MC component is held fixed).
Then, the dilepton background prediction is obtained by multiplying the
dilepton BG Monte Carlo by $SF_{pre}$.

The $t\bar{t}$ lepton $+$ jet BG is normalized to post-veto 
normalization region.  A post-veto scale factor  ($SF_{post}$) 
is defined in (almost) the same way as the pre-veto scale factor.
The difference here is that this scale factor applies only to 
the lepton $+$ jets components and not the dilepton component,
since that component is already rescaled by $SF_{pre}$.  
This procedure minimizes the reliance on the understanding of 
the isolated track veto.
Then the  $t\bar{t}$ lepton $+$ jet BG is obtained by taking the 
number of MC-predicted $t\bar{t}$ lepton $+$ jets in the post-veto
normalization region, scaling it by $SF_{post}$, multiplying 
it by the tail-to-peak ratio $R_{top}$ of Table~\ref{tab:ttp}, and
finally the data-MC scale factor $SFR_{top}$ from
Table~\ref{tab:cr2yields}.

The single top background is obtained in exactly the same way
as the  $t\bar{t}$ lepton $+$ jet BG, using the same tail-to-peak
ratio and the same data-MC scale-factor.  The $W$ background is
done in a similar way, but using a different tail-to-peak ratio 
($R_{wjets}$ of Table~\ref{tab:ttp}), and a different data-MC scale
factor
($SFR_{wjet}$ from
Table~\ref{tab:cr1yields}).

Other (small) backgrounds are taken straight from Monte Carlo, as 
described in Section~\ref{sec:bkg_other}.





\begin{table}[!h]
\begin{center}
\begin{tabular}{l||c|c|c|c|c}
\hline
Sample              & SRA & SRB & SRC & SRD & SRE\\
\hline
\hline
\multicolumn{6}{c}{Muon} \\
\hline
\ttdl\ 		 & $606 \pm 9$& $185 \pm 5$& $61 \pm 3$& $25 \pm 2$& $12 \pm 1$ \\
\ttsl\ \& single top (1\Lep) 		 & $5211 \pm 26$& $1450 \pm 14$& $459 \pm 8$& $157 \pm 5$& $60 \pm 3$ \\
\wjets\ 		 & $469 \pm 10$& $154 \pm 6$& $65 \pm 4$& $29 \pm 3$& $15 \pm 2$ \\
Rare 		 & $166 \pm 6$& $61 \pm 4$& $22 \pm 2$& $9 \pm 1$& $3 \pm 1$ \\
\hline
Total 		 & $6452 \pm 30$& $1851 \pm 16$& $606 \pm 9$& $219 \pm 6$& $89 \pm 4$ \\
\hline
\hline
Data 		 & $5872$& $1593$& $514$& $192$& $81$ \\
\hline
\hline
\hline
\multicolumn{6}{c}{Electron} \\
\hline
\ttdl\ 		 & $443 \pm 7$& $135 \pm 4$& $47 \pm 2$& $18 \pm 1$& $8 \pm 1$ \\
\ttsl\ \& single top (1\Lep) 		 & $3944 \pm 23$& $1108 \pm 12$& $347 \pm 7$& $125 \pm 4$& $51 \pm 3$ \\
\wjets\ 		 & $333 \pm 29$& $110 \pm 5$& $44 \pm 3$& $23 \pm 2$& $12 \pm 2$ \\
Rare 		 & $124 \pm 6$& $47 \pm 3$& $20 \pm 2$& $9 \pm 2$& $5 \pm 1$ \\
\hline
Total 		 & $4844 \pm 38$& $1400 \pm 14$& $457 \pm 8$& $176 \pm 5$& $76 \pm 3$ \\
\hline
\hline
Data 		 & $4405$& $1232$& $396$& $127$& $51$ \\
\hline
\hline
\hline
\multicolumn{6}{c}{Muon+Electron Combined} \\
\hline
\ttdl\ 		 & $1048 \pm 11$& $321 \pm 6$& $108 \pm 4$& $43 \pm 2$& $20 \pm 2$ \\
\ttsl\ \& single top (1\Lep) 		 & $9155 \pm 35$& $2559 \pm 19$& $805 \pm 10$& $282 \pm 6$& $111 \pm 4$ \\
\wjets\ 		 & $802 \pm 31$& $264 \pm 8$& $108 \pm 5$& $53 \pm 3$& $26 \pm 2$ \\
Rare 		 & $291 \pm 9$& $108 \pm 5$& $42 \pm 3$& $18 \pm 2$& $8 \pm 1$ \\
\hline
Total 		 & $11296 \pm 49$& $3251 \pm 22$& $1063 \pm 13$& $395 \pm 8$& $165 \pm 5$ \\
\hline
\hline
Data 		 & $10277$& $2825$& $910$& $319$& $132$ \\
\hline
\end{tabular}
\caption{ Preveto MC and data yields in \mt\ peak region. The
  n-jets k-factors have been applied to the \ttdl. The uncertainties are statistical only.
\label{tab:pvmtpeakyields}}
\end{center}
\end{table}


\begin{table}[!h]
\begin{center}
\begin{tabular}{l||c|c|c|c|c}
\hline
Sample              & SRA & SRB & SRC & SRD & SRE\\
\hline
\hline
\multicolumn{6}{c}{Muon} \\
\hline
\ttdl\ 		 & $241 \pm 5$& $77 \pm 3$& $25 \pm 2$& $10 \pm 1$& $5 \pm 1$ \\
\ttsl\ \& single top (1\Lep) 		 & $4654 \pm 25$& $1302 \pm 13$& $415 \pm 8$& $141 \pm 4$& $54 \pm 3$ \\
\wjets\ 		 & $433 \pm 10$& $143 \pm 6$& $60 \pm 4$& $27 \pm 2$& $13 \pm 2$ \\
Rare 		 & $119 \pm 5$& $41 \pm 3$& $15 \pm 2$& $5 \pm 1$& $2 \pm 1$ \\
\hline
Total 		 & $5447 \pm 28$& $1563 \pm 15$& $514 \pm 9$& $183 \pm 5$& $74 \pm 3$ \\
\hline
\hline
Data 		 & $4859$& $1330$& $437$& $163$& $69$ \\
\hline
\hline
\hline
\multicolumn{6}{c}{Electron} \\
\hline
\ttdl\ 		 & $174 \pm 5$& $55 \pm 3$& $18 \pm 1$& $8 \pm 1$& $3 \pm 1$ \\
\ttsl\ \& single top (1\Lep) 		 & $3517 \pm 22$& $994 \pm 12$& $316 \pm 7$& $114 \pm 4$& $47 \pm 3$ \\
\wjets\ 		 & $309 \pm 29$& $103 \pm 5$& $41 \pm 3$& $22 \pm 2$& $11 \pm 2$ \\
Rare 		 & $87 \pm 5$& $33 \pm 3$& $14 \pm 2$& $7 \pm 1$& $3 \pm 1$ \\
\hline
Total 		 & $4088 \pm 37$& $1185 \pm 13$& $389 \pm 8$& $151 \pm 5$& $64 \pm 3$ \\
\hline
\hline
Data 		 & $3644$& $1020$& $332$& $97$& $39$ \\
\hline
\hline
\hline
\multicolumn{6}{c}{Muon+Electron Combined} \\
\hline
\ttdl\ 		 & $416 \pm 7$& $132 \pm 4$& $43 \pm 2$& $18 \pm 1$& $8 \pm 1$ \\
\ttsl\ \& single top (1\Lep) 		 & $8171 \pm 33$& $2296 \pm 18$& $730 \pm 10$& $255 \pm 6$& $101 \pm 4$ \\
\wjets\ 		 & $742 \pm 31$& $246 \pm 7$& $101 \pm 5$& $49 \pm 3$& $24 \pm 2$ \\
Rare 		 & $207 \pm 7$& $75 \pm 4$& $28 \pm 3$& $12 \pm 2$& $5 \pm 1$ \\
\hline
Total 		 & $9535 \pm 46$& $2748 \pm 20$& $903 \pm 12$& $334 \pm 7$& $138 \pm 5$ \\
\hline
\hline
Data 		 & $8503$& $2350$& $769$& $260$& $108$ \\
\hline
\end{tabular}
\caption{ MC and data yields in \mt\ peak region after full selection. The
  n-jets k-factors have been applied to the \ttdl. The uncertainties are statistical only.
\label{tab:mtpeakyields}}
\end{center}
\end{table}


\begin{table}[!h]
\begin{center}
\begin{tabular}{l||c|c|c|c|c}
\hline
Sample              & SRA & SRB & SRC & SRD & SRE\\
\hline
\hline
\multicolumn{6}{c}{Muon} \\
\hline
\ttdl\ 		 & $364 \pm 7$& $216 \pm 5$& $74 \pm 3$& $26 \pm 2$& $10 \pm 1$ \\
\ttsl\ \& single top (1\Lep) 		 & $60 \pm 3$& $43 \pm 2$& $9 \pm 1$& $4 \pm 1$& $1 \pm 1$ \\
\wjets\ 		 & $20 \pm 2$& $9 \pm 1$& $3 \pm 1$& $2 \pm 1$& $0 \pm 0$ \\
Rare 		 & $37 \pm 3$& $26 \pm 2$& $10 \pm 1$& $5 \pm 1$& $3 \pm 1$ \\
\hline
Total 		 & $481 \pm 8$& $294 \pm 6$& $96 \pm 4$& $37 \pm 2$& $16 \pm 2$ \\
\hline
\hline
\hline
\hline
\multicolumn{6}{c}{Electron} \\
\hline
\ttdl\ 		 & $273 \pm 6$& $161 \pm 4$& $57 \pm 3$& $20 \pm 2$& $7 \pm 1$ \\
\ttsl\ \& single top (1\Lep) 		 & $43 \pm 2$& $30 \pm 2$& $8 \pm 1$& $3 \pm 1$& $1 \pm 0$ \\
\wjets\ 		 & $12 \pm 2$& $8 \pm 1$& $3 \pm 1$& $2 \pm 1$& $0 \pm 0$ \\
Rare 		 & $27 \pm 2$& $17 \pm 2$& $7 \pm 1$& $3 \pm 1$& $1 \pm 0$ \\
\hline
Total 		 & $355 \pm 7$& $216 \pm 5$& $75 \pm 3$& $28 \pm 2$& $9 \pm 1$ \\
\hline
\hline
\hline
\hline
\multicolumn{6}{c}{Muon+Electron Combined} \\
\hline
\ttdl\ 		 & $637 \pm 9$& $378 \pm 7$& $131 \pm 4$& $46 \pm 2$& $17 \pm 1$ \\
\ttsl\ \& single top (1\Lep) 		 & $103 \pm 4$& $73 \pm 3$& $16 \pm 2$& $7 \pm 1$& $2 \pm 1$ \\
\wjets\ 		 & $32 \pm 3$& $17 \pm 2$& $6 \pm 1$& $4 \pm 1$& $1 \pm 0$ \\
Rare 		 & $64 \pm 4$& $42 \pm 3$& $18 \pm 2$& $9 \pm 1$& $4 \pm 1$ \\
\hline
Total 		 & $837 \pm 10$& $510 \pm 8$& $171 \pm 5$& $65 \pm 3$& $24 \pm 2$ \\
\hline
\end{tabular}
\caption{ MC yields in \mt\ tail region after full selection. The
  n-jets k-factors have been applied to the \ttdl. The uncertainties
  are statistical only.
  Note these values are only used for the rare backgrounds prediction. 
\label{tab:mtpeakyields}}
\end{center}
\end{table}

\begin{table}[!h]
\begin{center}
\begin{tabular}{l||c|c|c|c|c}
\hline
Sample              & SRA & SRB & SRC & SRD & SRE\\
\hline
\hline
Muon pre-veto \mt-SF 	  & $0.91 \pm 0.01$ & $0.86 \pm 0.02$ & $0.84 \pm 0.04$ & $0.87 \pm 0.07$ & $0.91 \pm 0.11$ \\
Muon post-veto \mt-SF 	  & $0.89 \pm 0.01$ & $0.85 \pm 0.02$ & $0.85 \pm 0.04$ & $0.89 \pm 0.07$ & $0.93 \pm 0.12$ \\
\hline
Muon veto \mt-SF 	  & $0.98 \pm 0.01$ & $0.99 \pm 0.01$ & $1.00 \pm 0.02$ & $1.02 \pm 0.04$ & $1.03 \pm 0.06$ \\
\hline
\hline
Electron pre-veto \mt-SF 	  & $0.91 \pm 0.02$ & $0.88 \pm 0.03$ & $0.86 \pm 0.05$ & $0.71 \pm 0.07$ & $0.65 \pm 0.11$ \\
Electron post-veto \mt-SF 	  & $0.89 \pm 0.02$ & $0.86 \pm 0.03$ & $0.85 \pm 0.05$ & $0.62 \pm 0.07$ & $0.58 \pm 0.10$ \\
\hline
Electron veto \mt-SF 	  & $0.98 \pm 0.01$ & $0.98 \pm 0.01$ & $0.99 \pm 0.03$ & $0.88 \pm 0.05$ & $0.90 \pm 0.08$ \\
\hline
\end{tabular}
\caption{ \mt\ peak Data/MC scale factors. The pre-veto SFs are applied to the
  \ttdl\ sample, while the post-veto SFs are applied to the single
  lepton samples. The veto SF is shown for comparison across channels. 
  The raw MC is used for backgrounds from rare processes.
  The uncertainties are statistical only.
\label{tab:mtpeaksf}}
\end{center}
\end{table}


%POSSIBLY ADD SOME EQUATIONS TO MAKE IT MORE EXPLICIT HOW THE CALCULATION IS DERIVED
%We consider three samples:
%
%\begin{itemize}
%\item Dilepton sample (exactly 2 selected leptons): used to correct the \njets\ distribution in \ttll\ MC, which is not necessarily well-modelled since ISR jets 
%are needed to satisfy the \njets $\geq$ 4 requirement defining the signal region;
%\item Inclusive sample (at least 1 selected lepton): used to define a
%  scale factor which corrects for effects of integrated luminosity,
%  \ttbar\ cross section, jet energy scale and jet selection efficiencies, lepton selection and trigger efficiencies.
%\item Signal sample (exactly 1 selected lepton): this is the sample used to define the signal region. In addition, this sample is used to determine a scale 
%factor accounting for possible data vs. MC discrepancies in the isolated track fake rate for backgrounds which have only 1 genuine lepton.
%\end{itemize}
%
%\subsection{Step 1: Use dilepton control sample to correct the \njets\ distributon in \ttll\ MC}
%

% 
%\subsection{Step 2: Use the pre-veto sample to define a data-to-MC scale factor}
%
%The pre-veto sample is defined by the following requirements
%
%\begin{itemize}
%\item At least 1 selected electron (muon) with \pt $>$ 30 GeV and $|\eta|<2.5$ ($|\eta|<2.1$)
%\item At least 4 selected jets, with at least 1 b-tagged jet
%\item \met\ $>$ 50 GeV
%\end{itemize}
%
%Thus all selection criteria are applied with the exception of the veto on events containing an isolated track. 
%This sample is dominated by \ttlj\, with secondary contributions from \wjets\ and \ttll\ . 
%%The largest contribution to this sample is \ttlj, but \wjets\ and \ttll\ also have significant contributions.
%This sample is used to define an overall data over MC scale factor ($SF$) in the peak control region, 
%that accounts for differences between the data and the MC due to the luminosity estimate, the lepton 
%efficiencies and so on and is thus applied to the full MC cocktail. 
%To do so we define for the pre-veto sample (labeled `all'):
%
%\begin{itemize}
%\item $N_{\rm{peak}}^{\rm{all}}$ = data yield in the peak region $60<\mt<100$ GeV
%\item $M_{\rm{peak}}^{\rm{all}}$ = MC yield in the peak region $60<\mt<100$ GeV
%\item $SF^{\rm{all}} = N_{\rm{peak}}^{\rm{all}} / M_{\rm{peak}}^{\rm{all}}$
%\end{itemize}
%
%For all subsequent steps, the scale factor $SF^{\rm{all}}$ is applied to all MC contributions.
%
%\subsection{Step 3: Isolated Track Veto Efficiency Correction}
%
%The signal sample is defined by the same requirements as the pre-veto sample, except that we veto events containing
%an isolated track. The background is the inclusive sample can be split into 2 contributions:
%
%\begin{itemize}
%\item Dilepton background: mostly \ttll.
%\item Single lepton background: mostly \ttlj\ and \wjets;
%\end{itemize}
%
%The isolated track veto impacts these 2 contributions in different ways. For the dilepton background, the veto
%rejects events which have a genuine 2nd lepton, so applying the isolated track veto scales the dilepton background
%by the efficiency $\epsilon(trk)$ to identify the isolated track. For the single lepton background, the veto rejects events
%which do not have a genuine 2nd lepton but which have a ``fake'' isolated track, so the isolated track veto scales
%the single lepton background by (1-FR), where FR is the ``fake rate'' to identify an isolated track in events which 
%contain no genuine 2nd lepton. 
%
%The isolated track efficiency $\epsilon$ can be measured in data and MC using $Z\to\ell\ell$ tag-and-probe studies. 
%We therefore use the tag-and-probe studies to apply a correction to the \ttll\ MC. 
%A measurement of the FR in data is non-trivial, but we can account for differences between the data and the MC by scaling 
%only the single lepton background in the \mt\ peak region after applying the isolated track veto.
%
%In detail, the procedure to correct the dilepton background is:
%
%\begin{itemize}
%\item Using tag-and-probe studies, we plot the distribution of {\bf absolute} track isolation for identified probe electrons
%and muons {\bf TODO: need to compare the e vs. $\mu$ track iso distributions, they might differ due to e$\to$e$\gamma$}.
%\item We verify that the distribution of absolute track isolation does not depend on the \pt\ of the probe lepton.
%This is due to the fact that this isolation is from ambient PU and jet activity in the event, which is uncorrelated with
%the lepton \pt {\bf TODO: verify this in data and MC.}.
%\item Our requirement is {\bf relative} track isolation $<$ 0.1. For a given \ttll\ MC event, we determine the \pt of the 2nd
%lepton and translate this to find the corresponding requirement on the {\bf absolute} track isolation, which is simply $0.1\times$\pt.
%\item We measure the efficiency to satisfy this requirement in data and MC, and define a scale-factor $SF_{\epsilon(trk)}$ which
%is the ratio of the data-to-MC efficiencies. This scale-factor is applied to the \ttll\ MC event.
%\item {\bf THING 2 we are unsure about: we can measure this SF for electrons and for muons, but we can't measure it for hadronic 
%tracks from $\tau$ decays. Verena has showed that the absolute track isolation distribution in hadronic $\tau$ tracks is harder due 
%to $\pi^0\to\gamma\gamma$ with $\gamma\to e^+e^-$.}
%\end{itemize} 
%
%At this point we are done applying scale factors to the dilepton background. We next apply a scale factor to the single lepton background. We use the following quantities:
%
%%Use the signal sample to apply a correction factor to the single lepton background
%
%\begin{itemize}
%\item $N_{\rm{peak}}^{\rm{veto}}$ =  data yield in the peak region $60<\mt<100$ GeV
%\item $M^{\ell,\rm{veto}}_{\rm{peak}}$ =  single lepton background MC in the peak region $60<\mt<100$ GeV
%\item $M^{\ell\ell,\rm{veto}}_{\rm{peak}}$ =  dilepton background MC in the peak region $60<\mt<100$ GeV
%\item $SF_{\ell}^{\rm{veto}} = (N_{\rm{peak}}^{\rm{veto}} - M^{\ell\ell,\rm{veto}}_{\rm{peak}} SF_{\epsilon(trk)} )/ M^{\ell,\rm{veto}}_{\rm{peak}}$
%\end{itemize}
%
%The scale factor $SR_{\ell}$ is applied to the single lepton background to account for potential data vs. MC discrepancies 
%in the isolated track veto.
%
%To summarize, the dilepton and single lepton background prediction in the signal region (\mt\ $>150$ GeV) are given by
%
%\begin{itemize}
%\item $\rm{P}^{\ell\ell}_{\rm{sig}} = SF^{\rm{all}} \times SF_{\epsilon(trk)} \times M^{\ell \ell}_{\rm{sig}}$
%\item $\rm{P}^{\ell}_{\rm{sig}} = SF^{\rm{all}} \times SF_{\ell}^{\rm{veto}} \times M^{\ell}_{\rm{sig}}$
%\end{itemize}
%
%where $M_{\rm{sig}}$ correspond to the Monte Carlo predictions in the tail of the distributions and the $SF$ terms are data over MC scale factors to account 
%for differences between the data and the MC for the following effects
%
%\begin{itemize}
%\item $SF^{\rm{all}} \rightarrow $ effects impacting the normalization with the exception of the isolated track veto i.e. luminosity, 
%lepton identification and trigger efficiencies $\dots$
%\item $SF_{\epsilon(trk)} \rightarrow $ the track veto efficiency for real second leptons
%\item $SF_{\ell}^{\rm{veto}} \rightarrow $ the inefficiency introduced by the track veto on events without a second lepton (1-FR defined previously)
%\end{itemize}
%
%%SF_{\ell}^{\rm{post-veto}}$ 
%%\item $M^{\ell,\rm{sig}}_{\rm{peak}}$ =  single lepton background MC in the peak region $60<\mt<100$ GeV

