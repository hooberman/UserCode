
\section{Introduction}
\label{ref:intro}

The published analysis 
%``A Search For New Physics in Z + Jets + MET using MET Templates'' %AN/old title
%arxiv title
``Search for physics beyond the standard model in events with a Z boson, jets, 
and missing transverse energy in pp collisions at $\sqrt{s}$ = 7 TeV''
(SUS-11-021) 
searches for new physics in the final state of a 
leptonically ($ee$ and $\mu\mu$) decaying Z boson, two or more jets and 
missing transverse energy (\MET) 
\cite{ref:oszpaper} \cite{ref:osznote} \cite{ref:oszpas}.
This analysis will be referred to throughout this note as the ``nominal''
analysis. The basic analysis strategy is to select Z bosons and 
use data-driven methods to predict the \MET\ distribution in the signal 
regions.
The Z+Jets background is predicted using the 
\MET\ templates method \cite{ref:templates1}\cite{ref:templates2}, the 
\ttbar\ background is predicted using opposite flavor ($e\mu$) events, 
and the diboson (WZ, ZZ) background is taken from Monte Carlo.

The analysis presented in this note is a straightforward extension of
the nominal analysis in that the analysis strategy and methodology
remain unchanged. 
The only changes with respect to the nominal analysis are the addition
of cuts to increase sensitivity to new physics with 
diboson production (WZ and/or ZZ) and \MET .

An example of one such new physics scenario is the electroweak
production of SUSY particles.
In a generic SUSY framework, the neutralinos 
(for example, $\chi_2^0$ or $\chi_1^0$)
may decay to a Z boson and another neutral SUSY particle such as the LSP. 

%Although SUSY production involving strongly interacting particles 
%(such as gluinos and squarks) is normally targeted due to its expected 
%larger production cross section as compared with electroweak production, 
%such searches have as of yet failed to discover new physics. 
%Another logical search is for electroweak production, 
%and in this case, a final state involving Z bosons is a natural place 
%to start since leptonically decay Zs are an extremely clean signature.

In the case in which a neutralino is pair produced, the final state 
may be ZZ+\MET. In addition, production of a neutralino and chargino 
may lead to a final state of WZ+\MET.  
When the Z decays leptonically and the other boson (either 
a W or Z) decays to jets, the final state is Z plus two jets plus \MET,
to which the nominal \MET\ templates analysis is particularly well suited. 
Given that we are now searching for the specific final states WZ plus \MET\ or ZZ plus \MET ,
rather than the more general Z plus jets plus \MET\ signature,
we can apply additional cuts to increase the sensitivity.

In the nominal analysis, the search is performed in the high \MET\ tail.
\MET\ cuts used for signal regions are 100, 200, and 300 GeV.
At such high \MET\ cuts, \ttbar\ background in which (the same-flavor 
opposite-sign) dileptons happen to fall in the Z mass window dominate. 
Because all \ttbar\ events contain b jets, a b jet veto is 
very effective in suppressing this background.

Because the final state targeted involves the decay of W (Z) to jets, 
the dijet mass peaks at the W (Z) mass. In contrast, the jets from the
background processes Z+jets and \ttbar\ have a very broad distribution.
The dijet mass is therefore a variable which can further discriminate between 
signal and background (see section \ref{sec:eventSelection}).

The final sub-leading backgrounds in the nominal analysis are Z plus jets 
and dibosons (WZ and ZZ). In the case of WZ, real \MET\ is produced from 
the leptonically decaying W. In order to suppress this background, we 
introduce a third lepton veto.

In summary, we use the same selection as in the approved analysis SUS-11-021 and
place three additional requirements:
\begin{itemize}
\item Veto events containing a b-tagged jet
\item Require a dijet mass consistent with the hadronic decay of a W/Z boson
\item Veto events with a third lepton
\end{itemize}

This note is organized as follows. 
In Sec.~\ref{sec:datasets} we review the datasets used.
In Sec.~\ref{sec:eventSelection} we discuss the event selection.
In Sec.~\ref{sec:yields} we present the data and MC yields passing the event preselection.
In Sec.~\ref{sec:sigregion} we define the signal regions.
In Sec.~\ref{sec:results} we present the results.
In Sec.~\ref{sec:systematics} we discuss systematics on the background predictions.
In Sec.~\ref{sec:bsm} we provide a new physics interpretation of the results.
Additional material is included in the following appendices: 
supplemental results (App. \ref{app:results}),
supplemental interpretation (App. \ref{sec:app_bsm}),
kinmatical distributions (App. \ref{sec:appkin}),
combination of interpretation results (App. \ref{app:combo}),
and the \MET\ templates (App. \ref{sec:appendix_templates}).

%%%%%%%%%%%%%%%%%%%%%%%%%%%%%%%%%%%%%%%%%%%%%%%%%%%%%%%%%%%%%%%%%%%%%%

%OLD NOTE

\begin{comment}

In this note we describe a search for new physics in the 2011 
opposite sign isolated dilepton sample ($ee$, $e\mu$, and $\mu\mu$).  
The main sources of high \pt isolated dileptons at CMS are Drell Yan and \ttbar.
Here we concentrate on dileptons with invariant mass consistent
with $Z \to ee$ and $Z \to \mu\mu$.  A separate search for new physics in the non-\Z
sample is described in~\cite{ref:GenericOS}.

We search for new physics in the final state of \Z plus two or more jets plus missing 
transverse energy (MET). We reconstruct the \Z boson
in its decay to $e^+e^-$ or $\mu^+\mu^-$. Our search regions are defined as 
MET $\ge$ \signalmetl~GeV (loose signal region), MET $\ge$ \signalmett~GeV 
(medium signal region), MET $\ge$ 300~GeV (tight signal region), and two or more jets. We use data driven techniques to predict the
standard model background in these search regions. 
Contributions from Drell-Yan production combined with detector mis-measurements that 
produce fake MET are modeled via MET templates based on photon plus jets or QCD events. 
Top pair production backgrounds, as well as other backgrounds for which the lepton
flavors are uncorrelated such as $W^+W^-$, DY$\rightarrow\tau\tau$, and single top, are 
modeled via $e^\pm\mu^\mp$ subtraction.

As leptonically decaying \Z bosons are a signature that has very little background, 
they provide a clean final state in which to search for new physics. 
Because new physics is expected to be connected to the Standard Model Electroweak sector, 
it is likely that new particles will couple to W and Z bosons. 
For example, in mSUGRA, low $M_{1/2}$ can lead to a significant branching fraction 
for $\chi_2^0 \rightarrow Z \chi_1^0$. 
In addition, we are motivated by the existence of dark matter to search for new physics with MET.
Enhanced MET is a feature of many new physics scenarios, and R-parity conserving SUSY 
again provides a popular example. The main challenge of this search is therefore to 
understand the tail of the fake MET distribution in \Z plus jets events.

The basic idea of the MET template method~\cite{ref:templates1}\cite{ref:templates2} is 
to measure the MET distribution in data in a control sample which has no true MET 
and a similar topology to the signal events. 
%Start the qcd vs photon discussion
Templates can be derived from either a QCD sample (as was done in the original implementation) 
or a photon plus jets sample.
%In our case, we choose a photon sample with two or more jets as the control sample.
%Both the control sample and signal sample consist of a well measured object (either a 
%photon or a leptonically decaying $Z$), which recoils against a system of hadronic jets. 
In both cases, the instrumental MET is dominated by mismeasurements of the hadronic system,
and can be classified by the number of jets in the event and the scalar sum of their transverse
momenta.
The prediction is made such that the jet system in the control sample is similar to that of the
signal sample.
By using two independent control samples--QCD and photon plus jets--we are able to illustrate
the robustness of the MET templates method and to cross check the data driven background 
prediction.

This note is organized as follows. 
In sections \ref{sec:datasets} and \ref{sec:trigSel} we descibe
the datasets and triggers used, followed by the detailed object definitions (electrons, muons, photons,
jets, MET) and event selections in sections \ref{sec:evtsel} through \ref{sec:jetsel}.
We define a preselection and compare data vs. MC yields passing this preselection in 
Section~\ref{sec:yields}.
We then define the signal regions and show the number of observed events and MC expected 
yields in Section~\ref{sec:sigregion}.
Section~\ref{sec:templates} introduces the MET template method and discusses its derivation 
in some detail.
% and is followed by a demonstration in Section~\ref{sec:mc} 
%that the method works in Monte Carlo.
Section~\ref{sec:topbkg} introduces the top background estimate based on opposite flavor subtraction, 
and contributions from other backgrounds are discussed in Section~\ref{sec:othBG}.
Section~\ref{sec:results_combined} shows the results for applying these methods in data.
We analyze the systematic uncertainties in the background predictions and in signal acceptance
in Section~\ref{app:systematics}. 
We then proceed to calculate upper limit on the BSM physics processes
in Section~\ref{sec:bsm}. 
%Efficiencies which can be used to test specific models of new physics are given 
%in Section \ref{sec:outreach}.
%Finally, in Section~\ref{sec:models} we calculate upper limits on the quantity \sta\
%assuming efficiencies and uncertainties from sample benchmark SUSY processes. 

\end{comment}
