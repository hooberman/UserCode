\section{Example BG prediction calculation}
\label{BGexample}

The calculation of the background prediction is a bit complicated.
Here we walk the reader through a concrete example. 

 {\bf NB: the numbers
in this section corresponded to the numbers in V2 of the analysis note. 
They will not be updated, because this is meant as an illustration only.}.

The main background is $t\bar{t}$
The main idea is to normalize to the $M_T$ peak region ($50 < M_T < 80$ GeV).
This eliminates dependence on \ttbar\ cross-section, luminosity,
trigger efficiency, JES, lepton ID, etc.  This gets a bit complicated because
the $M_T$ peak region, while dominantly \ttbar\ lepton $+$ jets,
also includes some \wjets, \ttdl, rare processes, etc.  Also, we want
to minimize our need to understand the effect of the isolated 
track veto on \ttsl.  As a result we actually define two $M_T$ peak
regions: one before and one after applying the isolated track veto.
Then the \ttdl\ background is normalized the the ``before veto'' region,
and the \ttsl\ and \wjets\ background are normalized to the ``post veto''
region.

This complex procedure is important for the high statistics signal regions 
with relatively low \met\ requirements, eg, SRA.  For these SRs we want to keep the 
systematics low in order to be sensistive to low mass stop; for the signal regions
with hard cuts on \met\, this is less important.  However, we apply the same 
procedure to all SRs.

For concreteness, we show the calculation for SRA, electron channel.  The MC and data
event counts used in the background calculation are collected in Table~\ref{tab:bgexample}.
Note that the background uncertainties have already been described 
in Section~\ref{sec:systematics}.  The one tricky point to keep in mind is that 
when the \wjets\ and rare cross-sections are changed by their assumed uncertainties
(50\% each), the whole calculation describe below is repeated in order to take care
of all the correlations properly.


\begin{table}[!h]
\begin{center}
\begin{tabular}{l||c|c|c|}
\hline
Sample              &  $M_T$ peak, before trk veto  &  $M_T$ peak, after trk veto  & Signal Region A \\
\hline
\ttdl\                & $290 \pm 6$                          & $116 \pm 4$                       & $261 \pm 6$  \\
\ttsl\  (1\Lep)   & $2899 \pm 19$                      & $2595 \pm 18$                   &   not used    \\
\wjets\             & $252 \pm 28$                        & $236 \pm 28$                     &  not used   \\
Rare                  & $89 \pm 5$                            & $62 \pm 4$                         & $26 \pm 2$ \\
\hline 
Total                & $3530 \pm 35$                      & $3009 \pm 34$                    & not used \\
\hline
Data                & 3358                                       & 2787                                     & not used  \\
\hline
\end{tabular}
\caption{ Data and MC event counts used to predict the background in SRA, for electron events.
Uncertainties are statistical only.  The trigger efficiency has been applied to the MC samples.  In the
case of \ttdl\, the $K_3$ and $K_4$ factors of Section~\ref{sec:jetmultiplicity} have also been applied.
\label{tab:bgexample}}
\end{center}
\end{table}


\subsection{Central value of dilepton background}
A ``before veto'' scale factor is defined from the second column in Table~\ref{tab:bgexample} as the factor by which 
all MC except the ``rare'' need to be scaled up in order to have data/MC agreement.  This is 

\noindent $SF_{pre} = (3358 - 89)/(2899 + 252 + 290) = 0.950$.

Then the \ttdl\ background prediction is the number of events predicted by the MC  in SRA (261 from the
last column of SRA), rescaled by $SF_{pre}$.  The result for the central value is 248 events.

\subsection{Central value of the \ttsl\ background}
\label{sec:cenvttlj}
A ``post veto'' scale factor is defined from the third column in Table~\ref{tab:bgexample} as the factor by which
the \ttsl\ and the \wjets\ backgrounds need to be scaled to have data/MC agreement.  

\noindent $SF_{post} = (2787 - 62 - SF_{pre} \cdot 116) / (2595 + 236) = 0.924$.

Then the \ttsl\ background is obtained as the product of the following four factors

\begin{itemize}
\item $SF_{post} = 0.924$ as obtained above
\item 2595, from the third column of Table~\ref{tab:bgexample}
\item The tail-to-peak ratio $R_{top} = 0.015$ from Table~\ref{tab:ttp} (we use the average of electrons
and muons)
\item The tail-to-peak ratio scale factod $SFR_{top} = 1.89 \pm 0.56$ from Table~\ref{tab:cr2yields}.
\end{itemize}

The result for the central value is 68 events.

\subsection{Central value for the \wjets\ background}

It is calculated as the product of
\begin{itemize}
\item $SF_{post} = 0.924$ from Section~\ref{sec:cenvttlj}
\item 236, from the third column of Table~\ref{tab:bgexample}
\item The tail-to-peak ratio $R_{wjet} = 0.04$ from Table~\ref{tab:ttp} (we use the average of electrons and muons)
\item The tail-to-peak ratio scale factod $SFR_{wjet} = 1.64 \pm 0.38$ from Table~\ref{tab:cr1yields}.
\end{itemize}

The result for the central value is 14.3 events.

\subsection{Central value for the rare backgrounds}

This is 26 events from Table~\ref{tab:bgexample}

