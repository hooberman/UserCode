%%%%%%%%%%%%%%%%%%%%%%% file template.tex %%%%%%%%%%%%%%%%%%%%%%%%%
%
% This is a template file for Web of Conferences Journal
%
% Copy it to a new file with a new name and use it as the basis
% for your article
%
%%%%%%%%%%%%%%%%%%%%%%%%%% EDP Science %%%%%%%%%%%%%%%%%%%%%%%%%%%%
%
\documentclass[epj,twocolumn]{webofc}
\usepackage[varg]{txfonts}   % Web of Conferences font
\woctitle{Hadron Collider Physics symposium 2012}

%\newcommand{\met}{\mbox{$\raisebox{.3ex}{$\not$}E_T$\hspace*{0.5ex}}} 
\newcommand{\met}{\ensuremath{\rm{E_{T}^{miss}}}}
\newcommand{\lsp}{\ensuremath{\tilde{\chi}_{1}^{0}}}
\newcommand{\chip}{\ensuremath{\tilde{\chi}_{1}^{\pm}}}
\newcommand{\pt}{\ensuremath{p_T}}
\newcommand{\mt}{\ensuremath{M_T}}
\newcommand{\wjets}{\ensuremath{\rm{W+jets}}}
\newcommand{\ttljets}{\ensuremath{t\bar{t}\to\ell+\rm{jets}}}
\newcommand{\ttll}{\ensuremath{t\bar{t}\to\ell\ell}}
\newcommand{\alphat}{\ensuremath{\alpha_{T}}}
%
%
\begin{document}
%
\title{Searches for Top and Bottom Squarks at CMS}
%
% subtitle is optionnal
%
%%%\subtitle{Do you have a subtitle?\\ If so, write it here}

\author{Benjamin Hooberman\inst{1,3}\fnsep\thanks{\email{benhoob@fnal.gov}}
%        Second author\inst{2}\fnsep\thanks{\email{Mail address for second
%             author if necessary}} \and
%        Third author\inst{3}\fnsep\thanks{\email{Mail address for last
%             author if necessary}}
        % etc.
}

\institute{Fermi National Accelerator Laboratory
%\and
%           the second here 
%\and
%           Last address
          }

\abstract{%
  Supersymmetry is a popular extension to the standard problem, which may solve the hierarchy
  problem in a natural (not fine-tuned) way if it introduces top and bottom squarks with masses
  not more than several hundred GeV. This note presents the results from three searches for these particles, either
  produced directly in pairs or in the decays of gluinos. The searches are performed
  in the single lepton final state focusing on events with large transverse mass, the same-sign final state,
  and the all-hadronic final state. No evidence for the production of top and bottom squarks is observed.
  The results are used to place stringent constraints on the masses of these particles.
}
%
\maketitle
%


\section{Introduction}

In this note we describe a search for physics beyond the standard model (BSM) 
in a sample of proton-proton collisions at a centre-of-mass energy of 7~TeV. 
The data sample was collected with the Compact Muon Solenoid (CMS) detector~\cite{JINST} at 
the Large Hadron Collider (LHC) in 2011
and corresponds to an integrated luminosity of \lumi.
We search for new physics in events containing
opposite sign isolated lepton pairs ($ee$, $e\mu$, and $\mu\mu$).
The main sources of high \pt\ isolated dileptons at CMS are Drell Yan and \ttbar.
Here we concentrate on dileptons with invariant mass consistent
with $Z \to ee$ and $Z \to \mu\mu$.  A separate search for new physics in the non-\Z
sample is described in~\cite{ref:ospaper}.

We search for new physics in the final state of \Z plus two or more jets plus missing 
transverse energy (MET). We reconstruct the \Z boson
in its decay to $e^+e^-$ or $\mu^+\mu^-$. Our signal regions are defined as 
MET $>$ \signalmetl~GeV (loose signal region) and MET $>$ \signalmett~GeV 
(tight signal region). We use data driven techniques to predict the
standard model (SM) backgrounds in these search regions. 
Contributions from Drell-Yan production combined with detector mis-measurements that 
produce fake MET are modeled via the MET templates  method~\cite{ref:templates1,ref:templates2}
based on control samples of photon plus jets and QCD multijets events.
Top pair production backgrounds, as well as other backgrounds for which the lepton
flavors are uncorrelated such as $W^+W^-$ and DY$\rightarrow\tau^+\tau^-$, are 
modeled via $e^\pm\mu^\mp$ subtraction.

As leptonically decaying \Z bosons provide a signature that has very little background, 
they provide a clean final state in which to search for new physics. 
Because new physics is expected to be connected to the SM Electroweak sector, 
it is likely that new particles will couple to W and Z bosons. 
For example, in mSUGRA, low $M_{1/2}$ can lead to a significant branching fraction 
for $\chi_2^0 \rightarrow Z \chi_1^0$. 
In addition, we are motivated by the existence of dark matter to search for new physics with MET.
Enhanced MET is a feature of many new physics scenarios, and R-parity conserving SUSY 
again provides a popular example. The main challenge of this search is therefore to 
understand the tail of the fake MET distribution in \Z plus jets events.

%The search is not optimized in the context of any particular BSM physics, e.g. specific SUSY model.
No specific BSM  physics scenario, e.g.\ a particular  SUSY model, has
been  used  to  optimize  the  search.  In  order  to  illustrate  the
sensitivity of  the search, a  simplified and practical model  of SUSY
breaking,  the  constrained minimal  supersymmetric  extension of  the
standard  model  (CMSSM)~\cite{CMSSM,CMSSM2}, is  used.  The CMSSM  is
described by  five parameters: the  universal scalar and  gaugino mass
parameters   ($m_0$  and   $m_{1/2}$,  respectively),   the  universal
trilinear soft SUSY breaking parameter  $A_0$, the ratio of the vacuum
expectation values  of the two  Higgs doublets ($\tan\beta$),  and the
sign of  the Higgs mixing  parameter $\mu$.  Throughout the  note, two
CMSSM parameter sets, referred to  as LM4 and LM8~\cite{TDR}, are used
to  illustrate possible  CMSSM yields.   In these  scenarios, opposite
sign  leptons are  produced  in  the decays  of $Z$  bosons, which  are
produced  in  the  cascade  decays  of heavy,  colored  objects.   The
parameter  values defining  LM4  (LM8) are  $m_0  = 210~(500)$\GeVcc,
$m_{1/2} = 285~(300) \GeVcc$,  $A_0 = 0~(-300)\GeV$; both LM4 and
LM8 have $\tan\beta = 10$ and $\mu  > 0$. 
In this analysis, the  LM4 and LM8 scenarios serve as benchmarks
which may  be used to allow  comparison of the  sensitivity with other
analyses.

\section{Search for Top Squark Pair Production in the Single Lepton Final State}
\label{sec:stop}

This section presents the results of a dedicated search for the direct pair production of top squarks, based on an integrated luminosity of 9.7~fb$^{-1}$.
The decay of the top squark depends on the difference between its mass and that of the \lsp\ LSP,
$\Delta m = m_{\tilde{t}}-m_{\lsp}$. If $\Delta m > m_{t}$, the decay $\tilde{t}\to t\lsp$ is expected
to have a large branching fraction. If there is a light chargino \chip, the decay 
$\tilde{t}\to b\chip\to b W \lsp$ is expected to be significant, especially in the $\Delta m < m_{t}$ region.
The pair production of top squarks decaying to either of these channels leads to events with two b-jets, two W bosons,
and \met\ from the invisible LSPs. Our signal thus resembles SM $t\bar{t}$ production but with larger \met\ from
the invisible LSPs.
We focus on the single lepton final state, which has a significant branching fraction due to the two W bosons,
and smaller SM backgrounds than the all-hadronic final state.
We thus select events with a single lepton and jets and discriminate between
signal and background using \met\ and the transverse mass \mt, discussed below.

%\subsection{Event Selection}

We require the presence of exactly one well-identified and isolated lepton (e or $\mu$) with transverse
momentum \pt\ $>$ 30 GeV. 
We select events with at least four jets with \pt\ $>$ 30 GeV,
which must be well-separated from the selected leptons.
At least one of these jets is required to be consistent with coming from the decay of heavy flavor hadrons, as
identified by the Combined Secondary Vertex Medium Point (CSVM) b-tagging algorithm~\cite{ref:btag}.
The jet requirements suppress SM backgrounds from W bosons produced in association with jets from initial state
radiation (ISR), referred to as the \wjets\ background. 
The \met\ is required to exceed 50 GeV, suppressing the background from QCD multijet production.

%\subsection{Backgrounds and Estimation Strategy}

The SM background satisfying the above requirements is dominated by $t\bar{t}$ production where
one W boson decays hadronically and the other leptonically (\ttljets), or where both W bosons decay leptonically (\ttll).
There is a small contribution from \wjets, as well as a variety of rare SM
processes, dominated by $t\bar{t}$ produced in association with a vector boson
($t\bar{t}W$ and $t\bar{t}Z$).

To define signal regions, we require the events to have large transverse mass, defined as:

\begin{equation}
M_T = \sqrt{ 2 p_{T}^{\ell} \met ( 1-cos(\Delta\phi))},
\end{equation}

where $p_{T}^{\ell}$ is the lepton transverse momentum and $\Delta\phi$ is the difference in azimuthal angles between the lepton
and \met. This requirement strongly suppresses the background from \ttljets\ and \wjets, which have a kinematic endpoint
at \mt\ $=$ $M_W$ since the lepton and neutrino (which produces the \met) are produced together in the decay of the W.
For signal events, as well as the \ttll\ background, the presence of more than one invisible
particle in the final state leads to events with \mt\ $>>$ $M_W$. 
In addition to the \mt\ requirement, we make several 
\met\ requirements to achieve sensitivity to signals with different mass spectra.
Signal regions with large (small) \met\ requirements are more sensitive to signals with large (small) values of $\Delta m$.

The dominant background in our signal regions is \ttll, which may produce events with large \met\ and \mt\ due to the presence of
two invisible neutrinos. In order for \ttll\ events to pass the signal region selection, one of the two W leptons must not be identified,
which occurs if it is outside the acceptance, is a hadronic $\tau$ decaying to three charged particles (3-prong decay),
is a hadronic $\tau$ decaying to a single charged particle (1-prong decay), or is an electron or muon that fails the lepton identification
requirements. The latter two categories are suppressed by vetoing events that contain, in addition to the selected lepton, 
a charged particle with \pt\ $>$ 10 GeV that is isolated in space from other energetic charged particles. Furthermore, additional jets 
from initial state or final state radiation (ISR/FSR) are required to satisfy the jet multiplicity requirement $n_{jets}\geq4$. 
To validate and correct the MC modeling of jets from radiation, the MC is compared to data in a dilepton control region dominated by \ttll. The MC distribution of $n_{jets}$ is reweighted to 
match the corresponding data distribution, resulting in corrections of (1--7)\%.

The SM backgrounds are estimated from events simulated with Monte Carlo (MC) techniques, which are validated and 
(if necessary) corrected using comparisons to data in control regions. The MC expectation is normalized to data in the \mt\ peak region,
in order to remove systematic uncertainties from integrated luminosity and $t\bar{t}$ cross section, and then extrapolated to the 
large \mt\ region. Correction factors and corresponding systematic uncertainties on the MC extrapolation factors are evaluated by 
comparing MC to data in dedicated control regions dominated by \wjets\ (obtained by vetoing events with b-jets), \ttll\ 
(obtained by requiring two selected leptons), and a mixture of \ttll\ and \ttljets\ (obtained by requiring a selected lepton and 
an isolated track). The dominant systematic uncertainty in the background prediction is due to the limited statistical precision in
the data control samples used for these tests.


	\begin{table}[!h]																															
	\begin{center}																															
	{\footnotesize																															
	\begin{tabular}{l||c|c|c|c|c|c|c}																															
	\hline																															
	Sample		&	SRA			&	SRB			&	SRC			&	SRD			&	SRE			&	SRF			&	SRG\\				
	\hline																															
	\hline																															
	\multicolumn{8}{c}{Muon}	\\																														
	\hline																															
	\ttdl\  		&$	330.6	\pm	21.9	$&$	183.4	\pm	20.7	$&$	59.5	\pm	10.0	$&$	22.5	\pm	6.2	$&$	9.0	\pm	3.9	$&$	3.7	\pm	1.8	$&$	2.2	\pm	1.2	$	\\
	\ttsl\ \& single top (1\Lep) 		&$	92.8	\pm	27.5	$&$	41.0	\pm	8.6	$&$	11.5	\pm	3.5	$&$	7.7	\pm	3.4	$&$	0.7	\pm	0.6	$&$	0.3	\pm	0.2	$&$	0.2	\pm	0.2	$	\\
	\wjets\ 		&$	19.2	\pm	4.5	$&$	10.0	\pm	2.2	$&$	3.1	\pm	1.0	$&$	1.2	\pm	0.6	$&$	0.6	\pm	0.4	$&$	0.4	\pm	0.3	$&$	0.2	\pm	0.2	$	\\
	Rare 		&$	33.2	\pm	16.6	$&$	22.7	\pm	11.4	$&$	9.0	\pm	4.5	$&$	4.8	\pm	2.4	$&$	2.9	\pm	1.5	$&$	1.2	\pm	0.6	$&$	1.0	\pm	0.5	$	\\
	\hline																															
	Total 		&$	475.8	\pm	37.8	$&$	257.2	\pm	24.2	$&$	83.2	\pm	11.3	$&$	36.2	\pm	7.4	$&$	13.3	\pm	4.2	$&$	5.5	\pm	1.9	$&$	3.6	\pm	1.3	$	\\
	\hline																															
	\hline																															
	Data 		&$	?			$&$	?			$&$	?			$&$	?			$&$	?			$&$	?			$&$	?			$	\\
	\hline																															
	\hline																															
	\hline																															
	\multicolumn{8}{c}{Electron}	\\																														
	\hline																															
	\ttdl\  		&$	248.1	\pm	16.9	$&$	144.4	\pm	16.6	$&$	51.1	\pm	8.8	$&$	16.2	\pm	4.6	$&$	5.5	\pm	2.5	$&$	2.5	\pm	1.3	$&$	1.3	\pm	0.7	$	\\
	\ttsl\ \& single top (1\Lep) 		&$	68.0	\pm	20.2	$&$	31.2	\pm	6.6	$&$	9.3	\pm	2.8	$&$	4.9	\pm	2.1	$&$	0.5	\pm	0.4	$&$	0.2	\pm	0.2	$&$	0.2	\pm	0.2	$	\\
	\wjets\ 		&$	14.3	\pm	3.3	$&$	7.5	\pm	1.7	$&$	2.4	\pm	0.8	$&$	0.8	\pm	0.4	$&$	0.4	\pm	0.3	$&$	0.3	\pm	0.2	$&$	0.1	\pm	0.2	$	\\
	Rare 		&$	25.8	\pm	12.9	$&$	15.8	\pm	7.9	$&$	7.1	\pm	3.6	$&$	2.9	\pm	1.5	$&$	0.7	\pm	0.4	$&$	0.3	\pm	0.2	$&$	0.1	\pm	0.1	$	\\
	\hline																															
	Total 		&$	356.2	\pm	28.4	$&$	198.9	\pm	19.0	$&$	69.9	\pm	9.7	$&$	24.7	\pm	5.3	$&$	7.1	\pm	2.5	$&$	3.4	\pm	1.3	$&$	1.7	\pm	0.8	$	\\
	\hline																															
	\hline																															
	Data 		&$	?			$&$	?			$&$	?			$&$	?			$&$	?			$&$	?			$&$	?			$	\\
	\hline																															
	\hline																															
	\hline																															
	\multicolumn{8}{c}{Muon+Electron Combined}		\\																													
	\hline																															
	\ttdl\  		&$	578.7	\pm	38.1	$&$	327.8	\pm	36.6	$&$	110.6	\pm	18.3	$&$	38.7	\pm	10.5	$&$	14.5	\pm	6.2	$&$	6.2	\pm	2.9	$&$	3.5	\pm	1.8	$	\\
	\ttsl\ \& single top (1\Lep) 		&$	160.8	\pm	47.7	$&$	72.2	\pm	15.1	$&$	20.8	\pm	6.3	$&$	12.6	\pm	5.4	$&$	1.2	\pm	0.9	$&$	0.6	\pm	0.4	$&$	0.4	\pm	0.3	$	\\
	\wjets\ 		&$	33.5	\pm	8.0	$&$	17.5	\pm	4.1	$&$	5.5	\pm	1.9	$&$	2.0	\pm	1.2	$&$	1.0	\pm	0.7	$&$	0.7	\pm	0.5	$&$	0.3	\pm	0.4	$	\\
	Rare 		&$	59.0	\pm	29.5	$&$	38.5	\pm	19.3	$&$	16.1	\pm	8.1	$&$	7.7	\pm	3.9	$&$	3.6	\pm	1.8	$&$	1.5	\pm	0.8	$&$	1.1	\pm	0.6	$	\\
	\hline																															
	Total 		&$	832.0	\pm	65.7	$&$	456.1	\pm	42.5	$&$	153.0	\pm	20.6	$&$	60.9	\pm	12.4	$&$	20.3	\pm	6.5	$&$	8.9	\pm	3.0	$&$	5.3	\pm	1.9	$	\\
	\hline																															
	\hline																															
	Data 		&$	?			$&$	?			$&$	?			$&$	?			$&$	?			$&$	?			$&$	?			$	\\
	\hline																															
	\end{tabular}}																															
	\caption{The result of the search.																															
	\label{tab:result}																															
	\end{center}}																															
	\end{table}																															


%\subsection{Results}

The results of the search are summarized in Table~\ref{tab:stop}, which displays the SM background expectations and the observed data yields
in the signal regions. The distribution of \met\ after the requirement \mt\ $>$ 120 GeV for the SM background expectations is compared to
data in Fig.~\ref{fig:stop}. Good agreement between the data and the expected background is observed. We find no evidence
for the pair production of top squarks.

\begin{figure}
% Use the relevant command for your figure-insertion program
% to insert the figure file.
\centering
\includegraphics[width=0.4\textwidth]{HCPPlots/stopmet.pdf}
%\includegraphics[width=7cm,clip]{HCPPlots/stopmet.pdf}
\caption{The \met\ distribution in data, compared to the sum of expected backgrounds, for the top squark pair search.
Two example signal models are also indicated.}
\label{fig:stop}       % Give a unique label
\end{figure}

%\subsection{Interpretation}

%To interpret the results of our search, we consider two signal scenarios of top squark pair production, followed by the decays
%$\tilde{t}\to t\lsp$ and $\tilde{t}\to b\chip\to b W \lsp$. In the first scenario, the only SUSY particles which participate
%are the top squark and \lsp, and the model can thus be parameterized by the masses of these two particles. In the second case
%the chargino mass is also relevant, and we introduce a third parameter $x$, defined as $m_{\chip} = x m_{\lsp} + (1-x) m_{\tilde{t}}$.
%We consider $x=0.5$ and $x=0.75$ (we do not have sensitivity to the $x=0.25$ scenario).

To interpret the results of our search, we consider top squark pair production where both top squarks decay according to 
$\tilde{t}\to t\lsp$ as depicted in Fig.~\ref{fig:diagrams}(a).
The model is parameterized by the masses of the top squark and \lsp. We place upper limits on the signal
production cross section using, for each model point in the 2-dimensional parameter space, the signal region with the best expected
sensitivity. A region of the parameter space is excluded by comparing these cross section upper limits with the theoretical predictions 
for the signal cross section.
%, computed at next-to-leading order including the resummation of soft gluon emission at 
%next-to-leading-logarithmic
%accuracy (NLO+NLL)~\cite{ref:nlonll}. 
Our results probe top squarks with masses up to 430 GeV. For comparison, the requirement that SUSY
provides a natural solution to the hierarchy problem suggests top squarks with masses not exceeding 500--700 GeV~\cite{ref:naturalsusy}.
We also interpret our results assuming the top squark decays according to $\tilde{t}\to b\chip\to b W \lsp$,
as depicted in Fig.~\ref{fig:diagrams}(b)~\cite{ref:stop}.

The ATLAS experiment has presented a similar search for top squark pairs in the single lepton final state~\cite{ref:atlasstop}.
The constraints from ATLAS on the top squark mass are more stringent than those presented here. The ATLAS model assumes large 
right-handed top quark polarization, while we take the top quark in the $\tilde{t}\to t\lsp$ decay to be unpolarized, 
resulting in a lower signal selection efficiency in our analysis.

\begin{figure}
% Use the relevant command for your figure-insertion program
% to insert the figure file.
\centering
\includegraphics[width=0.5\textwidth]{HCPPlots/stop_interpretation.pdf}
%\includegraphics[width=7cm,clip]{HCPPlots/stop_interpretation.pdf}
\caption{Interpretation of the results of the top squark pair search in the $\tilde{t}\to t\lsp$ scenario of 
Fig.~\ref{fig:diagrams}(a). The color scale indicates the cross section upper limits. The solid black contour 
and dashed black contours indicate the observed excluded region and variation in this
excluded region due to the $\pm1\sigma$ uncertainties in the theoretical prediction of the signal cross section. The dashed blue
and dotted blue contours indicate the median and $\pm1\sigma$ expected excluded regions. }
\label{fig:stop_interpretation}       % Give a unique label
\end{figure}

\section{Search in the Same-sign Dilepton Final State}
\label{sec:ss}

This section presents a search in the same-sign (SS) dilepton final state, based on 10.5 fb$^{-1}$.
A wide variety of new physics processes may produce events with SS leptons, which provide a very clean
final state due to low SM background expectations. In particular, this final state is sensitive to 
direct pair production of bottom squarks with $\tilde{b}\to t \chi^{\pm}\to t W \lsp$ depicted in Fig.~\ref{fig:diagrams}(d),
as well as to gluino-mediated production of top and bottom squarks.

We select events with two leptons (e or $\mu$) with \pt\ $>$ 20 GeV and dilepton invariant mass $m_{\ell\ell}>8$ GeV. 
We reject events with a third lepton with \pt\ $>$ 10 GeV that forms an opposite-sign
same-flavor pair with $76 < m_{\ell\ell} < 106$ GeV with either selected lepton, to suppress
the background from WZ and ZZ. We require the presence of at least two jets with \pt\ $>$ 40 GeV.

This analysis is an extension of a previous search in the same-sign dilepton final state~\cite{ref:ss_inclusive}.
In that analysis, the background is dominated by \ttljets\ where one lepton is from the W decay and the other
lepton is produced in the decay of one of the b-jets. In this analysis we require the presence of at least two
b-tagged jets (using CSVM). The requirement that both b-jets are identified and well-separated from the selected leptons
reduces the \ttljets\ background by an order of magnitude. 

There are three sources of SM background passing the above preselection. 
The first background source is referred to as ``fake leptons'' and includes leptons from heavy-flavor decay, 
misidentified hadrons, muons from meson decay in flight, or electrons from unidentified photon conversions. 
This background is estimated from a sample of events with at least one lepton that passes a loose selection 
but fails the full analysis identification and isolation requirements. This sample is weighted by the probability 
for a fake lepton satisfying the loose selection to also pass the analysis selection, which is determined based 
on studies of fake 
leptons in jet events. The second background, estimated from MC, consists of rare SM processes and is dominated 
by $t\bar{t}$W and $t\bar{t}$Z. The systematic uncertainty on both the fake lepton and rare backgrounds is 50\%. 
A third, small background contributions is from ``charge flips'' and consists of events with opposite-sign leptons 
where one of the leptons is an electron whose charge is misreconstructed. This background is based on MC predictions,
which are validated using a sample of Z$\to e^+e^-$ events.

Signal regions are defined by placing additional requirements on the jet multiplicity, b-tagged jet multiplicity, 
\met, and $H_T$, defined as the scalar sum of the transverse momenta of selected jets. The observed data yields 
in these signal regions are compared to the SM background expectations in 
Table~\ref{tab:ss}. Good agreement is observed between the data and the expected background in all signal regions.

In Fig.~\ref{fig:ss_interpretation}(a),tThe results are interpreted using the model of bottom squark pair 
production with $\tilde{b}\to t \chi^{\pm}$ depicted in Fig.~\ref{fig:diagrams}(d).
The most sensitive signal region in most of the model parameter space is SR3 (see Table~\ref{tab:ss}).
These results probe bottom squarks with masses up to 450 GeV.
The constraint on the bottom squark mass from naturalness is similar to that on the top squark, requiring 
a mass less than 500-700~GeV. Several additional interpretations for models with gluino-mediated top and 
bottom squark production are presented in Ref.~\cite{ref:ss}.


%\begin{table}
%\centering
%\caption{Please write your table caption here}
%\label{tab-1}       % Give a unique label
%% For LaTeX tables you can use
%\begin{tabular}{lll}
%\hline
%first & second & third  \\\hline
%number & number & number \\
%number & number & number \\\hline
%\end{tabular}
%% Or use
%\vspace*{5cm}  % with the correct table height
%\end{table}


\begin{table*}[t]
\centering
  \caption{\label{tab:ss} Summary of the results of the search in the same-sign dilepton final state.
    Several signal regions (SR) are indicated, including the kinematic requirements, the prediction of the three background (BG) contributions,
    the total background, and the observed yield in data. The jet multiplicity requirement in the first row includes both b-tagged and untagged jets. The uncertainty includes the statistical and systematic components.}
  \tabcolsep 2.7pt
  \begin{scriptsize}
    \begin{tabular}{l|c|c|c|c|c|c|c|c|c}
\hline
\hline
& SR0 & SR1 & SR2 & SR3 & SR4 & SR5 & SR6 & SR7 & SR8 \\
      \hline
      No. of jets            & $\geq 2$               & $\geq 2$               & $\geq 2$               & $\geq 4$               & $\geq 4$               & $\geq 4$               & $\geq 4$               & $\geq 3$               & $\geq 4$       \\
      No. of btags           & $\geq 2$               & $\geq 2$               & $\geq 2$               & $\geq 2$               & $\geq 2$               & $\geq 2$               & $\geq 2$               & $\geq 3$               & $\geq 2$       \\
      Lepton charges         & $++/--$                & $++/--$                & $++$                   & $++/--$                & $++/--$                & $++/--$                & $++/--$                & $++/--$                & $++/--$        \\
      \met                 & $> 0$ GeV             & $> 30$ GeV            & $> 30$ GeV            & $> 120$ GeV           & $> 50$ GeV            & $> 50$ GeV            & $> 120$ GeV           & $> 50$ GeV            & $> 0$ GeV     \\
      $H_T$                  & $> 80$ GeV            & $> 80$ GeV            & $> 80$ GeV            & $> 200$ GeV           & $> 200$ GeV           & $> 320$ GeV           & $> 320$ GeV           & $> 200$ GeV           & $> 320$ GeV   \\
      \hline
      Charge-flip BG         & $3.35 \pm 0.67$ & $2.70 \pm 0.54$ & $1.35 \pm 0.27$ & $0.04 \pm 0.01$ & $0.21 \pm 0.05$ & $0.14 \pm 0.03$ & $0.04 \pm 0.01$ & $0.03 \pm 0.01$ & $0.21 \pm 0.05$\\
      Fake BG                & $24.77 \pm 12.62$ & $19.18 \pm 9.83$ & $9.59 \pm 5.02$ & $0.99 \pm 0.69$ & $4.51 \pm 2.85$ & $2.88 \pm 1.69$ & $0.67 \pm 0.48$ & $0.71 \pm 0.47$ & $4.39 \pm 2.64$  \\
      Rare SM BG             & $11.75 \pm 5.89$ & $10.46 \pm 5.25$ & $6.73 \pm 3.39$ & $1.18 \pm 0.67$ & $3.35 \pm 1.84$ & $2.66 \pm 1.47$ & $1.02 \pm 0.60$ & $0.44 \pm 0.39$ & $3.50 \pm 1.92$  \\
      \hline
      Total BG               & $39.87 \pm 13.94$ & $32.34 \pm 11.16$ & $17.67 \pm 6.06$ & $2.22 \pm 0.96$ & $8.07 \pm 3.39$ & $5.67 \pm 2.24$ & $1.73 \pm 0.77$ & $1.18 \pm 0.61$ & $8.11 \pm 3.26$  \\
      Event yield            & 43                & 38                & 14                & 1                & 10                & 7                & 1                & 1                & 9              \\
      \hline
%      $N_{{UL}}$ (13\% unc.) & 27.2   &26.0   &9.9    &3.6    &10.8   &8.6    &3.6    &3.7    &9.6 \\
%      $N_{{UL}}$ (20\% unc.) & 28.2   &27.2   &10.2   &3.6    &11.2   &8.9    &3.7    &3.8    &9.9 \\
%      $N_{{UL}}$ (30\% unc.) & 30.4   &29.6   &10.7   &3.8    &12.0   &9.6    &3.9    &4.0    &10.5 \\
      \hline
    \end{tabular}
  \end{scriptsize}
\end{table*}


\begin{figure*}
\centering
%\begin{center}
\begin{tabular}{cc}
\subfloat[] {\includegraphics[width=0.5\textwidth]{HCPPlots/SS_B1.pdf}} &
\subfloat[] {\includegraphics[width=0.4\textwidth]{HCPPlots/T2bb_interpretation.pdf}} \\
\end{tabular}
\caption{
Interpretation of the results of the search in (a) the same-sign dilepton final state for
bottom squark pair production with $\tilde{b}\to t\chip$ depicted in Fig.~\ref{fig:diagrams}(d), and (b)
the all-hadronic final state for bottom squark pair production with $\tilde{b}\to b\lsp$ depicted in Fig.~\ref{fig:diagrams}(c).
\label{fig:ss_interpretation}
}
%\end{center}
\end{figure*}


\section{Search in the All-Hadronic Final State}
\label{sec:alphat}

\begin{figure*}[!ht]
\centering
%\begin{center}
\begin{tabular}{cc}
\subfloat[] {\includegraphics[width=0.45\textwidth]{HCPPlots/AlphaT_le3j_prelim.pdf}} &
\subfloat[] {\includegraphics[width=0.4\textwidth]{HCPPlots/hadronic_2b_le3j_logy.pdf}} \\
\end{tabular}
\caption{
Distributions of \alphat\ (left) and $H_T$ (right) in data, compared to the SM background expectations.
An example signal scenario of bottom squark pair production with $\tilde{b}\to b\lsp$ is overlaid.
\label{fig:alphat}
}
%\end{center}
\end{figure*}

The production of bottom squark pairs, followed by the decay $\tilde{b}\to b\lsp$, leads to events with
two b-jets and \met. In this section we report the results from a search with 11.7 fb$^{-1}$ 
in the all-hadronic final state using the 
\alphat\ variable.%, discussed below, which discriminates between backgrounds with real and fake \met.

We count jets with \pt\ $>$ 50 GeV. The leading (highest \pt) jet is required to be in the tracker 
acceptance defined by $|\eta|<2.5$, and the leading two jets must satisfy \pt\ $>$ 100 GeV. Events with isolated 
electrons or muons with \pt\ $>$ 10 GeV are vetoed, in order to suppress backgrounds with neutrinos from the decays 
of W bosons. Events with an isolated photon with \pt\ $>$ 25 GeV are vetoed.
The remaining events are categorized based on the jet multiplicity, the number of b-tagged jets (using CSVM) and the event $H_T$, 
which is required to satisfy $H_T$ $>$ 275 GeV.

The background satisfying the above preselection is dominated by QCD multijet production with fake \met\ from mismeasurement effects. To suppress this background,
we require the events to have large $\alphat$. For dijet events this quantity is defined as $\alphat \equiv E_{T}^{j_2} / M_{T}$, where $E_{T}^{j_2}$ is the $E_T$
of the second leading jet and $M_T$ is the transverse mass of the dijet system. 
For events with perfectly measured jets, the reconstructed \pt\ values of the two jets are equal, leading to $\alphat=0.5$. 
%The key feature of the \alphat\ variable is that mismeasurement effects tend to decrease the value of \alphat, such that it is 
%extremely rare for events with fake \met\ to have \alphat\ much larger than 0.5. As shown in Fig.~\ref{fig:alphat}(a), 
%the \alphat\ distribution for the QCD multijet background falls off extremely rapidly near this endpoint value. 
The key feature of the \alphat\ variable is that mismeasurement effects tend to decrease the value of \alphat,
resulting in an endpoint at $\alpha_T\approx0.5$ for the QCD multijet background, which is evident in Fig.~\ref{fig:alphat}(a).
For events with three of more jets, an equivalent dijet 
system is formed by  clustering the jets into two pseudo-jets. In our search we strongly suppress the QCD multijet background with 
the requirement \alphat\ $>$ 0.55.

The background after the \alphat\ requirement is dominated by processes with genuine \met, including \ttljets\ and \wjets\ with a lepton and neutrino from W decay,
where the lepton is either not reconstructed or is a hadronically decaying $\tau$ lepton. 
These backgrounds are estimated using a $\mu+\rm{jets}$ data control sample.
The additional background from $\rm{Z}(\nu\nu)+\rm{jets}$ is estimated using two data control samples of $\rm{Z}(\ell\ell)+\rm{jets}$ and  $\gamma+\rm{jets}$ events. To estimate these backgrounds, the observed yields in the data control samples are extrapolated to the
signal region using translation factors derived from MC. The dominant systematic uncertainties in the background prediction stem from the uncertainties
in the MC translation factors, which are assessed by performing several closure tests in data. In these tests, the observed yields in one data control region
are used to predict the yields in another data control region.

Events are categorized based on the $H_T$, jet multiplicity, and b-tagged jet multiplicity. For the bottom squark scenario described above, the most sensitive
category is events with either two or three jets and exactly two b-tagged jets. The $H_T$ distribution for these events is indicated in Fig.~\ref{fig:alphat}(b),
which demonstrates good agreement between the data and the expected background. No evidence for an excess of events is observed.

The results are interpreted  using the model of bottom squark pair production with $\tilde{b}\to b\lsp$, in Fig.~\ref{fig:ss_interpretation}(b).
These results probe bottom squarks with masses up to approximately 600 GeV. Additional interpretations in models with gluino-mediated
top and bottom squark pair production are presented in Ref.~\cite{ref:alphat}.


\section{Summary}

This note presents a search for BSM physics in final states with leptonically-decaying Z bosons, jets, and \MET. 
Two strategies were pursued. The first is an inclusive approach which targets BSM scenarios with Z bosons produced
in the decays of strongly-interacting particles. The second is a targeted approach which focuses on BSM scenarios
where the Z bosons are produced in the decays of weakly-interacting particles. The main backgrounds are
estimated with data-driven techniques. Good agreement is observed between the data and the predicted backgrounds
over the full \MET\ range, for both searches. 
%The results are interpreted in the context of a simplified SUSY
%model where chargino-neutralino pairs decay to the \wzmet\ final state, and used to place constraints on the
%masses of these particles.


%\section{Introduction}
%\label{intro}
%Your text comes here. Separate text sections with

%
% BibTeX or Biber users please use (the style is already called in the class, ensure that the "woc.bst" style is in your local directory)
% \bibliography{name or your bibliography database}
%
% Non-BibTeX users please use
%
\begin{thebibliography}{}
%
% and use \bibitem to create references.
%
\bibitem{RefJ}
% Format for Journal Reference
Journal Author, Journal \textbf{Volume}, page numbers (year)
% Format for books
\bibitem{RefB}
Book Author, \textit{Book title} (Publisher, place, year) page numbers
% etc

\bibitem{ref:stop}
``Search for direct top squark pair production in events with a single isolated lepton, jets and missing transverse energy at $\sqrt{s} = 8$ TeV'', CMS-PAS-SUS-12-023.

\bibitem{ref:ss}
``Search for new physics in events with same-sign dileptons and b jets in pp collisions at $\sqrt{s} = 8$ TeV'', CMS-PAS-SUS-12-029, arXiv:1212.6194 [hep-ex].

\bibitem{ref:alphat}
``Search for supersymmetry in final states with missing transverse energy and 0, 1, 2, 3, or $\geq$4 b jets in 8 TeV pp collisions'', CMS-PAS-SUS-12-028.

\end{thebibliography}

\end{document}

% end of file template.tex

