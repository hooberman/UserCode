\clearpage

\section{Selection}
\label{sec:eventSelection}

In this section, we list the event selection, electron and muon objects selections, jets, \MET, and b-tagging selections
used in this analysis. These selections are based on those recommended by the relevant POG's. 

\subsection{Event Selection}

We require the presence of at least one primary vertex satisfying the standard quality criteria; namely,
vertex is not fake, $\rm{ndf}\geq4$, $\rho<2$ cm, and $|z|<24$ cm.

\subsection{Lepton Selection}

Because Z $\rightarrow\ell\ell$ ($\ell=e,\mu$) is a final state with very little background,
we restrict ourselves to events in which the Z boson decays to electrons or muons only.
Therefore opposite sign leptons passing the identification and isolation requirements described below are required in each event.

\begin{itemize}
\item \pt $> 20$~GeV and $|\eta|<2.4$;
\item Opposite-sign same-flavor (SF) ee and $\mu\mu$ lepton pairs (opposite-flavor (OF) e$\mu$ lepton pairs are retained in a control 
  sample used to estimate the FS contribution);
\item For SF events, the dilepton invariant mass is required to be consistent with the Z mass; namely $81<m_{\ell\ell}<101$ GeV.
\end{itemize}

\subsubsection{Electron Selection}

The electron selection is the loose working point recommended by the E/gamma POG, as documented at~\cite{ref:Egamma}. 
Electrons with \pt $>$ 20 GeV and $|\eta|<2.4$ are considered.
We use PF-based isolation with a cone size of $\Delta R<0.3$, using the effective area rho corrections documented at~\cite{ref:Egammaiso},
and we require a relative isolation $<$ 0.15. 
Electrons in the transition region defined by $1.4442 < |\eta_{SC}| < 1.566$ are rejected.
The electron selection requirements are listed in Table~\ref{table:electrons} for completeness.

\begin{table}[htb]
\begin{center}
\caption{\label{table:electrons} Summary of the electron selection requirements.}
\begin{tabular}{l|cc}
\hline
              Quantity   &     Barrel    &   Endcap   \\
\hline
$\delta\eta$                          &  $<0.007$ & $<0.009$ \\
$\delta\phi$                          &  $<0.15$  &  $<0.10$ \\
$\sigma_{i\eta i\eta}$                   &  $<0.01$  &  $<0.03$ \\
H/E                                    &  $<0.12$  & $<0.10$ \\
$d_{0}$ (w.r.t. 1st good PV)            & $<0.02$ cm & $<0.02$ cm \\
$d_{z}$ (w.r.t. 1st good PV)            & $<0.2$ cm  & $<0.2$ cm  \\
$|1/E-1/P|$                            &  $<0.05~\rm{GeV}^{-1}$ & $<0.05~\rm{GeV}^{-1}$ \\
PF isolation / \pt                     &  $<0.15$  & $<0.15$ \\
conversion rejection: fit probability  &  $<10^{-6}$ & $<10^{-6}$ \\
conversion rejection: missing hits     &  $\leq1$  & $\leq1$  \\
\hline
\end{tabular}
\end{center}
\end{table}

\subsubsection{Muon Selection}

We use the tight muon selection recommended by the muon POG, as documented at~\cite{ref:muon}.
Muons with \pt $>$ 20 GeV and $|\eta|<2.4$ are considered. We use PF-based isolation with a cone size
of $\Delta R<0.3$, using the $\Delta\beta$ PU correction scheme, and we require a relative isolation of $<$ 0.15.
The muon selection requirements are listed in Table~\ref{table:muons} for completeness.

\begin{table}[htb]
\begin{center}
\caption{\label{table:muons} Summary of the muons selection requirements.}
\begin{tabular}{l|c}
\hline
              Quantity   &     Requirement \\
\hline
muon type & global muon and PF muon \\
$\chi^2/\rm{ndf}$ & $<10$   \\
muon chamber hits & $\geq1$ \\
matched stations  & $\geq2$ \\
$d_{0}$ (w.r.t. 1st good PV)            & $<0.02$ cm  \\
$d_{z}$ (w.r.t. 1st good PV)            & $<0.5$ cm   \\
pixel hits & $\geq1$ \\
tracker layers & $\geq5$ \\
\hline
\end{tabular}
\end{center}
\end{table}

\subsubsection{PF Leptons}

For consistency with pfmet, both electrons and muons are required to be reconstructed as PF electrons and PF muons, respectively,
with \pt $>$ 20 GeV. For defining the dilepton invariant mass, the 4-momenta of the PF leptons are used.

\subsection{Photons}
\label{sec:phosel}

As will be explained later, it is not essential that we select real photons. 
What is needed are jets that are predominantly electromagnetic, well measured in the ECAL, and hence less likely to contribute to fake MET. We select photons with:

\begin{itemize}
\item \pt $ > 22$ GeV
\item $|\eta| < 2$
\item $H/E < 0.1$
\item No matching pixel track (pixel veto)
\item There must be a pfjet of \pt $ >$ 10 GeV matched to the photon within $dR < 0.3$. 
The matched jet is required to have a neutral electromagnetic energy fraction of at least 70\%.
\item 
  We require that the pfjet \pt matched to the photon satisfy
  (pfjet \pt - photon \pt) $>$ -5~GeV.
  This removes a few rare cases in which ``overcleaning" of a
  %ECAL recHit 
  pfjet
  generates fake MET.

\item We also match photons to calojets and require (calojet \pt - photon \pt) $>$ -5~GeV
  (the same requirement used for pfjets). This is to remove other rare cases in which fake
  energy is added to the photon object but not the calojet.

\item We reject photons which have an electron of at least \pt $>$ 10 GeV 
  within $dR < 0.2$
  in order to reject
  conversions from electrons from W decays which are accompanied by real MET.

\item We reject photons which are aligned with the MET to within 0.14 radians in phi.

\end{itemize}

\subsection{MET}

We use pfmet, henceforth referred to simply as \MET.

\subsection{Jets}
\label{sec:jetsel}

\begin{itemize}
\item PF jets with L1FastL2L3 corrections (MC), L1FastL2L3residual corrections (data), using the 52X jet energy corrections
\item $|\eta| < 2.5$
\item Passes loose PFJet ID
\item \pt $ > 30$ GeV for determining the jet multiplicity, \pt $ > 15$ GeV for calculation of \Ht
\item For the creation of photon templates, the jet matched to the photon passing the photon selection described above is vetoed
\item For the dilepton sample, jets are vetoed if they are within $\Delta R < 0.4$ from any lepton \pt $ > 20$~GeV passing analysis selection
\end{itemize}

