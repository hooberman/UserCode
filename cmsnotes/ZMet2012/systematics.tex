\clearpage

\section{Systematic Uncertainties in Signal Acceptance}

In this section we discuss systematic uncertainties in the signal acceptance. These efficiency
uncertainties are relevant for the interpretations in the \wzmet\ and the GMSB models, which 
are combined with the results of the trilepton and quadlepton analysis, respectively, in AN-2012/351.

\begin{table}[htb]
\begin{center}
\footnotesize
\caption{\label{tab:syst} Summary of uncertainties in the signal efficiency. }
\begin{tabular}{l|c|l}
\hline
\hline
Source & Value (\%) & Method \\
\hline
Luminosity & 4.4 & official CMS value \\
Trigger efficiency & 3 & efficiency measurements documented in Sec. 4, Table 9 of AN-2012/248 \\
Lepton ID/isolation & 2 (per lepton) & Z tag-and-probe measurements in AN-2012/257 \\
B-veto & 6 & dedicated measurement in AN-2012/248 Sec. 7.6 \\
Z mass window requirement & 3 & see text and Table~\ref{tab:mllsyst} \\
Jet selection, dijet mass, \MET & assessed at each model point & official JetMet POG recipe \\
\hline
\hline
\end{tabular}
\end{center}
\end{table}

A summary of the efficiency uncertainties is presented in Table~\ref{tab:syst}.
The CMS uncertainty in the luminosity is 4.4\%. The trigger efficiency is measured in AN-2102/248 with an uncertainty of 3\%.
The lepton identification and isolation requirements are measured in data and MC and found to be consistent within 2\%, for
the \pt\ $>$ 20 GeV region relevant for this analysis, in AN-2012/257. The impact of the b-veto on the signal acceptance
is quantified with a dedicated measurement performed in Sec. 7.6 of AN-2012/248. The uncertainty in the selection of dilepton
events satisfying the Z mass window requirement 81--101 GeV is performed as follows. In both data and MC, the Z mass window
in the inclusive preselection region (Z and at least 2 jets) is loosened to 60--120 GeV. 
The efficiency of the events in the loose window to satisfy the analysis dilepton mass selection
of 81--101 GeV is compared in data and MC, and found to be consistent within 3\% for both ee and $\mu\mu$ channels
(see Table~\ref{tab:mllsyst}), and a corresponding uncertainty on the signal efficiency is assessed.

The above uncertainties are the same for all SUSY model points. However the impact of the jet energy scale uncertainty, which
affects the selection efficiencies for all jets and \MET\ objects, varies significantly across the model parameter space and
is assessed separately at each point. The official JetMet POG recipe is used for this purpose. Each jet is assigned an uncertainty
based on its \pt\ and $\eta$. The jet energy is varied by this uncertainty, which is propagated to the efficiencies for the jet selection, 
dijet mass selection and \MET\ selection. In addition, for the \MET, a 10\% uncertainty on the unclustered energy is included.
The \MET\ variation alters the shape of the signal \MET\ distribution and causes a bin-to-bin migration of events, which is 
included in the limit setting procedure performed with LandS.

\begin{table}[htb]
\begin{center}
\footnotesize
\caption{\label{tab:mllsyst} Summary of the dilepton mass selection efficiency uncertainties. Loose and tight refer to dilepton
mass windows of 60--120 GeV and 81-101 GeV, respectively.}
\begin{tabular}{l|c|c}
\hline
\hline
 & ee & $\mu\mu$ \\
\hline
MC loose & 200694.0 & 285462.5 \\
MC tight & 183326.2 & 260960.0 \\  
MC tight/loose & 0.913 $\pm$ 0.005 & 0.914 $\pm$ 0.004 \\
\hline
data loose & 209540 & 263747 \\
data tight & 185555 & 234132 \\
data tight/loose & 0.886 $\pm$ 0.003 & 0.888 $\pm$ 0.003 \\
\hline
\hline
\end{tabular}
\end{center}
\end{table}
