\section{Introduction}
\label{sec:intro}

In this note we describe a search for new physics in the 2010 
opposite sign isolated dilepton sample ($ee$, $e\mu$, and $\mu\mu$).  
The main source of 
isolated dileptons at CMS is Drell Yan and $t\bar{t}$.
Here we concentrate on dileptons with invariant mass inconsistent
with $Z \to ee$ and $Z \to \mu\mu$.  Thus $t\bar{t}$ is the most
important background.  A separate search for new physics in the $Z$ 
sample is described in a separate note\cite{ref:Ztemplates}.
 
The search strategy is the following

\begin{itemize}

\item We start out with a pre-selection which is as close as 
possible to the published (or soon to be published) $t\bar{t}$
dilepton analysis\cite{ref:top} (same lepton ID, same jet definitions,
etc.).  We do make a couple of substantive modifications:

\begin{enumerate}
\item The top analysis requires two leptons of $P_T > 20$ GeV.  
 In this
analysis we lower the requirement on the second lepton to $P_T > 10$ 
GeV.  This is motivated by our desire to maintain sensitivity to possible
SUSY signals with relatively low $P_T$ leptons generated in the 
cascade decays of heavy objects.
\item The top analysis requires at least two jets of $P_T > 30$
GeV with \met $>30$ GeV ($ee$ and $e\mu$) or \met $>20$ GeV ($e \mu$).
We tighten the \met cut to 50 GeV and we 
also require that the scalar sum of the $P_T$ of all jets with $P_T > 30$
GeV be $> 100$ GeV.  These requirements considerably
reduce backgrounds to the $t\bar{t}$ sample, {\em e.g.}, backgrounds
from Drell Yan and $W+$jets.
\end{enumerate}

\item The pre-selection consists mostly of $t\bar{t}$ events.  We perform 
data $-$ Monte Carlo comparisons of kinematical distributions.  Assuming
reasonable agreeement for the bulk of $t\bar{t}$ we move on to a 
search for new physics in the tail of the $t\bar{t}$.

\item Our prejudice is that new physics would manifest itself in an
excess of events with high \met and significant hadronic activity.
We define an a-priori search region by tightening the \met and 
hadronic activity requirements such that we expect of order 1\% 
of $t\bar{t}$ events to pass the selection (as predicted by Monte Carlo).

\item We perform a counting experiment in the signal region.  We compare
observed yields with expectations from Monte Carlo and with two independent
data driven techniques (see Section~\ref{sec:abcd} and~\ref{sec:victory}).

\end{itemize}


