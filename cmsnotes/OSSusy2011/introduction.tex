\section{Introduction}
\label{sec:intro}

In this note we describe a search for new physics in the
opposite sign isolated dilepton sample ($ee$, $e\mu$, and $\mu\mu$)
based on \lumi\ of 2011 data.
The main source of 
isolated dileptons at CMS is Drell Yan and $t\bar{t}$.
Here we concentrate on dileptons with invariant mass inconsistent
with $Z \to ee$ and $Z \to \mu\mu$, thus $t\bar{t}$ is the most
important background.  A separate search for new physics in the $Z$ 
sample is described in a separate note~\cite{ref:Ztemplates}.
This is an update of a (soon to be) published analysis performed 
on 2010 data~\cite{ref:osnote,ref:ospaper}. 

The search strategy is the following

\begin{itemize}

\item We start out with a pre-selection which is as close as 
possible to the published $t\bar{t}$
dilepton analysis~\cite{ref:top} (same lepton ID, same jet definitions,
etc.).  We do make a couple of substantive modifications:

\begin{enumerate}
\item The top analysis requires two leptons of $P_T > 20$ GeV.  
In this analysis we search for new physics in 2 data samples. The first data
sample is collected with high \pt\ dilepton triggers; for this sample we require
leptons with \pt\ $>$ (20,10) GeV (leading lepton \pt\ $>$ 20~GeV,
trailing lepton \pt\ $>$ 10~GeV. The second data sample is collected
with dilepton-\Ht\ cross triggers; for this sample we require leptons
with \pt\ $>$ (10,5) GeV.  
This is motivated by our desire to maintain sensitivity to possible
SUSY signals with relatively low \pt\ leptons generated in the 
cascade decays of heavy objects.
\item The top analysis requires at least two jets of $p_T > 30$
GeV with \met\ $>30$ GeV ($ee$ and $e\mu$) or \met\ $>20$ GeV ($e\mu$).
We tighten the \met\ cut to 50 GeV and we 
also require that the scalar sum of the $p_T$ of all jets with $p_T > 30$
GeV be $> 100$ GeV.  These requirements considerably
reduce backgrounds to the $t\bar{t}$ sample, {\em e.g.}, backgrounds
from Drell Yan and $W+$jets.
\end{enumerate}

\item The pre-selection consists mostly of $t\bar{t}$ events.  We perform 
data $-$ Monte Carlo comparisons of kinematical distributions.  Assuming
reasonable agreeement for the bulk of $t\bar{t}$ we move on to a 
search for new physics in the tails of the $t\bar{t}$ \met\ and \Ht\ distributions.

\item Our prejudice is that new physics would manifest itself in an
excess of events with high \met\ and significant hadronic activity.
We define a a-priori search regions by tightening the \met\ and 
hadronic activity requirements.

\item We perform a counting experiment in the signal regions.  We compare
observed yields with expectations from Monte Carlo and with three independent
data driven techniques (see Sections~\ref{sec:abcd}, \ref{sec:victory} and \ref{sec:ofsubtraction}).

\end{itemize}



