\section{Derivation of Upper Limits}
\label{sec:limits}

We proceed to set upper limits on the non-SM contributions to the 3 signal regions. For all 3 regions,
we find reasonable agreement between the observed yields and the predictions from MC and from the 2
data-driven methods. We choose here to extract the upper limits using the weighted average of the 
2 data-driven predictions for the
background estimate. The 95\% CL upper limit is extracted using cl95cms, using a log-normal model of
nuissance parameter integration. The results are summarized in Table~\ref{tab:limit}. The
3 benchmark points LM1, LM3 and LM6 are excluded by these results.

\begin{table}[hbt]
\begin{center}
\caption{\label{tab:limit} 
Summary of 95\% CL upper limits on the non-SM yield. The background estimate $N_{bkg}$ is the weighted
average of the 2 data-driven estimates. The LM1 and LM3
yields are shown, with uncertainties from the hadronic energy scale, lepton selection and trigger efficiencies,
and integrated luminosity.
}
\begin{tabular}{l|c|c|c}
\hline
                                    &  2010 signal region  &   high \met\ signal region  &  high \Ht\ signal region              \\ 
\hline
Observed yield                      &         45           &                        8    &                        4              \\
\hline
$N_{bkg}$                            &    42 $\pm$ 8.5      &              4.2 $\pm$ 1.3  &           5.1 $\pm$ 1.7               \\
non-SM yield UL                     &         22           &                  10         &               5.8                     \\
\hline
LM1                                 &  85.7 $\pm$ 9.7      &    48.7 $\pm$ 11.2          &    37.7 $\pm$ 11.6                    \\
LM3                                 &  34.6 $\pm$ 4.2      &    18.0 $\pm$  5.0          &    18.8 $\pm$  6.2                    \\
LM6                                 &   9.7 $\pm$ 0.9      &     8.1 $\pm$  1.0          &     7.4 $\pm$  1.2                    \\
\hline
\end{tabular}
\end{center}
\end{table}

