\section{Data Driven Background Estimation Methods}
\label{sec:datadriven}
We have developed two data-driven methods to 
estimate the background in the signal region.
The first one exploits the fact that 
\Ht\ and $y$ are nearly 
uncorrelated for the $t\bar{t}$ background 
(Section~\ref{sec:abcd});  the second one 
is based on the fact that in $t\bar{t}$ the
$P_T$ of the dilepton pair is on average 
nearly the same as the $P_T$ of the pair of neutrinos
from $W$-decays, which is reconstructed as \met\ in the
detector.

%{\color{red} I took these
%numbers from the twiki, rescaling from 11.06 to 30/pb.
%They seem too large...are they really right?}


\subsection{ABCD method}
\label{sec:abcd}

We find that in $t\bar{t}$ events \Ht\ and 
$y$ are nearly uncorrelated, 
as demonstrated in Fig.~\ref{fig:uncor}.
Thus, we can use an ABCD method in the $y$ vs. \Ht
plane to estimate the background in a data driven way.

{\color{red} \bf HERE I WILL PUT THE ABCD MC CLOSURE STUDIES FOR THE 3 SIGNAL REGIONS}


\clearpage

\subsection{Dilepton $P_T$ method}
\label{sec:victory}
This method is based on a suggestion by V. Pavlunin\cite{ref:victory},
and was investigated by our group in 2009\cite{ref:ourvictory}.
The idea is that in dilepton $t\bar{t}$ events the lepton and neutrinos
from $W$ decays have the same $P_T$ spectrum (modulo $W$ polarization 
effects).  One can then use the observed 
$P_T(\ell\ell)$ distribution to model the sum of neutrino $P_T$'s which 
is identified with the \met.

Then, in order to predict the $t\bar{t} \to$ dilepton contribution to a 
selection with \met$+$X, one applies a cut on $P_T(\ell\ell)+$X instead.
In practice one has to rescale the result of the $P_T(\ell\ell)+$X selection
to account for the fact that any dilepton selection must include a 
moderate \met cut in order to reduce Drell Yan backgrounds.  This 
is discussed in Section 5.3 of Reference~\cite{ref:ourvictory}; for a \met
cut of 50 GeV, the rescaling factor is obtained from the MC as

\newcommand{\ptll} {\ensuremath{P_T(\ell\ell)}}
\begin{center}
$ K = \frac{\int_0^{\infty} {\cal N}(\ptll)~~d\ptll~}{\int_{50}^{\infty} {\cal N}(\ptll)~~d\ptll~} = 1.5$
\end{center}

{\color{red} \bf HERE I WILL PUT THE VICTORY MC CLOSURE STUDIES FOR THE 3 SIGNAL REGIONS}

\subsection{Signal Contamination}
\label{sec:sigcont}

All data-driven methods are in principle subject to signal contaminations
in the control regions, and the methods described in 
Sections~\ref{sec:abcd} and~\ref{sec:victory} are not exceptions.
Signal contamination tends to dilute the significance of a signal
present in the data by inflating the background prediction.

It is hard to quantify how important these effects are because we 
do not know what signal may be hiding in the data.  Having two
independent methods (in addition to Monte Carlo ``dead-reckoning'')
adds redundancy because signal contamination can have different effects
in the different control regions for the two methods.
For example, in the extreme case of a
new physics signal 
with $P_T(\ell \ell) = \met$, an excess of events would be seen 
in the ABCD method but not in the $P_T(\ell \ell)$ method.


The LM points are benchmarks for SUSY analyses at CMS.  The effects
of signal contaminations for a couple such points are summarized
in Table~\ref{tab:sigcont}. Signal contamination is definitely an important
effect for these two LM points, but it does not totally hide the
presence of the signal.


\begin{table}[htb]
\begin{center}
\caption{\label{tab:sigcont} Effects of signal contamination 
for the two data-driven background estimates. The three columns give
the expected yield in the signal region and the background estimates
using the ABCD and $P_T(\ell \ell)$ methods. Results are normalized to 204~pb$^{-1}$.
{\color{red} \bf UPDATE ME }
}
\begin{tabular}{lccc}
\hline
            &      Yield      &      ABCD    & $P_T(\ell \ell)$  \\
\hline
SM only     &       1.3      &      1.3    &       0.9        \\
SM + LM1    &       9.9      &      6.1    &       2.4        \\
SM + LM3    &       4.8      &      1.8    &       1.6        \\
\hline
\end{tabular}
\end{center}
\end{table}

