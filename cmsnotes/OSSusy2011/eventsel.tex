\section{Event Preselection}
\label{sec:eventSel}
The purpose of the preselection is to define a data sample rich 
in $t\bar{t} \to$ dileptons.  We compare the kinematical 
properties of this sample with expectations from $t\bar{t}$ 
Monte Carlo.

The preselection is based on the 
$t\bar{t}$ analysis~\cite{ref:top}.  
We select events with two opposite sign, well-identified and isolated
leptons ($ee$, $e\mu$, or $\mu\mu$); one of the leptons must 
have $P_T > 20$ GeV,
the other one must have $P_T > 10$ GeV. Events with dilepton mass
consistent with $Z \to ee/\mu\mu$ are rejected.
In case of events with 
more than two such leptons, we select the pair that maximizes the scalar 
sum of lepton $P_T$'s.
There must be two JPT
jets of $P_T > 30$ GeV and $|\eta| < 2.5$; the scalar sum of the 
$P_T$ of all such jets must exceed 100 GeV; jets must pass
{\tt caloJetId} and be separated by $\Delta R >$ 0.4 from any 
lepton with $P_T > 10$~GeV passing the selection.
Finally $\met > 50$ GeV (we use tcMet). More details are given in the subsections below.

\subsection{Event Cleanup}
\label{sec:cleanup}
\begin{itemize}
\item Scraping cut: if there are $\geq$ 10 tracks, require at
least 25\% of them to be high purity.
\item Require at least one good vertex:
\begin{itemize}
\item not fake
\item ndof $>$ 4
\item $|\rho| < 2$ cm
\item $|z| < 24$ cm.  
\end{itemize}
\end{itemize}


\subsection{Muon Selection}
\label{sec:muon}

Muon candidates are RECO muon objects passing the following
requirements:
\begin{itemize}

\item $|\eta| < 2.4$.

\item Global Muon and Tracker Muon.

\item $\chi^2$/ndof of global fit $<$ 10.

\item At least 11 hits in the tracker fit.

\item Transverse impact parameter with respect to the beamspot $<$ 200 $\mu$m.

\item $Iso \equiv $ $E_T^{\rm iso}$/Max(20 GeV, $P_T$) $<$ 0.15.  
$E_T^{\rm iso}$
is defined as the sum of transverse energy/momentum deposits in ecal,
hcal, and tracker, in a cone of 0.3.

\item At least one of the hits from the 
standalone muon must be used in the global fit.

\item Require tracker $\Delta P_T/P_T < 0.1$. This cut was not in the original top analysis.
It is motivated by the observation of
poorly measured muons in data with large
relative $P_T$ uncertainty, giving significant contributions to the \met.
%{\color{red} This is not applied to the 11 pb iteration.}


\end{itemize}



\subsection{Electron Selection}
\label{sec:electron}

Electron candidates are RECO GSF electrons passing the following
requirements:

\begin{itemize}

% \item $P_T > 10$ GeV.  (The $t\bar{t}$ analysis uses 20 GeV but for
% completeness we calculate FR down to 10 GeV).

\item $|\eta| < 2.5$.

\item SuperCluster $E_T > 10$ GeV.  

\item The electron must be ecal seeded.

\item VBTF90 identification\cite{ref:vbtf}.

\item Transverse impact parameter with respect to the beamspot $<$ 400 $\mu$m.

\item $Iso \equiv $ $E_T^{\rm iso}$/Max(20 GeV, $P_T$) $<$ 0.15.  
$E_T^{\rm iso}$
is defined as the sum of transverse energy/momentum deposits in ecal,
hcal, and tracker, in a 
cone of 0.3.  A 1 GeV pedestal is subtracted from the ecal energy 
deposition in the EB, however the ecal energy is never allowed to 
go negative.

\item Electrons with a tracker or global muon within $\Delta R$ of 
0.1 are vetoed.

\item The number of missing expected inner hits must be less than 
two\cite{ref:conv}.

\item Conversion removal via partner track finding: any electron
where an additional GeneralTrack is found with $Dist < 0.02$ cm 
and $\Delta \cot \theta < 0.02$ is vetoed\cite{ref:conv}.

\item Cleaning for ECAL spike (aka Swiss-Cross cleaning) has been applied
at the reconstruction level (CMSSW 38x).

\end{itemize}

\subsection{Invariant mass requirement}
\label{sec:zveto}

We remove $e^+e^-$ and $\mu^+ \mu^-$ events with invariant 
mass between 76 and 106 GeV.  We also remove events
with invariant mass $<$ 10 GeV, since this kinematical region is 
not well reprodced in CMS Monte Carlos.

In addition, we remove $Z \to \mu\mu\gamma$
candidates with the $\gamma$ collinear with one of the muons.  This is
done as follows:
if the ecal energy associated with one of the muons is greater than 6 GeV,
we add this energy to the momentum of the initial muon, and we recompute
the $\mu\mu$ mass.  If this mass is between 76 and 106 GeV, the event is rejected.


\subsection{Trigger Selection}
\label{sec:trigSel}

Because most of the triggers implemented in the 2nd half of the
2010 run were not implemented in the Monte Carlo, 
we do not make any requirements on HLT bits in the Monte Carlo.
Instead, as discussed in 
Section~\ref{sec:trgEff}, a trigger efficiency weight is applied
to each event, based on the trigger efficiencies measured on data.
Trigger efficiency weights are very close to 1.

%For data, we require the logical OR of all (or most?) unprescaled
%single and double lepton triggers that were deployed during the 2010
%run.  These are:
%{\color{red} Here we need to list the triggers, somehow.}

For data, we use a cocktail of unprescaled single 
and double lepton triggers. An event
in the $ee$ final state is required to pass at least 1 
single- or double-electron trigger, a
$\mu\mu$ event is required to pass at least 1 single 
or double-muon trigger, while an $e\mu$ event
is required to pass at least 1 single-muon, single-electron, 
or $e-\mu$ cross trigger. 
% We currently
% do not require MC events to pass any triggers.









\begin{itemize}
\item single-muon triggers
  \begin{itemize}
  \item \verb=HLT_Mu5=
  \item \verb=HLT_Mu7=       
  \item \verb=HLT_Mu9=        
  \item \verb=HLT_Mu11=      
  \item \verb=HLT_Mu13_v1=    
  \item \verb=HLT_Mu15_v1=    
  \item \verb=HLT_Mu17_v1=    
  \item \verb=HLT_Mu19_v1=    
  \end{itemize}
\item double-muon triggers
  \begin{itemize}
  \item \verb=HLT_DoubleMu3=
  \item \verb=HLT_DoubleMu3_v2=
  \item \verb=HLT_DoubleMu5_v1=
  \end{itemize}
\item single-electron triggers
  \begin{itemize}
  \item \verb=HLT_Ele10_SW_EleId_L1R=
  \item \verb=HLT_Ele10_LW_EleId_L1R=
  \item \verb=HLT_Ele10_LW_L1R=
  \item \verb=HLT_Ele10_SW_L1R=
  \item \verb=HLT_Ele15_SW_CaloEleId_L1R=
  \item \verb=HLT_Ele15_SW_EleId_L1R=
  \item \verb=HLT_Ele15_SW_L1R=
  \item \verb=HLT_Ele15_LW_L1R=
  \item \verb=HLT_Ele17_SW_TightEleId_L1R=
  \item \verb=HLT_Ele17_SW_TighterEleId_L1R_v1=
  \item \verb=HLT_Ele17_SW_CaloEleId_L1R=
  \item \verb=HLT_Ele17_SW_EleId_L1R=
  \item \verb=HLT_Ele17_SW_LooseEleId_L1R=
  \item \verb=HLT_Ele17_SW_TighterEleIdIsol_L1R_v1=
  \item \verb=HLT_Ele17_SW_TighterEleIdIsol_L1R_v2=
  \item \verb=HLT_Ele17_SW_TighterEleIdIsol_L1R_v3=
  \item \verb=HLT_Ele20_SW_L1R=
  \item \verb=HLT_Ele22_SW_TighterEleId_L1R_v2=
  \item \verb=HLT_Ele22_SW_TighterEleId_L1R_v3=
  \item \verb=HLT_Ele22_SW_TighterCaloIdIsol_L1R_v2=
  \item \verb=HLT_Ele27_SW_TightCaloEleIdTrack_L1R_v1=
  \item \verb=HLT_Ele32_SW_TightCaloEleIdTrack_L1R_v1=
  \item \verb=HLT_Ele32_SW_TighterEleId_L1R_v1=
  \item \verb=HLT_Ele32_SW_TighterEleId_L1R_v2=
  \end{itemize}
\item double-electron triggers
  \begin{itemize}
  \item \verb=HLT_DoubleEle15_SW_L1R_v1=                
  \item \verb=HLT_DoubleEle17_SW_L1R_v1=  
  \item \verb=HLT_Ele17_SW_TightCaloEleId_Ele8HE_L1R_v1=
  \item \verb=HLT_Ele17_SW_TightCaloEleId_Ele8HE_L1R_v2=
  \item \verb=HLT_Ele17_SW_TightCaloEleId_SC8HE_L1R_v1=
  \item \verb=HLT_DoubleEle10_SW_L1R=
  \item \verb=HLT_DoubleEle5_SW_L1R=
  \end{itemize}
\item e-$\mu$ cross triggers
  \begin{itemize}
  \item \verb=HLT_Mu5_Ele5_v1=
  \item \verb=HLT_Mu5_Ele9_v1=
  \item \verb=HLT_Mu11_Ele8_v1=
  \item \verb=HLT_Mu8_Ele8_v1=
  \item \verb=HLT_Mu5_Ele13_v1=
  \item \verb=HLT_Mu5_Ele13_v2=
  \item \verb=HLT_Mu5_Ele17_v1=
  \item \verb=HLT_Mu5_Ele17_v2=
  \end{itemize}
\end{itemize}
