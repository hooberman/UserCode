\section{Event Preselection}
\label{sec:eventSel}
The purpose of the preselection is to reject backgrounds other than 
$t\bar{t} \to$ dileptons.  We compare the kinematical 
properties of this sample with expectations from $t\bar{t}$ 
Monte Carlo.

The preselection is based on the 
$t\bar{t}$ analysis~\cite{ref:top}.  
We select events with two opposite sign, well-identified and isolated
leptons ($ee$, $e\mu$, or $\mu\mu$); one of the leptons must 
have $P_T > 20$ GeV,
the other one must have $P_T > 10$ GeV. Events with dilepton mass
consistent with $Z \to ee/\mu\mu$ are rejected.
In case of events with 
more than two such leptons, we select the pair that maximizes the scalar 
sum of lepton $P_T$'s.
There must be at least two 
pfjets of $P_T > 30$ GeV and $|\eta| < 3.0$;  jets must pass
loose {\tt pfJetId} and be separated by $\Delta R >$ 0.4 from any 
lepton with $P_T > 10$~GeV passing the selection.
The scalar sum \Ht\ of the 
$P_T$ of all such jets must exceed 100 GeV, for the dilepton-\Ht\ sample
this requirement is increased to 200 GeV since these triggers have large inefficiency
below this threshold.
Finally $\met > 50$ GeV (we use pfmet). More details are given in the subsections below.

\subsection{Event Cleanup}
\label{sec:cleanup}

\begin{itemize}
   \item Require at least one good deterministic annealing (DA) vertex
   \begin{itemize}
      \item not fake
      \item ndof $>$ 4
      \item $|\rho| < 2$ cm
      \item $|z| < 24$ cm.  
   \end{itemize}
\end{itemize}


\subsection{Muon Selection}
\label{sec:muon}

Muon candidates are RECO muon objects passing the following
requirements:

\begin{itemize}

\item $p_{T} > 5$~GeV and $|\eta| < 2.4$

\item Global Muon and Tracker Muon

\item $\chi^2$/ndof of global fit $<$ 10

\item At least 11 hits in the tracker fit

\item Impact parameter with respect to the first DA vertex $d_{0} < 200$~$\mu$m and $d_{z} < 1$~cm

\item $Iso \equiv E_T^{\rm iso}/p_T < $~0.15, $E_T^{\rm iso}$ is defined as the sum of 
transverse energy/momentum deposits in ecal, hcal, and tracker, in a cone of 0.3

\item At least one of the hits from the 
standalone muon must be used in the global fit

\item Require tracker $\Delta p_T/p_T < 0.1$. This cut was not in the original top analysis.
It is motivated by the observation of poorly measured muons in data with large
relative $p_T$ uncertainty, giving significant contributions to the \met

\end{itemize}



\subsection{Electron Selection}
\label{sec:electron}

Electron candidates are RECO GSF electrons passing the following requirements:

\begin{itemize}

\item $p_{T}>10$~GeV and $|\eta| < 2.5$.

\item Veto electrons with a supercluster in the transition region $1.4442 < |\eta| < 1.556$.

\item VBTF90 identification\cite{ref:vbtf} with requirements tightened to match the CaloIdT and TrkIdVL HLT requirements:

  \begin{itemize}
  \item $\sigma_{i\eta i\eta} < $ 0.01 (EB), 0.03 (EE)
  \item $\Delta\phi < $ 0.15 (EB), 0.10 (EE)
  \item $\Delta\eta < $ 0.007 (EB), 0.009 (EE)
  \item $H/E < $ 0.1 (EB), 0.075 (EE)
  \end{itemize}  

\item Impact parameter with respect to the first DA vertex $d_0 < 400$~$\mu$m and $d_z < 1$~cm.

\item $Iso \equiv $ $E_T^{\rm iso}/p_T < $ 0.15.  $E_T^{\rm iso}$
is defined as the sum of transverse energy/momentum deposits in ecal,
hcal, and tracker, in a 
cone of 0.3.  A 1 GeV pedestal is subtracted from the ecal energy 
deposition in the EB, however the ecal energy is never allowed to 
go negative.

\item Electrons with a tracker or global muon within $\Delta R$ of 
0.1 are vetoed.

\item The number of missing expected inner hits must be less than 
two~\cite{ref:conv}.

\item Conversion removal via partner track finding: any electron
where an additional GeneralTrack is found with $Dist < 0.02$ cm 
and $\Delta \cot \theta < 0.02$ is vetoed~\cite{ref:conv}.

%\item Cleaning for ECAL spike (aka Swiss-Cross cleaning) has been applied
%at the reconstruction level (CMSSW 38x).

\end{itemize}

The requirements defining the fakeable objects used to estimate the
contributions from fake leptons are listed in App.~\ref{sec:appendix_fo}.


\subsection{Invariant mass requirement}
\label{sec:zveto}

We remove $e^+e^-$ and $\mu^+ \mu^-$ events with invariant 
mass between 76 and 106 GeV.  We also remove events
with invariant mass $<$ 12 GeV, since this kinematical region is 
not well reproduced in CMS Monte Carlo and to remove Upsilons.

In addition, we remove $Z \to \mu\mu\gamma$
candidates with the $\gamma$ collinear with one of the muons.  This is
done as follows:
if the ecal energy associated with one of the muons is greater than 6 GeV,
we add this energy to the momentum of the initial muon, and we recompute
the $\mu\mu$ mass.  If this mass is between 76 and 106 GeV, the event is rejected.


\subsection{Trigger Selection}
\label{sec:trigSel}

We do not make any requirements on HLT bits in the Monte Carlo.
Instead, as discussed in 
Section~\ref{sec:trgEff}, a trigger efficiency weight is applied
to each event, based on the trigger efficiencies measured on data (see Sec.~\ref{sec:trgEff}).

We select data events using the following triggers. An event in the $ee$ channel is required
to pass a DoubleElectron trigger, an event in the $\mu\mu$ channel is required to pass a 
DoubleMu trigger, and an event in the $e\mu$ channel is required to pass a Ele-Mu trigger.

   \begin{itemize}
      \item High \pt\ dilepton trigger sample
      \begin{itemize}
         \item \verb=HLT_Ele17_CaloIdL_CaloIsoVL_Ele8_CaloIdL_CaloIsoVL=
         %\item {\footnotesize \verb=HLT_Ele17_CaloIdT_TrkIdVL_CaloIsoVL_TrkIsoVL_Ele8_CaloIdT_TrkIdVL_CaloIsoVL_TrkIsoVL=}
         \item \verb=HLT_DoubleMu7=
         \item \verb=HLT_Mu13_Mu7=
         \item \verb=HLT_Mu17_Ele8_CaloIdL=
         \item \verb=HLT_Mu8_Ele17_CaloIdL=
      \end{itemize}
      \item Lepton \Ht\ cross trigger sample
      \begin{itemize}
        \item \verb=HLT_DoubleMu3_HT150=
        \item \verb=HLT_DoubleMu3_HT160=
        \item \verb=HLT_Mu3_Ele8_CaloIdL_TrkIdVL_HT150=
        \item \verb=HLT_Mu3_Ele8_CaloIdT_TrkIdVL_HT150=
        \item \verb=HLT_Mu3_Ele8_CaloIdL_TrkIdVL_HT160=
        \item \verb=HLT_Mu3_Ele8_CaloIdT_TrkIdVL_HT160=
        \item \verb=HLT_DoubleEle8_CaloIdL_TrkIdVL_HT150=
        \item \verb=HLT_DoubleEle8_CaloIdT_TrkIdVL_HT150=
        \item \verb=HLT_DoubleEle8_CaloIdL_TrkIdVL_HT160=
        \item \verb=HLT_DoubleEle8_CaloIdT_TrkIdVL_HT160=
      \end{itemize}
   \end{itemize}








