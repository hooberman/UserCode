\section{Results}
\label{sec:results}

\subsection{Background estimate from the ABCD method}
\label{sec:abcdres}

We begin by applying the ABCD method to estimate the background in the 2010 signal region.
The data yields in the four regions are summarized in Tables~\ref{tab:datayield1} and
the $y$ vs. \Ht\ distributions are displayed in Fig.~\ref{fig:abcdData1}.
The ABCD background prediction is $N_A \times N_C / N_B = 12.7 \pm 2.4$ (stat), in
agreement with the MC expectation. We also apply the ABCD' method to estimate the
background in this region, following the procedure of App.~\ref{app:abcdprime},
and find a predicted background of $12.8 \pm 2.9$ (stat), in good agreement
with the ABCD prediction.

Next, we apply the ABCD' method to predict the yields in the high \met\ and high \Ht\
signal regions. The $y$ vs. \Ht\ distributions for data are displayed in 
Fig.~\ref{fig:abcdprimedata}. The signal regions are indicated, as well as the control 
regions used to measure the $f(y)$ and $g(H_T)$ distributions. For the high \met\
signal region, we find a predicted yield of $1.0 \pm 0.3$ (stat), in reasonable
agreement with the MC prediction. For the high \Ht\ signal region, we do not find
any events in the control region used to extract $g(H_T)$ with \Ht\ $>$ 600 GeV,
and the ABCD' background estimate is therefore 0. To assess the statistical uncertainty
in this prediction, we add a single event ``by hand'' to the $g(H_T)$ distributiion
at $H_T = 600$ GeV, leading to a predicted yield of 0.6. 
{\bf NEED TO THINK ABOUT THIS. THE PREDICTION DEPENDS ON WHERE IN HT YOU PUT THIS }
{\bf SINGLE EVENT. FOR EXAMPLE, PUTTING IT AT 1000 GIVES A PREDICTION OF 1.2      }
{\bf HOPEFULLY WITH ~3X MORE STATS WE WON'T BE IN THIS SITUATION                  }
These results are summarized in Table~\ref{tab:abcdprime}.

\begin{table}[hbt]
\begin{center}
\caption{\label{tab:abcdprime} 
Summary of results of the ABCD' method, applied to the 3 signal regions. The correction
factors are given in Table~\ref{tab:cor}.
}
\begin{tabular}{lccccc}
\hline
Signal Region             &     ABCD' pred      &  correction factor  &  prediction                                  \\ 
\hline
2010 signal region        &  $12.8 \pm 2.9$     & $1.0 \pm 0.2$      & 12.8 $\pm$ 2.9 (stat) $\pm$ 2.6 (syst)        \\
high \met\ signal region  &  $1.0  \pm 0.3$     & $1.3 \pm 0.2$      &  1.3 $\pm$ 0.4 (stat) $\pm$ 0.2 (syst)        \\
high \Ht\ signal region   &  $0.0  \pm 0.6$     & $1.0 \pm 0.2$      &  0.0 $\pm$ 0.6 (stat) $\pm$ 0.1 (syst)        \\
\hline
\end{tabular}
\end{center}
\end{table}

\begin{figure}[tbh]
\begin{center}
\includegraphics[width=0.6\linewidth]{plots/abcd_349pb.pdf}
\caption{\label{fig:abcdData1}\protect Distributions of $y$ 
vs. \Ht\ for SM Monte Carlo and data. The 2010 signal region boundaries are overlayed.}
\end{center}
\end{figure}

\begin{table}[hbt]
\begin{center}
\caption{\label{tab:datayield1} 
Data yields in the four
regions of Figure~\ref{fig:abcdData1} for the 2010 signal region, 
as well as the predicted yield in region D given
by $N_A \times N_C / N_B$.  The quoted uncertainty
on the prediction in data is statistical only, assuming Gaussian errors.
We also show the SM Monte Carlo expectations with statistical errors only.
}
\begin{tabular}{l||c|c|c|c||c}
\hline
           sample  &            $N_A$  &            $N_B$  &            $N_C$  &             $N_D$  &   $N_A \times N_C / N_B$ \\
\hline
            \ttll  & 63.9  $\pm$  2.0  &252.3  $\pm$  4.1  & 43.3  $\pm$  1.7  &  8.5  $\pm$  0.7  & 11.0  $\pm$  0.6        \\
           \tttau  & 18.5  $\pm$  1.1  & 55.9  $\pm$  1.9  & 10.3  $\pm$  0.8  &  3.4  $\pm$  0.5  &  3.4  $\pm$  0.4        \\
          \ttfake  &  1.0  $\pm$  0.3  &  6.6  $\pm$  0.7  &  1.1  $\pm$  0.3  &  0.4  $\pm$  0.2  &  0.2  $\pm$  0.1        \\
               DY  &  0.9  $\pm$  0.6  & 13.9  $\pm$  2.6  &  1.3  $\pm$  0.8  &  1.7  $\pm$  1.0  &  0.1  $\pm$  0.1        \\
              \WW  &  1.1  $\pm$  0.1  &  2.7  $\pm$  0.2  &  0.2  $\pm$  0.1  &  0.3  $\pm$  0.1  &  0.1  $\pm$  0.0        \\
              \WZ  &  0.1  $\pm$  0.0  &  0.6  $\pm$  0.0  &  0.1  $\pm$  0.0  &  0.0  $\pm$  0.0  &  0.0  $\pm$  0.0        \\
              \ZZ  &  0.0  $\pm$  0.0  &  0.2  $\pm$  0.0  &  0.0  $\pm$  0.0  &  0.0  $\pm$  0.0  &  0.0  $\pm$  0.0        \\
       single top  &  3.3  $\pm$  0.2  &  9.6  $\pm$  0.3  &  0.4  $\pm$  0.1  &  0.1  $\pm$  0.0  &  0.1  $\pm$  0.0        \\
           \wjets  &  1.0  $\pm$  1.0  &  2.2  $\pm$  1.1  &  0.0  $\pm$  0.0  &  0.0  $\pm$  0.0  &  0.0  $\pm$  0.0        \\
\hline
      Total SM MC  & 89.9  $\pm$  2.6  &344.0  $\pm$  5.3  & 56.6  $\pm$  2.1  & 14.5  $\pm$  1.4  & 14.8  $\pm$  0.7        \\
\hline
             data  &              110  &              381  &               44  &               19  & 12.7  $\pm$  2.4        \\
\hline
\end{tabular}
\end{center}
\end{table}


\begin{figure}[hbt]
\begin{center}
\includegraphics[width=0.48\linewidth]{plots/abcdprime_349pb_highmet.pdf}
\includegraphics[width=0.48\linewidth]{plots/abcdprime_349pb_highht.pdf}
\caption{\label{fig:abcdprimedata}\protect 
Distributions of $y$ vs. \Ht\ in data. The signal regions \met\ $>$ 275 GeV, \Ht\ $>$ 300 GeV (left)
and \met\ $>$ 200 GeV, \Ht\ $>$ 600 GeV (right) are indicated with thick black lines. 
The $f(y)$ and $g(H_T)$ 
functions are measured using events in the green and red shaded areas, respectively.
}
\end{center}
\end{figure}

\subsection{Background estimate from the $P_T(\ell\ell)$ method}
\label{sec:victoryres}

We begin by extracting the value of the \met\ acceptance scaling factor $K$ from data,
for the \Ht\ control region 125--300 GeV and for the 2 signal regions \Ht\ $>$ 300 and
\Ht\ $>$ 600 GeV. The quantity $1/K$ is the efficiency for events passing preselection
and falling in the given \Ht\ range to pass the requirement \ptll\ $>$ 50 GeV.
The values of $K$ extracted from data and \ttbar\ MC are given in Table~\ref{tab:K}.
For all 3 \Ht\ regions, the value of $K$ extracted from data agrees with the 
\ttbar\ MC prediction, but for the \Ht\ $>$ 600 GeV region the statistical uncertainty
in $K$ from data is $\sim$100\%. Therefore we use the value of $K$ extracted from
data for the control region 125--300 GeV and for the signal regions \Ht\ $>$ 300;
for the \Ht\ $>$ 600 GeV region we use $K$ from \ttbar\ MC.


\begin{table}[hbt]
\begin{center}
\caption{\label{tab:K} 
Summary of the \met\ acceptance scaling factor $K$, extracted from data and \ttbar\ MC.
}
\begin{tabular}{lccccc}
\hline
region                                    &  data              &   \ttbar\ MC      \\
\hline
control region: 125 $<$ \Ht\ $<$ 300 GeV  &  1.68 $\pm$ 0.14   &   1.67 $\pm$ 0.03 \\
signal region: \Ht\ $>$ 300 GeV           &  1.45 $\pm$ 0.29   &   1.50 $\pm$ 0.06 \\
signal region: \Ht\ $>$ 600 GeV           &  1.12 $\pm$ 1.19   &   1.32 $\pm$ 0.20 \\
\hline
\end{tabular}
\end{center}
\end{table}



\begin{table}[hbt]
\begin{center}
\caption{\label{tab:victory} 
Summary of results of the dilepton $p_{T}$ template method applied to the 3 signal regions.
The quantities indicated in the table are discussed in the text.
The quoted statistical uncertainty in the prediction $N_P$ is due to
that of $N(D')$, the quoted systematic uncertainty includes that of $N(DY)$, $K$, and $K_C$.
%{\bf K and KC taken from MC, do we want to take K and/or KC from data? }
%{\bf Need to add jet/met uncertainty here}
%{\color{red} \bf CURRENTLY USING 30\% UNCERTAINTY ON K_C IN 2010 REGION, WHICH IS WHAT WE HAD LAST YEAR (NEED TO REPEAT JET/MET UNCERTAINTIES?).}
%{\color{red} \bf FOR HIGH Y/HIGH HT REGION, TAKING SYST UNCERTAINTY ON KC FROM MC STATS, WHICH IS PROBABLY DOMINANT}
}
\begin{tabular}{lccccc}
\hline
Signal Region               &  $N(D')$   &   $N(DY)$         &          $K$   &   $K_C$        & $N_P$                                   \\ 
\hline
2010 signal region (UPDATE) &       9    &   0.8 $\pm$ 0.4   & 1.5 $\pm$ 0.3  & 1.4 $\pm$ 0.2  & 16.7 $\pm$ 6.1 (stat) $\pm$ 4.2 (syst)  \\
high \met\ signal region    &       3    &   0.5 $\pm$ 0.3   & 1.5 $\pm$ 0.3  & 1.5 $\pm$ 0.3  & 5.4 $\pm$ 3.8 (stat) $\pm$ 1.7 (syst)   \\
high \Ht\ signal region     &       1    &   0.0 $\pm$ 0.0   & 1.3 $\pm$ 0.2  & 1.3 $\pm$ 0.2  & 1.7 $\pm$ 1.7 (stat) $\pm$ 0.4 (syst)   \\
\hline
\end{tabular}
\end{center}
\end{table}

For each signal region D, we count the number of events falling in the region D', which is defined
using the same requirements as D but switching the $y$ requirement to a $\ptll/\sqrt{H_T}$ requirement (2010 signal region)
or switching the \met\ requirement to a \ptll\ requirement (high \met\ and high \Ht\ signal regions).
We subtract off the expected DY contribution using the data-driven $R_{out/in}$ technique, using $R_{out/in} = 0.13 \pm 0.07$.
%{\color{red} \bf add plot justifying this value}. 
We then scale this yield by 2 corrections factors:
$K$, the \met\ acceptance correction factor, and $K_C$, the correction factor determined in Sec.~\ref{sec:datadriven}.
Our final prediction $N_P$ is given by:

\begin{center}
$ N_P = (N(D')-N(DY)) \times K \times K_C$,
\end{center}

as summarized in Table~\ref{tab:victory}, and displayed in Figs.~\ref{fig:vic1}-\ref{fig:vic3}.
We also perform the \ptll\ method in the \Ht\ sideband region 125--300~GeV, as a validation of the technique in a high statistics
sample which is expected to be dominated by background. The results are summarized in Table~\ref{tab:victorycontrol}
and displayed in Fig.~\ref{fig:victorycontrol}.
The prediction is extractedd for the requirement $y > 8.5$~GeV$^{1/2}$ corresponding to the 2010 signal region, as well as
for \met\ $>$ 200 GeV corresponding to the high \Ht\ signal region. In both cases, we observe good agreement between
the predicted and observed yields. 


\begin{table}[hbt]
\begin{center}
\caption{\label{tab:victorycontrol} 
Summary of results of the dilepton $p_{T}$ template method applied to the \Ht\ sideband control region 125--300 GeV.
The quantities indicated in the table are discussed in the text.
The quoted statistical uncertainty in the prediction $N_P$ is due to
that of $N(D')$, the quoted systematic uncertainty includes that of $N(DY)$ and $K_C$.
The predictions are compare with the observed yield $N_O$.
}
\begin{tabular}{lcccccc}
\hline
Control Region                                   &  $N(D')$   &   $N(DY)$        &  $K$          &   $K_C$          & $N_P$                                     & $N_O$ \\ 
\hline                                           
125 $<$ \Ht\ $<$ 300~GeV, $y >$ 8.5~GeV$^{1/2}$  &     54      &  2.6 $\pm$ 1.2   & 1.7 $\pm$ 0.1 & 1.4 $\pm$ 0.1    & 120.9 $\pm$ 17.3 (stat) $\pm$ 13.6 (syst) & 110   \\
125 $<$ \Ht\ $<$ 300~GeV, \met\ $>$ 200 GeV     &      4      &  1.0 $\pm$ 0.5   & 1.7 $\pm$ 0.1 & 1.3 $\pm$ 0.2    &   6.5 $\pm$  4.4 (stat) $\pm$  1.6 (syst) &   6   \\
\hline
\end{tabular}
\end{center}
\end{table}



\begin{figure}[hbt]
\begin{center}
\includegraphics[width=0.48\linewidth]{plots/victory_y_control_349pb.pdf}
\includegraphics[width=0.48\linewidth]{plots/victory_met200_control_349pb.pdf}
\caption{\label{fig:victorycontrol}\protect 
Results of the \ptll\ method in the \Ht\ sideband region 125-300 GeV.
Left:  distributions of $\ptll/\sqrt{H_T}$ (predicted) and $y$ (observed) for 
SM MC and data. The vertical dashed lines indicate the requirement $y > 8.5$~GeV$^{1/2}$),
corresponding to the 2010 signal region.
Right: distributions of \ptll\ (predicted) and \met\ (observed) for 
SM MC and data. The vertical dashed lines indicate the requirement \met\ $>$ 200 GeV,
corresponding to the high \Ht\ signal region.
}
\end{center}
\end{figure}


\begin{figure}[tbh]
\begin{center}
\includegraphics[width=0.6\linewidth]{plots/victory_y_ht300_349pb.pdf}
\caption{\label{fig:vic1}\protect 
Distributions of $\ptll/\sqrt{H_T}$ (predicted) and $y$ (observed) for 
SM MC and data, for the \Ht $>$ 300 GeV. 
The vertical dashed lines indicate the requirement $y > 8.5$~GeV$^{1/2}$ corresponding to the 2010 signal region.
}
\end{center}
\end{figure}

\begin{figure}[tbh]
\begin{center}
\includegraphics[width=0.6\linewidth]{plots/victory_met275_ht300_349pb.pdf}
\caption{\label{fig:v2}\protect 
Distributions of \ptll\ (predicted) and \met\ (observed) for 
SM MC and data, for the region \Ht $>$ 300 GeV. 
The vertical dashed lines indicate the requirement \met\ $>$ 275 GeV, corresponding to the high \met\ signal region.
}
\end{center}
\end{figure}

\begin{figure}[tbh]
\begin{center}
\includegraphics[width=0.6\linewidth]{plots/victory_met200_ht600_349pb.pdf}
\caption{\label{fig:vic3}\protect 
Distributions of \ptll\ (predicted) and \met\ (observed) for 
SM MC and data, for the region \Ht $>$ 600 GeV. 
The vertical dashed lines indicate the requirement \met\ $>$ 200 GeV, corresponding to the high \Ht\ signal region.
}
\end{center}
\end{figure}

\subsection{Background estimate from OF subtraction}
\label{sec:ofres}

The results of the OF subtraction technique applied to the high \pt\ dilepton trigger sample are summarized in Table~\ref{tab:ofres}. 
We evaluate the quantity $\Delta = R_{\mu e}N(ee) + \frac{1}{R_{\mu e}}N(\mu\mu) - N(e\mu)$ with $R_{\mu e} = 1.12 \pm 0.05$
extracted from the ratio of $Z \to \mu^+\mu^-$ vs. $Z \to e^+e^-$ events in data.
We perform the OF subtraction first in the preselection region, and find $\Delta$ consistent with 0, as expected.
We then perform the OF subtraction in all 3 signal regions, and do not observe any excess of same-flavor vs. opposite-flavor events.

\begin{table}[hbt]
\begin{center}
\caption{\label{tab:ofres} Summary of results for the OF subtraction technique. 
The quantity $\Delta = R_{\mu e}N(ee) + \frac{1}{R_{\mu e}}N(\mu\mu) - N(e\mu)$ is quoted with $R_{\mu e} = 1.12 \pm 0.05$.
The quoted systematic uncertainty corresponds to that of $R_{\mu e}$. The $e\mu$ yields differ from those previously
quoted because the $Z$ mass veto is included here.
}
\begin{tabular}{l|ccc|c}
\hline
region                   &  $N(ee)$ & $N(\mu\mu)$ & $N(e\mu)$  &  $\Delta$   \\ 
\hline
preselection region      &      193 &         201 &      394   &    2.0 $\pm$ 28 (stat) $\pm$ 1.8 (syst) \\    
2010 signal region       &        4 &           4 &        9   &   -0.9 $\pm$ 4.2 (stat) $\pm$ 0.1 (syst)  \\
high \met\ signal region &        2 &           0 &        1   &    1.3 $\pm$ 1.9 (stat) $\pm$ 0.1 (syst)  \\
high \Ht\ signal region  &        1 &           0 &        1   &    0.1 $\pm$ 1.5 (stat) $\pm$ 0.0 (syst)  \\
\hline
\end{tabular}
\end{center}
\end{table}

For the dilepton-\Ht\ trigger sample, we observe only 1 event in the 2010 signal region, consistent with MC expectations,
and no events in either the high $y$ or high \Ht\ signal regions. In the case of an excess of events at low lepton \pt,
we will perform the OF subtraction technique of Sec.~\ref{sec:oflowpt}.

% \clearpage
\subsection{Summary of results}

\begin{table}[hbt]
\begin{center}
\caption{\label{tab:results} 
Summary of the observed and predicted yields in the 3 signal regions. MC errors are statistical only. The systematic uncertainty on the ABCD
and \ptll\ method is from the scaling factors from MC closure only. 
%{\bf need to put additional uncertainties, for example jet/met scale.}
For the OF subtraction, the quantity $\Delta = R_{\mu e}N(ee) + \frac{1}{R_{\mu e}}N(\mu\mu) - N(e\mu)$ is quoted; the systematic uncertainty
here is from the ratio of muon to electron selection efficiencies.
}
\begin{tabular}{l|c|c|c}
\hline
                                       &  2010 signal region                       &   high \met\ signal region             &  high \Ht\ signal region              \\ 
\hline
Observed yield                         &         19                                &                        4               &                        3              \\
\hline
MC prediction                          &    14.5 $\pm$ 1.4                         &            2.6 $\pm$ 0.8               &            2.5 $\pm$ 0.8              \\
ABCD prediction                        &    12.7 $\pm$ 2.4 (stat) $\pm$ 2.5 (syst) &                                        &                                       \\
ABCD' prediction                       &    12.8 $\pm$ 2.9 (stat) $\pm$ 2.6 (syst) & 1.3 $\pm$ 0.4 (stat) $\pm$ 0.2 (syst)  & 0.0 $\pm$ 0.6 (stat) $\pm$ 0.1 (syst) \\
\ptll\ prediction                      &    16.7 $\pm$ 6.1 (stat) $\pm$ 4.2 (syst) & 5.4 $\pm$ 3.8 (stat) $\pm$ 1.7 (syst)  & 1.7 $\pm$ 1.7 (stat) $\pm$ 0.4 (syst) \\
\hline
OF subtraction ($\Delta$)              &    -0.9 $\pm$ 4.2 (stat) $\pm$ 0.1 (syst) & 1.3 $\pm$ 1.9 (stat) $\pm$ 0.1 (syst)  & 0.1 $\pm$ 1.5 (stat) $\pm$ 0.0 (syst) \\
\hline
\end{tabular}
\end{center}
\end{table}

A summary of our results is presented in Table~\ref{tab:results}. In all 3 signal regions, we observe reasonable agreement
between the observed yields and the predictions from MC and data-driven background estimates. We therefore do not observe
evidence for an excess of events above SM expectations. After assessing systematic uncertainties in Sec.~\ref{sec:systematics},
we proceed to set upper limits on the non-SM contributions to the signal regions in Sec.~\ref{sec:limits}.
