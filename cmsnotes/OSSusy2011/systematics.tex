\section{Acceptance and efficiency systematics}
\label{sec:systematics}

This is a search for new physics contributions to 
events with high \met and lots of jet activity.
As seen in Section~\ref{sec:results}, there is no
evidence for a contribution beyond SM expectations.

Strictly speaking it is impossible to talk about 
``acceptance and efficiency systematics'' because these kinds of
systematics only apply to a well defined final state.
Nevertheless, we can make general statements about the 
systematic uncertainties, including quantitative
estimates of the systematic uncertainties associated with
a few specific processes. 

The systematic uncertainty on the lepton acceptance consists
of two parts: the trigger efficiency uncertainty and the 
ID and isolation uncertainty.  We discuss these in turn.

The trigger efficiency for the high \pt\ dilepton triggers has been studied
in~\cite{ref:HWW} in detail. The efficiency is found to be approximately
100\% for $ee$, 95\% for $e\mu$, and 90\% for $\mu\mu$. There is a small
dependence of the efficiency on the lepton \pt, which we do not attempt
to parameterize. Instead, we assign an uncertainty of 2\%, based on the
small variation of the efficiency on the lepton \pt.

To evaluate the uncertainty on the lepton ID and isolation efficiencies,
we perform a tag and probe technique on $Z$ data and MC. We calculate
the efficiency for leptons passing isolation criteria to also pass
the ID criteria, and the efficiency for leptons passing the ID criteria
to also pass the isolation criteria, as summarized in Table.~\ref{tab:tagandprobe}.
We observe agreement between data and MC within about 2\%, and assign 
a corresponding systematic uncertainty.

\begin{table}[hbt]
\begin{center}
\caption{\label{tab:tagandprobe} Tag and probe results on $Z \to \ell \ell$
on data and MC.  We quote ID efficiency given isolation and 
the isolation efficiency given ID. {\bf UPDATE TABLE WITH 2011 DATA/MC}}
\vspace{.25cm}
\begin{tabular}{|l||c|c|}
\hline
                             & Data  T\&P      & MC T\&P             \\  
\hline
$\epsilon(id|iso)$ electrons & $0.925 \pm 0.007$ & $0.934 \pm 0.004$ \\
$\epsilon(iso|id)$ electrons & $0.991 \pm 0.002$ & $0.987 \pm 0.002$ \\
$\epsilon(id|iso)$ muons     & $0.962 \pm 0.005$ & $0.984 \pm 0.002$ \\
$\epsilon(iso|id)$ muons     & $0.987 \pm 0.003$ & $0.982 \pm 0.002$ \\ 
\hline
\end{tabular}
\end{center}
\end{table}

Another significant source of systematic uncertainty is 
associated with the jet and \met\ energy scale.  The impact
of this uncertainty is final-state dependent.  Final
states characterized by lots of hadronic activity and \met\ are 
less sensitive than final states where the \met\ and \Ht\
are typically close to the requirement.  To be more quantitative,
we have used the method of Reference~\cite{ref:top} to evaluate
the systematic uncertainties on the acceptance for $t\bar{t}$ 
and two benchmark SUSY points.  The uncertainties are calculated
assuming a 5\% uncertainty to the hadronic energy scale in CMS.

\begin{table}[hbt]
\begin{center}
\caption{\label{tab:jetmet} 
Summary of efficiency uncertainties due to the 5\% uncertainty in the hadronic energy scale.
}
\vspace{.25cm}
\begin{tabular}{l|ccc}
\hline
Signal Region             & \ttbar  &   LM1   &   LM3    \\
\hline
2010 signal region        &  0.28   &  0.06   &  0.07    \\
high \met\ signal region  &  0.35   &  0.14   &  0.18    \\ 
high \Ht\ signal region   &  0.30   &  0.20   &  0.21    \\
\hline
\end{tabular}
\end{center}
\end{table}

The uncertainty in the integrated luminosity is 11\%~\cite{ref:lumi}.
