\section{Non $t\bar{t}$ Backgrounds}
\label{sec:othBG}

\subsection{Dilepton backgrounds from rare SM processes}
\label{sec:bgrare}
Backgrounds from divector bosons and single top
can be reliably estimated from Monte Carlo.
They are negligible compared to $t\bar{t}$. 


\subsection{Drell Yan background}
\label{sec:dybg}
 
%Backgrounds from Drell Yan are also expected
%to be negligible from MC.  However one always 
%worries about the modeling of tails of the \met. 
%In the context of other dilepton analyses we 
%have developed a data driven method to estimate 
%the number of Drell Yan events\cite{ref:dy}.
%The method is based on counting the number of
%$Z$ candidates passing the full selection, and 
%then scaling by the expected ratio of Drell Yan
%events outside vs. inside the $Z$ mass 
%window.\footnote{A correction based on $e\mu$ events 
%is also applied.}  This ratio is called $R_{out/in}$
%and is obtained from Monte Carlo.

%To estimate the Drell-Yan contribution in the four $ABCD$ 
%regions, we count the numbers of $Z \to ee$ and $Z \to \mu\mu$
%events falling in each region, we subtract off the number 
%of $e\mu$ events with $76 < M(e\mu) < 106$ GeV, and 
%we multiply the result by $R_{out/in}$ from Monte Carlo.
%The results are summarized in Table~\ref{tab:ABCD-DY}.

%\begin{table}[hbt]
%\begin{center}
%\caption{\label{tab:ABCD-DY} Drell-Yan estimations
%in the four 
%regions of Figure~\ref{fig:abcdData}.  The yields are
%for dileptons with invariant mass consistent with the $Z$.
%The factor
%$R_{out/in}$ is from MC.  All uncertainties 
%are statistical only.  In regions $A$ and $D$ there is no statistics
%in the Monte Carlo to calculate $R_{out/in}$.}
%\begin{tabular}{|l|c|c|c||c|}
%\hline
%Region   & $N(ee)+N(\mu\mu)$ & $N(e\mu)$ & $R_{out/in}$ & Estimated DY BG \\
%\hline
%$A$      &  0                & 0         & ??          & ??$\pm$xx   \\
%$B$      &  5                & 1         & 2.5$\pm$1.0 & 9$\pm$xx   \\
%$C$      &  0                & 0         & 1.$\pm$1.   & 0$\pm$xx   \\
%$D$      &  0                & 0         & ??          & 0$\pm$xx   \\
%\hline
%\end{tabular}
%\end{center}
%\end{table}



%When  find no dilepton events with invariant mass
%consistent with the $Z$ in the signal region.
%Using the value of 0.1 for the ratio described above, this 
%means that the Drell Yan background in our signal 
%region is $< 0.23\%$ events at the 90\% confidence level.
%{\color{red} (If we find 1 event this will need to be adjusted)}.

As discussed in Section~\ref{sec:victory}, residual Drell-Yan 
events can have a significant effect on the data driven background
prediction based on $P_T(\ell\ell)$.  This is taken into account,
based on MC expectations, 
by the $K_C$ factor described in that Section.  
As a cross-check, we use a separate data driven method to 
estimate the impact of Drell Yan events on the 
background prediction based on $P_T(\ell\ell)$. 
In this method\cite{ref:top} we count the number 
of $Z$ candidates\footnote{$e^+e^-$ and $\mu^+\mu^-$
with invariant mass between 76 and 106 GeV.}
passing the same selection as 
region $D$ except that the $\met/\sqrt{\rm SumJetPt}$ requirement is 
replaced by a $P_T(\ell\ell)/\sqrt{\rm SumJetPt}$ requirement
(we call this the ``region $D'$ selection'').
We subtract off the number of $e\mu$ events with 
invariant mass in the $Z$ passing the region $D'$ selection.
Finally, we multiply the result by ratio $R_{out/in}$ derived
from Monte Carlo as the ratio of Drell-Yan events
outside/inside the $Z$ mass window 
in the $D'$ region.

In the region D', we find no DY events in MC outside the Z mass window.
Our estimate of $R_{out/in}$ and our expected DY contribution to the region
D' in data are therefore zero.

%We find $N^{D'}(ee+\mu\mu)=2$, $N^{D'}(e\mu)=0$,
%$R^{D'}_{out/in}=0.18\pm0.16$ (stat.). 
%Thus we estimate the number of Drell Yan events in region $D'$ to 
%be $0.36\pm 0.36$. 

As a cross-check, we also perform the $P_{T}(\ell\ell)$ method to
predict the yield in region A using the yield in $A'$, and
we therefore also perform the DY estimate in the region $A'$. Here we find
$N^{A'}(ee+\mu\mu)=4$, $N^{A'}(e\mu)=0$,
$R^{A'}_{out/in}=0.33\pm0.17$ (stat), giving an estimated 
number of Drell Yan events in region $A'$ to 
be $1.3 \pm 0.9$. 


This Drell Yan method could also be used to estimate 
the number of DY events in the signal region (region $D$).
However, there is not enough statistics in the Monte 
Carlo to make a measurement of $R_{out/in}$ in region 
$D$.  In any case, no $Z \to \ell\ell$ candidates are
found in region $D$.  



%In the context of other dilepton analyses we 
%have developed a data driven method to estimate 
%the number of Drell Yan events\cite{ref:dy}.
%The method is based on counting the number of
%$Z$ candidates passing the full selection, and 
%then scaling by the expected ratio of Drell Yan
%events outside vs. inside the $Z$ mass 
%window.\footnote{A correction based on $e\mu$ events 
%is also applied.}  This ratio is called $R_{out/in}$
%and is obtained from Monte Carlo.




%As a cross-check, we use the same Drell Yan background
%estimation method described above to estimate the 
%number of DY events in the regions $A'B'C'D'$.
%The region $A'$ is defined in the same way as the region $A$ 
%except that the $\met/\sqrt{\rm SumJetPt}$ requirement is 
%replaced by a $P_T(\ell\ell)/\sqrt{\rm SumJetPt}$ requirement.
%The regions B', 
%C', and D' are defined in a similar way.  The results are
%summarized in Table~\ref{tab:ABCD-DYptll}.

%\begin{table}[hbt]
%\begin{center}
%\caption{\label{tab:ABCD-DYptll} Drell-Yan estimations
%in the four 
%regions $A'B'C'D'$ defined in the text.  The yields are
%for dileptons with invariant mass consistent with the $Z$.
%The factor
%$R_{out/in}$ is from MC.  All uncertainties 
%are statistical only.}
%\begin{tabular}{|l|c|c|c||c|}
%\hline
%Region    & $N(ee)+N(\mu\mu)$ & $N(e\mu)$ & $R_{out/in}$ & Estimated DY BG \\
%\hline
%$A'$      &  3               & 0        & 0.7$\pm$0.3   & 2.1$\pm$xx   \\
%$B'$      &  3               & 0        & 2.5$\pm$2.1   & 7.5$\pm$xx   \\
%$C'$      &  0               & 0        & 0.1$\pm$0.1   & 0$\pm$xx   \\
%$D'$      &  1               & 0        & 0.4$\pm$0.3   & 0.4$\pm$xx   \\
%\hline
%\end{tabular}
%\end{center}
%\end{table}


\subsection{Background from ``fake'' leptons}
\label{sec:bgfake}

Finally, we can use the ``Fake Rate'' method\cite{ref:FR} 
to predict 
the number of events with one fake lepton. We select
events where one of the leptons passes the full selection and 
the other one fails the full selection but passes the 
``Fakeable Object'' selection of 
Reference~\cite{ref:FR}.\footnote{For electrons we use
the V3 fakeable object definition to avoid complications 
associated with electron ID cuts applied in the trigger.} 
We then weight each event passing the full selection
by FR/(1-FR) where FR is the ``fake rate'' for the 
fakeable object.  

We first apply this method to events passing the preselection (we used
Spring10 MC samples for the following study, and we do not plan to update with Fall10
MC). The raw result is $6.7 \pm 1.7 \pm 3.4$, where the first uncertainty is 
statistical and the second uncertainty is from the 50\% systematic 
uncertainty associated with this method\cite{ref:FR}.  This has 
to be corrected for ``signal contamination'', {\em i.e.}, the 
contribution from true dilepton events with one lepton 
failing the selection.  This is estimated from Monte Carlo 
to be $2.3 \pm 0.05$, where the uncertainty is from MC statistics
only. Thus, the estimated number of events with one ``fake''
lepton after the preselection is $4.4 \pm 3.8$.  
The Monte Carlo expectation for this contribution can be obtained
by summing up the $t\bar{t}\rightarrow \mathrm{other}$ and
$W^{\pm}$ + jets entries from Table~\ref{tab:yields}.  This 
result is $2.0 \pm 0.2$ (stat. error only).  Thus, this study
confirms that the contribution of fake leptons to the event sample
after preselection is small, and consistent with the MC prediction.

We apply the same method to events in the signal region (region D).  
There are no events where one of the leptons passes the full selection and 
the other one fails the full selection but passes the 
``Fakeable Object'' selection.  Thus the background estimate
is $0.0^{+0.4}_{-0.0}$, where the upper uncertainty corresponds (roughly)
to what we would have calculated if we had found one such event.

We can also apply a similar technique to estimate backgrounds
with two fake leptons, {\em e.g.}, from QCD events.
In this case we select events with both
leptons failing the full selection but passing the 
``Fakeable Object'' selection.  For the preselection, the 
result is $0.2 \pm 0.2 \pm 0.2$, where the first uncertainty
is statistical and the second uncertainty is from the fake rate
systematics (50\% per lepton, 100\% total).  Note that this 
double fake contribution is already included in the $4.4 \pm 3.8$ 
single fake estimate discussed above $-$ in fact, it is double counted.  
Therefore the total fake estimate is $4.0 \pm 3.8$ (single fakes)
and $0.2 \pm 0.3$ (double fakes).

In the signal region (region D), the estimated double fake background
is $0.00^{+0.04}_{-0.00}$.  This is negligible.