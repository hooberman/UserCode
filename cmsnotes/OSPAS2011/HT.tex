\section{Same-flavour Dilepton Search}
\label{sec:HT} 

The result of Section~\ref{sec:results} is cross-checked in a similar kinematic region with an 
independent
search relying on a different trigger path, different methods for ``physics object'' reconstruction, and a
different background estimation method. 
This search is directed at BSM scenarios in which decay chains of a pair of new heavy particles
produce an excess of same-flavour ($e^{+}e^{-}$ and $\mu^{+}\mu^{-}$) events over opposite-flavour ($e^{\pm}\mu^{\mp}$) events. 
For example, in the context of the CMSSM, this excess may be caused by decays of neutralinos and Z bosons to same-flavour lepton pairs. 
For the benchmark scenario LM0 (LM1), the fraction of same-flavour events in the signal region discussed
below is 0.67 (0.86).

The dominant background in this search is also dilepton $t\bar{t}$, for which such an excess does not exist 
because the flavours of the two leptons are uncorrelated.
Therefore, the rate of $t\bar{t}$ decays with two  same-flavour leptons 
may be estimated from the number of opposite-flavour events, 
after correcting for the 
ratio of muon to electron selection efficiencies, $r_{\mu{}e}$. 
This method actually estimates the contribution of any uncorrelated pair of leptons, including
e.g.\ $Z\to\tau\tau$ events where the two $\tau$ leptons decay leptonically. 
This method will also subtract any BSM signal producing lepton pairs of uncorrelated flavour.

Events with two leptons with $\pt>10\GeVc$ are selected. Because the lepton triggers are not fully
efficient for events with two leptons of $\pt>10\GeVc$,
the data sample for this analysis is selected with hadronic triggers based on the
scalar sum of the transverse energies of all jets reconstructed from calorimeter signals with $\pt>20\GeVc$. 
The event is required to pass at least one of a set of hadronic triggers with transverse energy thresholds
ranging from 100 to 150~GeV. The efficiency of this set of triggers with respect to the analysis selection is
greater than 99\%.
In addition to the trigger, we require $\HT>350\GeV$, 
where \HT\ in this analysis is defined as the scalar sum of 
the transverse energies of all selected jets with $\pt>30\GeVc$
and within an increased pseudorapidity range $|\eta|<3$, in line with the trigger requirement.
The jets, \MET, and leptons are reconstructed with the Particle Flow technique~\cite{CMS-PAS-PFT-10-002}.
The resulting performance of the selection of leptons and jets does not differ 
significantly from the selection discussed  in Section~\ref{sec:eventSel}.  

The signal region is defined by additionally requiring $\MET>150\GeV$. 
This signal region is chosen such that approximately one SM event is expected in our 
current data sample. 

The  lepton   selection  efficiencies  are  measured   using  the  $Z$
resonance.    As   discussed   in  Section~\ref{sec:systematics},   these
efficiencies are  known with a  systematic uncertainty of  $2\%$.  The
selection efficiencies of isolated leptons are different in the $t\bar{t}$
and  $Z+\textrm{jets}$  samples.   The   ratio  of  muon  to  electron
efficiencies $r_{\mu{}e}$,  however, is found  to differ by  less than
5\%  in the MC simulations,  and  a corresponding  systematic
uncertainty is assigned to this ratio. This procedure gives $r_{\mu{}e} = 1.07 \pm
0.06$.

The $W+\textrm{jets}$ and QCD multijet  contributions, where at least one
of the two leptons is a  secondary lepton from a heavy flavour decay or
a jet misidentified as a lepton (non-$W/Z$ leptons) are estimated from
a  fit to the  lepton isolation distribution,  after relaxing
the  isolation requirement on  the leptons.   
%The estimated  purity is then applied to the number of  observed events in the signal region to
%infer  the number  of non-$W/Z$  leptons therein.   
Contributions from
other SM backgrounds,  such as DY or processes  with two gauge bosons,
are strongly suppressed  by the \MET\ requirement and  are expected to
be negligible.

We first estimate the number of SM events in a \ttbar-dominated region 
with $100 < \HT < 350\GeV$ and $\MET>80\GeV$. In order to cope with the lower \HT\ requirement,
we use the same high-\pt lepton trigger sample as described in Section~\ref{sec:eventSel}. 
In this region we observe $26$ opposite-flavour candidates and predict $1.0\pm0.5$ 
non-$W/Z$ lepton events from the fit to the lepton isolation distribution. This results in an 
estimate of $25.0 \pm 5.0$ \ttbar\ events in the $e\mu$ channel. Using the efficiency 
ratio  $r_{\mu{}e}$  this estimate is then converted into a prediction for the number
of same-flavour events in the $ee$ and $\mu\mu$ channels. 

\begin{table}[hbt]
\begin{center}
\caption{\label{tab:CRresults}
Number of predicted and observed $ee$ and $\mu\mu$ 
events in the control region, defined as $100 < \HT < 350\GeV$ and $\MET > 80\GeV$.
``SM MC'' indicates the sum of all MC samples ($t\bar{t}$, DY, $W+\textrm{jets}$, and $WW/WZ/ZZ$)
and includes statistical uncertainties only.
}
\vspace{2 mm}
\begin{tabular}{l|cc}
\hline
                                 & \multicolumn{2}{c}{Control region}               \\
\hline 
Process                          & $ee$          & $\mu\mu$        \\
\hline
$t\bar{t}$ predicted from $e\mu$ & $11.7\pm 2.4$ & $13.4\pm 2.8$   \\
Non-$W/Z$ leptons                & $0.5\pm 0.3$  & $0.4\pm0.2$ \\
\hline
Total predicted                  & $12.2\pm 2.4$ & $13.8 \pm 2.8$  \\
\hline\hline
Total observed                   & $10$          & $15$          \\
\hline \hline
SM MC                            & $8.4\pm 0.2$  & $10.5 \pm 0.3$    \\
%LM0                    &  $3.7\pm0.2$  & $4.2\pm0.2$     \\
%LM1                    &  $0.5\pm0.1$  & $0.7\pm0.1$     \\

\hline
\end{tabular}
\end{center}
\end{table}
 
Table~\ref{tab:CRresults} shows the number of expected SM background same-flavour events in the control region for the MC, 
as well as the prediction from the background estimation techniques based on data. There are a total of 25 same-flavour 
events, in good agreement with the prediction of $25.9 \pm 5.2$ events.
We thus proceed to the signal region selection. 
%It is worth noting that in this control region, we expect $7.9 \pm 1.4$ and $1.2 \pm 0.2$ from LM0 and LM1 respectively. 

The SM background predictions in the signal region from the opposite-flavour and non-$W/Z$ lepton methods 
are summarized in Table~\ref{tab:results}. 
We find one event in the signal region in the $e\mu$ channel with a prediction of non-$W/Z$ leptons
of $0.1\pm0.1$, and thus predict $0.9 {}_{-0.8}^{+2.2}$ same-flavour events using Poisson statistical uncertainties.
In the data we find no same-flavour events, in agreement with the prediction, in contrast with $7.3\pm1.6$ 
and $3.6\pm0.7$ expected events for the benchmark points LM0 and LM1, respectively. 
%With zero events observed, the non-$W/Z$ lepton prediction is also zero.
The predicted background from non-$W/Z$ leptons is negligible.

\begin{table}[hbt]
\begin{center}
\caption{\label{tab:results}
Number of predicted and observed events in the signal region, defined as $\HT > 350\GeV$ and $\MET> 150\GeV$.
``SM MC'' indicates the sum of all MC samples ($t\bar{t}$, DY, $W+\textrm{jets}$, and $WW/WZ/ZZ$)
and includes statistical uncertainties only.
}
\vspace{2 mm}
\begin{tabular}{l|cc}
\hline
                                    &   \multicolumn{2}{c}{Signal region}          \\
\hline 
Process                             & $ee$                   & $\mu\mu$  \\
\hline
$t\bar{t}$ predicted from $e\mu$    & $0.4 {}_{-0.4}^{+1.0}$ & $0.5 {}_{-0.4}^{+1.2}$  \\
Non-$W/Z$                           & 0                      & 0                        \\
\hline
Total predicted                     & $0.4 {}_{-0.4}^{+1.0}$ & $0.5 {}_{-0.4}^{+1.2}$   \\
\hline\hline   
Total observed                      & $0$                    & $0$  \\
\hline \hline
SM MC                               & $0.38\pm 0.08$         & $0.56 \pm 0.07$ \\
LM0                                 & $3.4\pm0.2$            & $3.9\pm0.2$  \\
LM1                                 & $1.6\pm0.1$            & $2.0\pm0.1$  \\

\hline
\end{tabular}
\end{center}
\end{table}

Table~\ref{tab:results} demonstrates the sensitivity of this approach. 
We observe comparable yields of the same benchmark points as for the high-\pt\  
lepton trigger search, where 35--60\% of the events are common to both 
searches for LM0 and LM1. 
Either approach would have given an excess in the presence of a signal.
