%\subsection{Non $t\bar{t}$ Backgrounds}
%\label{sec:othBG}

Backgrounds  in  which one  or  both  leptons  do not  originate  from
electroweak decays  (non-$W/Z$ leptons) are assessed  using the method
of  Ref.~\cite{ref:top}.  A non-$W/Z$  lepton is  a lepton  candidate
originating from within a jet,  such as a lepton from semileptonic $b$
or  $c$ decays,  a muon  decay-in-flight, a  pion misidentified  as an
electron,  or an  unidentified  photon conversion.   Estimates of  the
contributions to  the signal region  from pure multijet QCD,  with two
non-$W/Z$ leptons, and in $W+\mathrm{jets}$, with one non-$W/Z$ lepton
in  addition to  the lepton  from the  decay of  the $W$,  are derived
separately. We find $0.00^{+0.04}_{-0.00}$ and $0.0^{+0.5}_{-0.0}$ 
($0.00^{+0.04}_{-0.00}$ and $0.5 \pm 0.5$)
for the  multijet QCD  and $W$+jets  contributions to the high \MET\
(high \Ht) signal regions, respectively,  and thus
consider these backgrounds to be negligible.

Backgrounds from DY are estimated with the data-driven $R_{out/in}$ technique~\cite{ref:top},
which leads to an estimated DY contribution which is consistent with 0.
Backgrounds from processes with two vector bosons and single top 
are negligible compared to dilepton $t\bar{t}$. 
