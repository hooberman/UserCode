\section{Limits on New Physics}
\label{sec:limit}

We proceed to set upper limits on the non-SM contributions to the 2 signal regions. For both regions,
we find reasonable agreement between the observed yields and the predictions from MC and from the 2
data-driven methods. We choose here to extract the upper limits using the MC prediction for the
background estimate. The 95\% CL upper limit is extracted using a Bayesian technique~\cite{ref:cl95cms}, 
using a log-normal model of nuissance parameter integration. The results are summarized in 
Table~\ref{tab:results}. Based on these results, we exclude LM1 but not LM3.

\begin{table}[hbt]
\begin{center}
\caption{\label{tab:results} 
Summary of the observed and predicted yields in the 2 signal regions. MC errors are statistical only. 
}
\begin{tabular}{l|c|c|c}
\hline
                                       &     high \met\ signal region             &  high \Ht\ signal region              \\ 
\hline
Observed yield                         &                          4               &                        3              \\
\hline
MC prediction                          &              2.6 $\pm$ 0.8               &            2.5 $\pm$ 0.8              \\
ABCD' prediction                       &   1.2 $\pm$ 0.4 (stat) $\pm$ 0.5 (syst)  & 0.0 $\pm$ 0.6 (stat) $\pm$ 0.3 (syst) \\
\ptll\ prediction                      &   5.4 $\pm$ 3.8 (stat) $\pm$ 2.2 (syst)  & 1.7 $\pm$ 1.7 (stat) $\pm$ 0.6 (syst) \\
non-SM yield UL                        &                 5.2                      &               4.1                     \\
LM1 yield                              &                17 $\pm$ 3.1              &             14 $\pm$ 3.1              \\
LM3 yield                              &               4.3 $\pm$ 0.9              &            4.3 $\pm$ 1.0              \\
\hline
OF subtraction ($\Delta$)              &   1.3 $\pm$ 1.9 (stat) $\pm$ 0.1 (syst)  & 0.1 $\pm$ 1.5 (stat) $\pm$ 0.0 (syst) \\
\hline
\end{tabular}
\end{center}
\end{table}
