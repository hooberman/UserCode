Next, we search for a kinematic edge in the dilepton mass distribution. Such a feature is 
predicted to occur in SUSY models in which the opposite-sign leptons are produced by the
decay $\chi_2^0 \to \chi_1^0 \ell^+\ell^-$. Such models tend to produce an excess of pairs 
of same-flavor (SF) $ee$ or $\mu\mu$ leptons with respect to opposite-flavor (OF) $e\mu$
leptons. For the \ttbar\ background, the rates of SF and OF lepton pairs are the same;
furthermore, the kinematic properties of events containing SF and DF lepton pairs are the
same. Thus we can extract the shape of the \ttbar\ dilepton mass distribution from events 
containing OF leptons, and apply it to events containing SF leptons, which is where we expect 
our signal to be enhanced.

In order to further suppress DY, we increase the \MET\ requirement to \MET $>$ 100 GeV. 
We search for the kinematic edge in 2 regions.  The first region is a control region defined
as 100 $<$ \Ht\ $<$ 300 GeV which is expected to be dominated by the \ttbar\ background; we use 
this region to validate our fit methodology and verify that a signal yield consistent with 0 
is obtained. We then proceed to search for a kinematic edge in the signal region defined as 
\Ht\ $>$ 300 GeV. Since we do not observe a kinematic edge in this region, we perform a 
fit to the dilepton mass distribution assuming an example signal shape from the LM1 scenario.

For each region, we begin by extracting the \ttbar\ shape from the dilepton mass distribution 
in OF events, as shown in Fig.~\ref{fig:fit} (right). The background is described by:

\begin{equation}
B(M(ll)) = M(ll)^a e^{-b M(ll)}.
\end{equation}

For a potential signal, we use an edge model for a two-body decay, which comprises a triangular shape convoluted with a gaussian,
according to:

\begin{equation}
T(M(ll) = \frac{1}{\sqrt{2\pi}\sigma}\int_0^{M_{cut}} dy y e^{\frac{(-(M(ll)-y)^2}{2\sigma^2}}. 
\end{equation}

The position of the kinematic edge $M_{cut}$ is fixed based on the generator level
information for the signal model which is tested; for example, for LM1 
$M_{cut} = 78$~GeV. Finally, the $Z$ contribution is modelled by a Breit-Wigner 
convoluted with a Gaussian. We proceed to perform an extended, unbinned maximum 
likelihood (ML) fit to the distribution of dilepton mass for SF events. The shape of 
the \ttbar\ background is fixed, and the yield is allowed to vary within the 
statistical plus systematic uncertainty extracted from the events with $e\mu$ pairs. 

We begin by performing the fit in the control region 100 $<$ \Ht\ $<$ 300 GeV, in
which the \ttbar\ background, $Z$ background, and LM1 signal yields are allowed to vary in the fit. 
The extracted signal yield is $n_S = 2.4 \pm 4.7$, consistent with the background only 
hypothesis. The extracted $Z$ yield is $n_Z = 4.4 \pm 4.2$, which is used to constrain
the $Z$ yield in the signal region. 

Next, we proceed to perform the fit in the signal region \Ht\ $>$ 300 GeV. First, we
constrain the $Z$ yield in this region using an extrapolation in \Ht\ from the 
control region 100 $<$ \Ht\ $<$ 300 GeV. The $Z$ yield in the control region is
multiplied by a scale factor derived from $Z$ events in data with no requirement
on \MET, which quantifies the fraction of $Z$ events with \Ht\ $>$ 100 GeV which 
satisfy \Ht\ $>$ 300 GeV. Using this procedure we derive an upper limit on the
$Z$ yield in the signal region of $n_Z < 0.3$, which we use to constrain the
$Z$ yield in the ML fit. The extracted signal yield is $n_S = 4.6 \pm 3.2$,
which is consistent with zero yield within 1.4$\sigma$ and gives a 95\% confidence
level (CL) upper limit (UL) of X events. The expected LM1 yield
is $X \pm X$; hence we exclude LM1 based on these results.




