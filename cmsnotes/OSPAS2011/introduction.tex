\section{Introduction}
\label{sec:intro}

In this paper we describe a search for physics beyond the standard model (BSM) 
in a sample of proton-proton collisions at a centre-of-mass energy of 7~TeV. 
The data sample was collected with the Compact Muon Solenoid (CMS) detector~\cite{JINST} at 
the Large Hadron Collider (LHC) in 2011
and corresponds to an integrated luminosity of \lumifinal.
This is an update of a previous analysis performed with a data sample of 34~pb$^{-1}$
collected in 2010~\cite{ref:ospaper}.

The BSM signature in this search is motivated  by three general  considerations. 
First, new particles predicted by BSM
physics scenarios are expected to be heavy, since they have so far eluded detection.
Second, BSM physics  signals with  high
enough  cross sections to  be observed  in our  current dataset are expected to be
produced  strongly,  resulting  in  significant hadronic  activity. 
Third, astrophysical evidence for
dark matter  suggests~\cite{ref:DM,ref:DM2}  that the mass  of weakly-interacting
massive particles is of the  order of the electroweak symmetry breaking
scale. Such particles, if produced in pp collisions, could escape detection and give rise to
an apparent imbalance in the  event transverse energy. We therefore focus  on the
region  of  high missing transverse energy (\MET). An  example of  a  specific  BSM scenario  is
provided by R-parity conserving  supersymmetric (SUSY) models in which
new,  heavy  particles  are  pair-produced  and  subsequently  undergo
cascade       decays,      producing      hadronic       jets      and
leptons~\cite{Martin:fk,susy1,susy2,susy3,susy4,susy5,susy6}.  
These cascade decays  may terminate  in the
production  of weakly-interacting massive  particles,  resulting in large \MET.

The results reported in this paper are part of a broad program of BSM searches
in events with jets and \MET, characterized by the number and
type of leptons in the final state.  
Here we describe a search for events containing opposite-sign isolated
lepton pairs ($e^+e^-$, $e^{\pm}\mu^{\mp}$, $\mu^+\mu^-$) in addition to the jets
and \MET. Results from complementary searches with different final states have 
already been reported in Ref.~\cite{ref:RA1} {\bf ADD MORE REFS HERE}.

Our analysis strategy is as follows. In order to select dilepton events, we
use high \pt\ dilepton triggers and a preselection based 
on that of the $t\bar{t}$ cross section measurement in the dilepton channel~\cite{ref:top}.
%, which used a data sample corresponding to an integrated luminosity of 3.1\pbinv.
Good agreement is found between this
data sample and predictions from SM Monte Carlo (MC) simulations in terms of the event yields
and shapes of various kinematic distributions. 
We search for a kinematic edge in the dilepton mass distribution, which is a characteristic
feature of SUSY models in which the opposite-sign leptons are produced via the decay 
$\chi_2^0 \to \chi_1^0 \ell^+\ell^-$.
Because BSM physics is expected to have large hadronic activity and \MET\ as discussed
above, we  proceed to define 2 signal regions
with requirements on these quantities to select about 0.1\% 
of dilepton $t\bar{t}$ events, as predicted by MC.
The observed event yields in the signal regions are compared with the predictions from two 
independent background estimation techniques based on data control samples, 
as well as with SM and BSM MC expectations.

%The search is not optimized in the context of any particular BSM physics, e.g. specific SUSY model.
No specific BSM physics scenario, e.g.\ a particular SUSY model, has been used to optimize the search.
In order to illustrate the sensitivity of the search, a simplified and practical model of 
SUSY breaking, the constrained minimal supersymmetric
extension of the standard model (CMSSM)~\cite{CMSSM,CMSSM2}, is used. The CMSSM is described by
five parameters: the universal scalar and gaugino mass parameters ($m_0$ and $m_{1/2}$, respectively),
the universal trilinear soft SUSY breaking parameter $A_0$, the
ratio of the vacuum expectation values of the two Higgs doublets ($\tan\beta$), and the sign of the
Higgs mixing parameter $\mu$. 
Throughout the paper, two CMSSM parameter sets, referred
to as LM1 and LM3~\cite{TDR}, are used to illustrate possible CMSSM yields. The parameter values
defining LM1 (LM3) are $m_0 = 60~(330) \GeVcc$, $m_{1/2} = 250~(240) \GeVcc$, $\tan\beta = 10~(20)\GeV$; 
both LM1 and LM3 have $A_0 = 0$ and $\mu > 0$.  These two scenarios are beyond the exclusion reach 
of previous searches performed at the Tevatron and LEP. The LM1 scenario was recently excluded 
by a search performed at CMS in events with jets and \MET~\cite{ref:RA1}. 
In this analysis, the LM1 and LM3 scenarios serve as benchmarks which 
may be used to allow comparison of the sensitivity with other analyses.


%In the Standard Model (SM), the main sources of isolated dileptons are Drell-Yan production 
%($q\bar q$ annihilation into a virtual photon or a $Z$) and $t\bar{t}$ production.
%In this search we exclude lepton pairs with invariant mass consistent with a $Z$ boson, 
%leaving $t\bar{t}$ as the dominant SM background. A separate search for BSM physics in the 
%$Z+\textrm{jets}$ sample will be described in a forthcoming paper.
