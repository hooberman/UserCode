\section{Acceptance and Efficiency Systematic Uncertainties}
\label{sec:systematics}

%As seen in Section~\ref{sec:results} there is no
%evidence for a contribution to our signal region beyond SM expectations.

The acceptance and efficiency, as well as the systematic uncertainties in these quantities, 
depend on the signal model.
For some of the individual uncertainties, it is reasonable to quote values 
based on SM control samples with kinematic properties similar to the SUSY benchmark models. 
%In other cases, where the kinematics of the signal model kinematics show large variations with respect to control 
%samples, the systematic uncertainties must be quoted model by model.
For others that depend strongly on the kinematic properties of the event, the systematic
uncertainties must be quoted model by model.

%Strictly speaking it is impossible to talk about 
%``acceptance and efficiency systematics'' because these kinds of
%systematics only apply to a well defined final state.
%Nevertheless, we can make general statements about the 
%systematic uncertainties, including quantitative
%estimates of the systematic uncertainties associated with
%a few specific processes.

The systematic uncertainty in the lepton acceptance consists
of two parts: the trigger efficiency uncertainty and the 
identification and isolation uncertainty. The trigger efficiency 
for two leptons of $\pt>10\GeVc$, with one lepton of 
$\pt>20\GeVc$ is close to 100\%.
We estimate the efficiency uncertainty to be a few percent,
mostly in the low \pt\ region, using samples of $Z \to \ell\ell$. 
For dilepton $t\bar{t}$, LM0, and LM1,
the trigger efficiency uncertainties are found to be less than 1\%.
We verify that the MC reproduces the lepton identification and isolation efficiencies in data using
samples of $Z \to \ell\ell$; the data and MC efficiencies are found to be consistent within 2\%.

%The lepton identification and isolation efficiencies in data were measured
%in samples of $Z \to \ell\ell$; they were found to be consistent with those
%from the MC within 2\%.

%The isolation efficiency depends on the jet activity in the final state.  
%For example, in MC we find that the lepton isolation efficiency differs by $\approx 4\%$
%per lepton between $Z$ events and $t\bar{t}$ events~\cite{ref:top}.

Another significant source of systematic uncertainty is 
associated with the jet and $\MET$ energy scale.  The impact
of this uncertainty is final-state dependent.  Final
states characterized by very large hadronic activity and \MET\ are 
less sensitive than final states where the \MET\ and \HT\ 
are typically close to the minimum requirements applied to these quantities.  To be more quantitative,
we have used the method of Ref.~\cite{ref:top} to evaluate
the systematic uncertainties in the acceptance for $t\bar{t}$ 
and for the two benchmark SUSY points using a 5\% uncertainty in the hadronic 
energy scale~\cite{ref:jes}.
For $t\bar{t}$ the uncertainty is 27\%; for LM0 and LM1 the 
uncertainties are 14\% and 6\%, respectively.

%The uncertainty in the parton density function (PDF) for CTEQ6.6~\cite{Nadolsky:2008fk} 
%is estimated from  the  envelope  provided  by  the  CTEQ6.6  error  function.
%This gives uncertainties of X\%, X\% and X\% for $t\bar{t}$, LM0 and LM1, respectively.

The uncertainty in the integrated luminosity is 11\%~\cite{ref:lumi}.



