\section{Additional Information for Model Testing}
\label{sec:outreach}
{\bf SOME OF THE NUMBERS BELOW WILL BE UPDATED SHORTLY}
Other models of new physics in the dilepton final state can be confronted in an approximate way by simple 
generator-level studies that compare the expected number of events in \lumifinal\
with the upper limits from Section~\ref{sec:limit}.
The key ingredients of such studies are the kinematic requirements described 
in this note, the lepton efficiencies, and the detector responses for \HT\ and \MET.
%
The muon identification efficiency is $\approx 96\%$;
the electron identification efficiency varies approximately linearly from $\approx$ 60\% at 
$\pt = 10\GeVc$ to 90\% for $\pt > 30\GeVc$.  
%
The lepton isolation efficiency depends on the lepton momentum, as well as on the jet activity in the 
event.
In $t\bar{t}$ events, it varies approximately linearly from $\approx 73\%$ (muons)
and $\approx 82\%$ (electrons) at $\pt=10\GeVc$ to $\approx 97\%$ for $\pt>60\GeVc$. 
In LM1 (LM3) events, this efficiency is decreased by $\approx$5--10\% ($\approx$10\%)over the whole momentum spectrum.
%
The average detector responses (the reconstructed quantity divided by the generated quantity) 
for \HT\ and \MET\ are consistent with 1 within the 5\% jet energy scale uncertainty.
The experimental resolutions on these quantities are 9\% and 12\%, respectively.



