
\section{Model-Dependent Limits}
\label{sec:models}

As an example of how the upper limit presented in Sec.~\ref{sec:upperlimit} can be
used to test if a specific model is excluded, we consider the benchmark
SUSY processes LM4 and LM8 which contain
$Z$ bosons produced in the cascade decays of SUSY particles. 
We place upper limits on the quantity \sta,
assuming efficiencies and uncertainties from these processes, and compare them to 
the expected values of \sta.
Here $\sigma$ is the signal production cross section, %$BF$ is the branching fraction
%to the final state Z + jets + MET and $A$ is the signal acceptance.
and $A$ is the signal acceptance which includes the branching fraction to the final state Z + $\ge$ 2 jets + MET $\ge$ \signalmett.

The signal event yield $N_{SIG}$ can be expressed as:

\begin{equation}
%N_{SIG} = \sigma \times BF \times A \times \epsilon \times \mathcal{L},
N_{SIG} = \sigma \times A \times \epsilon \times \mathcal{L},
\end{equation}

and we therefore have:

\begin{equation}
N_{SIG}/( \epsilon \times \mathcal{L}) = \sigma \times A.
\end{equation}

Here  $\epsilon$ is the signal efficiency and $\mathcal{L}$ is the integrated luminosity. 
Since we wish to place an upper limit on the quantity %$\sigma \times BF \times A$
\sta, we must evaluate
the quantity $N_{SIG}/(\epsilon \times \mathcal{L})$.  Because of the efficiency
in the denominator, this upper limit cannot be calculated in an entirely model-independent
way; rather, it must be calculated with respect to a specific model. We therefore evaluate
the upper limit on %$\sigma \times BF \times A$ 
\sta\
with respect to LM4 and LM8.  
For each process, we calculate the efficiency as discussed
%in the following subsections.
in the next subsection, and
its uncertainty as described in section \ref{sec:signaleffuncer}.


\subsection{Signal Efficiencies}
\label{sec:sigeff}

We evaluate the signal efficiencies using MC. To evaluate these efficiencies for our 
sample processes, 
we take as our denominator the number of generated events which pass the following selection at
the generator level:

\begin{itemize}
\item Two electrons or muons with $p_T>20$~GeV and $|\eta|<2.5$
\item Opposite-sign, same-flavor pair with $81 < M(ll) < 101$~GeV
\item At least two genjets with $p_T > 30$~GeV, $|\eta|<3$, $\Delta R > 0.4$ from any selected lepton as defined above
\end{itemize}

For the %loose (tight) 
signal region efficiency,
we add to this the requirement genmet
%$>60$ (120)~GeV.
$>$ \signalmetl (\signalmett) for the loose (tight) signal regions.
We take as our numerator the number of events passing the above generator-level selection which
also pass the signal region selection at reco-level. 
Efficiencies are shown in table \ref{tab:models}.

%{\bf Should I also consider events passing reco-selection which do NOT pass gen-level selection? For ttbar this
%\bf is a large effect, it is small for LM4 and LM8. }
%For the loose (tight) signal region we find efficiencies of 43\% (42\%), 54\% (55\% )and 45\% (48\%) 
%for $t\bar{t}$, LM4 and LM8, respectively.
%This efficiency is dominated by the dilepton selection efficiency, which varies from $\sim0.6-0.7$ for the chosen 
%processes. 

%The efficiency for selected dileptons to satisfy the dilepton mass requirement is $\sim0.9$;
%the efficiency to satisfy the requirements on the jet multiplicity is $\sim0.8-0.9$, for the MET$>60$~GeV requirement
%this efficiency is $\sim0.9-1$. 



\subsection{Upper Limits on \sta}

Armed with the efficiencies and uncertainties calculated in the previous 2 sections, we proceed
to calculate upper limits on the quantity \sta.
We calculate Bayesian 95\% CL upper
limits using the cl95cms software, assuming a log-normal model of nuissance parameter integration.
We also calculate the quantity \sta ~for LM4 and LM8, for which we assume LO
cross-sections of 1.88~pb and 0.73~pb and calculate k-factors for each event, depending on the sub-process.
The quantity $A$ is taken to be the fraction of the total number of generated events which
pass the generator-level selection given in Sec.~\ref{sec:sigeff}. The results are summarized in
Table~\ref{tab:models}. 
These results show that the benchmark point LM4 is excluded, while
LM8 remains beyond the
sensitivity of this search with the current integrated luminosity.

\begin{table}[hbt]
  \begin{center}
	\caption{
	  \label{tab:models} 
	  Summary of efficiencies, efficiency uncertainties (quadrature sum
	  of jet/MET, dilepton and luminosity uncertainties), 
	  and upper limits on \sta\ %$\sigma \times BF \times A$ 
	  for the tight (MET $>$ \signalmett~GeV, top)
	  and loose (MET $>$ \signalmetl~GeV, bottom) signal regions.
	  We also show the quantity
	  \sta\ %$\sigma \times BF \times A$ 
	  for LM4 and LM8.}

	\medskip
	\begin{tabular}{l|cccc}
	  \hline
	  \bf MET $>$ \signalmett~GeV & efficiency (\%) & acceptance (\%) & UL(\sta)(fb) & \sta(fb) \\
	  \hline
          %BAYES
	  %LM4  &  50 $\pm$   6  &  0.84  &  14  &  23  \\
	  %LM8  &  43 $\pm$   5  &  0.98  &  16  &  11  \\
          %CLs
	  LM4  &  50 $\pm$   6  &  0.84  &  13  &  23  \\
	  LM8  &  43 $\pm$   5  &  0.98  &  15  &  11  \\

	  \hline
	\end{tabular}

	\medskip
	\begin{tabular}{l|cccc}
	  \hline
	  \bf MET $>$ \signalmetl~GeV & efficiency (\%) & acceptance (\%) & UL(\sta)(fb) & \sta(fb) \\
	  \hline
          %BAYES
	  %LM4 &  53 $\pm$   3  &  1.4  &   40  &   37  \\
	  %LM8 &  44 $\pm$   3  &  1.7  &   47  &   19  \\
          %CLs
	  LM4 &  53 $\pm$   3  &  1.4  &   39  &   37  \\
	  LM8 &  44 $\pm$   3  &  1.7  &   47  &   19  \\

	  \hline
	\end{tabular}

	%%%%%%%%%%%%%%%%%%%%%%%%%%%%%%%%%%%%%%%%%%%%%%
	%2010 results--35/pb
	\begin{comment}

	  \begin{tabular}{l|ccc}

		\hline
		& $t\bar{t}$    & LM4     & LM8       \\
		\hline
			{\bf Loose signal region}                &               &         &           \\
			Efficiency                               &       0.43    & 0.54    & 0.45      \\
			Efficiency Uncertainty                   &       0.15    & 0.12    & 0.12      \\
			UL($\sigma \times BF \times A$) (pb)     &       0.56    & 0.44    & 0.53      \\
			$\sigma \times BF \times A$ (pb)         &               & 0.045   & 0.025     \\

			\hline
				{\bf Tight signal region}                &               &         &           \\
				Efficiency                               &       0.42    &  0.55   & 0.48      \\
				Efficiency Uncertainty                   &       0.26    &  0.13   & 0.13      \\
				UL($\sigma \times BF \times A$) (pb)     &       0.23    &  0.17   & 0.19      \\
				$\sigma \times BF \times A$ (pb)         &               &  0.035  & 0.018     \\
				
				\hline

	  \end{tabular}

	\end{comment}

  \end{center}
\end{table}






%Double_t CL95(Double_t ilum, Double_t slum, Double_t eff, Double_t seff, 
%              Double_t bck, Double_t sbck, Int_t n, Bool_t gauss = kFALSE, Int_t nuisanceModel = 0)

%---------
%ttbar
%---------

%root [5] CL95( 34 , 0. , 0.43 , 0.15*0.43 , 6.04 , 0.96 , 7 , false , 1 )    
%Poisson 95% CL limit with Lognormal nuisance parameter integration will be used
%Likelihood function is evaluated over [0,2] 
%likelihood normalization: 0.0498808
%Info in <TCanvas::Print>: eps file Likelihood.eps has been created
%Upper 95% C.L. limit on signal = 0.561523 pb
%(Double_t)5.61523437500000000e-01

%---------
%LM4
%---------

%root [6] CL95( 34 , 0. , 0.54 , 0.12*0.54 , 6.04 , 0.96 , 7 , false , 1 )    
%Poisson 95% CL limit with Lognormal nuisance parameter integration will be used
%Likelihood function is evaluated over [0,2] 
%likelihood normalization: 0.0395872
%Info in <TCanvas::Print>: eps file Likelihood.eps has been created
%Upper 95% C.L. limit on signal = 0.44043 pb
%(Double_t)4.40429687500000000e-01

%---------
%LM8
%---------

%root [7] CL95( 34 , 0. , 0.45 , 0.12*0.45 , 6.04 , 0.96 , 7 , false , 1 )
%Poisson 95% CL limit with Lognormal nuisance parameter integration will be used
%Likelihood function is evaluated over [0,2] 
%likelihood normalization: 0.0475047
%Info in <TCanvas::Print>: eps file Likelihood.eps has been created
%Upper 95% C.L. limit on signal = 0.52832 pb
%(Double_t)5.28320312500000000e-01
