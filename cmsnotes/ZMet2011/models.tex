\section{Model-Dependent Limits}
\label{sec:models}

As an example of how the upper limit presented in Sec.~\ref{sec:upperlimit} can be
used to test if a specific model is excluded, in this section we consider the benchmark
SUSY processes LM4 and LM8, which contain
$Z$ bosons produced in the cascade decays of SUSY particles. 
We place upper limits on the quantity $\sigma \times BF \times A$,
assuming efficiencies and uncertainties from these processes, and compare them to 
the expected values of $\sigma \times BF \times A$.
Here $\sigma$ is the signal production cross section, $BF$ is the branching fraction
to the final state Z + jets + MET and $A$ is the signal acceptance.
%{\bf I am showing ttbar below, in addition to LM4 and LM8, but perhaps this should be removed...}

The signal event yield $N_{SIG}$ can be expressed as:

\begin{equation}
N_{SIG} = \sigma \times BF \times A \times \epsilon \times \mathcal{L},
\end{equation}

and we therefore have:

\begin{equation}
N_{SIG}/( \epsilon \times \mathcal{L}) = \sigma \times BF \times A.
\end{equation}

Here  $\epsilon$ is the signal efficiency and $\mathcal{L}$ is the integrated luminosity. 
Since we wish to place an upper limit on the quantity $\sigma \times BF \times A$, we must evaluate
the quantity $N_{SIG}/(\epsilon \times \mathcal{L})$.  Because of the efficiency
in the denominator, this upper limit cannot be calculated in an entirely model-independent
way; rather, it must be calculated with respect to a specific model. We therefore evaluate
the upper limit on $\sigma \times BF \times A$ with respect to $t\bar{t}$, LM4 and LM8.  
For each process, we must calculate both the efficiency and its uncertainty, as discussed
in the following subsections.



\subsection{Signal Efficiencies}
\label{sec:sigeff}

We evaluate the signal efficiencies using MC. To evaluate these efficiencies for our sample processes, 
we take as our denominator the number of generated events which pass the following selection at
the generator level:

\begin{itemize}
\item 2 electrons or muons with $p_T>20$~GeV and $|\eta|<2.5$,
\item Opposite-sign, same-flavor pair with $81 < M(ll) < 101$~GeV,
\item Count genjets with $p_T > 30$~GeV, $|\eta|<2.5$, $\Delta R > 0.4$ from any selected lepton as defined above,
\item At least 2 genjets.
\end{itemize}

For the loose (tight) signal region efficiency, we add to this the requirement genmet$>60$ (120)~GeV.
We take as our numerator the number of events passing the above generator-level selection which
also pass the signal region selection at reco-level. 
%{\bf Should I also consider events passing reco-selection which do NOT pass gen-level selection? For ttbar this
%\bf is a large effect, it is small for LM4 and LM8. }
For the loose (tight) signal region we find efficiencies of 43\% (42\%), 54\% (55\% )and 45\% (48\%) 
for $t\bar{t}$, LM4 and LM8, respectively.
This efficiency is dominated by the dilepton selection efficiency, which varies from $\sim0.6-0.7$ for the chosen 
processes. 

%The efficiency for selected dileptons to satisfy the dilepton mass requirement is $\sim0.9$;
%the efficiency to satisfy the requirements on the jet multiplicity is $\sim0.8-0.9$, for the MET$>60$~GeV requirement
%this efficiency is $\sim0.9-1$. 


%%%%%%%%%%%%%%%%%%%%%%%%%%%%%%%%%%%%%%%%%%%%%%%
%this section has been copied into systematics.tex


\subsection{Signal Efficiency Uncertainties}

Here we assess systematic uncertainties in the signal efficiencies for our sample processes, from
jet/MET, lepton identification, and luminosity uncertainties. 

\begin{itemize}
  
\item Jets and MET selection efficiency: we assess this uncertainty by varying the hadronic energy scale by $\pm 5\%$
  following the procedure used in the ttdilepton cross-section measurement.  For the loose (tight) signal regions 
  we find uncertainties of 9\% (23\%), 2\% (4\%), and 2\%(4\%) for  $t\bar{t}$, LM4 and LM8, respectively.
\item Lepton ID and isolation efficiencies: we perform a tag-and-probe technique
  on Z data and MC and find that the simulation agrees with data within about 2\% as shown in Table~\ref{tab:tagandprobe}.
  We apply a conservative 5\% uncertainty on the dilepton selection efficiency.
\item Trigger efficiency: the trigger efficiency is very close to 1 since there are 2 leptons with \pt $> 20$ GeV.
  We apply the simplified model of the trigger efficiency documented in~\cite{ref:GenericOS} and assess the uncertainty
  as the relative difference in the yields between assuming 100\% trigger efficiency vs. using the trigger model. This
  procedure gives differences of less than 1\% and the trigger efficiency uncertainty is therefore regarded as negligible.
\item Luminosity: we assess an uncertainty of 11\%.

\end{itemize}

\begin{table}[hbt]
\begin{center}
\caption{\label{tab:tagandprobe} Tag and probe results on $Z \to \ell \ell$
on data and MC.  We quote ID efficiency given isolation and 
the isolation efficiency given ID. }
\begin{tabular}{|l||c|c|}
\hline
                       & Data  T\&P        & MC T\&P               \\  
\hline
$\epsilon(id|iso)$ els & 0.918 $\pm$ 0.003 & 0.936 $\pm$ 0.0038    \\ 
$\epsilon(iso|id)$ els & 0.986 $\pm$ 0.001 & 0.987 $\pm$ 0.0018    \\
$\epsilon(id|iso)$ mus & 0.962 $\pm$ 0.002 & 0.964 $\pm$ 0.0027    \\
$\epsilon(iso|id)$ mus & 0.987 $\pm$ 0.001 & 0.984 $\pm$ 0.0018    \\
\hline

\end{tabular}
\end{center}
\end{table}

\subsection{Upper Limits on $\sigma \times BF \times A$}

Armed with the efficiencies and uncertainties calculated in the previous 2 sections, we proceed
to calculate upper limits on the quantity $\sigma \times BF \times A$. We calculate Bayesian 95\% CL upper
limits using the cl95cms software, assuming a log-normal model of nuissance parameter integration.
We also calculate the quantity $\sigma \times BF \times A$ for LM4 and LM8, for which we assume LO
cross-sections of 1.88~pb and 0.73~pb and calculate k-factors for each event, depending on the sub-process.
The quantity $BF \times A$ is taken to be the fraction of the total number of generated events which
pass the generator-level selection given in Sec.~\ref{sec:sigeff}. The results are summarized in
Table~\ref{tab:models}. These results show that the LM4 and LM8 benchmark points are beyond the
sensitivity of this search with the current integrated luminosity.

\begin{table}[hbt]
\begin{center}
\caption{\label{tab:models} Summary of efficiencies, efficiency uncertainties (quadrature sum
of jet/MET, dilepton and luminosity uncertainties), and upper limits on $\sigma \times BF \times A$ 
for the loose (MET$>60$~GeV) and tight (MET$>120$~GeV) signal regions. We also show the quantity
$\sigma \times BF \times A$ for LM4 and LM8.}
\begin{tabular}{l|ccc}

\hline
                                         & $t\bar{t}$    & LM4     & LM8       \\
\hline
{\bf Loose signal region}                &               &         &           \\
Efficiency                               &       0.43    & 0.54    & 0.45      \\
Efficiency Uncertainty                   &       0.15    & 0.12    & 0.12      \\
UL($\sigma \times BF \times A$) (pb)     &       0.56    & 0.44    & 0.53      \\
$\sigma \times BF \times A$ (pb)         &               & 0.045   & 0.025     \\

\hline
{\bf Tight signal region}                &               &         &           \\
Efficiency                               &       0.42    &  0.55   & 0.48      \\
Efficiency Uncertainty                   &       0.26    &  0.13   & 0.13      \\
UL($\sigma \times BF \times A$) (pb)     &       0.23    &  0.17   & 0.19      \\
$\sigma \times BF \times A$ (pb)         &               &  0.035  & 0.018     \\
 
\hline

\end{tabular}
\end{center}
\end{table}






%Double_t CL95(Double_t ilum, Double_t slum, Double_t eff, Double_t seff, 
%              Double_t bck, Double_t sbck, Int_t n, Bool_t gauss = kFALSE, Int_t nuisanceModel = 0)

%---------
%ttbar
%---------

%root [5] CL95( 34 , 0. , 0.43 , 0.15*0.43 , 6.04 , 0.96 , 7 , false , 1 )    
%Poisson 95% CL limit with Lognormal nuisance parameter integration will be used
%Likelihood function is evaluated over [0,2] 
%likelihood normalization: 0.0498808
%Info in <TCanvas::Print>: eps file Likelihood.eps has been created
%Upper 95% C.L. limit on signal = 0.561523 pb
%(Double_t)5.61523437500000000e-01

%---------
%LM4
%---------

%root [6] CL95( 34 , 0. , 0.54 , 0.12*0.54 , 6.04 , 0.96 , 7 , false , 1 )    
%Poisson 95% CL limit with Lognormal nuisance parameter integration will be used
%Likelihood function is evaluated over [0,2] 
%likelihood normalization: 0.0395872
%Info in <TCanvas::Print>: eps file Likelihood.eps has been created
%Upper 95% C.L. limit on signal = 0.44043 pb
%(Double_t)4.40429687500000000e-01

%---------
%LM8
%---------

%root [7] CL95( 34 , 0. , 0.45 , 0.12*0.45 , 6.04 , 0.96 , 7 , false , 1 )
%Poisson 95% CL limit with Lognormal nuisance parameter integration will be used
%Likelihood function is evaluated over [0,2] 
%likelihood normalization: 0.0475047
%Info in <TCanvas::Print>: eps file Likelihood.eps has been created
%Upper 95% C.L. limit on signal = 0.52832 pb
%(Double_t)5.28320312500000000e-01
