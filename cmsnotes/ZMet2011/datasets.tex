
\section{Datasets}
\label{sec:datasets}

\subsection{Datasets}

%We use a combination of rereco and prompt reco for both leptons and photons.
We use the May 10 ReReco data for both leptons and photons.
\\
For selecting the dilepton sample, the following datasets are used (the pythia DY samples are used only for generator level Z mass values less than 50 to avoid overlap with the madgraph DYJets sample):


\begin{itemize}
\item Data 
\begin{itemize}
\item \verb=/DoubleElectron/Run2011A-May10ReReco-v1/AOD=
\item \verb=/DoubleMu/Run2011A-May10ReReco-v1/AOD=
\item \verb=/MuEG/Run2011A-May10ReReco-v1/AOD=

%we do not plan to use these
%\item \verb=/DoubleElectron/Run2011A-PromptReco-v4/AOD=
%\item \verb=/DoubleMu/Run2011A-PromptReco-v4/AOD=
%\item \verb=/MuEG/Run2011A-PromptReco-v4/AOD=
\end{itemize}

\item Monte Carlo
\begin{itemize} 
\item \verb=/DYJetsToLL_TuneD6T_M-50_7TeV-madgraph-tauola/Spring11-PU_S1_START311_V1G1-v1/AODSIM=
\item \verb=/TTJets_TuneZ2_7TeV-madgraph-tauola/Spring11-PU_S1_START311_V1G1-v1/AODSIM=
\item \verb=/WJetsToLNu_TuneZ2_7TeV-madgraph-tauola/Spring11-PU_S1_START311_V1G1-v1/AODSIM=
\item \verb=/WWTo2L2Nu_TuneZ2_7TeV-pythia6/Spring11-PU_S1_START311_V1G1-v1/AODSIM=
\item \verb=/WZtoAnything_TuneZ2_7TeV-pythia6-tauola/Spring11-PU_S1_START311_V1G1-v1/AODSIM=
\item \verb=/ZZtoAnything_TuneZ2_7TeV-pythia6-tauola/Spring11-PU_S1_START311_V1G1-v1/AODSIM=
\item \verb=/TToBLNu_TuneZ2_s-channel_7TeV-madgraph/Spring11-PU_S1_START311_V1G1-v1/AODSIM=
\item \verb=/TToBLNu_TuneZ2_t-channel_7TeV-madgraph/Spring11-PU_S1_START311_V1G1-v1/AODSIM=
\item \verb=/TToBLNu_TuneZ2_tW-channel_7TeV-madgraph/Spring11-PU_S1_START311_V1G1-v1/AODSIM=
\item Pythia samples:
\item \verb=/DYToEE_M-20_CT10_TuneZ2_7TeV-powheg-pythia/Spring11-PU_S1_START311_V1G1-v1/AODSIM=
\item \verb=/DYToMuMu_M-20_CT10_TuneZ2_7TeV-powheg-pythia/Spring11-PU_S1_START311_V1G1-v1/AODSIM=
\item \verb=/DYToTauTau_M-20_CT10_TuneZ2_7TeV-powheg-pythia-tauola/Spring11-PU_S1_START311_V1G1-v1/AODSIM=
\item \verb=/DYToEE_M-10To20_TuneZ2_7TeV-pythia6/Spring11-PU_S1_START311_V1G1-v1/AODSIM=
\item \verb=/DYToMuMu_M-10To20_TuneZ2_7TeV-pythia6/Spring11-PU_S1_START311_V1G1-v1/AODSIM=
\end{itemize}
\end{itemize}

For the creation of photon templates, we use:

\begin{itemize}
\item \verb=/Photon/Run2011A-May10ReReco-v1/AOD=
%we do not plan to use these
%\item \verb=/Photon/Run2011A-PromptReco-v4/AOD=
%\item \verb==
\end{itemize}

The integrated luminosity used corresponds to \lumi, and the JSON used is 
the official May 10 ReReco JSON:
\\
Cert\_160404-163869\_7TeV\_May10ReReco\_Collisions11\_JSON.txt
%Cert_160404-163869_7TeV_May10ReReco_Collisions11_JSON.txt

%last year
%the official json PVT v3: 
%Cert\_132440-149442\_7TeV\_StreamExpress\_Collisions10\_JSON\_v3.txt