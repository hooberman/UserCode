
\section{Introduction}

In this note we describe a search for new physics in the 2010 
opposite sign isolated dilepton sample ($ee$, $e\mu$, and $\mu\mu$).  
The main sources of high \pt isolated dileptons at CMS are Drell Yan and $t\bar{t}$.
Here we concentrate on dileptons with invariant mass consistent
with $Z \to ee$ and $Z \to \mu\mu$.  A separate search for new physics in the non-\Z
sample is described in~\cite{ref:GenericOS}.

We search for new physics in the final state of \Z plus two or more jets plus missing transverse energy (MET). We reconstruct the \Z boson
in its decay to $e^+e^-$ or $\mu^+\mu^-$. Our search regions are defined as MET $\ge$ 60~GeV (loose signal region) and MET $\ge$ 120~GeV (tight signal region), and two or more jets. We use data driven techniques to predict the
standard model background in this search region. 
Contributions from Drell-Yan production combined with detector mis-measurements that produce fake MET are modeled via MET templates based on photon plus jets events. 
Top pair production backgrounds, as well as other backgrounds for which the lepton
flavors are uncorrelated such as $VV$ and DY$\rightarrow\tau\tau$, are modeled via $e^\pm\mu^\mp$ subtraction.

As leptonically decaying \Z bosons is a signature that has very little background, they provide a clean final state in which to search for new physics. 
Because new physics is expected to be connected to the Standard Model Electroweak sector, it is likely that new particles will couple to W and Z bosons. 
For example, in mSUGRA, low $M_{1/2}$ can lead to a significant branching fraction for $\chi_2^0 \rightarrow Z \chi_1^0$. 
In addition, we are motivated by the existence of dark matter to search for new physics with MET.
Enhanced MET is a feature of many new physics scenarios, and R-parity conserving SUSY again provides a popular example. The main challenge of this search is therefore to 
understand the tail of the fake MET distribution in \Z plus jets events.

The basic idea of the MET template method~\cite{ref:templates1}\cite{ref:templates2} is to measure the MET distribution in a control sample which has no true MET and a similar topology to the signal events. 
In our case, we choose a photon sample with two or more jets as the control sample.
Both the control sample and signal sample consist of a well measured object (either a photon or a leptonically decaying $Z$), which recoils against a system of hadronic jets. 
In both cases, the instrumental MET is dominated by mismeasurements of the hadronic system. 

This note is organized as follows. 
In Sections~\ref{sec:datasets} and ~\ref{sec:trigSel} we start by describing 
the triggers and datasets used, followed by the detailed object definitions (electrons, muons, photons,
jets, MET) and event selection which is described in Section~\ref{sec:eventSelection}. 
We define a preselection and compare data vs. MC yields passing this preselection in Section~\ref{sec:yields}.
We then define the signal regions and show the number of observed events and MC expected yields in Section~\ref{sec:sigregion}.
Section~\ref{sec:templates} then introduces the MET template method and discusses its derivation 
in some detail, followed by a demonstration in Section~\ref{sec:mc} that the method works in Monte Carlo.
Section~\ref{sec:topbkg} introduces the top background estimate based on opposite flavor subtraction, and contributions from other backgrounds are discussed in Section~\ref{sec:othBG}.
Section~\ref{sec:results} shows the results for applying these methods in data.
We analyze the systematic uncertainties in the background prediction in Section~\ref{sec:systematics} 
and proceed to calculate an upper limit on the non SM contributions to our signal regions in Section~\ref{sec:upperlimit}. In Section~\ref{sec:models} we calculate upper limits on the quantity $\sigma \times BF \times A$, assuming
efficiencies and uncertainties from sample benchmark SUSY processes. We conclude in Section~\ref{sec:conclusion}.