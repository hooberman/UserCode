\section{Acceptance Systematics}

Strictly speaking it is impossible to talk about 
``acceptance and efficiency systematics'' because these kinds of
systematics only apply to a well defined final state.
Nevertheless, we can make general statements about the 
systematic uncertainties, including quantitative
estimates of the systematic uncertainties associated with
a few specific processes. 
\begin{itemize}
  
\item Jets and MET selection efficiency: we assess this uncertainty by varying the hadronic energy scale by $\pm 5\%$ using
  following the procedure used in the ttdilepton cross-section measurement.   {\bf NEED TO EVALUATE THIS}
\item Lepton ID and isolation efficiencies: we perform a tag-and-probe technique
  on Z data and MC and find that the simulation agrees with data within about 2\%.  {\bf NEED TO VERIFY THIS}
\item Trigger efficiency: the trigger efficiency is very close to 1 since there are 2 leptons with \pt $> 20$ GeV.
  We apply a simplified model of the trigger efficiency as documented in App.~\ref{sec:trgEff} and assess the uncertainty
  as the relative difference in the yields between assuming 100\% trigger efficiency vs. using the trigger model. This
  procedure gives differences of less than 1\%: the trigger efficiency uncertainty is therefore regarded as negligible.
  {\bf NEED TO VERIFY THIS}
\item Dilepton mass requirement: {\bf NEED TO THINK ABOUT HOW TO DO THIS}
\item {\bf IS ANYTHING MISSING FROM THIS LIST}

\end{itemize}

\begin{table}[hbt]
\begin{center}
\caption{\label{tab:tagandprobe} Tag and probe results on $Z \to \ell \ell$
on data and MC.  We quote ID efficiency given isolation and 
the isolation efficiency given ID. }
\begin{tabular}{|l||c|c|}
\hline
                             & Data  T\&P      & MC T\&P             \\  
\hline
$\epsilon(id|iso)$ electrons & $0.X \pm 0.X$ & $0.X \pm 0.X$ \\
$\epsilon(iso|id)$ electrons & $0.X \pm 0.X$ & $0.X \pm 0.X$ \\
$\epsilon(id|iso)$ muons     & $0.X \pm 0.X$ & $0.X \pm 0.X$ \\
$\epsilon(iso|id)$ muons     & $0.X \pm 0.X$ & $0.X \pm 0.X$ \\ 
\hline
\end{tabular}
\end{center}
\end{table}