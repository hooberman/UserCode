\documentclass{cmspaper}
\usepackage{color}
\usepackage{graphicx}
\usepackage{wrapfig}
\usepackage{slashbox}
\usepackage{epstopdf}  %added for MAC compiler
\usepackage{pdfpages}
\usepackage{lineno}
\usepackage{verbatim}
\RequirePackage{lineno} 
%\RequirePackage{lineno} 

\usepackage{wrapfig}

%Note to authors: these definitions are suggested to be used for consistency
\newcommand{\pt}{$p_T$ }
\newcommand{\mll}{\ensuremath{M_{\ell\ell}}}
\newcommand{\z}{$Z$ }
\newcommand{\Z}{$Z$ } %redundancy can be good
\newcommand{\ttbar}{\ensuremath{t\bar{t}} }
\newcommand{\eepm}{\ensuremath{e^+ e^-}}
\newcommand{\mmpm}{\ensuremath{\mu^+ \mu^-}}
\newcommand{\empm}{\ensuremath{e^\pm \mu^\mp}}
\newcommand{\ase}[2]{\ensuremath{_{~- #1}^{~+ #2}}}
\newcommand{\met}{\mbox{$\raisebox{.3ex}{$\not$}E_T$\hspace*{0.5ex}}} 

%The lumi so that if it changes it's easier to update
\newcommand{\lumi}{204~pb$^{-1}$}

%loose(tight) signal region met cut
\newcommand{\signalmetl}{100}
\newcommand{\signalmett}{200}


\begin{document}

\begin{titlepage}

  %\internalnote{2011/000}
  \cmsnote{2011/000}

  \date{\today}
 
  \title{A Search For New Physics in Z + Jets + MET using MET Templates}

  \begin{Authlist}
    D.~Barge, C.~Campagnari, P.~Kalavase, D.~Kovalskyi, V.~Krutelyov, J.~Ribnik
    \Instfoot{ucsb}{University of California, Santa Barbara, USA}

    W.~Andrews, G.~Cerati, D.~Evans, F.~Golf, I.~MacNeill, S.~Padhi, Y.~Tu, F.~W\"urthwein, A.~Yagil, J.~Yoo
    \Instfoot{ucsd}{University of California, San Diego, USA}

	L.~Bauerdick, I.~Bloch, K.~Burkett, I.~Fisk, Y.~Gao, O.~Gutsche, B.~Hooberman, S.~Jindariani, J.~Linacre
    \Instfoot{fnal}{Fermi National Accelerator Laboratory, Batavia, Illinois, USA}
  \end{Authlist}

  \begin{abstract}

We search for new physics in the dilepton final state of \Z plus two or more jets plus missing transverse 
energy (MET) in the $\sqrt{s}$ = 7~TeV data in 2011 (\lumi). 
The \Z boson is reconstructed in its decay to $e^+e^-$ or $\mu^+\mu^-$, and
the search regions are defined as MET $\ge$ \signalmetl~GeV (loose signal region) and 
MET $\ge$ \signalmett~GeV (tight signal region). 
We use data driven techniques to predict the standard model background in these
search regions. 
Contributions from Drell-Yan production combined with detector mis-measurements that produce 
fake MET are modeled via MET templates.
% based on photon plus jets events. 
%omit the photon vs qcd decision
Top pair production background, as well as other backgrounds for which the lepton
flavors are uncorrelated, are modeled via $e^\pm\mu^\mp$ subtraction.


  \end{abstract}
\end{titlepage}


\setcounter{page}{2}%JPP
\newpage
\tableofcontents

\newpage
\linenumbers

\section{Introduction}

In this note we describe a search for physics beyond the standard model (BSM) 
in a sample of proton-proton collisions at a centre-of-mass energy of 7~TeV. 
The data sample was collected with the Compact Muon Solenoid (CMS) detector~\cite{JINST} at 
the Large Hadron Collider (LHC) in 2011
and corresponds to an integrated luminosity of \lumi.
We search for new physics in events containing
opposite sign isolated lepton pairs ($ee$, $e\mu$, and $\mu\mu$).
The main sources of high \pt\ isolated dileptons at CMS are Drell Yan and \ttbar.
Here we concentrate on dileptons with invariant mass consistent
with $Z \to ee$ and $Z \to \mu\mu$.  A separate search for new physics in the non-\Z
sample is described in~\cite{ref:ospaper}.

We search for new physics in the final state of \Z plus two or more jets plus missing 
transverse energy (MET). We reconstruct the \Z boson
in its decay to $e^+e^-$ or $\mu^+\mu^-$. Our signal regions are defined as 
MET $>$ \signalmetl~GeV (loose signal region) and MET $>$ \signalmett~GeV 
(tight signal region). We use data driven techniques to predict the
standard model (SM) backgrounds in these search regions. 
Contributions from Drell-Yan production combined with detector mis-measurements that 
produce fake MET are modeled via the MET templates  method~\cite{ref:templates1,ref:templates2}
based on control samples of photon plus jets and QCD multijets events.
Top pair production backgrounds, as well as other backgrounds for which the lepton
flavors are uncorrelated such as $W^+W^-$ and DY$\rightarrow\tau^+\tau^-$, are 
modeled via $e^\pm\mu^\mp$ subtraction.

As leptonically decaying \Z bosons provide a signature that has very little background, 
they provide a clean final state in which to search for new physics. 
Because new physics is expected to be connected to the SM Electroweak sector, 
it is likely that new particles will couple to W and Z bosons. 
For example, in mSUGRA, low $M_{1/2}$ can lead to a significant branching fraction 
for $\chi_2^0 \rightarrow Z \chi_1^0$. 
In addition, we are motivated by the existence of dark matter to search for new physics with MET.
Enhanced MET is a feature of many new physics scenarios, and R-parity conserving SUSY 
again provides a popular example. The main challenge of this search is therefore to 
understand the tail of the fake MET distribution in \Z plus jets events.

%The search is not optimized in the context of any particular BSM physics, e.g. specific SUSY model.
No specific BSM  physics scenario, e.g.\ a particular  SUSY model, has
been  used  to  optimize  the  search.  In  order  to  illustrate  the
sensitivity of  the search, a  simplified and practical model  of SUSY
breaking,  the  constrained minimal  supersymmetric  extension of  the
standard  model  (CMSSM)~\cite{CMSSM,CMSSM2}, is  used.  The CMSSM  is
described by  five parameters: the  universal scalar and  gaugino mass
parameters   ($m_0$  and   $m_{1/2}$,  respectively),   the  universal
trilinear soft SUSY breaking parameter  $A_0$, the ratio of the vacuum
expectation values  of the two  Higgs doublets ($\tan\beta$),  and the
sign of  the Higgs mixing  parameter $\mu$.  Throughout the  note, two
CMSSM parameter sets, referred to  as LM4 and LM8~\cite{TDR}, are used
to  illustrate possible  CMSSM yields.   In these  scenarios, opposite
sign  leptons are  produced  in  the decays  of $Z$  bosons, which  are
produced  in  the  cascade  decays  of heavy,  colored  objects.   The
parameter  values defining  LM4  (LM8) are  $m_0  = 210~(500)$\GeVcc,
$m_{1/2} = 285~(300) \GeVcc$,  $A_0 = 0~(-300)\GeV$; both LM4 and
LM8 have $\tan\beta = 10$ and $\mu  > 0$. 
In this analysis, the  LM4 and LM8 scenarios serve as benchmarks
which may  be used to allow  comparison of the  sensitivity with other
analyses.


The datasets used for this analysis are summarized in 
Tables~\ref{tab:DatasetsData} (data) and~\ref{tab:DatasetsMC} (MC).
The total integrated luminosity is \lumi after
applying the official good run list. The main Monte Carlo
samples are generated with Madgraph, though samples with
alternative generators such as Powheg and MC@NLO are also used for
the derivation of systematic uncertainties in the \ttbar~background prediction. 
The triggers used to select both the signal and control samples are
also summarized in Table.~\ref{tab:TrigData}.

\begin{table}[!ht]
\begin{center}
\begin{tabular}{l}
\hline
Dataset Name   \\
\hline
\hline
Single Lepton Samples \\
\hline
/ElectronHad/Run2011A-May10ReReco-v1/AOD \\
/SingleMu/Run2011A-May10ReReco-v1/AOD \\
/ElectronHad/Run2011A-05Aug2011-v1/AOD \\
/SingleMu/Run2011A-05Aug2011-v1/AOD \\
/ElectronHad/Run2011A-PromptReco-v*/AOD \\
/SingleMu/Run2011A-PromptReco-v*/AOD \\
/ElectronHad/Run2011B-PromptReco-v1/AOD \\
/SingleMu/Run2011B-PromptReco-v1/AOD \\
\hline
\hline
Dilepton Samples (only used for dilepton control region)\\
\hline
/MuEG/Run2011A-PromptReco-v*/AOD   \\
/DoubleMu/Run2011A-PromptReco-v*/AOD   \\
/SingleMu/Run2011A-PromptReco-v*/AOD   \\
/DoubleElectron/Run2011A-PromptReco-v*/AOD   \\
/DoubleElectron/Run2011B-PromptReco-v1/AOD   \\
/DoubleMu/Run2011B-PromptReco-v1/AOD   \\
/MuEG/Run2011B-PromptReco-v1/AOD   \\
/SingleMu/Run2011B-PromptReco-v1/AOD   \\
/SingleElectron/Run2011B-PromptReco-v1/AOD   \\
\hline
\end{tabular}
\caption{Summary of data datasets used.\label{tab:DatasetsData}}
\end{center}
\end{table}

\begin{table}[!ht]
\begin{center}
{\footnotesize
\begin{tabular}{l|l|c}
\hline
\multicolumn{3}{c}{With Pileup: Processed dataset name is} \\
\multicolumn{3}{c}{(S) Summer11-PU\_S4\_START42\_V11-v*/AODSIM} \\
\multicolumn{3}{c}{(F) Fall11-PU\_S6\_START42\_V14B-v*/AODSIM} \\
\multicolumn{3}{c}{(S3) Summer11-PU\_S3\_START42\_V11-v*/AODSIM} \\
\hline
 Description                     &   Primary Dataset Name   & cross-section [pb]\\
\hline
$\ttbar$                              	 &   /TTJets\_TuneZ2\_7TeV-madgraph-tauola (F)                            & 157.5 \\
W $\rightarrow$ $\ell\nu$           	 &   /WJetsToLNu\_TuneZ2\_7TeV-madgraph-tauola (S)                        &  31314.0 \\
 WW                               	 &  /WW\_TuneZ2\_7TeV\_pythia6\_tauola (S)                       &   45.6\\
WZ                               	 &   /WZ\_TuneZ2\_7TeV\_pythia6\_tauola (S)                       &  18.2 \\
ZZ                               	 &  /ZZ\_TuneZ2\_7TeV\_pythia6\_tauola (S)                               &  7.4 \\
$t$ ($s$-chan)                 	 	 &   /T\_TuneZ2\_s-channel\_7TeV-powheg-tauola (S)                        &  3.19 \\
$\bar{t}$ ($s$-chan)                 	 &   /Tbar\_TuneZ2\_s-channel\_7TeV-powheg-tauola (S)                      &  1.44 \\
$t$ ($t$-chan)             	 	 &   /T\_TuneZ2\_t-channel\_7TeV-powheg-tauola (S)                         &  41.92 \\
$\bar{t}$ ($t$-chan)                 	 &   /Tbar\_TuneZ2\_t-channel\_7TeV-powheg-tauola (S)                      &  22.65 \\
$tW$                                     &   /T\_TuneZ2\_tW-channel-DR\_7TeV-powheg-tauola (S)                     &  7.87 \\
$\bar{t} W$                               &   /Tbar\_TuneZ2\_tW-channel-DR\_7TeV-powheg-tauola (S)                  &  7.87 \\
$\dy \To \ell \ell$      & /DYJetsToLL\_TuneZ2\_M-50\_7TeV-madgraph-tauola (S)                   &  3048.0 \\
$\ttbar W$                              	 &   /TTW\_TuneZ2\_7TeV-madgraph (S)                   &  0.1633 \\
$\ttbar Z$                              	 &   /TTZ\_TuneZ2\_7TeV-madgraph (S)                   &  0.139 \\
$\ttbar \gamma$                      &   /TTPhoton\_TuneZ2\_7TeV-madgraph (S)            &  0.6545 \\
WW$\gamma$	& /WWPhoton\_TuneZ2\_7TeV-madgraph (S) 	&	0.177 \\
WWZ	& /WWZNoGstar\_TuneZ2\_7TeV-madgraph (S) 	&	0.0268\\
WWW	& /WWW\_TuneZ2\_7TeV-madgraph (S) 	&	0.038 \\
WZZ		& /WZZNoGstar\_TuneZ2\_7TeV-madgraph (S) 	& 0.0088 \\
ZZZ		& /ZZZNoGstar\_TuneZ2\_7TeV-madgraph (S) 	& 0.00288 \\
$\tilde{t}\bar{\tilde{t}}\rightarrow t\bar{t}\chi^0_1\chi^0_1$ & /SMS-T2tt\_Mstop-225to1200\_mLSP-50to1025\_7TeV-Pythia6Z (S)            &  scan\\
$\tilde{t}\bar{\tilde{t}}\rightarrow b\bar{b}\chi^+_1\chi^-_1$ & /SMS-T2bw\_x-0p25to0p75\_mStop-50to850\_mLSP-50to800\_7TeV-Pythia6Z (S)            &  scan \\
\hline
\hline
$\ttbar$ ($Q^2 \times 2$)                              	 &   /TTjets\_TuneZ2\_scaleup\_7TeV-madgraph-tauola (F)                            & 157.5 \\
$\ttbar$ ($Q^2 \times 0.5$)                             &   /TTjets\_TuneZ2\_scaledown\_7TeV-madgraph-tauola (F)                            & 157.5 \\
$\ttbar$ ($x_q>40~\GeV$)                               &   /TTjets\_TuneZ2\_matchingup\_7TeV-madgraph-tauola (F)                             & 157.5 \\
$\ttbar$ ($x_q>10~\GeV$)                               &   /TTjets\_TuneZ2\_matchingdown\_7TeV-madgraph-tauola (S)                             & 157.5 \\
$\ttbar$ ($m_{\mathrm{top}} = 178.5~\GeV$)       &   /TTJets\_TuneZ2\_mass178\_5\_7TeV-madgraph-tauola (S)                             & 157.5 \\
$\ttbar$ ($m_{\mathrm{top}} = 166.5~\GeV$)       &   /TTJets\_TuneZ2\_mass166\_5\_7TeV-madgraph-tauola (S)                             & 157.5 \\
$\ttbar$                                                  	 &   /TT\_TuneZ2\_7TeV-powheg-tauola (S)                            & 157.5 \\
$\ttbar \To \ell \ell \nu \nu$                          &   /TTTo2L2Nu2B\_7TeV-powheg-pythia6 (S)                              &  16.5 \\
$\ttbar$                                                  	 &   /TT\_TuneZ2\_7TeV-mcatnlo (F)                            & 157.5 \\
$\ttbar$                                                  	 &   /TT\_TuneZ2\_7TeV-pythia6-tauola (S3)                      & 157.5 \\
\hline
\end{tabular}
}
\caption{Summary of Monte Carlo datasets used.\label{tab:DatasetsMC}}
\end{center}
\end{table}

\begin{table}[!ht]
\begin{center}
\begin{tabular}{l}
\hline
\hline
Triggers   \\
\hline
\hline
Single Muon Sample\\
\hline
\footnotesize{\verb=HLT_IsoMu17_v*=}\\
\footnotesize{\verb=HLT_IsoMu24_v*=}\\
\footnotesize{\verb=HLT_IsoMu30_eta2p1_v*=}\\
\hline
Single Electron Sample\\
\hline
\footnotesize{\verb=HLT_Ele25_CaloIdVT_TrkIdT_CentralTriJet30_v*=}\\
\footnotesize{\verb=HLT_Ele25_CaloIdVT_TrkIdT_TriCentralJet30_v*=}\\
\footnotesize{\verb=HLT_Ele25_CaloIdVT_CaloIsoT_TrkIdT_TrkIsoT_TriCentralJet30_v*=}\\
\footnotesize{\verb=HLT_Ele25_CaloIdVT_CaloIsoT_TrkIdT_TrkIsoT_TriCentralPFJet30_v*=}\\
\hline
Dimuon Sample (only used for dilepton control region)\\
\hline
\footnotesize{\verb=HLT_DoubleMu7_v*=}\\
\footnotesize{\verb=HLT_Mu13_Mu7_v*=}\\
\footnotesize{\verb=HLT_Mu13_Mu8_v*=}\\
\footnotesize{\verb=HLT_Mu17_Mu8_v*=}\\
\hline
Electron-Muon Sample (only used for dilepton control region)\\
\hline
\footnotesize{\verb=HLT_Mu17_Ele8_CaloIdL_v*=}\\
\footnotesize{\verb=HLT_Mu8_Ele17_CaloIdL_v*=}\\
\footnotesize{\verb=HLT_Mu17_Ele8_CaloIdT_CaloIsoVL_v*=}\\
\footnotesize{\verb=HLT_Mu8_Ele17_CaloIdT_CaloIsoVL_v*=}\\
\hline
Dielectron Sample (only used for dilepton control region)\\
\hline
\footnotesize{\verb=HLT_Ele17_CaloIdL_CaloIsoVL_Ele8_CaloIdL_CaloIsoVL_v*=}\\
\footnotesize{\verb=HLT_Ele17_CaloIdT_TrkIdVL_CaloIsoVL_TrkIsoVL_Ele8_CaloIdT_TrkIdVL_CaloIsoVL_TrkIsoVL_v*=}\\
\footnotesize{\verb=HLT_Ele17_CaloIdT_CaloIsoVL_TrkIdVL_TrkIsoVL_Ele8_CaloIdT_CaloIsoVL_TrkIdVL_TrkIsoVL_v*=}\\
\hline
\end{tabular}
\caption{Summary of triggers used.\label{tab:TrigData}}
\end{center}
\end{table}




Here we define the selections of leptons, jets, and \met.
We also describe our measurements of the lepton and trigger efficiency.
The analysis uses several different Control Regions (CRs) in addition
to the Signal 
Regions (SRs).
All of these different regions are defined in this section.
This section also includes some information on the basic MC
corrections that we apply.  
%Figure~\ref{fig:venndiagram} illustrates the relationship between these regions.

\subsection{Single Lepton Selection}
\label{sec:singlelepselection}

The single lepton selection is based on the following criteria, starting from the requirements described 
on \url{https://twiki.cern.ch/twiki/bin/viewauth/CMS/SUSYstop#SINGLE_LEPTON_CHANNEL} (revision r20)
\begin{itemize}
\item satisfy the trigger requirement (see
  Table~\ref{tab:TrigData}). 
Note that the analysis triggers are inclusive single lepton triggers.
Dilepton triggers are used only for the dilepton control region.
\item select events with one high \pt\ electron or muon, requiring
  \begin{itemize}
  \item $\pt>30~\GeVc$  and $|\eta|<1.4442 (2.1)$ for electrons (muons). The restriction to the barrel for electrons
is motivated by an observed excess of events with large \mt\ with endcap electrons in the b-veto control region, 
and does not significantly reduce the signal acceptance since the leptons tend to be central.
  \item muon ID criteria is based on the 2012 POG recommended tight working point
  \item electron ID critera is based on the 2012 POG recommended medium working point
  \item PF-based isolation ($\Delta R < 0.3$) relative isolation $<$ 0.15 and absolute isolation $<$ 5~GeV. PU corrections 
are performed with the $\Delta\beta$ scheme for muons and effective-area fastjet rho scheme for electrons (as recommended by the relevant POGs).
  \item $|\pt(\rm{PF}_{lep}) - \pt(\rm{RECO}_{lep})| < 10~\GeV$
  \item $E/p_{\rm{in}} < 4$ (electrons only)
  \item We remove electron events with $\met > 50$ GeV and $M_T > 100$
    GeV with at least one crystal in the supercluster with laser
    correction $>$2.\footnote{This is an ad-hoc removal based on
      run-event numbers, since the
      problem was found very recently and the filter was not available
      when we processed the events.}
  \end{itemize} 
  \item require at least 4 PF jets in the event with $\pt>30~\GeV$
    within $|\eta|<2.5$ out of which at least 1 satisfies the CSV
    medium working point b-tagging requirement
  \item require moderate $\met>50~\GeV$  (type1-corrected pfmet with $\phi$ corrections applied as described in Sec.~\ref{sec:JetMet}).
 \item Isolated track veto, see Section~\ref{sec:tkveto}

\end{itemize}

%Table~\ref{tab:preselectionyield} shows the yields in data and MC without any corrections for this preselection region.

%\begin{table}[!h]
%\begin{center}
%\begin{tabular}{c|c}
%\hline
%\hline
%\end{tabular}
%\caption{  Raw Data and MC predictions without any corrections are shown after preselection. \label{tab:preselectionyield}}
%\end{center}
%\end{table}

\subsection{Isolated track veto}
\label{sec:tkveto}

The isolated track veto is intended to remove top dilepton events.
Looking for an isolated track is an effective way of identifying $W
\to e$, $W \to \mu$, $W \to \tau \to \ell$, and $W \to \tau \to
h^{\pm} + n\pi^{0}$.  The requirements on the track are

\begin{itemize}
\item $P_T > 10$ GeV
\item Relative track isolation $< 10\%$  computed from charged PF
  candidates with $d_Z<$ 0.05 cm from the primary vertex.
\end{itemize}


\subsection{Signal Region Selection}
\label{sec:SR}

The signal regions (SRs) are selected to improve the sensitivity for the
single lepton requirements and cover a range of scalar top
scenarios. The \mt\ and \met\ variables are used to define the signal
regions and the requirements are listed in Table~\ref{tab:srdef}. 

\begin{table}[!h]
\begin{center}
\begin{tabular}{l|c|c}
\hline
Signal Region & Minimum \mt\ [GeV] & Minimum \met\ [GeV] \\
\hline
\hline
SRA & 150 & 100 \\
SRB & 120 & 150 \\
SRC & 120 & 200 \\
SRD & 120 & 250 \\
SRE & 120 & 300 \\
SRF & 120 & 350 \\
SRG & 120 & 400 \\
\hline
\end{tabular}
\caption{ Signal region definitions based on \mt\ and \met\
  requirements. These requirements are applied in addition to the
  baseline single lepton selection.
\label{tab:srdef}}
\end{center}
\end{table}

Table~\ref{tab:srrawmcyields} shows the expected number of SM
background yields for the SRs. A few stop signal yields for four
values of the parameters are also shown for comparison. The signal
regions with looser requirements are sensitive to lower stop masses
M(\sctop), while those with tighter requirements are more sensitive to
higher M(\sctop). Kinematic distributions for a few sample signal
points can be found in Appendix~\ref{app:sigkin}.

\begin{table}[!h]
\begin{center}
\footnotesize
\begin{tabular}{l||c|c|c|c|c|c|c}
\hline
Sample              & SRA & SRB & SRC & SRD & SRE & SRF & SRG\\
\hline
\hline
\ttdl\ 		 & $619 \pm 9$& $366 \pm 7$& $127 \pm 4$& $44 \pm 2$& $17 \pm 1$& $7 \pm 1$& $4 \pm 1$ \\
\ttsl\ \& single top (1\Lep) 		 & $95 \pm 3$& $67 \pm 3$& $15 \pm 1$& $6 \pm 1$& $2 \pm 1$& $1 \pm 1$& $1 \pm 0$ \\
\wjets\ 		 & $29 \pm 2$& $15 \pm 2$& $6 \pm 1$& $3 \pm 1$& $1 \pm 0$& $0 \pm 0$& $0 \pm 0$ \\
Rare 		 & $59 \pm 3$& $38 \pm 3$& $16 \pm 2$& $8 \pm 1$& $4 \pm 1$& $2 \pm 0$& $1 \pm 0$ \\
\hline
Total 		 & $802 \pm 10$& $486 \pm 8$& $164 \pm 5$& $62 \pm 3$& $23 \pm 2$& $10 \pm 1$& $6 \pm 1$ \\
\hline
Yield UL (optimistic)  & 147 (10\%) & 94 (10\%)  & 47 (15\%) & 25 (20\%) & 14 (25\%) & 8.6 (30\%) & 7.5 (50\%)  \\
Yield UL (pessimistic) & 200 (15\%) & 152 (20\%) & 64 (25\%) & 30 (30\%) & 15 (35\%) & 9.7 (50\%) & 8.2 (100\%) \\
\hline
T2tt m(stop) = 250 m($\chi^0$) = 0 & $315 \pm 18$& $193 \pm 14$& $53 \pm 8$& $13 \pm 4$& $2 \pm 2$& $0 \pm 0$& $0 \pm 0$ \\ \hline
T2tt m(stop) = 300 m($\chi^0$) = 50 & $296 \pm 11$& $236 \pm 10$& $88 \pm 6$& $28 \pm 3$& $10 \pm 2$& $2 \pm 1$& $0 \pm 0$ \\ \hline
T2tt m(stop) = 300 m($\chi^0$) = 100 & $128 \pm 7$& $93 \pm 6$& $29 \pm 3$& $10 \pm 2$& $5 \pm 1$& $2 \pm 1$& $1 \pm 1$ \\ \hline
T2tt m(stop) = 350 m($\chi^0$) = 0 & $224 \pm 6$& $206 \pm 6$& $119 \pm 4$& $52 \pm 3$& $20 \pm 2$& $8 \pm 1$& $3 \pm 1$ \\ \hline
T2tt m(stop) = 450 m($\chi^0$) = 0 & $71 \pm 2$& $71 \pm 2$& $53 \pm 1$& $36 \pm 1$& $21 \pm 1$& $11 \pm 1$& $5 \pm 0$ \\
\hline
\end{tabular}
\caption{ Expected SM background contributions and signal yields for a few sample points, 
including both muon and electron channels. This is ``dead reckoning'' MC with no
correction. It is meant only as a general guide. The uncertainties are statistical only.
The signal yield expected upper limits are also shown for two values of the total background systematic uncertainty, indicated in parentheses.
%[{\bf VERENA} THESE SIGNAL YIELDS NEED TO BE UPDATED. Do you have a point with larger stop mass to illustrate why we use SRF and SRG? ].
%HOOBERMAN
\label{tab:srrawmcyields}}
\end{center}
\end{table}

\subsection{Control Region Selection}
\label{sec:CRsel}

Control regions (CRs) are used to validate the background estimation
procedure and derive systematic uncertainties for some
contributions. The CRs are selected to have similar
kinematics to the SRs, but have a different requirement in terms of
number of b-tags and number of leptons, thus enhancing them in
different SM contributions. The four CRs used in this analysis are
summarized in Table~\ref{tab:crdef}.  

\begin{table}
\begin{center}
{\small
\begin{tabular}{l|c|c|c}
\hline
Selection 	& \multirow{2}{*}{exactly 1 lepton}	& \multirow{2}{*}{exactly 2
	leptons}		& \multirow{2}{*}{1 lepton + isolated
        track}\\
      Criteria & & & \\
\hline
\hline
\multirow{4}{*}{0 b-tags} 	 
& 	 CR1) W+Jets dominated:
& 	 CR2) apply \Z-mass constraint			 
& 	 CR3) not used \\  
& 	 
&       $\rightarrow$ Z+Jets dominated: Validate 
&      \\
&      Validate W+Jets \mt\ tail
& 	 \ttsl\ \mt\ tail comparing 
& 	 \\  
&
& 	 data vs. MC ``pseudo-\mt ''
& 	 \\  
\hline
\multirow{4}{*}{$\ge$ 1 b-tags} 	 
& 	
& 	CR4) Apply \Z-mass veto 
&      CR5) \ttdl, \ttlt\ and \\
&     SIGNAL 
&      $\rightarrow$ \ttdl\ dominated: Validate 
&	\ttlf\ dominated:  Validate \\
&     REGION 
&      ``physics'' modelling of \ttdl\     
&      \Tau\  and fake lepton modeling/\\
&
&
&      detector effects in \ttdl\     \\
\hline
\end{tabular}
}
\caption{Summary of signal and control regions.
  \label{tab:crdef}%\protect
}
\end{center}
\end{table}

\subsection{Definition of $M_T$ peak region}
\label{sec:mtpeakdef}

This region is defined as $50 < M_T < 80$ GeV.


\subsection{Default \ttbar\  MC sample}

Our default \ttbar\ MC sample is Powheg.

\subsection{MC Corrections}
\label{sec:MCCorr}

All MC samples are corrected for trigger efficiency.  In the case of
single lepton selections, we apply the $\pt$ and $\eta$-dependent
scale factors that we measure ourselves, see Section~\ref{sec:trg}.
In the case of dilepton selections that require the dilepton triggers,
we apply overall scale factors of 0.95, 0.88, and 0.92 for $ee$,
$\mu\mu$,
and $e\mu$ respectively~\cite{didar}.

The leptonic branching fraction used in some of the \ttbar\ MC samples
differs from the value listed in the PDG $(10.80 \pm 0.09)\%$. 
Table~\ref{tab:wlepbf} summarizes the branching fractions used in
the generation of the various \ttbar\ MC samples. 
For \ttbar\ samples with the incorrect leptonic branching fraction, event
weights are applied based on the number of true leptons and the ratio
of the corrected and incorrect branching fractions. 

\begin{table}[!h]
\begin{center}
\begin{tabular}{c|c}
\hline
         \ttbar\ Sample - Event Generator & Leptonic Branching Fraction\\
\hline
\hline
Madgraph     &       0.111\\
MC@NLO       &       0.111\\
Pythia       &       0.108\\
Powheg       &       0.108\\
\hline
\end{tabular}
\caption{Leptonic branching fractions for the various \ttbar\ samples
  used in the analysis. The \ttbar\ MC samples produced with
  Madgraph and MC@NLO has a branching fraction that is almost $3\%$ higher than
  the PDG value. \label{tab:wlepbf}}
\end{center}
\end{table}

All \ttbar\ dilepton samples are corrected (when needed and
appropriate) 
in order to have the correct jet multiplicity distribution.  This
correction procedure is described in Section~\ref{sec:jetmultiplicity}.


\subsubsection{Corrections to Jets and \met}
\label{sec:JetMet}

The official recommendations from the Jet/MET group are used for 
the data and MC samples. In particular, the jet
energy corrections (JEC) are updated using the official recipe.
L1FastL2L3Residual (L1FastL2L3) corrections are applied for data (MC),
based on the global tags GR\_R\_52\_V9 (START52\_V9B) for
data (MC). In addition, these jet energy corrections are propagated to
the \met\ calculation, following the official prescription for
deriving the Type I corrections. 

Events with anomalous ``rho'' pile-up corrections are excluded from the sample since these 
can correspond to events with unphysically large \met\ and \mt.
%tail signal region.
In addition, the recommended MET filters are applied. 
A correction to remove the $\phi$ modulation in \met\ is also applied
to the data.


\subsection{Lepton Selection Efficiency Measurements}
\label{sec:lepEff}

In this section we measure the identification and isolation efficiencies for muons and electrons in data and MC using tag-and-probe studies. 
The tag is required to pass the full offline analysis selection and have \pt\ $>$ 30 GeV, $|\eta|<2.1$, and be matched to the single
lepton trigger, HLT\_IsoMu24(\_eta2p1) for muons and HLT\_Ele27\_WP80 for electrons. 
The probe is required to have $|\eta|<2.1$ and \pt\ $>$ 20 GeV. To measure the identification efficiency we require the probe to pass the isolation requirement,
to measure the isolation efficiency we require the probe to pass the
identification requirement.

The tag-probe pair is required to have opposite-sign and an invariant mass in the range 76--106 GeV.
In order to suppress lepton pairs from sources other than Z boson
decays, we require the event to have \met\ $<$ 30 GeV and no b-tagged
jets (CSV loose working point).

The muon efficiencies are summarized in Table~\ref{tab:mutnpeff} for inclusive events (i.e. no jet requirements). These efficiencies are displayed in Fig.~\ref{fig:mutnpeff} for
several different jet multiplicity requirements. 
We currently observe good agreement for muons with \pt\ up to about 300 GeV. 
For high \pt\ muons we observe a source of background in the data with large impact parameters, which we suppress by requiring muon $d_0<0.02$~cm and $d_Z<0.5$~cm.
%For muons with \pt\ $>$ 200 GeV the data efficiency
%begins to drop, and the effect is especially pronounced for muons with \pt\ $>$ 300 GeV. 
%We are currently investigating the source of this inefficiency.
The electron efficiencies are summarized in Table~\ref{tab:eltnpeff} for inclusive events (i.e. no jet requirements). These efficiencies are displayed in Fig.~\ref{fig:eltnpeff} 
for several different jet multiplicity requirements. In general we observe good agreement between the data and MC identification and isolation efficiencies.

% Pending a better understanding of the very high \pt\ muon efficiency, 
We
do not correct the MC for differences in lepton efficiency.  In the
background calculation, we do not take any systematics due to lepton
selection
efficiency uncertainties.  This is because all backgrounds except the 
rare MC background are normalized to the $M_T$ peak, thus the lepton
identification uncertainty cancels out.  For the rare MC these
uncertainties
are negligible compared to the assumed cross-section uncertainty
(Section~\ref{sec:bkg_other}).




\begin{table}[htb]
\begin{center}
\scriptsize
\caption{\label{tab:mutnpeff}
Summary of the data and MC muon identification and isolation efficiencies measured with tag-and-probe studies.}
\begin{tabular}{c|c|c|c}

%%%UPDATED WITH FULL 2012 DATA SAMPLE

%-------------------
%Doing muons
%-------------------
%DOING MUON ETA BINS
%Selection  : ((((((((abs(tagAndProbeMass-91)<15)&&(qProbe*qTag<0))&&(abs(tag->eta())<2.1))&&(tag->pt()>30.0))&&(abs(probe->eta())<2.1))&&(met<30))&&(nbl==0))&&((eventSelection&2)==2))&&(HLT_IsoMu24_tag > 0)
%Ndata      : 14334405
%NMC        : 4365306
%ID cut     : (leptonSelection&65536)==65536
%iso cut    : (leptonSelection&131072)==131072

\hline
\hline
MC ID & & & \\
\pt\ range [GeV] & $|\eta|<0.8$ & $0.8<|\eta|<1.5$ & $1.5<|\eta|<2.1$ \\
\hline
    20 -   30  & 	0.9559 $\pm$ 0.0005 & 	0.9511 $\pm$ 0.0006 & 	0.9354 $\pm$ 0.0008 \\
    30 -   40  & 	0.9578 $\pm$ 0.0002 & 	0.9543 $\pm$ 0.0003 & 	0.9356 $\pm$ 0.0005 \\
    40 -   50  & 	0.9605 $\pm$ 0.0002 & 	0.9589 $\pm$ 0.0002 & 	0.9359 $\pm$ 0.0004 \\
    50 -   60  & 	0.9568 $\pm$ 0.0005 & 	0.9517 $\pm$ 0.0006 & 	0.9318 $\pm$ 0.0009 \\
    60 -   80  & 	0.9573 $\pm$ 0.0010 & 	0.9514 $\pm$ 0.0012 & 	0.9245 $\pm$ 0.0018 \\
    80 -  100  & 	0.9610 $\pm$ 0.0024 & 	0.9515 $\pm$ 0.0030 & 	0.9151 $\pm$ 0.0052 \\
   100 -  150  & 	0.9607 $\pm$ 0.0031 & 	0.9534 $\pm$ 0.0039 & 	0.9205 $\pm$ 0.0069 \\
   150 -  200  & 	0.9490 $\pm$ 0.0078 & 	0.9428 $\pm$ 0.0097 & 	0.8844 $\pm$ 0.0213 \\
   200 -  300  & 	0.9560 $\pm$ 0.0124 & 	0.9183 $\pm$ 0.0190 & 	0.8676 $\pm$ 0.0411 \\
   300 - 10000  & 	0.9394 $\pm$ 0.0294 & 	0.8293 $\pm$ 0.0588 & 	0.5455 $\pm$ 0.1501 \\
\hline
\hline
MC ISO  & & & \\
\pt\ range [GeV] & $|\eta|<0.8$ & $0.8<|\eta|<1.5$ & $1.5<|\eta|<2.1$ \\
\hline
    20 -   30  & 	0.8966 $\pm$ 0.0007 & 	0.9155 $\pm$ 0.0008 & 	0.9304 $\pm$ 0.0008 \\
    30 -   40  & 	0.9609 $\pm$ 0.0002 & 	0.9630 $\pm$ 0.0003 & 	0.9705 $\pm$ 0.0003 \\
    40 -   50  & 	0.9876 $\pm$ 0.0001 & 	0.9897 $\pm$ 0.0001 & 	0.9913 $\pm$ 0.0002 \\
    50 -   60  & 	0.9919 $\pm$ 0.0002 & 	0.9928 $\pm$ 0.0002 & 	0.9938 $\pm$ 0.0003 \\
    60 -   80  & 	0.9926 $\pm$ 0.0004 & 	0.9937 $\pm$ 0.0004 & 	0.9946 $\pm$ 0.0005 \\
    80 -  100  & 	0.9912 $\pm$ 0.0012 & 	0.9916 $\pm$ 0.0013 & 	0.9925 $\pm$ 0.0017 \\
   100 -  150  & 	0.9897 $\pm$ 0.0016 & 	0.9928 $\pm$ 0.0016 & 	0.9937 $\pm$ 0.0021 \\
   150 -  200  & 	0.9894 $\pm$ 0.0037 & 	0.9909 $\pm$ 0.0041 & 	0.9950 $\pm$ 0.0050 \\
   200 -  300  & 	0.9924 $\pm$ 0.0054 & 	1.0000 $\pm$ 0.0000 & 	0.9833 $\pm$ 0.0165 \\
   300 - 10000  & 	1.0000 $\pm$ 0.0000 & 	1.0000 $\pm$ 0.0000 & 	1.0000 $\pm$ 0.0000 \\
\hline
\hline
DATA ID & & & \\
\pt\ range [GeV] & $|\eta|<0.8$ & $0.8<|\eta|<1.5$ & $1.5<|\eta|<2.1$ \\
\hline
    20 -   30  & 	0.9370 $\pm$ 0.0003 & 	0.9349 $\pm$ 0.0004 & 	0.9164 $\pm$ 0.0005 \\
    30 -   40  & 	0.9412 $\pm$ 0.0002 & 	0.9391 $\pm$ 0.0002 & 	0.9213 $\pm$ 0.0003 \\
    40 -   50  & 	0.9457 $\pm$ 0.0001 & 	0.9460 $\pm$ 0.0002 & 	0.9234 $\pm$ 0.0002 \\
    50 -   60  & 	0.9374 $\pm$ 0.0003 & 	0.9339 $\pm$ 0.0004 & 	0.9135 $\pm$ 0.0006 \\
    60 -   80  & 	0.9359 $\pm$ 0.0006 & 	0.9303 $\pm$ 0.0008 & 	0.9079 $\pm$ 0.0012 \\
    80 -  100  & 	0.9267 $\pm$ 0.0018 & 	0.9246 $\pm$ 0.0022 & 	0.9036 $\pm$ 0.0033 \\
   100 -  150  & 	0.9256 $\pm$ 0.0024 & 	0.9201 $\pm$ 0.0030 & 	0.8823 $\pm$ 0.0050 \\
   150 -  200  & 	0.9102 $\pm$ 0.0061 & 	0.9204 $\pm$ 0.0072 & 	0.8624 $\pm$ 0.0127 \\
   200 -  300  & 	0.8593 $\pm$ 0.0124 & 	0.8574 $\pm$ 0.0146 & 	0.7755 $\pm$ 0.0298 \\
   300 - 10000  & 	0.5622 $\pm$ 0.0365 & 	0.4578 $\pm$ 0.0547 & 	0.1406 $\pm$ 0.0435 \\
\hline
\hline
DATA ISO  & & & \\
\pt\ range [GeV] & $|\eta|<0.8$ & $0.8<|\eta|<1.5$ & $1.5<|\eta|<2.1$ \\
\hline
    20 -   30  & 	0.8912 $\pm$ 0.0004 & 	0.9127 $\pm$ 0.0004 & 	0.9361 $\pm$ 0.0004 \\
    30 -   40  & 	0.9586 $\pm$ 0.0001 & 	0.9638 $\pm$ 0.0002 & 	0.9742 $\pm$ 0.0002 \\
    40 -   50  & 	0.9865 $\pm$ 0.0001 & 	0.9898 $\pm$ 0.0001 & 	0.9921 $\pm$ 0.0001 \\
    50 -   60  & 	0.9910 $\pm$ 0.0001 & 	0.9931 $\pm$ 0.0001 & 	0.9952 $\pm$ 0.0001 \\
    60 -   80  & 	0.9922 $\pm$ 0.0002 & 	0.9931 $\pm$ 0.0003 & 	0.9955 $\pm$ 0.0003 \\
    80 -  100  & 	0.9919 $\pm$ 0.0006 & 	0.9918 $\pm$ 0.0008 & 	0.9945 $\pm$ 0.0009 \\
   100 -  150  & 	0.9918 $\pm$ 0.0009 & 	0.9936 $\pm$ 0.0009 & 	0.9973 $\pm$ 0.0009 \\
   150 -  200  & 	0.9926 $\pm$ 0.0019 & 	0.9939 $\pm$ 0.0021 & 	0.9969 $\pm$ 0.0022 \\
   200 -  300  & 	1.0000 $\pm$ 0.0000 & 	1.0000 $\pm$ 0.0000 & 	1.0000 $\pm$ 0.0000 \\
   300 - 10000  & 	1.0000 $\pm$ 0.0000 & 	1.0000 $\pm$ 0.0000 & 	1.0000 $\pm$ 0.0000 \\
\hline
\hline
 Scale Factor ID  & & & \\
\pt\ range [GeV] & $|\eta|<0.8$ & $0.8<|\eta|<1.5$ & $1.5<|\eta|<2.1$ \\
\hline
    20 -   30  & 	0.9803 $\pm$ 0.0006 & 	0.9830 $\pm$ 0.0008 & 	0.9797 $\pm$ 0.0010 \\
    30 -   40  & 	0.9826 $\pm$ 0.0003 & 	0.9841 $\pm$ 0.0004 & 	0.9847 $\pm$ 0.0006 \\
    40 -   50  & 	0.9846 $\pm$ 0.0003 & 	0.9866 $\pm$ 0.0003 & 	0.9866 $\pm$ 0.0005 \\
    50 -   60  & 	0.9797 $\pm$ 0.0006 & 	0.9813 $\pm$ 0.0007 & 	0.9804 $\pm$ 0.0011 \\
    60 -   80  & 	0.9776 $\pm$ 0.0012 & 	0.9778 $\pm$ 0.0014 & 	0.9820 $\pm$ 0.0023 \\
    80 -  100  & 	0.9643 $\pm$ 0.0030 & 	0.9716 $\pm$ 0.0038 & 	0.9874 $\pm$ 0.0066 \\
   100 -  150  & 	0.9635 $\pm$ 0.0040 & 	0.9651 $\pm$ 0.0050 & 	0.9585 $\pm$ 0.0090 \\
   150 -  200  & 	0.9590 $\pm$ 0.0102 & 	0.9763 $\pm$ 0.0126 & 	0.9751 $\pm$ 0.0275 \\
   200 -  300  & 	0.8988 $\pm$ 0.0175 & 	0.9337 $\pm$ 0.0250 & 	0.8938 $\pm$ 0.0545 \\
   300 - 10000  & 	0.5984 $\pm$ 0.0431 & 	0.5521 $\pm$ 0.0767 & 	0.2578 $\pm$ 0.1067 \\
\hline
\hline
Scale Factor ISO & & & \\
\pt\ range [GeV] & $|\eta|<0.8$ & $0.8<|\eta|<1.5$ & $1.5<|\eta|<2.1$ \\
\hline
    20 -   30  & 	0.9940 $\pm$ 0.0009 & 	0.9970 $\pm$ 0.0010 & 	1.0061 $\pm$ 0.0010 \\
    30 -   40  & 	0.9975 $\pm$ 0.0003 & 	1.0008 $\pm$ 0.0003 & 	1.0038 $\pm$ 0.0004 \\
    40 -   50  & 	0.9989 $\pm$ 0.0001 & 	1.0001 $\pm$ 0.0001 & 	1.0009 $\pm$ 0.0002 \\
    50 -   60  & 	0.9991 $\pm$ 0.0003 & 	1.0003 $\pm$ 0.0003 & 	1.0014 $\pm$ 0.0003 \\
    60 -   80  & 	0.9996 $\pm$ 0.0005 & 	0.9993 $\pm$ 0.0005 & 	1.0009 $\pm$ 0.0006 \\
    80 -  100  & 	1.0007 $\pm$ 0.0013 & 	1.0002 $\pm$ 0.0015 & 	1.0020 $\pm$ 0.0019 \\
   100 -  150  & 	1.0021 $\pm$ 0.0019 & 	1.0007 $\pm$ 0.0019 & 	1.0036 $\pm$ 0.0023 \\
   150 -  200  & 	1.0032 $\pm$ 0.0043 & 	1.0031 $\pm$ 0.0046 & 	1.0019 $\pm$ 0.0055 \\
   200 -  300  & 	1.0077 $\pm$ 0.0054 & 	1.0000 $\pm$ 0.0000 & 	1.0169 $\pm$ 0.0171 \\
   300 - 10000  & 	1.0000 $\pm$ 0.0000 & 	1.0000 $\pm$ 0.0000 & 	1.0000 $\pm$ 0.0000 \\
\hline
\hline


\end{tabular}
\end{center}
\end{table}

\begin{figure}[hbt]
  \begin{center}
  	\includegraphics[width=0.3\linewidth]{plots/mu_id_njets0.pdf}%
	\includegraphics[width=0.3\linewidth]{plots/mu_iso_njets0.pdf}
	\includegraphics[width=0.3\linewidth]{plots/mu_id_njets1.pdf}%
	\includegraphics[width=0.3\linewidth]{plots/mu_iso_njets1.pdf}
	\includegraphics[width=0.3\linewidth]{plots/mu_id_njets2.pdf}%
	\includegraphics[width=0.3\linewidth]{plots/mu_iso_njets2.pdf}
	\includegraphics[width=0.3\linewidth]{plots/mu_id_njets3.pdf}%
	\includegraphics[width=0.3\linewidth]{plots/mu_iso_njets3.pdf}
	\includegraphics[width=0.3\linewidth]{plots/mu_id_njets4.pdf}%
	\includegraphics[width=0.3\linewidth]{plots/mu_iso_njets4.pdf}
	\caption{
	  \label{fig:mutnpeff} Comparison of the muon identification and isolation efficiencies in data and MC for various jet multiplicity requirements. }  
      \end{center}
\end{figure}

\clearpage

\begin{table}[htb]
\begin{center}
\scriptsize
\caption{\label{tab:eltnpeff}
Summary of the data and MC electron identification and isolation efficiencies measured with tag-and-probe studies.}
\begin{tabular}{c|c|c}

%%% THESE EFFICIENCIES HAVE BEEN UPDATED WITH THE FULL 2012 DATA SAMPLE

%-------------------
%Doing electrons
%-------------------
%DOING ELECTRON ETA BINS
%Selection  : ((((((((abs(tagAndProbeMass-91)<15)&&(qProbe*qTag<0))&&(abs(tag->eta())<2.1))&&(tag->pt()>30.0))&&(abs(probe->eta())<2.1))&&(met<30))&&(nbl==0))&&((eventSelection&1)==1))&&(HLT_Ele27_WP80_tag > 0)
%Ndata      : 10157846
%NMC        : 3351234
%ID cut     : (leptonSelection&8)==8
%iso cut    : (leptonSelection&16)==16

\hline
\hline
MC ID & & \\
\pt\ range [GeV] & $|\eta|<0.8$ &  $1.5<|\eta|<2.1$ \\
\hline
    20 -   30  & 	0.8147 $\pm$ 0.0008 & 	0.6570 $\pm$ 0.0019 \\
    30 -   40  & 	0.8667 $\pm$ 0.0004 & 	0.7465 $\pm$ 0.0010 \\
    40 -   50  & 	0.8921 $\pm$ 0.0003 & 	0.7849 $\pm$ 0.0008 \\
    50 -   60  & 	0.9017 $\pm$ 0.0006 & 	0.7964 $\pm$ 0.0017 \\
    60 -   80  & 	0.9097 $\pm$ 0.0011 & 	0.8149 $\pm$ 0.0034 \\
    80 -  100  & 	0.9195 $\pm$ 0.0028 & 	0.8275 $\pm$ 0.0087 \\
   100 -  150  & 	0.9163 $\pm$ 0.0037 & 	0.8418 $\pm$ 0.0114 \\
   150 -  200  & 	0.9093 $\pm$ 0.0086 & 	0.8368 $\pm$ 0.0268 \\
   200 -  300  & 	0.9264 $\pm$ 0.0121 & 	0.8852 $\pm$ 0.0408 \\
   300 - 10000  & 	0.8734 $\pm$ 0.0374 & 	1.0000 $\pm$ 0.0000 \\
\hline
\hline
MC ISO  & & \\
\pt\ range [GeV] & $|\eta|<0.8$ &  $1.5<|\eta|<2.1$ \\
\hline
    20 -   30  & 	0.9330 $\pm$ 0.0006 & 	0.9410 $\pm$ 0.0011 \\
    30 -   40  & 	0.9719 $\pm$ 0.0002 & 	0.9714 $\pm$ 0.0004 \\
    40 -   50  & 	0.9895 $\pm$ 0.0001 & 	0.9871 $\pm$ 0.0003 \\
    50 -   60  & 	0.9919 $\pm$ 0.0002 & 	0.9902 $\pm$ 0.0005 \\
    60 -   80  & 	0.9926 $\pm$ 0.0004 & 	0.9929 $\pm$ 0.0008 \\
    80 -  100  & 	0.9920 $\pm$ 0.0010 & 	0.9905 $\pm$ 0.0024 \\
   100 -  150  & 	0.9929 $\pm$ 0.0012 & 	0.9897 $\pm$ 0.0034 \\
   150 -  200  & 	0.9931 $\pm$ 0.0026 & 	0.9937 $\pm$ 0.0062 \\
   200 -  300  & 	0.9930 $\pm$ 0.0040 & 	1.0000 $\pm$ 0.0000 \\
   300 - 10000  & 	1.0000 $\pm$ 0.0000 & 	1.0000 $\pm$ 0.0000 \\
\hline
\hline
DATA ID & & \\
\pt\ range [GeV] & $|\eta|<0.8$ &  $1.5<|\eta|<2.1$ \\
\hline
    20 -   30  & 	0.8095 $\pm$ 0.0005 & 	0.6475 $\pm$ 0.0011 \\
    30 -   40  & 	0.8656 $\pm$ 0.0002 & 	0.7431 $\pm$ 0.0006 \\
    40 -   50  & 	0.8936 $\pm$ 0.0002 & 	0.7911 $\pm$ 0.0005 \\
    50 -   60  & 	0.9024 $\pm$ 0.0004 & 	0.8035 $\pm$ 0.0011 \\
    60 -   80  & 	0.9091 $\pm$ 0.0007 & 	0.8138 $\pm$ 0.0021 \\
    80 -  100  & 	0.9165 $\pm$ 0.0017 & 	0.8329 $\pm$ 0.0055 \\
   100 -  150  & 	0.9229 $\pm$ 0.0022 & 	0.8589 $\pm$ 0.0070 \\
   150 -  200  & 	0.9242 $\pm$ 0.0050 & 	0.8254 $\pm$ 0.0214 \\
   200 -  300  & 	0.9232 $\pm$ 0.0085 & 	0.8409 $\pm$ 0.0390 \\
   300 - 10000  & 	0.9137 $\pm$ 0.0238 & 	1.0000 $\pm$ 0.0000 \\
\hline
\hline
DATA ISO  & & \\
\pt\ range [GeV] & $|\eta|<0.8$ &  $1.5<|\eta|<2.1$ \\
\hline
    20 -   30  & 	0.9263 $\pm$ 0.0003 & 	0.9356 $\pm$ 0.0007 \\
    30 -   40  & 	0.9693 $\pm$ 0.0001 & 	0.9706 $\pm$ 0.0003 \\
    40 -   50  & 	0.9882 $\pm$ 0.0001 & 	0.9872 $\pm$ 0.0002 \\
    50 -   60  & 	0.9913 $\pm$ 0.0001 & 	0.9920 $\pm$ 0.0003 \\
    60 -   80  & 	0.9923 $\pm$ 0.0002 & 	0.9928 $\pm$ 0.0005 \\
    80 -  100  & 	0.9912 $\pm$ 0.0006 & 	0.9945 $\pm$ 0.0012 \\
   100 -  150  & 	0.9927 $\pm$ 0.0007 & 	0.9986 $\pm$ 0.0008 \\
   150 -  200  & 	0.9913 $\pm$ 0.0018 & 	0.9886 $\pm$ 0.0065 \\
   200 -  300  & 	0.9946 $\pm$ 0.0024 & 	1.0000 $\pm$ 0.0000 \\
   300 - 10000  & 	0.9922 $\pm$ 0.0078 & 	1.0000 $\pm$ 0.0000 \\
\hline
\hline
 Scale Factor ID  & & \\
\pt\ range [GeV] & $|\eta|<0.8$ &  $1.5<|\eta|<2.1$ \\
\hline
    20 -   30  & 	0.9936 $\pm$ 0.0012 & 	0.9855 $\pm$ 0.0033 \\
    30 -   40  & 	0.9987 $\pm$ 0.0005 & 	0.9955 $\pm$ 0.0015 \\
    40 -   50  & 	1.0017 $\pm$ 0.0004 & 	1.0079 $\pm$ 0.0012 \\
    50 -   60  & 	1.0008 $\pm$ 0.0008 & 	1.0089 $\pm$ 0.0026 \\
    60 -   80  & 	0.9993 $\pm$ 0.0015 & 	0.9987 $\pm$ 0.0049 \\
    80 -  100  & 	0.9968 $\pm$ 0.0036 & 	1.0065 $\pm$ 0.0125 \\
   100 -  150  & 	1.0073 $\pm$ 0.0047 & 	1.0203 $\pm$ 0.0161 \\
   150 -  200  & 	1.0164 $\pm$ 0.0111 & 	0.9863 $\pm$ 0.0406 \\
   200 -  300  & 	0.9965 $\pm$ 0.0159 & 	0.9499 $\pm$ 0.0621 \\
   300 - 10000  & 	1.0461 $\pm$ 0.0525 & 	1.0000 $\pm$ 0.0000 \\
\hline
\hline
Scale Factor ISO & & \\
\pt\ range [GeV] & $|\eta|<0.8$ &  $1.5<|\eta|<2.1$ \\
\hline
    20 -   30  & 	0.9928 $\pm$ 0.0007 & 	0.9942 $\pm$ 0.0014 \\
    30 -   40  & 	0.9974 $\pm$ 0.0002 & 	0.9992 $\pm$ 0.0005 \\
    40 -   50  & 	0.9988 $\pm$ 0.0001 & 	1.0001 $\pm$ 0.0003 \\
    50 -   60  & 	0.9993 $\pm$ 0.0002 & 	1.0018 $\pm$ 0.0005 \\
    60 -   80  & 	0.9997 $\pm$ 0.0004 & 	0.9999 $\pm$ 0.0010 \\
    80 -  100  & 	0.9991 $\pm$ 0.0011 & 	1.0041 $\pm$ 0.0028 \\
   100 -  150  & 	0.9998 $\pm$ 0.0014 & 	1.0090 $\pm$ 0.0036 \\
   150 -  200  & 	0.9981 $\pm$ 0.0032 & 	0.9948 $\pm$ 0.0091 \\
   200 -  300  & 	1.0015 $\pm$ 0.0047 & 	1.0000 $\pm$ 0.0000 \\
   300 - 10000  & 	0.9922 $\pm$ 0.0078 & 	1.0000 $\pm$ 0.0000 \\
\hline
\hline

\end{tabular}
\end{center}
\end{table}

\begin{figure}[hbt]
  \begin{center}
    	\includegraphics[width=0.3\linewidth]{plots/el_id_njets0.pdf}%
	\includegraphics[width=0.3\linewidth]{plots/el_iso_njets0.pdf}
	\includegraphics[width=0.3\linewidth]{plots/el_id_njets1.pdf}%
	\includegraphics[width=0.3\linewidth]{plots/el_iso_njets1.pdf}
	\includegraphics[width=0.3\linewidth]{plots/el_id_njets2.pdf}%
	\includegraphics[width=0.3\linewidth]{plots/el_iso_njets2.pdf}
	\includegraphics[width=0.3\linewidth]{plots/el_id_njets3.pdf}%
	\includegraphics[width=0.3\linewidth]{plots/el_iso_njets3.pdf}
	\includegraphics[width=0.3\linewidth]{plots/el_id_njets4.pdf}%
	\includegraphics[width=0.3\linewidth]{plots/el_iso_njets4.pdf}
	\caption{
	  \label{fig:eltnpeff} Comparison of the electron identification and isolation efficiencies in data and MC for various jet multiplicity requirements. }  
      \end{center}
\end{figure}

\clearpage


\subsection{Trigger Efficiency Measurements}
\label{sec:trg}

In this section we measure the efficiencies of the single lepton triggers, HLT\_IsoMu24(\_eta2p1) for muons and HLT\_Ele27\_WP80 for electrons, using a tag-and-probe
approach. The tag is required to pass the full offline analysis selection and have \pt\ $>$ 30 GeV, $|\eta|<2.1$, and be matched to the single
lepton trigger. The probe is also required to pass the full offline analysis selection and have $|\eta|<2.1$, but the \pt\ requirement is relaxed to 20 GeV
in order to measure the \pt\ turn-on curve. The tag-probe pair is
required to have opposite-sign and an invariant mass in the range
76--106 GeV.

The measured trigger efficiencies are displayed in Fig.~\ref{fig:trigeff} and summarized in Table~\ref{tab:mutriggeff} (muons) and Table~\ref{tab:eltriggeff} (electrons).
These trigger efficiencies are applied to the MC when used to predict data yields selected by single lepton triggers.


\begin{figure}[!ht]
\begin{center}
\begin{tabular}{cc}
\includegraphics[width=0.4\textwidth]{plots/mutrig_20fb.pdf} &
\includegraphics[width=0.4\textwidth]{plots/eltrig_20fb.pdf} \\
\end{tabular}
\caption{\label{fig:trigeff}
Efficiency for the single muon trigger HLT\_IsoMu24(\_eta2p1) (left) and single electron trigger HLT\_Ele27\_WP80 (right) as a function of lepton \pt,
for several bins in lepton $|\eta|$.
}
\end{center}
\end{figure}

\clearpage

\begin{table}[htb]
\begin{center}
\footnotesize
\caption{\label{tab:mutriggeff}
Summary of the single muon trigger efficiency HLT\_IsoMu24(\_eta2p1). Uncertainties are statistical.}
\begin{tabular}{c|c|c|c}

%%% UPDATED WITH FULL 2012 DATA

%-------------------
%Doing muons
%-------------------
%USING MUON ETA BINS

%----------------------------------------------------------
% Selection            : (((((((((abs(tagAndProbeMass-91)<15)&&(qProbe*qTag<0))&&((eventSelection&2)==2))&&(HLT_IsoMu24_tag > 0))&&(tag->pt()>30.0))&&(abs(tag->eta())<2.1))&&(probe->pt()>20))&&(abs(probe->eta())<2.1))&&((leptonSelection&65536)==65536))&&((leptonSelection&131072)==131072)
% Probe trigger        : HLT_IsoMu24_probe > 0
% Total data yield 	 : 15561350
%----------------------------------------------------------

\hline
\hline
  \pt\ range [GeV] & $|\eta|<0.8$ & $0.8<|\eta|<1.5$ & $1.5<|\eta|<2.1$ \\
\hline
  20 -  22  & 	0.00 $\pm$ 0.000 & 	0.00 $\pm$ 0.000 & 	0.00 $\pm$ 0.000 \\
  22 -  24  & 	0.02 $\pm$ 0.001 & 	0.05 $\pm$ 0.001 & 	0.10 $\pm$ 0.001 \\
  24 -  26  & 	0.87 $\pm$ 0.001 & 	0.78 $\pm$ 0.001 & 	0.76 $\pm$ 0.002 \\
  26 -  28  & 	0.90 $\pm$ 0.001 & 	0.80 $\pm$ 0.001 & 	0.78 $\pm$ 0.001 \\
  28 -  30  & 	0.91 $\pm$ 0.001 & 	0.81 $\pm$ 0.001 & 	0.79 $\pm$ 0.001 \\
  30 -  32  & 	0.91 $\pm$ 0.001 & 	0.82 $\pm$ 0.001 & 	0.80 $\pm$ 0.001 \\
  32 -  34  & 	0.92 $\pm$ 0.000 & 	0.82 $\pm$ 0.001 & 	0.81 $\pm$ 0.001 \\
  34 -  36  & 	0.93 $\pm$ 0.000 & 	0.82 $\pm$ 0.001 & 	0.81 $\pm$ 0.001 \\
  36 -  38  & 	0.93 $\pm$ 0.000 & 	0.83 $\pm$ 0.001 & 	0.82 $\pm$ 0.001 \\
  38 -  40  & 	0.93 $\pm$ 0.000 & 	0.83 $\pm$ 0.001 & 	0.82 $\pm$ 0.001 \\
  40 -  50  & 	0.94 $\pm$ 0.000 & 	0.84 $\pm$ 0.000 & 	0.83 $\pm$ 0.000 \\
  50 -  60  & 	0.95 $\pm$ 0.000 & 	0.84 $\pm$ 0.001 & 	0.83 $\pm$ 0.001 \\
  60 -  80  & 	0.95 $\pm$ 0.000 & 	0.84 $\pm$ 0.001 & 	0.84 $\pm$ 0.001 \\
  80 - 100  & 	0.94 $\pm$ 0.001 & 	0.84 $\pm$ 0.002 & 	0.84 $\pm$ 0.003 \\
 100 - 150  & 	0.94 $\pm$ 0.002 & 	0.84 $\pm$ 0.003 & 	0.84 $\pm$ 0.004 \\
 150 - 200  & 	0.93 $\pm$ 0.004 & 	0.83 $\pm$ 0.007 & 	0.82 $\pm$ 0.010 \\
  $>$200    & 	0.92 $\pm$ 0.005 & 	0.83 $\pm$ 0.010 & 	0.83 $\pm$ 0.018 \\
\hline
\hline

\end{tabular}
\end{center}
\end{table}

\begin{table}[htb]
\begin{center}
\footnotesize
\caption{\label{tab:eltriggeff}
Summary of the single electron trigger efficiency HLT\_Ele27\_WP80. Uncertainties are statistical.}
\begin{tabular}{c|c|c}

%%% UPDATED WITH FULL 2012 DATA

%-------------------
%Doing electrons
%-------------------
%USING ELECTRON ETA BINS

%----------------------------------------------------------
% Selection            : (((((((((abs(tagAndProbeMass-91)<15)&&(qProbe*qTag<0))&&((eventSelection&1)==1))&&(HLT_Ele27_WP80_tag > 0))&&(tag->pt()>30.0))&&(abs(tag->eta())<2.1))&&(probe->pt()>20))&&(abs(probe->eta())<2.1))&&((leptonSelection&8)==8))&&((leptonSelection&16)==16)
% Probe trigger        : HLT_Ele27_WP80_probe > 0
% Total data yield 	 : 9620002
%----------------------------------------------------------

\hline
\hline
  \pt\ range [GeV] & $|\eta|<1.5$ & $1.5<|\eta|<2.1$ \\
\hline
  20 -  22  & 	0.00 $\pm$ 0.000 & 	0.00 $\pm$ 0.000 \\
  22 -  24  & 	0.00 $\pm$ 0.000 & 	0.00 $\pm$ 0.000 \\
  24 -  26  & 	0.00 $\pm$ 0.000 & 	0.03 $\pm$ 0.001 \\
  26 -  28  & 	0.07 $\pm$ 0.001 & 	0.22 $\pm$ 0.002 \\
  28 -  30  & 	0.57 $\pm$ 0.001 & 	0.52 $\pm$ 0.002 \\
  30 -  32  & 	0.85 $\pm$ 0.001 & 	0.65 $\pm$ 0.002 \\
  32 -  34  & 	0.88 $\pm$ 0.001 & 	0.70 $\pm$ 0.002 \\
  34 -  36  & 	0.89 $\pm$ 0.000 & 	0.72 $\pm$ 0.001 \\
  36 -  38  & 	0.91 $\pm$ 0.000 & 	0.74 $\pm$ 0.001 \\
  38 -  40  & 	0.92 $\pm$ 0.000 & 	0.75 $\pm$ 0.001 \\
  40 -  50  & 	0.94 $\pm$ 0.000 & 	0.77 $\pm$ 0.001 \\
  50 -  60  & 	0.95 $\pm$ 0.000 & 	0.79 $\pm$ 0.001 \\
  60 -  80  & 	0.96 $\pm$ 0.000 & 	0.79 $\pm$ 0.002 \\
  80 - 100  & 	0.96 $\pm$ 0.001 & 	0.80 $\pm$ 0.005 \\
 100 - 150  & 	0.97 $\pm$ 0.001 & 	0.82 $\pm$ 0.006 \\
 150 - 200  & 	0.97 $\pm$ 0.002 & 	0.83 $\pm$ 0.014 \\
 $>$200     & 	0.97 $\pm$ 0.003 & 	0.85 $\pm$ 0.020 \\
\hline
\hline


\end{tabular}
\end{center}
\end{table}

\clearpage

\section{Preselection yields}
\label{sec:yields}

Based on the event and trigger selections described in~Section \ref{sec:eventSelection}, 
we define a preselection as follows:
\begin{itemize}
\item Number of jets $\geq$ 2
\item Same flavor dileptons (opposite flavor yields will be shown since 
  they are used in data for \ttbar background estimation)
%\item All dilepton flavor combinations (SF as well as OF) %I think this is misleading
\item Dilepton mass within 10~GeV of the Z mass
%\item Dilepton mass within 15~GeV of the \Z mass
\end{itemize}

The resulting dilepton mass spectra for the $ee$ and $\mu\mu$ final states are shown in 
Figure~\ref{fig:dilmass}.
For all MC plots and tables, the yields are normalized to \lumi\ using the cross-sections
from Reference~\cite{ref:xsec} 
%assuming 100\% 
using trigger efficiency as described in section \ref{sec:trigSel}.
%Add statement about overflow
For these plots and all others in this note, the last bin contains the overflow.

\begin{figure}[hbt]
  \begin{center}
	\includegraphics[width=0.48\linewidth]{plots/hdilmass_ee_allj.pdf}
	\includegraphics[width=0.48\linewidth]{plots/hdilmass_mm_allj.pdf}
	\caption{
	  \label{fig:dilmass}\protect 
	  Dilepton mass distribution for events passing the pre-selection for \lumi~
	  in the $ee$ (left) and $\mu\mu$ (right) final states. Backgrounds from 
	  single top and $W$+jets are omitted
	  since they are negligible.}
  \end{center}
\end{figure}


%\begin{figure}[htb]
%  \begin{center}
%    \resizebox{0.6\linewidth}{!}{\includegraphics{plots/dilmass_34pb.png}}
%    \caption{ \label{fig:dilmass} Dilepton mass distribution for events passing the pre-selection for \lumi.}
%  \end{center}
%\end{figure}


The data yields and the MC predictions are given in Table~\ref{preselyieldtable}.
Dilepton mass and MET distributions for \emu ~events are shown in appendix \ref{app:emu}.

As anticipated, the MC predicts that the preselection is dominated by Z+jets in the same-flavor 
case and by \ttbar in the opposite-flavor case.  
%The data yield is in reasonable agreement with the predictions for the $ee$, $\mu\mu$ and $e\mu$ channels.
We also show the %LO 
next-to-leading order (NLO) 
yields for the LM4 and LM8 processes, which are benchmark
SUSY processes in which $Z$ bosons are produced via cascade decays of SUSY particles. 

%, while an excess of 
%data events is observed in the $\mu\mu$ case. 
%We believe that this is due to a slightly harder njet distribution in data vs MC, combined with a slight 
%deficiency of electron yield with respect to MC. {\bf Reevaluate w new mc}

%I'm replacing below with above

%{\bf NOTE: This excess was not understood, and we investigated it a bit.
%In the 11/pb sample it is a difference between the Njet spectra between e/mu. It mostly went away in the 35/pb sample. We do see a difference in yield in the later data between e/mu, though - To be investigated. Maybe reReco will fix that. }

\begin{table}[htb]
\begin{center}
\caption{\label{preselyieldtable} Data and Monte Carlo yields for the preselection for \lumi. 
  The NLO yields for the SUSY benchmark processes LM4 and LM8 are also shown.}
\begin{tabular}{lccccc}
\hline
              Sample   &                $ee$   &            $\mu\mu$   &              $e\mu$   &         tot  \\
\hline
       Z+Jets & 8805.8 $\pm$  101.5  &  8991.1 $\pm$   97.3  &     5.6 $\pm$    2.5  &  17802.5 $\pm$  140.6 \\ 
       \ttbar &  117.7 $\pm$    4.1  &   111.8 $\pm$    3.8  &   233.6 $\pm$    5.7  &   463.1 $\pm$    8.0 \\ 
        WJets &    4.1 $\pm$    2.9  &     0.0 $\pm$    0.0  &     1.9 $\pm$    1.9  &     6.0 $\pm$    3.5 \\ 
           WW &    1.1 $\pm$    0.2  &     1.9 $\pm$    0.3  &     2.7 $\pm$    0.3  &     5.7 $\pm$    0.5 \\ 
           WZ &   66.1 $\pm$    0.7  &    68.0 $\pm$    0.7  &     0.3 $\pm$    0.0  &   134.4 $\pm$    1.0 \\ 
           ZZ &   47.7 $\pm$    0.4  &    48.5 $\pm$    0.3  &     0.1 $\pm$    0.0  &    96.2 $\pm$    0.5 \\ 
   Single Top &    3.5 $\pm$    0.3  &     3.3 $\pm$    0.3  &     7.7 $\pm$    0.4  &    14.5 $\pm$    0.5 \\ 
\hline
     Total MC & 9046.0 $\pm$  101.6  &  9224.6 $\pm$   97.4  &   251.8 $\pm$    6.5  &  18522.3 $\pm$  140.9 \\ 
\hline
         Data &  10100  &   10869  &     299  &   21268 \\ 
\hline
          LM4 &   13.6 $\pm$    0.4  &    12.9 $\pm$    0.4  &     2.0 $\pm$    0.2  &    28.4 $\pm$    0.6 \\ 
          LM8 &    6.2 $\pm$    0.2  &     6.2 $\pm$    0.2  &     2.0 $\pm$    0.1  &    14.3 $\pm$    0.3 \\ 
\hline
\end{tabular}
\end{center}
\end{table}



\begin{comment}
204/pb

       Z+Jets & 1840.566 $\pm$ 21.213  &  2088.019 $\pm$ 22.592  &   1.467 $\pm$  0.599  &  3930.052 $\pm$ 30.996 \\ 
   $t\bar{t}$ & 24.515 $\pm$  0.860  &  25.965 $\pm$  0.885  &  51.295 $\pm$  1.244  &  101.775 $\pm$  1.753 \\ 
        WJets &  0.852 $\pm$  0.603  &   0.000 $\pm$  0.000  &   0.426 $\pm$  0.426  &   1.278 $\pm$  0.738 \\ 
           WW &  0.217 $\pm$  0.043  &   0.442 $\pm$  0.061  &   0.593 $\pm$  0.070  &   1.252 $\pm$  0.102 \\ 
           WZ & 13.947 $\pm$  0.157  &  16.001 $\pm$  0.168  &   0.111 $\pm$  0.014  &  30.059 $\pm$  0.230 \\ 
           ZZ & 10.005 $\pm$  0.076  &  11.364 $\pm$  0.082  &   0.022 $\pm$  0.004  &  21.391 $\pm$  0.112 \\ 
   Single Top &  0.725 $\pm$  0.057  &   0.778 $\pm$  0.059  &   1.694 $\pm$  0.088  &   3.196 $\pm$  0.120 \\ 
\hline
     Total MC & 1890.827 $\pm$ 21.240  &  2142.569 $\pm$ 22.610  &  55.607 $\pm$  1.450  &  4089.003 $\pm$ 31.056 \\ 
\hline
         Data &   2051                 &    2277                 &      66               &    4394 \\ 
\hline
          LM4 &  2.842 $\pm$  0.086  &   2.986 $\pm$  0.087  &   0.434 $\pm$  0.036  &   6.262 $\pm$  0.127 \\ 
          LM8 &  1.286 $\pm$  0.037  &   1.443 $\pm$  0.039  &   0.430 $\pm$  0.023  &   3.159 $\pm$  0.059 \\ 

%LO
%          LM4 &  2.027 $\pm$  0.060  &   2.175 $\pm$  0.062  &   0.291 $\pm$  0.023  &   4.493 $\pm$  0.089 \\ 
%          LM8 &  0.889 $\pm$  0.025  &   1.004 $\pm$  0.026  &   0.273 $\pm$  0.014  &   2.167 $\pm$  0.038 \\ 

\end{comment}
\section{Definition of the signal regions}
\label{sec:sigregion}

We define signal regions to look for possible
new physics contributions by adding the requirement of large MET to the preselection. 
Our choice of MET requirements to define the signal regions is driven by the 
MET distributions expected from $Z$ and $t\bar{t}$ MC with the preselection applied, 
as shown in Fig.~\ref{fig:metdist}. The equivalent luminosity of the $Z$ MC is 
approximately 834 pb$^{-1}$, and that of the \ttbar sample is approximately 6.8 fb$^{-1}$.

\begin{figure}[tbh]
  \begin{center}
	\includegraphics[width=0.75\linewidth]{plots/met_ttbar_Z.pdf}
	\caption{
	  \label{fig:metdist}\protect 
	  Distributions of MET in $Z$ and $t\bar{t}$ MC normalized to 1~fb$^{-1}$.}
  \end{center}
\end{figure}


In addition to our two signal discussed below regions, 
we use the regions MET $>$ 30 and MET $>$ 60 
as very loose signal regions. The mass distributions for these regions are shown in
figures \ref{fig:dilmass30} and \ref{fig:dilmass60}.
For all mass plots in this section, backgrounds from single top 
and $W$+jets are omitted since they are negligible.

\begin{figure}[hbt]
  \begin{center}
	\includegraphics[width=0.48\linewidth]{plots/hdilmass_pfmet30_ee_allj.pdf}
	\includegraphics[width=0.48\linewidth]{plots/hdilmass_pfmet30_mm_allj.pdf}
	\caption{
	  \label{fig:dilmass30}\protect 
	  Dilepton mass distribution for events passing the pre-selection 
	  and MET $>$ 30 GeV for \lumi\ in the $ee$ (left) and $\mu\mu$ (right) final states. 
	  %Backgrounds from single top and $W$+jets are omitted since they are negligible.
	}
  \end{center}
\end{figure}

\begin{figure}[hbt]
  \begin{center}
	\includegraphics[width=0.48\linewidth]{plots/hdilmass_pfmet60_ee_allj.pdf}
	\includegraphics[width=0.48\linewidth]{plots/hdilmass_pfmet60_mm_allj.pdf}
	\caption{
	  \label{fig:dilmass60}\protect 
	  Dilepton mass distribution for events passing the pre-selection 
	  and MET $>$ 60 GeV for \lumi\ in the $ee$ (left) and $\mu\mu$ (right) final states. 
	  %Backgrounds from single top and $W$+jets are omitted since they are negligible.
	}
  \end{center}
\end{figure}


We define two signal regions for our search:

\begin{itemize}
\item MET $>$ \signalmetl GeV (loose signal region):
  %In this region of MET there is a contribution from the tail of the MET distribution 
  In this region of MET there is a small contribution from the tail of the MET distribution 
  in \Z plus jets events. 
  %There is also a contribution from \ttbar events where the leptons happen to be in the \Z 
  The bulk of the events in this region are from \ttbar events where the leptons happen to be in the \Z 
  mass window.

  %We expect the MC simulation to do a good job on the second source, as it is well within the 
  %bulk of the \ttbar phase space. However, for the first we must rely on data driven procedures. 

  The MC and data yields for this signal region are given in Table~\ref{sigyieldtableloose} and the dilepton
  mass distributions are shown in Fig.~\ref{fig:dilmass100}.

  %until i fix the lumi, don't know what signal events i have

  %More information on the data events in this signal region is given in Table~\ref{sig60events} and information 
  %on the muons in these events is given in Table~\ref{sig60muons}.

\item MET $>$ \signalmett GeV (tight signal region):
  %Though this wording worked when we had a fixed lumi, it doesn't work when we don't know how much we'll get by the end of the year. Just reword.
  This signal region was selected by picking a region where the SM 
  % prediction for the dataset we have is  $\approx$ 1 event. 
  expectation is very low.
  At this kinematical region the dominant background contribution is expected to be from \ttbar.

  The MC and data yields for this signal region are given in Table~\ref{sigyieldtabletight}.

\end{itemize}

In the two signal regions above, the dominant background is \ttbar. 
However, it is still essential to have a data driven estimation of the \Z contribution
in the signal regions to demonstrate that we understand our background composition (see section \ref{sec:templates}).
This is important both in case when an excess is observed or when placing a limit on new physics.

%It's right below this, so why bother -- just put the reference in above
%The data driven technique used to predict the missing transverse energy accompanying 
%a \Z event is described in Section~\ref{sec:templates}.

To estimate the \ttbar background we will use the opposite flavor subtraction
described in Section~\ref{sec:topbkg}.


\begin{figure}[hbt]
  \begin{center}
	\includegraphics[width=0.48\linewidth]{plots/hdilmass_pfmet100_ee_allj.pdf}
	\includegraphics[width=0.48\linewidth]{plots/hdilmass_pfmet100_mm_allj.pdf}
	\caption{
	  \label{fig:dilmass100}\protect 
	  Dilepton mass distribution for events passing the pre-selection 
	  and MET $>$ \signalmetl~GeV for \lumi\ in the $ee$ (left) and $\mu\mu$ (right) final states. 
	  %Backgrounds from single top and $W$+jets are omitted since they are negligible.
	}
  \end{center}
\end{figure}


\begin{table}[htb]
\begin{center}
\caption{\label{sigyieldtableloose} Data and Monte Carlo yields for the loose signal region MET $>$ \signalmetl~GeV  for \lumi.}
\begin{tabular}{lccccc}
\hline
     Sample   &                $ee$   &            $\mu\mu$   &              $e\mu$   &                 tot  \\
\hline
       Z+Jets &  0.489 $\pm$  0.346  &   0.000 $\pm$  0.000  &   0.000 $\pm$  0.000  &   0.489 $\pm$  0.346 \\ 
       \ttbar &  3.714 $\pm$  0.335  &   4.317 $\pm$  0.361  &   8.454 $\pm$  0.505  &  16.485 $\pm$  0.705 \\ 
        WJets &  0.000 $\pm$  0.000  &   0.000 $\pm$  0.000  &   0.000 $\pm$  0.000  &   0.000 $\pm$  0.000 \\ 
           WW &  0.025 $\pm$  0.014  &   0.058 $\pm$  0.022  &   0.100 $\pm$  0.029  &   0.184 $\pm$  0.039 \\ 
           WZ &  0.155 $\pm$  0.017  &   0.144 $\pm$  0.016  &   0.004 $\pm$  0.002  &   0.303 $\pm$  0.023 \\ 
           ZZ &  0.088 $\pm$  0.007  &   0.096 $\pm$  0.007  &   0.000 $\pm$  0.000  &   0.184 $\pm$  0.010 \\ 
   Single Top &  0.102 $\pm$  0.021  &   0.133 $\pm$  0.024  &   0.256 $\pm$  0.034  &   0.490 $\pm$  0.047 \\ 
\hline
     Total MC &  4.572 $\pm$  0.482  &   4.749 $\pm$  0.363  &   8.814 $\pm$  0.507  &  18.134 $\pm$  0.788 \\ 
\hline
         Data &      7               &       7               &      12               &      26 \\ 
\hline
          LM4 &  1.522 $\pm$  0.052  &   1.603 $\pm$  0.053  &   0.232 $\pm$  0.020  &   3.356 $\pm$  0.077 \\ 
          LM8 &  0.597 $\pm$  0.020  &   0.686 $\pm$  0.022  &   0.219 $\pm$  0.012  &   1.502 $\pm$  0.032 \\ 
\hline
\end{tabular}
\end{center}
\end{table}


\begin{table}[htb]
\begin{center}
  \caption{
	\label{sigyieldtabletight} 
	Data and Monte Carlo yields for the tight signal region MET $>$ \signalmett~GeV  for \lumi.}
\begin{tabular}{lccccc}
\hline
     Sample   &                $ee$   &            $\mu\mu$   &              $e\mu$   &                 tot  \\
\hline
       Z+Jets &  0.000 $\pm$  0.000  &   0.000 $\pm$  0.000  &   0.000 $\pm$  0.000  &   0.000 $\pm$  0.000 \\ 
     \ttbar   &  0.181 $\pm$  0.074  &   0.181 $\pm$  0.074  &   0.483 $\pm$  0.121  &   0.845 $\pm$  0.160 \\ 
        WJets &  0.000 $\pm$  0.000  &   0.000 $\pm$  0.000  &   0.000 $\pm$  0.000  &   0.000 $\pm$  0.000 \\ 
           WW &  0.000 $\pm$  0.000  &   0.008 $\pm$  0.008  &   0.008 $\pm$  0.008  &   0.017 $\pm$  0.012 \\ 
           WZ &  0.025 $\pm$  0.007  &   0.021 $\pm$  0.006  &   0.000 $\pm$  0.000  &   0.046 $\pm$  0.009 \\ 
           ZZ &  0.013 $\pm$  0.003  &   0.013 $\pm$  0.003  &   0.000 $\pm$  0.000  &   0.026 $\pm$  0.004 \\ 
   Single Top &  0.000 $\pm$  0.000  &   0.013 $\pm$  0.008  &   0.009 $\pm$  0.006  &   0.022 $\pm$  0.010 \\ 
          LM4 &  0.922 $\pm$  0.040  &   0.969 $\pm$  0.041  &   0.128 $\pm$  0.015  &   2.019 $\pm$  0.060 \\ 
          LM8 &  0.345 $\pm$  0.015  &   0.387 $\pm$  0.016  &   0.144 $\pm$  0.010  &   0.876 $\pm$  0.024 \\ 
\hline
     Total MC &  0.219 $\pm$  0.074  &   0.237 $\pm$  0.075  &   0.500 $\pm$  0.121  &   0.956 $\pm$  0.161 \\ 
\hline
         Data &      0               &       2               &       2               &       4 \\ 
\hline
          LM4 &  0.922 $\pm$  0.040  &   0.969 $\pm$  0.041  &   0.128 $\pm$  0.015  &   2.019 $\pm$  0.060 \\ 
          LM8 &  0.345 $\pm$  0.015  &   0.387 $\pm$  0.016  &   0.144 $\pm$  0.010  &   0.876 $\pm$  0.024 \\ 
\hline
\end{tabular}
\end{center}
\end{table}



%2011 event table

\begin{table}[htb]
\begin{center}
\caption{\label{sig60events} Details of data events for the loose signal region 
  MET $>$ \signalmetl GeV for \lumi. The SimpleSecondaryVertexTagger high efficiency medium 
  working point is used for b-tagging.}

  %IS THIS STILL TRUE???
  %, for which the efficiency is $\sim40$\% and the mistag rate is $\sim1$\%.}
  \begin{tabular}{rrrrrrrrrrr}

	\hline
Run & Lumi & Event & Lep & Njet & N B Tag & pfMET & tcMET & Dilep & Sum & Z \pt\\
 &  Section &  & Type &  &  &  &  & Mass  & jet \pt & \\
\hline
161312 & 963 &  366557890 &    $e$ & 4 & 2 & 117.1 & 117.9 &  89.0 & 218.9 &  18.1\\
163069 & 432 &  236566401 &    $e$ & 2 & 1 & 107.4 &  92.3 &  83.6 & 213.9 &  90.9\\
163758 &  86 &   64208683 &    $e$ & 2 & 0 & 119.5 & 114.1 &  82.6 & 114.0 &  58.6\\
163758 &  63 &   46610542 &    $e$ & 4 & 0 & 124.1 &  41.4 &  91.1 & 300.1 & 261.9\\
163332 & 615 &  415350358 &    $e$ & 4 & 0 & 115.6 & 108.6 & 100.1 & 316.7 & 108.3\\
163332 & 730 &  489406176 &    $e$ & 2 & 0 & 100.4 &  88.4 &  92.6 &  98.6 &  52.8\\
163817 & 680 &  633297354 &    $e$ & 4 & 0 & 165.0 & 155.0 &  89.7 & 437.8 &  26.0\\
163659 & 320 &  243264615 &  $\mu$ & 2 & 0 & 150.0 & 130.2 &  86.1 & 192.7 &  71.1\\
163663 & 181 &  126212285 &  $\mu$ & 3 & 1 & 102.3 &  79.9 &  85.4 & 210.5 &  49.8\\
163584 &  54 &   39429230 &  $\mu$ & 3 & 0 & 209.6 & 217.4 &  92.5 & 349.6 &  29.2\\
160873 & 115 &   30074254 &  $\mu$ & 2 & 1 & 102.8 &  87.8 &  89.1 &  89.9 &  78.5\\
163255 & 673 &  435619707 &  $\mu$ & 2 & 2 & 169.4 & 160.6 &  96.0 & 140.4 & 108.0\\
163759 &  80 &   52493009 &  $\mu$ & 2 & 0 & 239.3 & 229.0 &  87.5 & 198.7 & 109.4\\
163233 &   8 &    3963812 &  $\mu$ & 2 & 0 & 106.2 & 103.3 &  82.9 & 123.0 &  58.0\\

	\hline
  \end{tabular}
\end{center}
\end{table}




%%%%%%%%%%%%%%%%%%%%%%%%%%%%%%%%%%%%%%%%%

%2010

%%%%%%%%%%%%%%%%%%%%%%%%%%%%%%%%%%%%%%%%%

\begin{comment}

%2010 loose--met 30
%%%official PVT json v3, 38X MC 
              Sample   &                $ee$   &            $\mu\mu$   &              $e\mu$   &                 tot  \\
\hline
               ZJets   &     0.21 $\pm$ 0.09   &     0.21 $\pm$ 0.09   &     0.04 $\pm$ 0.04   &     0.46 $\pm$ 0.14  \\
               TTbar   &     1.89 $\pm$ 0.09   &     2.04 $\pm$ 0.09   &     4.17 $\pm$ 0.13   &     8.11 $\pm$ 0.18  \\
               WJets   &     0.00 $\pm$ 0.00   &     0.00 $\pm$ 0.00   &     0.00 $\pm$ 0.00   &     0.00 $\pm$ 0.00  \\
                  WW   &     0.01 $\pm$ 0.00   &     0.02 $\pm$ 0.00   &     0.03 $\pm$ 0.01   &     0.06 $\pm$ 0.01  \\
                  WZ   &     0.06 $\pm$ 0.00   &     0.06 $\pm$ 0.00   &     0.00 $\pm$ 0.00   &     0.12 $\pm$ 0.00  \\
                  ZZ   &     0.03 $\pm$ 0.00   &     0.03 $\pm$ 0.00   &     0.00 $\pm$ 0.00   &     0.06 $\pm$ 0.00  \\
                  tW   &     0.06 $\pm$ 0.01   &     0.06 $\pm$ 0.01   &     0.14 $\pm$ 0.01   &     0.26 $\pm$ 0.01  \\
\hline
           tot SM MC   &     2.26 $\pm$ 0.13   &     2.43 $\pm$ 0.13   &     4.39 $\pm$ 0.13   &     9.07 $\pm$ 0.22  \\
\hline
                data   &                   0   &                   7   &                   3   &                  10  \\
\hline
                 LM4   &     0.42 $\pm$ 0.01   &     0.44 $\pm$ 0.01   &     0.07 $\pm$ 0.01   &     0.93 $\pm$ 0.02  \\
                 LM8   &     0.19 $\pm$ 0.01   &     0.22 $\pm$ 0.01   &     0.07 $\pm$ 0.00   &     0.48 $\pm$ 0.01  \\


%2010 tight--met 60
%%%official PVT json v3, 38X MC
              Sample   &                $ee$   &            $\mu\mu$   &              $e\mu$   &                 tot  \\
\hline
               ZJets   &     0.00 $\pm$ 0.00   &     0.00 $\pm$ 0.00   &     0.00 $\pm$ 0.00   &     0.00 $\pm$ 0.00  \\
               TTbar   &     0.26 $\pm$ 0.03   &     0.28 $\pm$ 0.03   &     0.55 $\pm$ 0.05   &     1.09 $\pm$ 0.06  \\
               WJets   &     0.00 $\pm$ 0.00   &     0.00 $\pm$ 0.00   &     0.00 $\pm$ 0.00   &     0.00 $\pm$ 0.00  \\
                  WW   &     0.00 $\pm$ 0.00   &     0.00 $\pm$ 0.00   &     0.00 $\pm$ 0.00   &     0.01 $\pm$ 0.00  \\
                  WZ   &     0.01 $\pm$ 0.00   &     0.01 $\pm$ 0.00   &     0.00 $\pm$ 0.00   &     0.02 $\pm$ 0.00  \\
                  ZZ   &     0.01 $\pm$ 0.00   &     0.01 $\pm$ 0.00   &     0.00 $\pm$ 0.00   &     0.02 $\pm$ 0.00  \\
                  tW   &     0.00 $\pm$ 0.00   &     0.01 $\pm$ 0.00   &     0.02 $\pm$ 0.00   &     0.03 $\pm$ 0.00  \\
\hline
           tot SM MC   &     0.29 $\pm$ 0.03   &     0.31 $\pm$ 0.03   &     0.58 $\pm$ 0.05   &     1.17 $\pm$ 0.07  \\
\hline
                data   &                   0   &                   0   &                   0   &                   0  \\
\hline
                 LM4   &     0.33 $\pm$ 0.01   &     0.35 $\pm$ 0.01   &     0.06 $\pm$ 0.01   &     0.74 $\pm$ 0.02  \\
                 LM8   &     0.14 $\pm$ 0.01   &     0.16 $\pm$ 0.01   &     0.06 $\pm$ 0.00   &     0.36 $\pm$ 0.01  \\

\end{comment}


%The 2010 tables

%\begin{table}[htb]
%\begin{center}
%\caption{\label{sig60events} Details of data events for the loose signal region MET $>$ \signalmetl~GeV for \lumi. 
%All 7 events are in the $\mu\mu$ final states. The SimpleSecondaryVertexTagger high efficiency medium 
%working point is used for b-tagging, for which the efficiency is $\sim40$\% and the mistag rate is $\sim1$\%.}
%\begin{tabular}{rrrrrrrrrr}
%
%\hline
%
%Run & Lumi & Event & Njet & N B Tag & pfMET & tcMET & Dilep Mass & Sum & Z \pt\\
% &  Section &  &  &  &  &  &  & jet \pt &  \\
%\hline
%147216 & 48 & 35885648 & 2 & 0 & 78.9 & 72.9 & 95.0 & 216.4 & 116.4\\
%147217 & 75 & 55188718 & 2 & 0 & 79.7 & 67.8 & 90.7 & 75.7 & 39.1\\
%147450 & 82 & 29253181 & 5 & 0 & 63.6 & 70.8 & 97.9 & 429.5 & 312.0\\
%148862 & 350 & 522383338 & 4 & 1 & 90.0 & 75.7 & 82.4 & 373.4 & 87.0\\
%149181 & 1769 & 1675896175 & 2 & 1 & 67.8 & 64.4 & 97.2 & 163.9 & 128.6\\
%149291 & 205 & 199787369 & 4 & 0 & 74.5 & 92.9 & 85.0 & 303.1 & 64.3\\
%149291 & 232 & 235101408 & 2 & 1 & 87.3 & 90.0 & 88.7 & 315.4 & 32.4\\
%
%\hline
%\end{tabular}
%\end{center}
%\end{table}
%
%
%\begin{table}[htb]
%\begin{center}
%\caption{\label{sig60muons} Details of the muons in events in the loose signal region MET $>$ \signalmetl~GeV for \lumi. 
%Shown are the transverse momentum, the relative error in the transverse momentum, d0 calculated with respect to the beamspot,
%the number of hits in the Silicon track, the number of layers crossed by the Silicon track, the normalized $\chi^2$,
%whether a muon segment was found in the last chamber traversed by the muon, and the number of hits in the muon chambers used in the global fit. }
%\begin{tabular}{rrrrrrrrrr}
%\hline
%
%Event & Si \pt &  Si \pt & gfit \pt & d0 & N Si Hits & N Si & Si $\chi^2$ & TMLast & gfit STA \\
% &  &  rel err &  &  &  &  Layers &  & Station & hits \\
% &  &  &  &  &  &  &  & Loose &  \\
%\hline
%35885648 & 110.2 & 0.030 & 111.3 & -0.001 & 29 & 15 & 0.65 & 1 & 18 \\
%35885648 & 27.5 & 0.012 & 27.5 & 0.005 & 16 & 12 & 0.26 & 1 & 18 \\
%55188718 & 22.4 & 0.017 & 22.4 & -0.003 & 15 & 10 & 0.93 & 1 & 25 \\
%55188718 & 46.7 & 0.016 & 46.9 & 0.007 & 19 & 13 & 0.79 & 1 & 35 \\
%29253181 & 71.2 & 0.026 & 70.5 & -0.011 & 25 & 15 & 0.78 & 1 & 13 \\
%29253181 & 255.7 & 0.068 & 301.0 & -0.011 & 24 & 15 & 0.69 & 1 & 13 \\
%522383338 & 89.1 & 0.018 & 89.4 & -0.000 & 16 & 12 & 0.43 & 1 & 26 \\
%522383338 & 27.1 & 0.011 & 27.1 & -0.004 & 16 & 12 & 0.13 & 1 & 36 \\
%1675896175 & 80.3 & 0.017 & 80.1 & 0.002 & 19 & 13 & 0.60 & 1 & 32 \\
%1675896175 & 77.6 & 0.016 & 77.5 & -0.000 & 17 & 13 & 0.21 & 1 & 25 \\
%199787369 & 22.6 & 0.018 & 22.5 & -0.000 & 21 & 14 & 0.59 & 1 & 12 \\
%199787369 & 82.0 & 0.033 & 80.9 & -0.002 & 12 & 10 & 0.61 & 1 & 14 \\
%235101408 & 30.6 & 0.014 & 30.9 & -0.007 & 16 & 11 & 0.14 & 1 & 29 \\
%235101408 & 59.7 & 0.013 & 59.9 & 0.006 & 20 & 13 & 0.37 & 1 & 16 \\
%
%
%\hline
%\end{tabular}
%\end{center}
%\end{table}

\clearpage


\section{MET Templates}
\label{sec:templates}

The premise of this data driven technique is that MET in \Z plus jets events
is produced by the hadronic recoil system and {\it not} by the leptons making up the \Z.
Therefore, the basic idea of the MET template method is to measure the MET distribution in 
a control sample which has no true MET and the same general attributes regarding
fake MET as in \Z plus jets events.
In our case, we choose a photon-like sample. These are not necessarily photons, but jets with predominantly 
electromagnetic energy deposition in a good fiducial volume. This ensures that 
they are well measured and do not contribute to fake MET.

Both the control sample and the \Z plus jets background
consist of a well measured object (either a photon or a leptonically decaying \Z) which recoils 
against a system of hadronic jets. The MET in these events 
is then dominated by mismeasurements of the hadronic system. To account for kinematic 
differences between the hadronic systems in the control vs. signal samples, 
we measure the MET distributions in the control sample in bins of the number of jets and the 
scalar sum of jet \pt. 

In order to track conditions which change over the course of data-taking
(most notably the increased pile-up) and to increase the statistics at large photon \pt, we
use templates obtained from multiple photon triggers ``stitched'' together. 
Each photon event enters the template for the
highest \pt photon trigger which fired in the event. The resulting MET 
distributions in each bin of njets, sumjetpt and photon trigger are then 
normalized to unit area, yielding an array of MET templates. 

Each \Z event is then assigned one such 
unit area template based on its number of jets, the scalar sum of jet \pt and the \Z \pt. 
The \Z \pt enters only in choosing which bin of photon trigger to use. 
This is done because the \Z \pt is the analogous variable in the \Z sample 
to the photon \pt. 
In order to account for the different prescales of the different photon triggers (which 
change during running), 
each template is filled with a weight equal to the prescale of the corresponding trigger.
After this procedure, the templates are normalized to unity.
If this is not done, the relative contribution of different parts of the photon
\pt specturm will not be sampled properly. In other words, this procedure corrects and 
smoothes the photon \pt distribution.% so that it matches the \Z \pt distribution.
The ranges of \Z \pt chosen for each photon trigger 
take into account the turn-on of the photon trigger and are shown below.
%{\bf Claudio: currently the Z pt is used to determine which trigger template set to use. You argued earlier that instead
%\bf we should use the pt of the hadronic recoil. I have verified that this makes a negligible difference, but if you prefer,
%\bf I can switch to using hadronic recoil pt.} 

%Below was commented out, but since I notice this procedure has a large effect due to the large prescale, I think it's important to include this text.

%In order to select a photon sample whose \pt distribution matches the distribution of \Z \pt
% in our preselection, a combination of photon triggers is used.
% must be used (see below). 
%In order to account for varying prescales of the triggers in different runs,
% separate templates are created for each trigger line. 
%Each event enters the template for the highest \pt trigger which fires in the event,
% and the template is filled with a weight equal to the prescale of the corresponding trigger.
%After this procedure, the templates are normalized to unity.

The sum of the templates for all selected \Z events then forms the 
prediction of the MET distribution for the \Z sample. Integrating this prediction for our 
signal regions  thus provides a data driven prediction for the \Z plus jets yields in the 
signal regions. 
%The prediction is then formed by selecting the photon trigger based on the \pt of the \Z in the event which is being predicted

%The MET templates are derived from a set of photon triggered samples. 
The Njet binning used is 2 jets and $\ge$ 3 jets. 
The sum jet \pt binning is defined by the boundaries {0, 30, 60, 90, 120, 150, 250, 5000}~GeV.
The photon triggers, and the ranges of \Z \pt for which they are selected, are (where the ``\verb=v*='' stands for a verion number which increments with trigger menu changes):

\begin{itemize}
\item \verb=HLT_Photon20_CaloIdVL_IsoL_v*= (\Z \pt $<$ 33~GeV)
\item \verb=HLT_Photon30_CaloIdVL_IsoL_v*= (33 GeV $<$ \Z \pt $<$ 55~GeV)
\item \verb=HLT_Photon50_CaloIdVL_IsoL_v*= (55 GeV $<$ \Z \pt $<$ 81~GeV)
\item \verb=HLT_Photon75_CaloIdVL_IsoL_v*= (\Z \pt $>$ 81~GeV)
\end{itemize}

The HLT requirements (as described in \cite{ref:eghlt}) on the above triggers are listed in appendix~\ref{app:photrig}.

%\begin{itemize}
%\item CaloIdVL: \\
%  H/E $<$ 0.15 (0.1), $\sigma_{i\eta i\eta} <$ 0.024 (0.04) in barrel (endcap)
%\item IsoL: \\
%  Ecal ET $<$ 5.5 + 0.012*ET, %\\
%  Hcal ET $<$ 3.5 + 0.005*ET, %\\
%  Trk PT $<$ 3.5 + 0.002*ET 
%\end{itemize}

%last year for the thresholds we used trigger value +2. This year I observe a bit slower turn-on:

%  //notes on turn on curves from drawing photon pts from data babies
%  //the 20 trig is nearly fully efficient at 21, but use 22 to be conservative
%  //the 30 trig is nearly fully efficient at 32, but use 33 to be conservative
%  //the 50 trig is nearly fully efficient at 54, but use 55 to be conservative
%  //the 75 trig is nearly fully efficient at 80, but use 81 to be conservative

We show all the templates used in appendix~\ref{sec:appendix_templates}.


\section{Closure Test of Templates in MC}
\label{sec:mc}

The above procedure is applied to MC to test its effectiveness under `ideal' conditions. Templates are derived from PhotonJet MC, 
and these templates are used to predict the MET distribution in ZJets MC. The MC samples used are:
\begin{itemize}
\item PhotonJet MC
  \begin{itemize}
  \item \verb=/G_Pt_15to30_TuneZ2_7TeV_pythia6/Spring11-PU_S1_START311_V1G1-v1/AODSIM  =
  \item \verb=/G_Pt_30to50_TuneZ2_7TeV_pythia6/Spring11-PU_S1_START311_V1G1-v1/AODSIM  =
  \item \verb=/G_Pt_50to80_TuneZ2_7TeV_pythia6/Spring11-PU_S1_START311_V1G1-v1/AODSIM  =
  \item \verb=/G_Pt_80to120_TuneZ2_7TeV_pythia6/Spring11-PU_S1_START311_V1G1-v1/AODSIM =
  \item \verb=/G_Pt_120to170_TuneZ2_7TeV_pythia6/Spring11-PU_S1_START311_V1G1-v1/AODSIM=
  \item \verb=/G_Pt_170to300_TuneZ2_7TeV_pythia6/Spring11-PU_S1_START311_V1G1-v1/AODSIM= 	  
  \end{itemize}
\item ZJet MC
  \begin{itemize}
  \item \verb=/DYToEE_M-20_CT10_TuneZ2_7TeV-powheg-pythia/Spring11-PU_S1_START311_V1G1-v1/AODSIM=
  \item \verb=/DYToMuMu_M-20_CT10_TuneZ2_7TeV-powheg-pythia/Spring11-PU_S1_START311_V1G1-v1/AODSIM=
  \item \verb=/DYToTauTau_M-20_CT10_TuneZ2_7TeV-powheg-pythia-tauola/Spring11-PU_S1_START311_V1G1-v1/AODSIM=
	%use pythia
	%\item \verb=/DYJetsToLL_TuneD6T_M-50_7TeV-madgraph-tauola/Spring11-PU_S1_START311_V1G1-v1/AODSIM=
  \end{itemize}
\end{itemize}

Good agreement between the observed and predicted MET distributions is observed, as shown in Fig.~\ref{fig:mcclosure}.

\begin{figure}[hbt]
  \begin{center}
    \resizebox{0.8\linewidth}{!}{\includegraphics{plots/mcclosure.png}}
	\\ \medskip
    %\resizebox{\linewidth}{!}{
    \begin{tabular}{r|r|r|r|r}
      MET        & $>$ 30 GeV       & $>$ 60 GeV        & $>$ 100 GeV       & $>$ 200  \\ \hline
	  Z MC       &   919            &    15             &     1             &     0 \\
	  Prediction & 950.35 $\pm$  2.37 &  17.89 $\pm$   0.07 &   2.44 $\pm$   0.02 &   0.11 $\pm$   0.01 \\

    \end{tabular}
	%}
	\\ \medskip
    \caption{The MET distribution in Z+jets MC (black) and prediction (blue) for Njet $\ge$ 2. 
	  Below the plot is tabulated the integral of the observed Z+jets MC MET and the predicted 
	  MET from $\gamma$+jets MC for 
	  MET $>$ 30 GeV, $>$ 60 GeV, $>$ 100 GeV, and $>$ 200 GeV. 
	  %MET $>$ 30 GeV, $>$ 60 GeV and $>$ 120 GeV. 
	}
%to add:
%The quantity (observed-predicted)/predicted as a function of MET is shown above the plot.}
    \label{fig:mcclosure}
  \end{center}
\end{figure}


%2010 results
%      MET                   & $>$ 30 GeV & $>$ 60 GeV  & $>$ 120 GeV \\ \hline
%      Z+jets observed         &      184   &    10       & 0          \\
%      $\gamma$+jets predicted &   182.21   &    11.52    & 1.40       \\


%\begin{wrapfigure}{r}{0.6\textwidth}
%\vspace{-25pt}
%\begin{center}
%\includegraphics[width=0.8\textwidth]{plots/mcclosure}
% \caption{\label{fig:mcclosure} The MET distribution in Z+jets MC (black) and prediction (blue) for Njet $\ge$ 2. Below the plot is tabulated the integral of the observed Z+jets MC MET and the predicted MET from $\gamma$+jets MC for MET $>$ 30 GeV, $>$ 60 GeV and $>$ 120 GeV. The quantity (observed-predicted)/predicted as a function of MET is shown above the plot.}
%\end{center}
%\vspace{-20pt}
%\end{wrapfigure}

\clearpage

\section{Top Background Estimation}
\label{sec:topbkg}

The \ttbar\ contribution to the signal region is estimated using an opposite-flavor (OF) subtraction technique.
This technique takes advantage of the fact that the \ttbar\ yield in the 
OF final state ($e\mu$) is the same as in the same-flavor (SF) final state
($ee+\mu\mu$), modulo differences in efficiency in the $e$ vs. $\mu$ selection.
Hence the \ttbar\ yield in the same-flavor final state can be estimated
using the corresponding yield in the opposite-flavor final state. 
Other backgrounds for which the lepton flavors are
uncorrelated (for example, $W^+W^-$ and DY$\rightarrow \tau^+\tau^-$) will also be included in
this estimate.

To predict the SF yield in a signal region defined by a requirement on the MET, we take the 
OF yield passing the same MET requirement. This yield is corrected using the ratio of
muon to electron selection efficiencies $R_{\mu e}=1.07 \pm 0.03$.
This quantity is evaluated as the square root of the ratio of $Z\to\mu^+\mu^-$ to $Z\to e^+e^-$
events in data, with no jets or MET requirements. To improve the statistical precision
of the background estimate, we do not require the OF events to lie in the $Z$ mass region,
and we apply a scale factor $K=0.16 \pm 0.01$ accounting for the fraction of \ttbar\ events
which lie in the region $81 < \mathrm{M(\ell\ell)} < 101$\GeVcc, extracted from MC.

Backgrounds from pair production of vector bosons are negligible compared to $t\bar{t}$.
Backgrounds from fake leptons are negligible due to the requirement of two \pt$ > 20$~GeV leptons
in the \Z mass window, accompanied by jets and large MET.

%\subsection{Non $t\bar{t}$ Backgrounds}
%\label{sec:othBG}

Backgrounds  in  which one  or  both  leptons  do not  originate  from
electroweak decays  (fake leptons) are assessed  using the method
of  Ref.~\cite{ref:top}.  A fake  lepton is  a lepton  candidate
originating from within a jet,  such as a lepton from semileptonic $b$
or  $c$ decays,  a muon  decay-in-flight, a  pion misidentified  as an
electron,  or an  unidentified  photon conversion.   Estimates of  the
contributions to  the signal region  from pure multijet QCD,  with two
fake leptons, and in $W+\mathrm{jets}$, with one fake lepton
in  addition to  the lepton  from the  decay of  the $W$,  are derived
separately. We find $0.00^{+0.04}_{-0.00}$ and $0.0^{+0.5}_{-0.0}$ 
($0.00^{+0.04}_{-0.00}$ and $0.5 \pm 0.5$)
for the  multijet QCD  and $W$+jets  contributions to the high \MET\
(high \Ht) signal regions, respectively,  and thus
consider these backgrounds to be negligible.

Backgrounds from DY are estimated with the data-driven $R_{out/in}$ technique~\cite{ref:top},
which leads to an estimated DY contribution which is consistent with 0.
Backgrounds from processes with two vector bosons and single top 
are negligible compared to dilepton $t\bar{t}$. 

%\clearpage

\section{Results}
\label{sec:results}

\begin{figure}[tbh]
\begin{center}
\includegraphics[width=0.75\linewidth]{abcd_jsonv3.png}
\caption{\label{fig:abcdData}\protect Distributions of SumJetPt 
vs. MET$/\sqrt{\rm SumJetPt}$ for SM Monte Carlo and data.}
\end{center}
\end{figure}

The data, together with SM expectations is presented 
in Figure~\ref{fig:abcdData}.  We see 1 event in the 
signal region (region $D$).  For more information about
this one candidate events, see Appendix~\ref{sec:cand}.
The Standard Model MC expectation is 1.3 events.

\subsection{Background estimate from the ABCD method}
\label{sec:abcdres}

The data yields in the 
four regions are summarized in Table~\ref{tab:datayield}.
The prediction of the ABCD method is is given by $A \times C / B = 1.3 \pm 0.8({\rm stat}) \pm 0.3({\rm syst})$.

\begin{table}[hbt]
\begin{center}
\caption{\label{tab:datayield} Data yields in the four
regions of Figure~\ref{fig:abcdData}, as well as the predicted yield in region D given
by A $\times$ C / B.  The quoted uncertainty
on the prediction in data is statistical only, assuming Gaussian errors.
We also show the SM Monte Carlo expectations, scaled to 34.0~pb$^{-1}$.}
\begin{tabular}{l||c|c|c|c||c}
%%%official json v3 33.96/pb, 38X MC (D6T for ttbar and DY)
\hline
              sample                     &                   A   &                   B   &                   C   &                   D   &                PRED  \\
\hline
$t\bar{t}\rightarrow \ell^{+}\ell^{-}$   &   8.44  $\pm$  0.18   &  32.83  $\pm$  0.35   &   4.78  $\pm$  0.14   &   1.07  $\pm$  0.06   &   1.23  $\pm$  0.05  \\
$t\bar{t}\rightarrow \mathrm{other}$     &   0.12  $\pm$  0.02   &   0.78  $\pm$  0.05   &   0.16  $\pm$  0.02   &   0.02  $\pm$  0.01   &   0.02  $\pm$  0.01  \\
$Z^0 \rightarrow \ell^{+}\ell^{-}$       &   0.17  $\pm$  0.08   &   1.18  $\pm$  0.22   &   0.04  $\pm$  0.04   &   0.12  $\pm$  0.07   &   0.01  $\pm$  0.01  \\
    $W^{\pm}$ + jets                     &   0.00  $\pm$  0.00   &   0.09  $\pm$  0.09   &   0.00  $\pm$  0.00   &   0.00  $\pm$  0.00   &   0.00  $\pm$  0.00  \\
            $W^+W^-$                     &   0.11  $\pm$  0.01   &   0.29  $\pm$  0.02   &   0.02  $\pm$  0.01   &   0.03  $\pm$  0.01   &   0.01  $\pm$  0.00  \\
        $W^{\pm}Z^0$                     &   0.01  $\pm$  0.00   &   0.04  $\pm$  0.00   &   0.00  $\pm$  0.00   &   0.00  $\pm$  0.00   &   0.00  $\pm$  0.00  \\
            $Z^0Z^0$                     &   0.01  $\pm$  0.00   &   0.02  $\pm$  0.00   &   0.00  $\pm$  0.00   &   0.00  $\pm$  0.00   &   0.00  $\pm$  0.00  \\
          single top                     &   0.29  $\pm$  0.01   &   1.04  $\pm$  0.03   &   0.04  $\pm$  0.01   &   0.01  $\pm$  0.00   &   0.01  $\pm$  0.00  \\
\hline
         total SM MC                     &   9.14  $\pm$  0.20   &  36.26  $\pm$  0.43   &   5.05  $\pm$  0.14   &   1.27  $\pm$  0.10   &   1.27  $\pm$  0.05  \\
\hline
                data                     &                  12   &                  37   &                   4   &                   1   &   1.30  $\pm$  0.78  \\
\hline
                 LM0                     &   4.04  $\pm$  0.19   &   4.45  $\pm$  0.20   &  13.92  $\pm$  0.36   &   8.63  $\pm$  0.27   &  12.63  $\pm$  0.88  \\
                 LM1                     &   0.52  $\pm$  0.02   &   0.26  $\pm$  0.02   &   1.64  $\pm$  0.04   &   3.56  $\pm$  0.06   &   3.33  $\pm$  0.27  \\
\hline
\end{tabular}
\end{center}
\end{table}

%As a cross-check, we can subtract from the yields in 
%Table~\ref{tab:datayield} the expected DY contributions
%from Table~\ref{tab:ABCD-DY} in order to get a ``purer''
%estimate of the $t\bar{t}$ contribution.  The result
%of this exercise is {\color{red} xx} events.

%\clearpage

\subsection{Background estimate from the $P_T(\ell\ell)$ method}
\label{sec:victoryres}

We first use the $P_T(\ell \ell)$ method to predict the number of events 
in control region A, defined in Sec.~\ref{sec:abcd} as 
$125<{\rm SumJetPt}>300$~GeV and $\met/\sqrt{\rm SumJetPt}>$8.5~GeV$^{1/2}$.
We count the number of events in region
$A'$, defined in Sec.~\ref{sec:othBG} by replacing the above $\met/\sqrt{\rm SumJetPt}$
cut with the same cut on the quantity $P_T(\ell\ell)/\sqrt{\rm SumJetPt}$,
and find $N_{A'}=5$. We subtract off the expected DY contribution in this region
$N_{DY} = 1.3 \pm 0.9$, derived in Sec.~\ref{sec:othBG}.
To predict the yield in region A we take 
$N_A = K \cdot K_C \cdot ( N_{A'} - N_{DY} ) = 9.0 \pm 6.0$
where we have taken $K = 1.7$ and $K_C = 1.4$.
This uncertainty takes into account the statistical uncertainties in $N_{A'}$ and $N_{DY}$,
assuming Gaussian errors. This yield is consistent
with the observed yield of 12 events, as shown in 
Table~\ref{tab:victory} and displayed in Fig.~\ref{fig:victory} (left).

Encouraged by the good agreement between predicted and observed yields
in the control region, we proceed to perform the $P_T(\ell \ell)$ method 
in the signal region ${\rm SumJetPt}>300$~GeV.
The number of data events in region $D'$, which is defined in 
Section~\ref{sec:othBG} to be the same as region $D$ but with the
$\met/\sqrt{\rm SumJetPt}$ requirement 
replaced by a $P_T(\ell\ell)/\sqrt{\rm SumJetPt}$ requirement,
is $N_{D'}=1$.  
%We next subtract off the expected DY contribution of 
%$N_{DY}$ = $0.4 \pm 0.4$ events, as calculated 
%in Sec.~\ref{sec:othBG}. 
The BG prediction is 
$N_D = K \cdot K_C \cdot (N_{D'}-N_{DY}) = 2.1 \pm 2.1({\rm stat}) \pm 0.6({\rm syst})$ 
where $K=1.5$ as derived in Sec.~\ref{sec:victory} and $K_C = 1.4 \pm 0.4$.
This prediction is consistent with the observed yield of 1 event, as summarized 
in Table~\ref{tab:victory} and Fig.~\ref{fig:victory} (right).


\begin{figure}[hbt]
\begin{center}
\includegraphics[width=0.48\linewidth]{victory_control_jsonv3.png}
\includegraphics[width=0.48\linewidth]{victory_signal_jsonv3.png}
\caption{\label{fig:victory}\protect Distributions of 
tcMet/$\sqrt{\rm SumJetPt}$ for the control and signal region.
We show the oberved distributions in both Monte Carlo and data.
We also show the distributions predicted from 
${P_T(\ell\ell)}/\sqrt{\rm SumJetPt}$ in both MC and data.}
\end{center}
\end{figure}


\begin{table}[hbt]
\begin{center}
\caption{\label{tab:victory}Results of the dilepton $p_{T}$ template method in the control region
($125 < \mathrm{SumJetPt} < 300$~GeV) and signal region ($\mathrm{SumJetPt} > 300$~GeV). The predicted and 
observed yields for the region $\mathrm{tcmet}/\sqrt{\mathrm{sumJetPt}} > 8.5$~GeV$^{1/2}$. The errors are
statistical only, assuming Gaussian errors. Note that the correction factor $K_C$ has been applied to
the data but not to the MC.  }
\begin{tabular}{l|cc|cc}
\hline
              &    Control Region   &                        &   Signal Region    &               \\
\hline
              & Predicted           &   Observed             &   Predicted        &  Observed     \\              
\hline
total SM   MC &      6.45           &       9.14             &   0.92             &  1.27         \\
         data &  $9.0 \pm 6.0$      &         12             &   $2.1 \pm 2.1$    &  1            \\
\hline
\end{tabular}
\end{center}
\end{table}



% \clearpage
\subsection{Summary of results}

In summary, in the signal region defined as $\mathrm{SumJetPt}>300$~GeV and $\met/\sqrt{\rm SumJetPt} > 8.5$~GeV$^{1/2}$:\\ 
We observe 1 event. \\
We expect 1.3 events from Standard Model MC prediction. \\
The ABCD data driven method predicts $1.3 \pm 0.8({\rm stat}) \pm 0.3({\rm syst})$ events. \\
The  $P_T(\ell\ell)$ method predicts $2.1 \pm 2.1({\rm stat}) \pm 0.6({\rm syst})$ events. \\
  
All three background estimates are consistent within their uncertainties.
We thus take as our best estimate of the Standard Model yield in 
the signal region the average of the predicted yields from the 2 data-driven methods, 
weighted by their uncertainties.
This procedure gives an expected background yield $N_{BG}=1.4 \pm 0.8$.

We conclude that we see no evidence for an anomalous 
rate of opposite sign isolated dilepton events
at high \met and high SumJetPt.  The extraction of 
quantitative limits on new physics models is discussed
in Section~\ref{sec:limit}.
\clearpage

%syst on bkgnd preds
%\section{Systematics Uncertainties on the Background Prediction}
%\label{sec:systematics}

[DESCRIBE HERE ONE BY ONE THE UNCERTAINTIES THAT ARE PRESENT IN THE SPREADSHHET
FROM WHICH WE CALCULATE THE TOTAL UNCERTAINTY. WE KNOW HOW TO DO THIS
AND
WE HAVE THE TECHNOLOGY FROM THE 7 TEV ANALYSIS TO PROPAGATE ALL
UNCERTAINTIES
CORRECTLY THROUGH.  WE WILL DO IT ONCE WE HAVE SETTLED ON THE
INDIVIDUAL PIECES WHICH ARE STILL IN FLUX]

In this Section we discuss the systematic uncertainty on the BG
prediction.  This prediction is assembled from the event
counts in the peak region of the transverse mass distribution as
well as Monte Carlo 
with a number of correction factors, as described previously.
The
final uncertainty on the prediction is built up from the uncertainties in these
individual 
components. 
The calculation is done for each signal
region,
for electrons and muons separately.

The choice to normalizing to the peak region of $M_T$ has the
advantage that some uncertainties, e.g., luminosity, cancel.
It does however introduce complications because it couples
some of the uncertainties in non-trivial ways.  For example, 
the primary effect of an uncertainty on the rare MC cross-section
is to introduce an uncertainty in the rare MC background estimate
which comes entirely from MC.   But this uncertainty also affects,
for example,
the $t\bar{t} \to$ dilepton BG estimate because it changes the 
$t\bar{t}$ normalization to the peak region (because some of the 
events in the peak region are from rare processes).  These effects
are carefully accounted for.  The contribution to the overall
uncertainty from each BG source is tabulated in
Section~\ref{sec:bgunc-bottomline}.
First, however, we discuss the uncertainties one-by-one and we comment 
on their impact on the overall result, at least to first order.
Second order effects, such as the one described, are also included.

\subsection{Statistical uncertainties on the event counts in the $M_T$
peak regions}
These vary between XX and XX \%, depending on the signal region
(different
signal regions have different \met\ requirements, thus they also have
different $M_T$ regions used as control.
Since 
the major BG, eg, $t\bar{t}$ are normalized to the peak regions, this 
fractional uncertainty is pretty much carried through all the way to
the end.  There is also an uncertainty from the finite MC event counts
in the $M_T$ peak regions.  This is also included, but it is smaller.

\subsection{Uncertainty from the choice of $M_T$ peak region}
IN 7 TEV DATA WE HAD SOME SHAPE DIFFERENCES IN THE MTRANS REGION THAT 
LED US TO CONSERVATIVELY INCLUDE THIS UNCERTAINTY.  WE NEED TO LOOK
INTO THIS AGAIN

\subsection{Uncertainty on the Wjets cross-section and the rare MC cross-sections}
These are taken as 50\%, uncorrelated.  
The primary effect is to introduce a 50\%
uncertainty
on the $W +$ jets and rare BG 
background predictions, respectively.  However they also
have an effect on the other BGs via the $M_T$ peak normalization
in a way that tends to reduce the uncertainty.  This is easy
to understand: if the $W$ cross-section is increased by 50\%, then
the $W$ background goes up.  But the number of $M_T$ peak events 
attributed to $t\bar{t}$ goes down, and since the $t\bar{t}$ BG is
scaled to the number of $t\bar{t}$ events in the peak, the $t\bar{t}$ 
BG goes down.  

\subsection{Scale factors for the tail-to-peak ratios for lepton +
  jets top and W events}
These tail-to-peak ratios are described in Section~\ref{sec:ttp}.
They are studied in CR1 and CR2.  The studies are described
in Sections~\ref{sec:cr1} and~\ref{sec:cr2}), respectively, where 
we also give the uncertainty on the scale factors.

\subsection{Uncertainty on extra jet radiation for dilepton
  background}
As discussed in Section~\ref{sec:jetmultiplicity}, the 
jet distribution in
$t\bar{t} \to$
dilepton MC is rescaled by the factors $K_3$ and $K_4$ to make 
it agree with the data.  The XX\% uncertainties on $K_3$ and $K_4$
comes from data/MC statistics.  This  
result directly in a XX\% uncertainty on the dilepton BG, which is by far 
the most important one.


\subsection{Uncertainty on the \ttll\ Acceptance}

The \ttbar\ background prediction is obtained from MC, with corrections
derived from control samples in data. The uncertainty associated with
the theoretical modeling of the \ttbar\ production and decay is
estimated by comparing the background predictions obtained using 
alternative MC samples. It should be noted that the full analysis is
performed with the alternative samples under consideration, 
including the derivation of the various data-to-MC scale factors. 
The variations considered are

\begin{itemize}
\item Top mass: The alternative values for the top mass differ
  from the central value by $5~\GeV$: $m_{\mathrm{top}} = 178.5~\GeV$ and $m_{\mathrm{top}}
  = 166.5~\GeV$.
\item Jet-parton matching scale: This corresponds to variations in the
  scale at which the Matrix Element partons from Madgraph are matched
  to Parton Shower partons from Pythia. The nominal value is
  $x_q>20~\GeV$. The alternative values used are $x_q>10~\GeV$ and
  $x_q>40~\GeV$.
\item Renormalization and factorization scale: The alternative samples
  correspond to variations in the scale $\times 2$ and $\times 0.5$. The nominal
  value for the scale used is $Q^2 = m_{\mathrm{top}}^2 +
  \sum_{\mathrm{jets}} \pt^2$.
\item Alternative generators: Samples produced with different
  generators include MC@NLO and Powheg (NLO generators) and
  Pythia (LO). It may also be noted that MC@NLO uses Herwig6 for the 
  hadronisation, while POWHEG uses Pythia6.
\item Modeling of taus: The alternative sample does not include
  Tauola and is otherwise identical to the Powheg sample.
  This effect was studied earlier using 7~TeV samples and found to be negligible.
\item The PDF uncertainty is estimated following the PDF4LHC
  recommendations[CITE]. The events are reweighted using alternative
  PDF sets for CT10 and MSTW2008 and the uncertainties for each are derived using the
  alternative eigenvector variations and the ``master equation''. In
  addition, the NNPDF2.1 set with 100 replicas. The central value is
  determined from the mean and the uncertainty is derived from the
  $1\sigma$ range. The overall uncertainty is derived from the envelope of the
  alternative predictions and their uncertainties.
  This effect was studied earlier using 7~TeV samples and found to be negligible.
  \end{itemize}


\begin{figure}[hbt]
  \begin{center}
	\includegraphics[width=0.8\linewidth]{plots/n_dl_syst_comp.png}
	\caption{
	  \label{fig:ttllsyst}%\protect 
          Comparison of the \ttll\ central prediction with those using
          alternative MC samples. The blue band corresponds to the
          total statistical error for all data and MC samples. The
          alternative sample predictions are indicated by the
          datapoints. The uncertainties on the alternative predictions
          correspond to the uncorrelated statistical uncertainty from
          the size of the alternative sample only.
        [TO BE UPDATED WITH THE LATEST SELECTION AND SFS]}
      \end{center}
    \end{figure}

\clearpage

%
%
%The methodology for determining the systematics on the background
%predictions has not changed with respect to the nominal analysis.
%Because the template method has not changed, the same 
%systematic uncertainty is assessed on this prediction (32\%).
%The 50\% uncertainty on the WZ and ZZ background is also unchanged.
%The systematic uncertainty in the OF background prediction based on 
%e$\mu$ events has changed, due to the different composition of this
%sample after vetoing events containing b-tagged jets.
%
%As in the nominal analysis, we do not require the e$\mu$ events
%to satisfy the dilepton mass requirement and apply a scaling factor K,
%extracted from MC, to account for the fraction of e$\mu$ events
%which satisfy the dilepton mass requirement. This procedure is used
%in order to improve the statistical precision of the OF background estimate.
%
%For the selection used in the nominal analysis, 
%the e$\mu$ sample is completely dominated by $t\bar{t}$
%events, and we observe that K is statistically consistent with constant with
%respect to the \MET\ requirement. However, in this analysis, the $t\bar{t}$
%background is strongly suppressed by the b-veto, and hence the non-$t\bar{t}$
%backgrounds (specifically, $Z\to\tau\tau$ and VV) become more relevant. 
%At low \MET, the $Z\to\tau\tau$ background is pronounced, while $t\bar{t}$
%and VV dominate at high \MET\ (see App.~\ref{app:kinemu}).
%Therefore, the sample composition changes
%as the \MET\ requirement is varied, and as a result K depends
%on the \MET\ requirement. 
%
%We thus measure K in MC separately for each
%\MET\ requirement, as displayed in Fig.~\ref{fig:kvmet} (left).
%%The systematic uncertainty on K is determined separately for each \MET\
%%requirement by comparing the relative difference in K in data vs. MC.
%The values of K used are the MC predictions 
%%and the total systematic uncertainty on the OF prediction 
%%as shown in 
%(Table \ref{fig:kvmettable}).
%The contribution to the total OF prediction systematic uncertainty 
%from K is assessed from the ratio of K in data and MC,
%shown in Fig.~\ref{fig:kvmet} (right).
%The ratio is consistent with unity to roughly 17\%, 
%so we take this value as the systematic from K.
%17\% added in quadrature with 7\% from 
%the electron to muon efficieny ratio 
%(as assessed in the inclusive analysis)
%yields a total systematic of $\sim$18\% 
%which we round up to 20\%.
%For \MET\ $>$ 150, there are no OF events in data inside the Z mass window
%so we take a systematic based on the statistical uncertainty
%of the MC prediction for K. 
%This value is 25\% for \MET\ $>$ 150 GeV and 60\% for \MET\ $>$ 200 GeV.
%%Although we cannot check the value of K in data for \MET\ $>$ 150
%%because we find no OF events inside the Z mass window for this \MET\ 
%%cut, the overall OF yields with no dilepton mass requirement 
%%agree to roughly 20\% (9 data vs 7.0 $\pm$ 1.1 MC).
%
%
%%Below Old
%
%%In reevaluating the systematics on the OF prediction, however,
%%we observed a different behavior of K as a function of \MET\ 
%%as was seen in the inclusive analysis. 
%
%%Recall that K is the ratio of the number of \emu\ events
%%inside the Z window to the total number of \emu\ events.
%%In the inclusive analysis, it is taken from \ttbar\ MC
%%and used to scale the inclusive \emu\ yield in data.
%%The yield scaled by K is then corrected for 
%%the $e$ vs $\mu$ efficiency difference to obtain the 
%%final OF prediction.
%
%%Based on the plot in figure \ref{fig:kvmet}, 
%%we choose to use a different
%%K for each \MET\ cut and assess a systematic uncertainty
%%on the OF prediction based on the difference between 
%%K in data and MC. 
%%The variation of K as a function of \MET\ is caused 
%%by a change in sample composition with increasing \MET.
%%At \MET\ $<$ 60 GeV, the contribution of Z plus jets is
%%not negligible (as it was in the inclusive analysis)
%%because of the b veto. (See appendix \ref{app:kinemu}.)
%%At higher \MET, \ttbar\ and diboson backgrounds dominate.
%
%
%
%
%\begin{figure}[hbt]
%  \begin{center}
%	\includegraphics[width=0.48\linewidth]{plots/kvmet_data_ttbm.pdf}
%	\includegraphics[width=0.48\linewidth]{plots/kvmet_ratio.pdf}
%	\caption{
%	  \label{fig:kvmet}\protect 
%	  The left plot shows
%	  K as a function of \MET\ in MC (red) and data (black). 
%	  The bin low edge corresponds to the \MET\ cut, and the 
%	  bins are inclusive.
%	  The MC used is a sum of all SM MC used in the yield table of
%	  section \ref{sec:yields}.
%	  The right plot is the ratio of K in data to MC.
%	  The ratio is fit to a line whose slope is consistent with zero
%	  (the fit parameters are 
%	  0.9 $\pm$  0.4 for the intercept and
%      0.001 $\pm$ 0.005 for the slope).
%	}
%  \end{center}
%\end{figure}
%
%
%
%\begin{table}[htb]
%\begin{center}
%\caption{\label{fig:kvmettable} The values of K used in the OF background prediction. 
%The uncertainties shown are the total relative systematic used for the OF prediction,
%which is the systematic uncertainty from K added in quadrature with
%a 7\% uncertainty from the electron to muon efficieny ratio as assessed in the
%inclusive analysis.
%}
%\begin{tabular}{lcc}
%\hline
%\MET\ Cut    &    K        &  Relative Systematic \\
%\hline
%%the met zero row is used only for normalization of the money plot.
%%0    &  0.1   &        \\  
%30   &  0.12  &  20\%  \\  
%60   &  0.13  &  20\%  \\  
%80   &  0.12  &  20\%  \\  
%100  &  0.12  &  20\%  \\  
%150  &  0.09  &  25\%  \\  
%200  &  0.06  &  60\%  \\  
%\hline
%\end{tabular}
%\end{center}
%\end{table}

\subsection{Uncertainty from the isolated track veto}
This is the uncertainty associated with how well the isolated track
veto performance is modeled by the Monte Carlo.  This uncertainty
only applies to the fraction of dilepton BG events that have 
a second e/$\mu$ or a one prong $\tau \to h$, with 
$P_T > 10$ GeV in $|\eta| < 2.4$.  This fraction is 1/3 (THIS WAS THE
7 TEV NUMBER, CHECK).  The uncertainty for these events
is XX\% and is obtained from Tag and Probe studies of Section~\ref{sec:trkveto}

\subsubsection{Isolated Track Veto: Tag and Probe Studies}
\label{sec:trkveto}

[EVERYTHING IS 7TEV HERE, UPDATE WITH NEW RESULTS \\
ADD TABLE WITH FRACTION OF EVENTS THAT HAVE A TRUE ISOLATED TRACK]

In this section we compare the performance of the isolated track veto in data and MC using tag-and-probe studies
with samples of Z$\to$ee and Z$\to\mu\mu$. The purpose of these studies is to demonstrate that the efficiency
to satisfy the isolated track veto requirements is well-reproduced in the MC, since if this were not the case 
we would need to apply a data-to-MC scale factor in order to correctly predict the \ttll\ background. This study
addresses possible data vs. MC discrepancies for the {\bf efficiency} to identify (and reject) events with a 
second {\bf genuine} lepton (e, $\mu$, or $\tau\to$1-prong). It does not address possible data vs. MC discrepancies
in the fake rate for rejecting events without a second genuine lepton; this is handled separately in the top normalization
procedure by scaling the \ttlj\ contribution to match the data in the \mt\ peak after applying the isolated track veto. 
Furthermore, we test the data and MC
isolated track veto efficiencies for electrons and muons since we are using a Z tag-and-probe technique, but we do not
directly test the performance for hadronic tracks from $\tau$ decays. The performance for hadronic $\tau$ decay products
may differ from that of electrons and muons for two reasons. First, the $\tau$ may decay to a hadronic track plus one
or two $\pi^0$'s, which may decay to $\gamma\gamma$ followed by a photon conversion. As shown in Figure~\ref{fig:absiso},
the isolation distribution for charged tracks from $\tau$ decays that are not produced in association with $\pi^0$s are 
consistent with that from $\E$s and $\M$s. Since events from single prong $\tau$ decays produced in association with 
$\pi^0$s comprise a small fraction of the total sample, and since the kinematics of $\tau$, $\pi^0$ and $\gamma\to e^+e^-$
decays are well-understood, we currently demonstrate that the isolation is well-reproduced for electrons and muons only.
Second, hadronic tracks may undergo nuclear interactions and hence their tracks may not be reconstructed.
As discussed above, independent studies show that the MC reproduces the hadronic tracking efficiency within 4\%,
leading to a total background uncertainty of less than 0.5\% (after taking into account the fraction of the total background
due to hadronic $\tau$ decays with \pt\ $>$ 10 GeV tracks), and we hence regard this effect as neglgigible.

The tag-and-probe studies are performed in the full 2011 data sample, and compared with the DYJets madgraph sample.
All events must contain a tag-probe pair (details below) with opposite-sign and satisfying the Z mass requirement 76--106 GeV.
We compare the distributions of absolute track isolation for probe electrons/muons in data vs. MC. The contributions to
this isolation sum are from ambient energy in the event from underlying event, pile-up and jet activitiy, and hence do
not depend on the \pt\ of the probe lepton. We therefore restrict the probe \pt\ to be $>$ 30 GeV in order to suppress
fake backgrounds with steeply-falling \pt\ spectra. To suppress non-Z backgrounds (in particular \ttbar) we require 
\met\ $<$ 30 GeV and 0 b-tagged events. 
The specific criteria for tags and probes for electrons and muons are:

%We study the isolated track veto efficiency in bins of \njets.
%We are interested in events with at least 4 jets to emulate the hadronic activity in our signal sample. However since
%there are limited statistics for Z + $\geq$4 jet events, we study the isolated track performance in events with


\begin{itemize}
  \item{Electrons}

    \begin{itemize}
    \item{Tag criteria}

      \begin{itemize}
      \item Electron passes full analysis ID/iso selection 
      \item \pt\ $>$ 30 GeV, $|\eta|<2.5$
 
      \item Matched to 1 of the 2 electron tag-and-probe triggers
        \begin{itemize}
        \item \verb=HLT_Ele17_CaloIdVT_CaloIsoVT_TrkIdT_TrkIsoVT_SC8_Mass30_v*=
        \item \verb=HLT_Ele17_CaloIdVT_CaloIsoVT_TrkIdT_TrkIsoVT_Ele8_Mass30_v*=
        \end{itemize}
      \end{itemize}

    \item{Probe criteria}
      \begin{itemize}
      \item Electron passes full analysis ID selection
      \item \pt\ $>$ 30 GeV
      \end{itemize}
      \end{itemize}
  \item{Muons}
    \begin{itemize}
    \item{Tag criteria}
      \begin{itemize}
      \item Muon passes full analysis ID/iso selection
      \item \pt\ $>$ 30 GeV, $|\eta|<2.1$
      \item Matched to 1 of the 2 electron tag-and-probe triggers
        \begin{itemize}
        \item \verb=HLT_IsoMu30_v*=
        \item \verb=HLT_IsoMu30_eta2p1_v*=
        \end{itemize}
      \end{itemize}
    \item{Probe criteria}
      \begin{itemize}
      \item Muon passes full analysis ID selection
      \item \pt\ $>$ 30 GeV
      \end{itemize}
    \end{itemize}
\end{itemize}

The absolute track isolation distributions for passing probes are displayed in Fig.~\ref{fig:tnp}. In general we observe
good agreement between data and MC. To be more quantitative, we compare the data vs. MC efficiencies to satisfy
absolute track isolation requirements varying from $>$ 1 GeV to $>$ 5 GeV, as summarized in Table~\ref{tab:isotrk}.
In the $\geq$0 and $\geq$1 jet bins where the efficiencies can be tested with statistical precision, the data and MC
efficiencies agree within 7\%, and we apply this as a systematic uncertainty on the isolated track veto efficiency.
For the higher jet multiplicity bins the statistical precision decreases, but we do not observe any evidence for
a data vs. MC discrepancy in the isolated track veto efficiency.


%This is because our analysis requirement is relative track isolation $<$ 0.1, and m
%This requirement is chosen because most of the tracks rejected by the isolated
%track veto have a \pt\ near the 10 GeV threshold, and our analysis requirement is relative track isolation $<$ 1 GeV.

\begin{figure}[hbt]
  \begin{center}
	%\includegraphics[width=0.3\linewidth]{plots/el_tkiso_0j.pdf}%
	%\includegraphics[width=0.3\linewidth]{plots/mu_tkiso_0j.pdf}
	%\includegraphics[width=0.3\linewidth]{plots/el_tkiso_1j.pdf}%
	%\includegraphics[width=0.3\linewidth]{plots/mu_tkiso_1j.pdf}
	%\includegraphics[width=0.3\linewidth]{plots/el_tkiso_2j.pdf}%
	%\includegraphics[width=0.3\linewidth]{plots/mu_tkiso_2j.pdf}
	%\includegraphics[width=0.3\linewidth]{plots/el_tkiso_3j.pdf}%
	%\includegraphics[width=0.3\linewidth]{plots/mu_tkiso_3j.pdf}
	%\includegraphics[width=0.3\linewidth]{plots/el_tkiso_4j.pdf}%
	%\includegraphics[width=0.3\linewidth]{plots/mu_tkiso_4j.pdf}
	\caption{
	  \label{fig:tnp} Comparison of the absolute track isolation in data vs. MC for electrons (left) and muons (right)
for events with the \njets\ requirement varied from \njets\ $\geq$ 0 to \njets\ $\geq$ 4. 
}  
      \end{center}
\end{figure}

\clearpage

\begin{table}[!ht]
\begin{center}
\caption{\label{tab:isotrk} Comparison of the data vs. MC efficiencies to satisfy the indicated requirements
on the absolute track isolation, and the ratio of these two efficiencies. Results are indicated separately for electrons and muons and for various
jet multiplicity requirements.}
\begin{tabular}{l|l|c|c|c|c|c}
\hline
\hline
 e + $\geq$0 jets            &           $>$ 1 GeV   &           $>$ 2 GeV   &           $>$ 3 GeV   &           $>$ 4 GeV   &           $>$ 5 GeV  \\
\hline
      data   &  0.088 $\pm$ 0.0003   &  0.030 $\pm$ 0.0002   &  0.013 $\pm$ 0.0001   &  0.007 $\pm$ 0.0001   &  0.005 $\pm$ 0.0001  \\
        mc   &  0.087 $\pm$ 0.0001   &  0.030 $\pm$ 0.0001   &  0.014 $\pm$ 0.0001   &  0.008 $\pm$ 0.0000   &  0.005 $\pm$ 0.0000  \\
   data/mc   &     1.01 $\pm$ 0.00   &     0.99 $\pm$ 0.01   &     0.97 $\pm$ 0.01   &     0.95 $\pm$ 0.01   &     0.93 $\pm$ 0.01  \\
\hline
\hline
 $\mu$ + $\geq$0 jets            &           $>$ 1 GeV   &           $>$ 2 GeV   &           $>$ 3 GeV   &           $>$ 4 GeV   &           $>$ 5 GeV  \\
\hline
      data   &  0.087 $\pm$ 0.0002   &  0.031 $\pm$ 0.0001   &  0.015 $\pm$ 0.0001   &  0.008 $\pm$ 0.0001   &  0.005 $\pm$ 0.0001  \\
        mc   &  0.085 $\pm$ 0.0001   &  0.030 $\pm$ 0.0001   &  0.014 $\pm$ 0.0000   &  0.008 $\pm$ 0.0000   &  0.005 $\pm$ 0.0000  \\
   data/mc   &     1.02 $\pm$ 0.00   &     1.06 $\pm$ 0.00   &     1.06 $\pm$ 0.01   &     1.03 $\pm$ 0.01   &     1.02 $\pm$ 0.01  \\
\hline
\hline 
 e + $\geq$1 jets            &           $>$ 1 GeV   &           $>$ 2 GeV   &           $>$ 3 GeV   &           $>$ 4 GeV   &           $>$ 5 GeV  \\
\hline
      data   &  0.099 $\pm$ 0.0008   &  0.038 $\pm$ 0.0005   &  0.019 $\pm$ 0.0004   &  0.011 $\pm$ 0.0003   &  0.008 $\pm$ 0.0002  \\
        mc   &  0.100 $\pm$ 0.0004   &  0.038 $\pm$ 0.0003   &  0.019 $\pm$ 0.0002   &  0.012 $\pm$ 0.0002   &  0.008 $\pm$ 0.0001  \\
   data/mc   &     0.99 $\pm$ 0.01   &     1.00 $\pm$ 0.02   &     0.99 $\pm$ 0.02   &     0.98 $\pm$ 0.03   &     0.97 $\pm$ 0.03  \\
\hline
\hline
 $\mu$ + $\geq$1 jets            &           $>$ 1 GeV   &           $>$ 2 GeV   &           $>$ 3 GeV   &           $>$ 4 GeV   &           $>$ 5 GeV  \\
\hline
      data   &  0.100 $\pm$ 0.0006   &  0.041 $\pm$ 0.0004   &  0.022 $\pm$ 0.0003   &  0.014 $\pm$ 0.0002   &  0.010 $\pm$ 0.0002  \\
        mc   &  0.099 $\pm$ 0.0004   &  0.039 $\pm$ 0.0002   &  0.020 $\pm$ 0.0002   &  0.013 $\pm$ 0.0001   &  0.009 $\pm$ 0.0001  \\
   data/mc   &     1.01 $\pm$ 0.01   &     1.05 $\pm$ 0.01   &     1.05 $\pm$ 0.02   &     1.06 $\pm$ 0.02   &     1.06 $\pm$ 0.03  \\
\hline
\hline
 e + $\geq$2 jets            &           $>$ 1 GeV   &           $>$ 2 GeV   &           $>$ 3 GeV   &           $>$ 4 GeV   &           $>$ 5 GeV  \\
\hline
      data   &  0.105 $\pm$ 0.0020   &  0.042 $\pm$ 0.0013   &  0.021 $\pm$ 0.0009   &  0.013 $\pm$ 0.0007   &  0.009 $\pm$ 0.0006  \\
        mc   &  0.109 $\pm$ 0.0011   &  0.043 $\pm$ 0.0007   &  0.021 $\pm$ 0.0005   &  0.013 $\pm$ 0.0004   &  0.009 $\pm$ 0.0003  \\
   data/mc   &     0.96 $\pm$ 0.02   &     0.97 $\pm$ 0.03   &     1.00 $\pm$ 0.05   &     1.01 $\pm$ 0.06   &     0.97 $\pm$ 0.08  \\
\hline
\hline
 $\mu$ + $\geq$2 jets            &           $>$ 1 GeV   &           $>$ 2 GeV   &           $>$ 3 GeV   &           $>$ 4 GeV   &           $>$ 5 GeV  \\
\hline
      data   &  0.106 $\pm$ 0.0016   &  0.045 $\pm$ 0.0011   &  0.025 $\pm$ 0.0008   &  0.016 $\pm$ 0.0007   &  0.012 $\pm$ 0.0006  \\
        mc   &  0.108 $\pm$ 0.0009   &  0.044 $\pm$ 0.0006   &  0.024 $\pm$ 0.0004   &  0.016 $\pm$ 0.0004   &  0.011 $\pm$ 0.0003  \\
   data/mc   &     0.98 $\pm$ 0.02   &     1.04 $\pm$ 0.03   &     1.04 $\pm$ 0.04   &     1.04 $\pm$ 0.05   &     1.06 $\pm$ 0.06  \\
\hline
\hline
 e + $\geq$3 jets            &           $>$ 1 GeV   &           $>$ 2 GeV   &           $>$ 3 GeV   &           $>$ 4 GeV   &           $>$ 5 GeV  \\
\hline
      data   &  0.117 $\pm$ 0.0055   &  0.051 $\pm$ 0.0038   &  0.029 $\pm$ 0.0029   &  0.018 $\pm$ 0.0023   &  0.012 $\pm$ 0.0019  \\
        mc   &  0.120 $\pm$ 0.0031   &  0.052 $\pm$ 0.0021   &  0.027 $\pm$ 0.0015   &  0.018 $\pm$ 0.0012   &  0.013 $\pm$ 0.0011  \\
   data/mc   &     0.97 $\pm$ 0.05   &     0.99 $\pm$ 0.08   &     1.10 $\pm$ 0.13   &     1.03 $\pm$ 0.15   &     0.91 $\pm$ 0.16  \\
\hline
\hline
 $\mu$ + $\geq$3 jets            &           $>$ 1 GeV   &           $>$ 2 GeV   &           $>$ 3 GeV   &           $>$ 4 GeV   &           $>$ 5 GeV  \\
\hline
      data   &  0.111 $\pm$ 0.0044   &  0.050 $\pm$ 0.0030   &  0.029 $\pm$ 0.0024   &  0.019 $\pm$ 0.0019   &  0.014 $\pm$ 0.0017  \\
        mc   &  0.115 $\pm$ 0.0025   &  0.051 $\pm$ 0.0017   &  0.030 $\pm$ 0.0013   &  0.020 $\pm$ 0.0011   &  0.015 $\pm$ 0.0009  \\
   data/mc   &     0.97 $\pm$ 0.04   &     0.97 $\pm$ 0.07   &     0.95 $\pm$ 0.09   &     0.97 $\pm$ 0.11   &     0.99 $\pm$ 0.13  \\
\hline
\hline
 e + $\geq$4 jets            &           $>$ 1 GeV   &           $>$ 2 GeV   &           $>$ 3 GeV   &           $>$ 4 GeV   &           $>$ 5 GeV  \\
\hline
      data   &  0.113 $\pm$ 0.0148   &  0.048 $\pm$ 0.0100   &  0.033 $\pm$ 0.0083   &  0.020 $\pm$ 0.0065   &  0.017 $\pm$ 0.0062  \\
        mc   &  0.146 $\pm$ 0.0092   &  0.064 $\pm$ 0.0064   &  0.034 $\pm$ 0.0048   &  0.024 $\pm$ 0.0040   &  0.021 $\pm$ 0.0037  \\
   data/mc   &     0.78 $\pm$ 0.11   &     0.74 $\pm$ 0.17   &     0.96 $\pm$ 0.28   &     0.82 $\pm$ 0.30   &     0.85 $\pm$ 0.34  \\
\hline
\hline
 $\mu$ + $\geq$4 jets            &           $>$ 1 GeV   &           $>$ 2 GeV   &           $>$ 3 GeV   &           $>$ 4 GeV   &           $>$ 5 GeV  \\
\hline
      data   &  0.130 $\pm$ 0.0128   &  0.052 $\pm$ 0.0085   &  0.028 $\pm$ 0.0063   &  0.019 $\pm$ 0.0052   &  0.019 $\pm$ 0.0052  \\
        mc   &  0.105 $\pm$ 0.0064   &  0.045 $\pm$ 0.0043   &  0.027 $\pm$ 0.0034   &  0.019 $\pm$ 0.0028   &  0.014 $\pm$ 0.0024  \\
   data/mc   &     1.23 $\pm$ 0.14   &     1.18 $\pm$ 0.22   &     1.03 $\pm$ 0.27   &     1.01 $\pm$ 0.32   &     1.37 $\pm$ 0.45  \\
\hline
\hline

\end{tabular}
\end{center}
\end{table}



%Figure.~\ref{fig:reliso} compares the relative track isolation
%for events with a track with $\pt > 10~\GeV$ in addition to a selected
%muon for $\Z+4$ jet events and various \ttll\ components. The
%isolation distributions show significant differences, particularly
%between the leptons from a \W\ or \Z\ decay and the tracks arising
%from $\tau$ decays. As can also be seen in the figure, the \pt\
%distribution for the various categories of tracks is different, where
%the decay products from $\tau$s are significantly softer. Since the
%\pt\ enters the denominator of the isolation definition and hence
%alters the isolation variable...

%\begin{figure}[hbt]
%  \begin{center}
%	\includegraphics[width=0.5\linewidth]{plots/pfiso_njets4_log.png}%
%	\includegraphics[width=0.5\linewidth]{plots/pfpt_njets4.png}
%	\caption{
%	  \label{fig:reliso}%\protect 
%          Comparison of relative track isolation variable for PF cand probe in Z+jets and ttbar 
%          Z+Jets and ttbar dilepton have similar isolation distributions
%          ttbar with leptonic and single prong taus tend to be less
%          isolated. The difference in the isolation can be attributed
%          to the different \pt\ distribution of the samples, since
%          $\tau$ decay products tend to be softer than leptons arising
%          from \W\ or \Z\ decays.}  
%      \end{center}
%\end{figure}

%	\includegraphics[width=0.5\linewidth]{plots/pfabsiso_njets4_log.png}


%BEGIN SECTION TO WRITE OUT 
%In detail, the procedure to correct the dilepton background is:

%\begin{itemize}
%\item Using tag-and-probe studies, we plot the distribution of {\bf absolute} track isolation for identified probe electrons
%and muons {\bf TODO: need to compare the e vs. $\mu$ track iso distributions, they might differ due to e$\to$e$\gamma$}.
%\item We verify that the distribution of absolute track isolation does not depend on the \pt\ of the probe lepton.
%This is due to the fact that this isolation is from ambient PU and jet activity in the event, which is uncorrelated with
%the lepton \pt {\bf TODO: verify this in data and MC.}.
%\item Our requirement is {\bf relative} track isolation $<$ 0.1. For a given \ttll\ MC event, we determine the \pt of the 2nd
%lepton and translate this to find the corresponding requirement on the {\bf absolute} track isolation, which is simply $0.1\times$\pt.
%\item We measure the efficiency to satisfy this requirement in data and MC, and define a scale-factor $SF_{\epsilon(trk)}$ which
%is the ratio of the data-to-MC efficiencies. This scale-factor is applied to the \ttll\ MC event.
%\item {\bf THING 2 we are unsure about: we can measure this SF for electrons and for muons, but we can't measure it for hadronic 
%tracks from $\tau$ decays. Verena has showed that the absolute track isolation distribution in hadronic $\tau$ tracks is harder due 
%to $\pi^0\to\gamma\gamma$ with $\gamma\to e^+e^-$.}
%\end{itemize} 
%END SECTION TO WRITE OUT 


{\bf fix me: What you have written in the next paragraph does not explain how $\epsilon_{fake}$ is measured.
Why not measure $\epsilon_{fake}$ in the b-veto region?}

%A measurement of the $\epsilon_{fake}$ in data is non-trivial. However, it is
%possible to correct for differences in the $\epsilon_{fake}$ between data and MC by
%applying an additional scale factor for the single lepton background
%alone, using the sample in the \mt\ peak region. This scale factor is determined after applying the isolated track
%veto and after subtracting the \ttll\ component, corrected for the
%isolation efficiency derived previously. 
%As shown in Figure~\ref{fig:vetoeffcomp}, the efficiency for selecting an
%isolated track in single lepton events is independent of \mt\, so the use of
%an overall scale factor is justified to estimate the contribution in
%the \mt\ tail. 
%
%\begin{figure}[hbt]
%  \begin{center}
%	\includegraphics[width=0.5\linewidth]{plots/vetoeff_comp.png}
%	\caption{
%	  \label{fig:vetoeffcomp}%\protect 
%          Efficiency for selecting an isolated track comparing
%          single lepton \ttlj\ and dilepton \ttll\ events in MC and 
%          data as a function of \mt. The
%          efficiencies in \ttlj\ and \ttll\ exhibit no dependence on
%          \mt\, while the data ranges between the two. This behavior
%          is expected since the low \mt\ region is predominantly \ttlj, while the
%          high \mt\ region contains mostly \ttll\ events.}  
%      \end{center}
%\end{figure}

\subsection{Summary of uncertainties}
\label{sec:bgunc-bottomline}.

THIS NEEDS TO BE WRITTEN

%UL on non-SM yeild
\section{Limits on New Physics}
\label{sec:limit}

As discussed in Sec.~\ref{sec:results}, we do not observe any excess in the signal regions.
We use these results to place Bayesian 95\% confidence level upper limits~\cite{ref:cl95cms} on 
the non-SM contributions to the yields in the signal regions, using a log-normal model
of nuissance parameter integration. The results are summarized in Table~\ref{resultsyieldtable}.  

\begin{figure}[hbt]
\begin{center}
\includegraphics[width=0.78\linewidth]{plots/lep_metPredicted.pdf}
\caption{\label{fig:results}\protect 
  The observed MET distribution for data in the (black points),
  predicted $t\bar{t}$ MET distribution (red line), the sum of predicted %
  $t\bar{t}$ MET distribution and
  $Z$  MET  distribution  predicted  from photon  MET  templates
  (solid blue line),  and MC (solid histograms). 
  }
\end{center}
\end{figure}

\begin{table}[htb]
\begin{center}
\caption{\label{resultsyieldtable} 
Summary of the yields in the regions MET $>$ 30, 60, 100 and 200 GeV. The total predicted yield is the sum of the
predicted background from \Z plus jets from the MET templates method (\Z prediction) plus the \ttbar\ contribution
predicted from OF subtraction (\ttbar\ prediction). Here the first uncertainty is statistical, the second uncertainty is systematic.
The 95\% CL Bayesian UL is indicated, as well as the expected NLO yields for the
LM4 and LM8 scenarios, including the uncertainties from lepton identification and isolation efficiency,
trigger efficiency, hadronic energy scale, and integrated luminosity.
}
\begin{tabular}{lcccc}
\hline
                  &   N(MET $>30$)  GeV          &   N(MET $>60$)  GeV          &   N(MET $>100$) GeV          &   N(MET $>200$) GeV \\
\hline
$Z$ prediction    & 406.2 $\pm$ 7.1 $\pm$ 101.6  &    13.1 $\pm$ 1.2 $\pm$ 3.3  &     1.4 $\pm$ 0.6 $\pm$ 0.4  &     0.05 $\pm$ 0.02 $\pm$ 0.01  \\
\ttbar prediction &  54.8 $\pm$ 3.0 $\pm$ 2.0    &    34.7 $\pm$ 2.4 $\pm$ 1.3  &    11.8 $\pm$ 1.4 $\pm$ 0.4  &      1.1 $\pm$  0.5 $\pm$ 0.04  \\
total prediction  & 461.0 $\pm$ 7.7 $\pm$ 101.6  &    47.9 $\pm$ 2.7 $\pm$ 3.5  &    13.2 $\pm$ 1.5 $\pm$ 0.6  &      1.2 $\pm$  0.5 $\pm$ 0.04  \\
      observed    &                         488  &                          39  &                          14  &                    2            \\
UL                &                         179  &                        11.5  &                        10.0  &                   5.3           \\
LM4               &   5.6 $\pm$ 0.3              &     5.0 $\pm$ 0.3            &     4.4 $\pm$ 0.3            &      2.7 $\pm$ 0.3              \\
LM8               &   2.6 $\pm$ 0.2              &     2.4 $\pm$ 0.2            &     1.9 $\pm$ 0.1            &      1.1 $\pm$ 0.1              \\

\hline
\end{tabular}
\end{center}
\end{table}



%Additional Information for Model Testing
\section{Additional Information for Model Testing}
\label{sec:outreach}

{\bf UPDATE NUMBERS BELOW}

Other models of new physics in the dilepton final state can be confronted in an approximate way by simple 
generator-level studies that compare the expected number of events in \lumi
with the upper limits from Section~\ref{sec:limit}.
The key ingredients of such studies are the kinematic requirements described 
in this note, the lepton efficiencies, and the detector responses for and \met.
%
The muon identification efficiency is $\approx 96\%$;
the electron identification efficiency varies approximately linearly from $\approx$ 60\% at 
$\pt = 10\GeVc$ to 90\% for $\pt > 30\GeVc$.  
%
The lepton isolation efficiency depends on the lepton momentum, as well as on the jet activity in the 
event.
In $t\bar{t}$ events, it varies approximately linearly from $\approx 73\%$ (muons)
and $\approx 82\%$ (electrons) at $\pt=10\GeVc$ to $\approx 97\%$ for $\pt>60\GeVc$. 
In LM4 (LM8) events, this efficiency is decreased by $\approx$X\% ($\approx$X\%)over the whole momentum spectrum.
%
The average detector responses (the reconstructed quantity divided by the generated quantity) 
for \met\ is consistent with 1 within the 5\% jet energy scale uncertainty.
The experimental resolutions on this quantity is X\%.





%model-dependent limits
%
\section{Model-Dependent Limits}
\label{sec:models}

As an example of how the upper limit presented in Sec.~\ref{sec:upperlimit} can be
used to test if a specific model is excluded, we consider the benchmark
SUSY processes LM4 and LM8 which contain
$Z$ bosons produced in the cascade decays of SUSY particles. 
We place upper limits on the quantity \sta,
assuming efficiencies and uncertainties from these processes, and compare them to 
the expected values of \sta.
Here $\sigma$ is the signal production cross section, %$BF$ is the branching fraction
%to the final state Z + jets + MET and $A$ is the signal acceptance.
and $A$ is the signal acceptance which includes the branching fraction to the final state Z + $\ge$ 2 jets + MET $\ge$ \signalmett.

The signal event yield $N_{SIG}$ can be expressed as:

\begin{equation}
%N_{SIG} = \sigma \times BF \times A \times \epsilon \times \mathcal{L},
N_{SIG} = \sigma \times A \times \epsilon \times \mathcal{L},
\end{equation}

and we therefore have:

\begin{equation}
N_{SIG}/( \epsilon \times \mathcal{L}) = \sigma \times A.
\end{equation}

Here  $\epsilon$ is the signal efficiency and $\mathcal{L}$ is the integrated luminosity. 
Since we wish to place an upper limit on the quantity %$\sigma \times BF \times A$
\sta, we must evaluate
the quantity $N_{SIG}/(\epsilon \times \mathcal{L})$.  Because of the efficiency
in the denominator, this upper limit cannot be calculated in an entirely model-independent
way; rather, it must be calculated with respect to a specific model. We therefore evaluate
the upper limit on %$\sigma \times BF \times A$ 
\sta\
with respect to LM4 and LM8.  
For each process, we calculate the efficiency as discussed
%in the following subsections.
in the next subsection, and
its uncertainty as described in section \ref{sec:signaleffuncer}.


\subsection{Signal Efficiencies}
\label{sec:sigeff}

We evaluate the signal efficiencies using MC. To evaluate these efficiencies for our 
sample processes, 
we take as our denominator the number of generated events which pass the following selection at
the generator level:

\begin{itemize}
\item Two electrons or muons with $p_T>20$~GeV and $|\eta|<2.5$
\item Opposite-sign, same-flavor pair with $81 < M(ll) < 101$~GeV
\item At least two genjets with $p_T > 30$~GeV, $|\eta|<3$, $\Delta R > 0.4$ from any selected lepton as defined above
\end{itemize}

For the %loose (tight) 
signal region efficiency,
we add to this the requirement genmet
%$>60$ (120)~GeV.
$>$ \signalmetl (\signalmett) for the loose (tight) signal regions.
We take as our numerator the number of events passing the above generator-level selection which
also pass the signal region selection at reco-level. 
Efficiencies are shown in table \ref{tab:models}.

%{\bf Should I also consider events passing reco-selection which do NOT pass gen-level selection? For ttbar this
%\bf is a large effect, it is small for LM4 and LM8. }
%For the loose (tight) signal region we find efficiencies of 43\% (42\%), 54\% (55\% )and 45\% (48\%) 
%for $t\bar{t}$, LM4 and LM8, respectively.
%This efficiency is dominated by the dilepton selection efficiency, which varies from $\sim0.6-0.7$ for the chosen 
%processes. 

%The efficiency for selected dileptons to satisfy the dilepton mass requirement is $\sim0.9$;
%the efficiency to satisfy the requirements on the jet multiplicity is $\sim0.8-0.9$, for the MET$>60$~GeV requirement
%this efficiency is $\sim0.9-1$. 



\subsection{Upper Limits on \sta}

Armed with the efficiencies and uncertainties calculated in the previous 2 sections, we proceed
to calculate upper limits on the quantity \sta.
We calculate Bayesian 95\% CL upper
limits using the cl95cms software, assuming a log-normal model of nuissance parameter integration.
We also calculate the quantity \sta ~for LM4 and LM8, for which we assume LO
cross-sections of 1.88~pb and 0.73~pb and calculate k-factors for each event, depending on the sub-process.
The quantity $A$ is taken to be the fraction of the total number of generated events which
pass the generator-level selection given in Sec.~\ref{sec:sigeff}. The results are summarized in
Table~\ref{tab:models}. 
These results show that the benchmark point LM4 is excluded, while
LM8 remains beyond the
sensitivity of this search with the current integrated luminosity.

\begin{table}[hbt]
  \begin{center}
	\caption{
	  \label{tab:models} 
	  Summary of efficiencies, efficiency uncertainties (quadrature sum
	  of jet/MET, dilepton and luminosity uncertainties), 
	  and upper limits on \sta\ %$\sigma \times BF \times A$ 
	  for the tight (MET $>$ \signalmett~GeV, top)
	  and loose (MET $>$ \signalmetl~GeV, bottom) signal regions.
	  We also show the quantity
	  \sta\ %$\sigma \times BF \times A$ 
	  for LM4 and LM8.}

	\medskip
	\begin{tabular}{l|cccc}
	  \hline
	  \bf MET $>$ \signalmett~GeV & efficiency (\%) & acceptance (\%) & UL(\sta)(fb) & \sta(fb) \\
	  \hline
          %BAYES
	  %LM4  &  50 $\pm$   6  &  0.84  &  14  &  23  \\
	  %LM8  &  43 $\pm$   5  &  0.98  &  16  &  11  \\
          %CLs
	  LM4  &  50 $\pm$   6  &  0.84  &  13  &  23  \\
	  LM8  &  43 $\pm$   5  &  0.98  &  15  &  11  \\

	  \hline
	\end{tabular}

	\medskip
	\begin{tabular}{l|cccc}
	  \hline
	  \bf MET $>$ \signalmetl~GeV & efficiency (\%) & acceptance (\%) & UL(\sta)(fb) & \sta(fb) \\
	  \hline
          %BAYES
	  %LM4 &  53 $\pm$   3  &  1.4  &   40  &   37  \\
	  %LM8 &  44 $\pm$   3  &  1.7  &   47  &   19  \\
          %CLs
	  LM4 &  53 $\pm$   3  &  1.4  &   39  &   37  \\
	  LM8 &  44 $\pm$   3  &  1.7  &   47  &   19  \\

	  \hline
	\end{tabular}

	%%%%%%%%%%%%%%%%%%%%%%%%%%%%%%%%%%%%%%%%%%%%%%
	%2010 results--35/pb
	\begin{comment}

	  \begin{tabular}{l|ccc}

		\hline
		& $t\bar{t}$    & LM4     & LM8       \\
		\hline
			{\bf Loose signal region}                &               &         &           \\
			Efficiency                               &       0.43    & 0.54    & 0.45      \\
			Efficiency Uncertainty                   &       0.15    & 0.12    & 0.12      \\
			UL($\sigma \times BF \times A$) (pb)     &       0.56    & 0.44    & 0.53      \\
			$\sigma \times BF \times A$ (pb)         &               & 0.045   & 0.025     \\

			\hline
				{\bf Tight signal region}                &               &         &           \\
				Efficiency                               &       0.42    &  0.55   & 0.48      \\
				Efficiency Uncertainty                   &       0.26    &  0.13   & 0.13      \\
				UL($\sigma \times BF \times A$) (pb)     &       0.23    &  0.17   & 0.19      \\
				$\sigma \times BF \times A$ (pb)         &               &  0.035  & 0.018     \\
				
				\hline

	  \end{tabular}

	\end{comment}

  \end{center}
\end{table}






%Double_t CL95(Double_t ilum, Double_t slum, Double_t eff, Double_t seff, 
%              Double_t bck, Double_t sbck, Int_t n, Bool_t gauss = kFALSE, Int_t nuisanceModel = 0)

%---------
%ttbar
%---------

%root [5] CL95( 34 , 0. , 0.43 , 0.15*0.43 , 6.04 , 0.96 , 7 , false , 1 )    
%Poisson 95% CL limit with Lognormal nuisance parameter integration will be used
%Likelihood function is evaluated over [0,2] 
%likelihood normalization: 0.0498808
%Info in <TCanvas::Print>: eps file Likelihood.eps has been created
%Upper 95% C.L. limit on signal = 0.561523 pb
%(Double_t)5.61523437500000000e-01

%---------
%LM4
%---------

%root [6] CL95( 34 , 0. , 0.54 , 0.12*0.54 , 6.04 , 0.96 , 7 , false , 1 )    
%Poisson 95% CL limit with Lognormal nuisance parameter integration will be used
%Likelihood function is evaluated over [0,2] 
%likelihood normalization: 0.0395872
%Info in <TCanvas::Print>: eps file Likelihood.eps has been created
%Upper 95% C.L. limit on signal = 0.44043 pb
%(Double_t)4.40429687500000000e-01

%---------
%LM8
%---------

%root [7] CL95( 34 , 0. , 0.45 , 0.12*0.45 , 6.04 , 0.96 , 7 , false , 1 )
%Poisson 95% CL limit with Lognormal nuisance parameter integration will be used
%Likelihood function is evaluated over [0,2] 
%likelihood normalization: 0.0475047
%Info in <TCanvas::Print>: eps file Likelihood.eps has been created
%Upper 95% C.L. limit on signal = 0.52832 pb
%(Double_t)5.28320312500000000e-01


\section{Conclusion}
\label{sec:conclusion}

We have performed a search for BSM physics in the Z plus jets plus MET final state.
Backgrounds from SM \Z production were estimated using the data-driven
MET templates method, and backgrounds from $t\bar{t}$ were estimated using
the data-driven opposite-flavor subtraction technique. We found no evidence
for anomalous yield beyond SM expectations and placed \statistics\ 95\% CL upper limits
on the non SM yields in the loose (MET$>$\signalmetl~GeV) and tight signal regions (MET$>$\signalmett~GeV)
of \ulloose~and \ultight~events, respectively. 
%We also quoted expected yields for the 
We also quote upper limits on the quantity
%$\sigma \times BF \times A$,
\sta\ for the
%assuming efficiencies and uncertainties from the 
benchmark SUSY processes LM4 and LM8
including estimated signal efficiencies and systematics, 
and conclude that LM4 is not compatible with the data and therefore ruled out.


\clearpage
\begin{thebibliography}{10}
%  \bibitem {NOTE000} {\bf CMS Note 2005/000},
%    X.Somebody et al.,
%    {\em "CMS Note Template"}.

\bibitem{ref:GenericOS} D. Barge {\em et al.}, AN-CMS2010/370  

\bibitem{ref:templates1} V.~Pavlunin, Phys. Rev. {\bf D81}, 035005 (2010).
    
\bibitem{ref:templates2} V.~Pavlunin, CMS AN-2009/125

\bibitem{ref:top} A reference to the top paper, once it is submitted.  Also
D. Barge {\em et al.}, AN-CMS2010/258.  

\bibitem{ref:toptwiki} Changes to the selection for the 38x CMSSW release are given in 
{\tt https://twiki.cern.ch/twiki/bin/viewauth/CMS/TopDileptonRefAnalysis2010Pass5}.

\bibitem{ref:vbtf} https://twiki.cern.ch/twiki/bin/viewauth/CMS/SimpleCutBasedEleID

\bibitem{ref:eghlt} https://twiki.cern.ch/twiki/bin/viewauth/CMS/EgammaWorkingPointsv3

\bibitem{ref:conv} D.~Barge {\em at al.}, AN-CMS2009/159.

\bibitem{ref:xsec} https://twiki.cern.ch/twiki/bin/viewauth/CMS/CrossSections\_3XSeries, 
%https://twiki.cern.ch/twiki/bin/view/CMS/ProductionReProcessingSpring10 
https://twiki.cern.ch/twiki/bin/view/CMS/ProductionSpring2011
% {\color{red} Is this the right reference?}


\bibitem{ref:ourvictory} D.~Barge {\em at al.}, AN-CMS2009/130.

\bibitem{ref:dy} W.~Andrews {\em et al.}, AN-CMS2009/023.

\bibitem{ref:FR} D.~Barge {\em at al.}, AN-CMS2010/257.

\end{thebibliography}


\clearpage
\appendix
\section{Triggers}
\label{app:trigsel}

The detailed list of triggers for selecting dilepton events is (with the version number omitted for simplicity):


\begin{itemize}
\item double-muon triggers
  \begin{itemize}
  \item \verb=HLT_DoubleMu7_v=
  \item \verb=HLT_Mu13_Mu7_v=
  \end{itemize}
\item double-electron triggers
  \begin{itemize}
  \item \verb=HLT_Ele17_CaloIdL_CaloIsoVL_Ele8_CaloIdL_CaloIsoVL_v=
  \item \verb=HLT_Ele17_CaloIdT_TrkIdVL_CaloIsoVL_TrkIsoVL_Ele8_CaloIdT_TrkIdVL_CaloIsoVL_TrkIsoVL_v=
  \end{itemize}
\item e-$\mu$ cross triggers
  \begin{itemize}
  \item \verb=HLT_Mu17_Ele8_CaloIdL_v=
  \item \verb=HLT_Mu8_Ele17_CaloIdL_v=
  \end{itemize}
\end{itemize}


%%%%%%%%%%%%%%%%%%%%%%%%%%%%%%%%%%%%%%
%2010 triggers

%\begin{itemize}
%\item single-muon triggers
%  \begin{itemize}
%  \item \verb=HLT_Mu5=
%  \item \verb=HLT_Mu7=        
%  \item \verb=HLT_Mu9=         
%  \item \verb=HLT_Mu11=       
%  \item \verb=HLT_Mu13_v1=     
%  \item \verb=HLT_Mu15_v1=     
%  \item \verb=HLT_Mu17_v1=     
%  \item \verb=HLT_Mu19_v1=     
%  \end{itemize}
%\item double-muon triggers
%  \begin{itemize}
%  \item \verb=HLT_DoubleMu3=
%  \item \verb=HLT_DoubleMu3_v2=
%  \item \verb=HLT_DoubleMu5_v1=
%  \end{itemize}
%\item single-electron triggers
%  \begin{itemize}
%  \item \verb=HLT_Ele10_SW_EleId_L1R=
%  \item \verb=HLT_Ele10_LW_EleId_L1R=
%  \item \verb=HLT_Ele10_LW_L1R=
%  \item \verb=HLT_Ele10_SW_L1R=
%  \item \verb=HLT_Ele15_SW_CaloEleId_L1R=
%  \item \verb=HLT_Ele15_SW_EleId_L1R=
%  \item \verb=HLT_Ele15_SW_L1R=
%  \item \verb=HLT_Ele15_LW_L1R=
%  \item \verb=HLT_Ele17_SW_TightEleId_L1R=
%  \item \verb=HLT_Ele17_SW_TighterEleId_L1R_v1=
%  \item \verb=HLT_Ele17_SW_CaloEleId_L1R=
%  \item \verb=HLT_Ele17_SW_EleId_L1R=
%  \item \verb=HLT_Ele17_SW_LooseEleId_L1R=
%  \item \verb=HLT_Ele17_SW_TighterEleIdIsol_L1R_v2=
%  \item \verb=HLT_Ele20_SW_L1R=
%  \item \verb=HLT_Ele22_SW_TighterEleId_L1R_v2=
%  \item \verb=HLT_Ele32_SW_TightCaloEleIdTrack_L1R_v1=
%  \item \verb=HLT_Ele32_SW_TighterEleId_L1R_v2=
%  \item \verb=HLT_Ele27_SW_TightCaloEleIdTrack_L1R_v1=
%  \item \verb=HLT_Ele22_SW_TighterCaloIdIsol_L1R_v2=
%  \item \verb=HLT_Ele22_SW_TighterEleId_L1R_v3=
%  \item \verb=HLT_Ele22_SW_TighterCaloIdIsol_L1R_v2=
%  \end{itemize}
%\item double-electron triggers
%  \begin{itemize}
%  \item \verb=HLT_DoubleEle15_SW_L1R_v1=                 
%  \item \verb=HLT_DoubleEle17_SW_L1R_v1=   
%  \item \verb=HLT_Ele17_SW_TightCaloEleId_Ele8HE_L1R_v1=
%  \item \verb=HLT_Ele17_SW_TightCaloEleId_SC8HE_L1R_v1=
%  \item \verb=HLT_DoubleEle10_SW_L1R=
%  \item \verb=HLT_DoubleEle5_SW_L1R=
%  \end{itemize}
%\item e-$\mu$ cross triggers
%  \begin{itemize}
%  \item \verb=HLT_Mu5_Ele5_v1=
%  \item \verb=HLT_Mu5_Ele9_v1=
%  \item \verb=HLT_Mu11_Ele8_v1=
%  \item \verb=HLT_Mu8_Ele8_v1=
%  \item \verb=HLT_Mu5_Ele13_v2=
%  \item \verb=HLT_Mu5_Ele17_v1=
%  \end{itemize}
%\end{itemize}

%%\section{Trigger efficiency}
%\label{sec:trgEff}

We rely on a
mixture of single- and double-lepton triggers.  The trigger
efficiency is very high (typically in excess of 99\%) since
each of the two leptons can fire a single-lepton trigger.
%We apply to MC events a simplified model of the trigger efficiency 
%as a function of dilepton species, rapidity and $p_T$.

% Add a paragraph:
% Describe MC generation & simulation, event reconstruction program:
%
% PYTHIA and MADGRAPH event generators
% GEANT4 simulation
% CMS event reconstruction program applied to both data and MC
\clearpage

\section{\MET\ Templates from \gjets\ Sample}
\label{app:templates}

In this section we display the templates used for the inclusive analysis (red) and the targeted analysis (blue).

\begin{figure}[!h]
\begin{center}
\begin{tabular}{cc}
\includegraphics[width=0.5\textwidth]{plots/template_targeted_0_19p5fb.pdf}
\end{tabular}
\caption{
\MET\ templates collected with the \pt $>$ 22 GeV single photon trigger.
The number in red (blue) indicates the number of entries in the template for the inclusive (targeted) analysis.
}
\end{center}
\end{figure}

\clearpage

\begin{figure}[!h]
\begin{center}
\begin{tabular}{cc}
\includegraphics[width=0.5\textwidth]{plots/template_targeted_1_19p5fb.pdf}
\end{tabular}
\caption{
\MET\ templates collected with the \pt $>$ 36 GeV single photon trigger.
The number in red (blue) indicates the number of entries in the template for the inclusive (targeted) analysis.
}
\end{center}
\end{figure}

\clearpage

\begin{figure}[!h]
\begin{center}
\begin{tabular}{cc}
\includegraphics[width=0.5\textwidth]{plots/template_targeted_2_19p5fb.pdf}
\end{tabular}
\caption{
\MET\ templates collected with the \pt $>$ 50 GeV single photon trigger.
The number in red (blue) indicates the number of entries in the template for the inclusive (targeted) analysis.
}
\end{center}
\end{figure}

\clearpage

\begin{figure}[!h]
\begin{center}
\begin{tabular}{cc}
\includegraphics[width=0.5\textwidth]{plots/template_targeted_3_19p5fb.pdf}
\end{tabular}
\caption{
\MET\ templates collected with the \pt $>$ 75 GeV single photon trigger.
The number in red (blue) indicates the number of entries in the template for the inclusive (targeted) analysis.
}
\end{center}
\end{figure}

\clearpage

\begin{figure}[!h]
\begin{center}
\begin{tabular}{cc}
\includegraphics[width=0.5\textwidth]{plots/template_targeted_4_19p5fb.pdf}
\end{tabular}
\caption{
\MET\ templates collected with the \pt $>$ 90 GeV single photon trigger.
The number in red (blue) indicates the number of entries in the template for the inclusive (targeted) analysis.
}
\end{center}
\end{figure}


\end{document}



